\chapter{Practice Sessions}

\section{General Procedures and Rules}

	Buggy freeroll and push practice sessions shall be governed by any common rules
	between Sweepstakes races and practices as well as the following general
	procedures and rules:
	\newline

	No buggy freeroll or street push practice sessions of any type shall be held at
	Carnegie Mellon University at any time without the approval and authority
	of the following: The Police Department of the City of Pittsburgh, The
	Department of Parks and Recreation of the City of Pittsburgh, The Carnegie
	Mellon University Campus Police, The Sweepstakes Advisor, The Sweepstakes
	Chairman, The Safety Chairman.

	Buggy practice sessions shall only be held at those times and places authorized
	by the Sweepstakes Advisor.

	All persons participating in any activity directly or indirectly related to the
	Sweepstakes Competition at Carnegie Mellon University are considered to be
	representatives of Carnegie Mellon University. As such, these persons are
	expected and required to act in a responsible and orderly manner whenever they
	are involved in any such activity, especially when they are interacting with
	any segment of the public at large, the Pittsburgh Police, or the Carnegie
	Mellon University Campus Police.

	Sweepstakes races can only be held with the cooperation and approval of the
	City of Pittsburgh and the people of Pittsburgh. Any irresponsible action on
	the part of anyone involved with the Sweepstakes Competition could jeopardize
	this cooperation and approval. Any individual or organization found to be
	acting in an irresponsible manner while participating in any activity directly
	or indirectly related to the Sweepstakes Competition at Carnegie Mellon
	University may be barred from any further participation in that activity by the
	Dean of Student Affairs or the Sweepstakes Advisor.

	Any organizations found to be conducting or otherwise participating in any
	unauthorized freeroll practice or push practice on the public streets in
	Schenley Park and/or on the Carnegie Mellon University campus shall be
	penalized by being barred from ALL Sweepstakes activities (including both
	practices and competitions) for a period of one year from the date of the
	occurrence of that violation.

\subsection{Buggies}
\label{subsec:Freeroll Buggies}

	The following procedures and rules shall apply to buggies during practice
	sessions:
	\newline

	\noindent No buggy shall be permitted to participate in any freeroll practice or push
	practice session at Carnegie Mellon University unless it successfully satisfies all of
	the following requirements:

	\begin{itemize}

		\item Safety inspection.

		\item Braking Capability and Field of Vision test with the driver for that practice.

		\item Driver familiarity with the buggy by being pushed around prior to freeroll practice on sidewalks, in parking lots, etc.

	\end{itemize}

	Any organization that permits any of its buggies to participate in any freeroll
	practice or push practice in any semester, before that buggy has successfully
	completed a safety inspection that same semester shall be penalized by having
	one entry withdrawn, as described in Disciplinary Actions
	(sec. \ref{sec:DiciplinaryActions} pg. \pageref{sec:DiciplinaryActions}).
	In addition, the entry fee for that entry shall be forfeited.

	Any organization that fails to provide all of the required lights and
	reflectors, in proper working order, for any of its buggies that are being used
	between sunset and sunrise shall be fined the amount of \$20.00 for each buggy
	that is not properly equipped.

\subsection{Safety}

\subsubsection{Safety Lighting}

	Each buggy shall have an operating light attached to it at a location that is
	within 12 inches of the highest point of that buggy (usually the pushbar or the
	top of the buggy shell), whenever that buggy is being used during any type of
	practice session that takes place at Carnegie Mellon University between sunset
	and sunrise. This requirement is applicable to push practices, activities
	before freeroll practices, drop brake tests, braking capability tests, and
	whenever drivers are being familiarized with their buggies by being pushed
	around on sidewalks, in parking lots, etc. The light shall be white in color
	and it shall be aimed approximately parallel to the ground and in the forward
	direction. The light shall have sufficient brightness such that it is visible
	from a distance of 550 feet, which is the approximate distance from the finish
	line to the end of the Hill 4-5 Transition zone.

	Each buggy being used between sunset and sunrise shall also have some type of
	reflective material that is at least four square inches in area attached to the
	rear of that buggy in order to reflect the headlights of vehicles that approach
	it from behind. This material may be white, red, orange, or yellow in color and
	it must be approved by the Safety Chairman.

	Each organization is encouraged to provide additional safety lights for each of
	its buggies, such as red lights aimed toward the rear, flashing red or orange
	lights anywhere on the buggies, and flashing strobe type lights on the tops of
	the buggies.

\subsubsection{Pushers}

	All pushers are encouraged to wear reflective vests or other clothing which has
	some type of reflective material on it whenever they are participating in a
	night time push practice, so that they will be more visible to vehicular
	traffic while they are pushing a buggy.

\subsection{Cancellation}

	Any push practice session may be canceled at any time by the Sweepstakes
	Chairman, or anyone designated by that Chairman, due to inclement weather,
	inadequate police protection, inadequate communications, lack of medical
	personnel, vehicles obstructing the course or any other condition which might
	endanger the participants or spectators of that practice session.


\section{Freeroll Practice Procedures and Rules}
\label{sec:FreerollRules}

\subsection{Common Rules and Procedures}
	Some rules and procedures are common to freeroll practice and Raceday.
	Please see Race and Freeroll Common Rules and Procedures, in sec. \ref{ch:CommonRules} pg \pageref{ch:CommonRules}.
\subsection{Time and Place}

	Freeroll practices will generally be held between the hours of 6:00 am and
	9:00 am on Saturday and Sunday mornings, starting as soon as there is enough
	light to roll safely. Fall freeroll practices will usually be scheduled for
	six weekends sometime between late September and late November each school
	year. Spring freeroll practices will usually be scheduled for the six weekends
	immediately preceding the weekend scheduled for the races.

\subsection{Rollboard}

	At the discretion of the Sweepstakes Chairman, the responsibility of moving a
	portable chalkboard or dry-erase board to the sidewalk near the top of Hill 2,
	before the start of each freeroll practice, and returning it after the practice
	is over, may be delegated to one organization. This board shall be used to post
	the rolling order for freeroll practices.

\subsection{Drop Tests and Safety Equipment Check}

	Before each freeroll practice session starts, each buggy is required to take a
	drop brake test as follows:

	No buggy shall be permitted to participate in a freeroll practice unless it has
	successfully completed a drop brake test earlier that same day. Each buggy's
	performance in the drop brake test shall be recorded by the test administrator
	at the time that the buggy is tested.

	Any buggy that fails the drop brake test on each of two consecutive attempts
	before a freeroll practice session, shall be required to again successfully
	complete a braking capability test, with any of its drivers, before that buggy
	will be permitted to participate in any subsequent freeroll or push practice
	sessions.

	At the time that the drop brake tests are given, each participating buggy and
	driver may be checked by the test administrator to ensure that they have proper
	head protection, eye protection, hand protection, adequate field of vision, and
	safety harness.

	Drop brake tests will be administered by the Safety Chairman, or anyone
	designated by that Chairman, on the morning of each scheduled freeroll
	practice. The starting time for these tests shall be specified by the Safety
	Chairman, or anyone designated by that Chairman, sometime before the day of
	that freeroll practice.

\subsection{Rolling Order}

	The order in which the organizations who are participating in a freeroll
	practice session will be permitted to freeroll their buggies should usually be
	determined by the following procedure:

	Before the first scheduled freeroll practice of each school semester, the
	Sweepstakes Chairman shall compile a list of all of the organizations wishing
	to participate in freeroll practices. This list shall be in a random order
	which shall be determined by a lottery method chosen by the Chairman. The order
	of this list shall be the rolling order for the first freeroll practice session
	of that semester. For each subsequent freeroll practice session of that
	semester, the rotation of the rolling order shall remain the same and the
	organization that rolls first shall be the organization that was next in the
	order to roll at the previous freeroll practice session when that practice
	session ended.

	Changes to this rolling order procedure may be made for individual freeroll
	practices at the discretion of the Sweepstakes Chairman with the approval of
	the Sweepstakes Committee.

\subsection{Rolling Procedure}

	Before and after each organization freerolls its buggies, the Sweepstakes
	Chairman, or anyone designated by that Chairman, will announce which
	organization is next in order to freeroll its buggies. The next organization in
	the rolling order shall be designated as being ``UP'', the second as being ``ON
	DECK'', and the third ss being ``IN THE HOLE''. When the Chairman determines
	that the course is clear of all vehicles and buggies, he or she will signal the
	``UP'' organization that it may roll its buggies. After this signal, that
	organization will have 15 seconds in which to get its first buggy started, (if
	that organization is running Hills 1 and 2 it will have 45 seconds from the
	signal to get its first buggy over Hill 2 and into the freeroll portion of the
	course).

	After an organization's first buggy is into the freeroll portion of the course,
	that organization will have an additional 30 seconds for each of its remaining
	buggies in order to get all of those buggies into the freeroll portion of the
	course. After an organization's last buggy is into the freeroll portion of the
	course, an additional 15 seconds will be allowed to get its follow car into the
	freeroll portion of the course. If an organization exceeds this total time
	limit, that organization will not be permitted to freeroll any of its buggies
	on its next turn in the rolling order, even if that turn isn't until a freeroll
	practice session at a later date.

	After an organization rolls its buggies at a freeroll practice session, the
	next organization in the rolling order will not be given the signal to start
	freerolling its buggies until the previous organization's follow car is past
	the chute.

\subsection{Follow Cars}
\label{subsec:Follow Cars}

	During each freeroll practice session, each participating organization may
	provide a motor vehicle known as a follow car which will drive around the buggy
	course directly behind that organization's last buggy each time that
	organization freerolls its buggies.

    If an organization does not provide a follow car, they must inform Sweepstakes
    before they are cleared. The team must have a member of the organization at
    the top of the hill with an extraction kit while their buggies are rolling. In
    the event of a stop or accident, this member will drive with the Safety
    Chairman to the site of the incident. A team's roll will be scratched for
    the remainder of the day and fined \$15.00 unless Sweepstakes has confirmed
    that there is a representative with an extraction kit at the top of the hill.

    For teams using a follow car, the follow car must be ready to leave when the organization
    receives the signal to start rolling its buggies from the Sweepstakes Chairman. The
    follow car must leave the Hill 2 lanes area of the course and be following the last buggy within
    15 seconds of when that buggy is released into freeroll by its pusher. If this time
    limit is exceeded, the rolling organization will not be permitted to freeroll any
    of its buggies on its next turn in the rolling order, even if that turn isn't until
    a freeroll practice session at a later date.

	The follow car should stay approximately one hundred feet behind the last buggy.
    Several members of the freerolling organization should ride in the follow car so
    that they may observe the buggies from that vantage point and
	provide any assistance that may be necessary during that freeroll. In the event
	of an accident during a freeroll, the follow car should stop near the scene of
	the accident in order to render assistance to those involved in the accident.
	The members of the freerolling organization riding in the follow car MUST have
	any tools or devices that are necessary to quickly remove any of their drivers
	from their buggies. If an organization fails to have these tools or devices in
	the car following the buggies during any freeroll, that organization shall be
	fined the amount of \$15.00 and shall not be permitted to freeroll any of its
	buggies for the remainder of that day.

	Follow cars should not be stopped on the buggy course during freeroll practice
	sessions unless a buggy has stopped during its roll. Stopping a follow car on
	the course in order to talk to people or to allow people to enter or leave the
	car is strictly prohibited. Stopping a follow car while a freeroll practice is
	in progress creates a potential safety hazard and also causes delays to the
	buggies waiting to roll.

	If the vehicle used as the follow car is a pick-up truck and there are people
	riding in the back of that truck, its tail gate must always be closed while the
	vehicle is moving in order to reduce the chances of anyone falling out of it.

	At the discretion of the Sweepstakes Chairman, one or more organizations may be
	appointed to provide follow cars (with drivers) to follow the buggies of other
	organizations, either because they do not have access to a follow car vehicle
	during freeroll practices, or in order to reduce confusion and safety hazards
	caused by too many different follow cars during freeroll practices.

\subsection{Medical Personnel}

	If medical personnel are available during freeroll practices, such as off-duty
	paramedics from the City of Pittsburgh or members of an active Emergency
	Medical Service that might exist on the Carnegie Mellon University campus, they
	should be stationed near the chute portion of the buggy course, since that is
	the most likely area in which an accident might occur.

\subsection{Safety Checks}

	Each freeroll practice session will be observed by the Safety Chairman, and/or
	anyone designated by that Chairman. At any time during a freeroll practice
	session this observer may check for any or all of the following:

	\begin{itemize}

		\item Safety of any practicing buggy by performing a spot safety check.

		\item All required barricades and warning signs are in place.

		\item No-Parking signs are in place.

		\item Each organization has provided the correct number of
		properly equipped barricaders to help control vehicular traffic.

		\item Each practicing organization has an adequate number of
		people to properly attend to all of the buggies that they are using.

		\item All practicing buggy drivers are properly qualified to
		drive.

		\item An adequate number of hay bales are in place.

		\item Each practicing organization has a properly equipped chute flagger.

	\end{itemize}

\subsection{Drivers}

	The following procedures and rules shall apply to buggy drivers during freeroll
	practice sessions:
	\newline

	\noindent No driver shall be permitted to participate in a freeroll practice unless he or
	she complies with all of the following requirements:

	\begin{itemize}

		\item
		He or she has walked the buggy course earlier that same day. Each driver must
		check in with the Assistant Sweepstakes Chairman, or anyone designated by that
		Chairman, immediately before or after they walk the course. Drivers are only
		permitted to walk the course BEFORE a freeroll practice session starts.

        \item
        If a driver is new, they must complete their first course walk with a veteran driver.

        \item
        If a driver is new, they must observe another driver's complete freeroll from
        the view of a follow car.

		\item
		He or she has previously successfully completed a braking capability test in
		the buggy that he or she will be driving in that freeroll practice.

		\item
		He or she has previously successfully completed a field of vision test in the
		buggy that he or she will be driving in that freeroll practice.

		\item
		He or she has previously become familiar with the operation of the buggy that
		he or she will be driving in that freeroll practice by being pushed around (on
		sidewalks, in parking lots, etc.) in that same buggy.

		\item
		He or she has either attended all of the driver education meetings that have
		been held that school year, or has reviewed the content of those meetings with
		someone designated by the Sweepstakes Chairman.

	\end{itemize}

	Each driver is required to complete a drop brake test before he or she
	participates in a freeroll practice session in the buggy that he or she will be
	driving at that session. The purpose of this test is to ensure
	that the buggy's braking system is both properly adjusted for the
	current driver and is in proper working order.

\section{Push Practice Procedures and Rules}
\label{sec:PushPRules}

\subsection{Common Rules and Procedures}
	Some rules and procedures are common to freeroll practice, Raceday, and Push Practice.
	Please see Race and Freeroll Common Rules and Procedures, in sec. \ref{ch:CommonRules} pg. \pageref{ch:CommonRules}.

\subsection{Time and Place}

	Street push practices shall only be held on Tech Street and Frew Street and
	only on the dates and at the times specified by the Sweepstakes Advisor.

	Street push practices will generally be held between the hours of 12:00
	midnight and 6:00 am, Mondays through Fridays, for approximately six weeks
	immediately preceding the date scheduled for the preliminary races each year.
	These days and times shall be determined by the Sweepstakes Advisor.

\subsection{Notification}

	The Sweepstakes Chairman shall notify the Carnegie Mellon University Campus
	Police Department when Push practice is scheduled to begin for the year. Each
	organization shall have a copy of the permits for push practice in case they
	are questioned by campus or city police. In addition, the campus community
	itself must be made aware of push practices. A post should be made to
	misc.market and a memo should be distributed to the faculty and staff so that
	they are knowledgeable concerning when and where push practices will be in
	progress.


	Several times during the first two weeks of street push practices,
	informational flyers shall be distributed to all of the motor vehicles parked
	along Tech Street and Frew Street, by placing these flyers under the windshield
	wipers of these vehicles. These informational flyers shall be provided by the
	Sweepstakes Chairman or the Sweepstakes Advisor, and they shall inform the
	drivers of the parked vehicles that street push practices are now in progress
	and that caution must be observed when driving in that area until the races are
	over. These flyers shall be distributed by members of one or more
	organizations, as assigned by the Sweepstakes Chairman.

\subsection{Barricades and Warning Signs}
\label{subsec:Freeroll Barricades}

	Portable wooden barricades with warning signs shall be placed at several
	locations near Tech Street and Frew Street while street push practices are in
	progress, to warn and/or stop vehicular traffic when this traffic tries to
	approach these streets. The signs shall indicate that the road is, or may be,
	closed, and that there will be barricaders to stop and/or redirect traffic. The
	barricades and signs shall be put in place immediately before any street push
	practices begin, by the participants of those practices. They shall be removed
	within 15 minutes of the end of those practices by those same participants.
	The last organization to finish practicing on that hill is required to put
	the barricades and signs away, regardless of who set them out.

	In order to provide the maximum amount of protection to the push practice
	participants, barricades should be placed at the following locations:

	\begin{itemize}

		\item Margaret Morrison Street, at its intersection with Tech Street.

		\item Frew Street on the eastern side of its intersection with Tech
		Street.

		\item Tech Street, just north of its intersection with Schenley Drive.

		\item Frew Street, just north of its intersection with Schenley Drive.

		\item Scaife Hall driveway at its intersection with Frew Street.

	\end{itemize}

	Any organization found to be conducting a push practice session without a
	sufficient number of barricades in place, shall be fined the amount of \$15.00.

\subsection{Barricaders}

	Each organization conducting a street or sidewalk push practice session must
	have a minimum of four barricaders present in order to control vehicular traffic
	on Tech Street and Frew Street. Each organization shall equip their barricaders
	with reflective vests and flags which must be used while they are acting as
	barricaders. All vests and flags must be approved by the Safety Chairman or anyone
	designated by that Chairman. In addition, all barricaders at night time practices
	are advised to use flashlights, preferably with illuminated orange extensions,
	in order to make themselves more visible to vehicular traffic. barricaders are
	also advised to wear orange, yellow, or white construction type hard hats, so
	that they may appear to be more official-looking to vehicular traffic.

	When more than one organization is practicing on a hill or on adjacent hills,
	they may combine their barricaders in order to save manpower. However, if one
	organization leaves, any barricaders that leave with them must be replaced by the
	remaining organizations before they can continue practicing. In any event,
	barricaders from different organizations MUST cooperate with each other in order
	to control traffic. Barricaders should be positioned such that there is at least
	one barricader at each end, and one barricader in the middle, of each hill that is
	being used for practice. The barricader in the middle of the hill should be
	responsible for any vehicles that pull out of parking spaces along the hill.

	Any organization that fails to provide the required number of properly equipped
	barricaders during any push practice shall be fined the amount of \$15.00 for each
	missing or improperly equipped barricader, shall not be permitted to continue
	their push practice session that day, and shall not be permitted to participate
	in a push practice session on the next night that they are scheduled to
	practice.

\subsection{Pushing Order}

	Each school year during the spring semester, before any street push practices
	are scheduled, the Sweepstakes Chairman, and/or anyone designated by that
	Chairman, shall assign days, times, and hills for each organization's street
	push practices. Each organization shall only be permitted to hold push
	practices on the street, on the days, at the times, and on the hills, that are
	assigned to them. The Sweepstakes Chairman shall determine the method by which
	these assignments shall be made. No more than four organizations shall be
	assigned to any one hill at any given time. No organization may practice on a
	hill they have not been assigned to unless they receive permission from all
	present organizations assigned to the hill. If an assigned organization
	arrives at a particular hill, all organizations not assigned to that hill
	must ask permission to remain.

\subsection{Spot Safety Checks}

	Each push practice session will be observed by the Safety Chairman, and/or
	anyone designated by that Chairman, for at least the first hour of that
	session. At any time during a push practice session, this observer may check
	for any or all of the following:

	\begin{itemize}

		\item Safety of any practicing buggy by performing a spot safety check.

		\item All required barricades and warning signs are in place.

		\item An adequate number of properly equipped barricaders are present to help control vehicular traffic.

		\item Only hills assigned or permitted for that particular practice
		are in use.

		\item an adequate number of people are present to properly attend
		to all of the buggies that they are using.

	\end{itemize}


