\chapter{Practice Sessions}

\section{General Procedures and Rules}

	Buggy freeroll and push practice sessions shall be governed by the following
	general procedures and rules:
	\newline

	No buggy freeroll or street push practice sessions of any type shall be held at
	Carnegie Mellon University at any time without the approval and authority
	of the following: The Police Department of the City of Pittsburgh, The
	Department of Parks and Recreation of the City of Pittsburgh, The Carnegie
	Mellon University Campus Police, The Sweepstakes Advisor, The Sweepstakes
	Chairman, The Safety Chairman.

	Buggy practice sessions shall only be held at those times and places authorized
	by the Sweepstakes Advisor.

	All persons participating in any activity directly or indirectly related to the
	Sweepstakes Competition at Carnegie Mellon University are considered to be
	representatives of Carnegie Mellon University. As such, these persons are
	expected and required to act in a responsible and orderly manner whenever they
	are involved in any such activity, especially when they are interacting with
	any segment of the public at large, the Pittsburgh Police, or the Carnegie
	Mellon University Campus Police.

	Sweepstakes races can only be held with the cooperation and approval of the
	City of Pittsburgh and the people of Pittsburgh. Any irresponsible action on
	the part of anyone involved with the Sweepstakes Competition could jeopardize
	this cooperation and approval. Any individual or organization found to be
	acting in an irresponsible manner while participating in any activity directly
	or indirectly related to the Sweepstakes Competition at Carnegie Mellon
	University may be barred from any further participation in that activity by the
	Dean of Student Affairs or the Sweepstakes Advisor.

	Any organizations found to be conducting or otherwise participating in any
	unauthorized freeroll practice or push practice on the public streets in
	Schenley Park and/or on the Carnegie Mellon University campus shall be
	penalized by being barred from ALL Sweepstakes activities (including both
	practices and competitions) for a period of one year from the date of the
	occurrence of that violation.

\subsection{Buggies}

	The following procedures and rules shall apply to buggies during practice
	sessions:
	\newline

	\noindent No buggy shall be permitted to participate in any freeroll practice or push
	practice session at Carnegie Mellon University unless it successfully satisfies all of the	following requirements:

	\begin{itemize}

		\item Safety inspection.

		\item Braking Capability and Field of Vision test with the driver for that practice.

		\item Driver familiarity with the buggy by being pushed around prior to freeroll practice on sidewalks, in parking lots, etc.

	\end{itemize}

	Any organization that permits any of its buggies to participate in any freeroll
	practice or push practice in any semester, before that buggy has successfully
	completed a safety inspection that same semester shall be penalized by having
	one entry withdrawn, as described in the ``DISCIPLINARY ACTIONS'' section of
	this document. In addition, the entry fee for that entry shall be forfeited.

	Any organization that fails to provide all of the required lights and
	reflectors, in proper working order, for any of its buggies that are being used
	between sunset and sunrise shall be fined the amount of \$20.00 for each buggy
	that is not properly equipped.

	Any buggy that is involved in any type of accident during any type of practice
	session at Carnegie Mellon University may not be used in any type of practice
	session again until that buggy's sponsoring organization has submitted an
	accident report to the Safety Chairman, or anyone designated by that Chairman,
	and had that report approved by the Sweepstakes Chairman or the Safety
	Chairman. In any case this report must be submitted within 48 hours of the time
	that the accident occurred. The form to be used for this accident report can be
	obtained from the Sweepstakes Advisor through the Safety Chairman. Accident
	reports will not be approved unless the Sweepstakes Chairman or the Safety
	Chairman consider that the problem that caused the accident has been corrected
	and that a similar accident is not likely to occur again. The Sweepstakes
	Chairman or the Safety Chairman may require any buggy that has been involved in
	an accident to again successfully complete a safety inspection before that
	buggy is permitted to participate in any future practice session.
	\newline

	\noindent A buggy shall be considered to have been involved in an accident when any of the following events occur:

	\begin{itemize}

		\item Departure from the course while freerolling

		\item Contact with any border of the buggy course, such as a hay-bale or curb.

		\item Contact with any material obstruction, such as another buggy, a person, a bicycle, a motor vehicle, etc.

		\item Any complete stop during the freeroll portion of the buggy course.

		\item Any part such as the shell, a hatch, a cover, etc., falls	off while freerolling.

	\end{itemize}

\subsection{Safety}

\subsubsection{Buggies With Drivers}

	No buggy that has a driver in it may be left unattended at ANY time. When any
	buggy is outdoors with a driver in it and it is not being used in a practice
	session or a brake test, (i.e. it is not being pushed by a pusher, is not
	rolling down the buggy course during a freeroll practice session, or is not
	rolling freely during a braking capability or drop brake test), someone MUST be
	within three feet of the buggy (preferably holding the pushbar), watching and
	attending it so that preventative action can be taken in the event that the
	buggy starts to move.

\subsubsection{Safety Lighting}

	Each buggy shall have an operating light attached to it at a location that is
	within 12 inches of the highest point of that buggy (usually the pushbar or the
	top of the buggy shell), whenever that buggy is being used during any type of
	practice session that takes place at Carnegie Mellon University between sunset
	and sunrise. This requirement is applicable to push practices, activities
	before freeroll practices, drop brake tests, braking capability tests, and
	whenever drivers are being familiarized with their buggies by being pushed
	around on sidewalks, in parking lots, etc. The light shall be white in color
	and it shall be aimed approximately parallel to the ground and in the forward
	direction. The light shall have sufficient brightness such that it is visible
	from a distance of 550 feet, which is the approximate distance from the finish
	line to the end of the Hill 4-5 Transition zone.

	Each buggy being used between sunset and sunrise shall also have some type of
	reflective material that is at least four square inches in area attached to the
	rear of that buggy in order to reflect the headlights of vehicles that approach
	it from behind. This material may be white, red, orange, or yellow in color and
	it must be approved by the Safety Chairman.

	Each organization is encouraged to provide additional safety lights for each of
	its buggies, such as red lights aimed toward the rear, flashing red or orange
	lights anywhere on the buggies, and flashing strobe type lights on the tops of
	the buggies.

\subsubsection{Pushers}

	All pushers are encouraged to wear reflective vests or other clothing which has
	some type of reflective material on it whenever they are participating in a
	night time push practice, so that they will be more visible to vehicular
	traffic while they are pushing a buggy.

\subsection{Permits}

	Practices Sessions shall only be held with the approval of the City of
	Pittsburgh and the Department of Parks and Recreation. This approval shall be
	in the form of permits to use the public streets on campus and in Schenley
	Park, issued by
	both the City of Pittsburgh and the Department of Parks and Recreation.

	Applications for these permits should be made by the Sweepstakes Advisor, in
	cooperation with the Sweepstakes Chairman, approximately six to eight weeks
	prior to the first scheduled Practice Session in both the fall and the spring.

\subsection{Cancellation}

	Any push practice session may be canceled at any time by the Sweepstakes
	Chairman, or anyone designated by that Chairman, due to inclement weather,
	inadequate police protection, inadequate communications, lack of medical
	personnel, vehicles obstructing the course or any other condition which might
	endanger the participants or spectators of that practice session.

\subsection{Clean-Up}

	After each organization has finished its push practice session, all debris on
	the buggy course and on the sidewalks around the course must be cleaned up and
	disposed of properly. Special care must be taken to ensure that all debris left
	by push practice participants is removed, such as empty food and beverage
	containers, duct tape, buggy preparation materials, etc.

\section{Freeroll Practice Procedures and Rules}

\subsection{Time and Place}

	Freeroll practices will generally be held between the hours of 6:00 am and
	9:00 am on Saturday and Sunday mornings, starting as soon as there is enough
	light to roll safely.  Fall freeroll practices will usually be scheduled for
	six weekends sometime between late September and late November each school
	year. Spring freeroll practices will usually be scheduled for the six weekends
	immediately preceding the weekend scheduled for the races.

\subsection{Police}

	Off-duty City of Pittsburgh police officers, usually from the Park Police
	Department, are hired by the Sweepstakes Committee and Carnegie Mellon
	University to provide police protection during all freeroll practice sessions.
	Arrangements to have these officers present during freeroll practices should be
	made by the Sweepstakes Chairman, in cooperation with the Sweepstakes Advisor,
	at the same time that the permits to use the streets are applied for.

	Usually a minimum of four officers are needed to provide protection at a
	freeroll practice. They should be available for the entire freeroll practice,
	and should report to the Sweepstakes Chairman at least 30 minutes before the
	practice is scheduled to begin that day. One officer should be stationed at
	each of the following locations:

	\begin{itemize}

		\item Schenley Drive near the clubhouse for the Schenley Park Golf Course.

		\item Circuit Road at its intersection with Schenley Drive, near the 
		George Westinghouse Memorial Pond.

		\item Panther Hollow Road at its intersection with Schenley Drive,
		near the north end of the Panther Hollow Bridge.

		\item Schenley Drive at the eastern end of the Schenley Bridge, near its 
		intersection with Frew Street, or alternatively the western end of the 
		Schenley Bridge.

	\end{itemize}

\subsection{No-Parking Signs}

	Before each scheduled freeroll practice, No-Parking signs shall be placed
	around the buggy course in order to prevent cars and other motor vehicles from
	parking there. The signs shall be obtained from the Police Department of the
	City of Pittsburgh, with the assistance of the Sweepstakes Advisor, if
	necessary. The signs shall be put in place around the buggy course as early as
	8:00 pm, and NO LATER THAN 11:00 pm, the night before each freeroll practice is
	scheduled. They shall be removed as soon as the course is officially closed for
	freeroll practice to begin each day, and NO LATER THAN 8:00 am, including dates
	on which free rolls have been officially cancelled due to inclement weather.

	At the discretion of the Sweepstakes Chairman, the responsibility of obtaining,
	storing, and placing No-Parking signs in position for freeroll practices may be
	delegated to one organization. The organization charged with this
	responsibility shall not have to provide sweepers or flaggers for freeroll
	practices.

	If the organization responsible for the No-Parking signs fails to provide them
	or remove them for any freeroll practice, that organization shall be fined the
	amount of \$25.00.

\subsection{Course Inspection and Official Notification}

	Approximately two and one half hours before the scheduled start of each
	freeroll practice, the Sweepstakes Chairman, the Assistant Chairman, the Safety
	Chairman, and/or anyone designated by any of these Chairmen, shall inspect the
	buggy course (if necessary) and decide if a freeroll practice can be held that
	day. After a decision has been made, the Carnegie Mellon University Campus
	Police Department, and the Police Department of the City of Pittsburgh shall be
	notified of that decision.

	Starting two hours before the scheduled start of a freeroll practice, all of
	the participating organizations may call the Campus Police Department in order
	to find out if a freeroll practice is to be held that day.

\subsection{Sweepers}

	Each organization shall provide two sweepers for each freeroll practice, to
	help clean debris from the buggy course. These sweepers must be provided even
	if an organization is not participating in that freeroll practice. Each
	organization shall equip their sweepers with brooms (preferably large push type
	brooms) and/or shovels. Sweepers must also wear reflective vests so that they
	can be seen by any motorists.

	These sweepers MUST be available from a time that is two hours before the
	freeroll practice is scheduled to start, until they are dismissed for the day,
	which should usually be when the freeroll practice starts. However, some
	sweepers and brooms should be available during the entire freeroll practice in
	case they are needed to clean up debris after an accident or some similar
	incident. The Sweepstakes Chairman, or anyone designated by that Chairman,
	shall determine when and where the sweepers must report for duty and when they
	can leave.

	Any organizations that fails to provide the required number of properly
	equipped sweepers for any freeroll practice shall be fined the amount of
	\$15.00 for each missing or improperly equipped sweeper. In addition, any
	organization failing to provide at least one properly equipped sweeper for a
	freeroll practice shall not be permitted to participate in that freeroll
	practice.

	At the discretion of the Sweepstakes Chairman, some organizations may not be
	required to provide sweepers, in lieu of providing other services for freeroll
	practices.

\subsection{Hay Bales}

	Approximately 300 fall and an additional 300 spring hay bales should be 
	obtained for use during freeroll practices, with two layers lining both sides 
	of the chute. Bales of hay, as opposed to bales of straw, are usually used 
	because they tend to hold up better and therefore can be used more times before 
	they start to fall apart. Freeroll practices shall only be held when an adequate 
	number of hay bales are in place around the buggy course. The Safety Chairman, 
	or anyone designated by that Chairman, shall determine how many hay bales are 
	required in order to have a freeroll practice, and where around the buggy course 
	those hay bales shall be placed, in order to provide the maximum amount of 
	protection to the freeroll practice participants.

	At the discretion of the Sweepstakes Chairman, the responsibility of obtaining,
	storing, and placing hay bales in position for freeroll practices may be
	delegated to one organization. The hay bales shall be put in place no later
	than one hour before the freeroll practice is scheduled to start, and they
	should be in place before the sweepers clean that part of the course. They
	shall be removed within 30 minutes after the freeroll practice has ended. The
	organization charged with this responsibility shall not have to provide
	sweepers or flaggers for freeroll practices.

	If the organization responsible for the hay bales fails to provide them or
	remove them for any freeroll practice, that organization shall be fined the
	amount of \$25.00.

\subsection{Flaggers}

	Each organization shall provide two flaggers for each freeroll practice, to
	help control vehicular traffic on the buggy course. These flaggers must be
	provided even if an organization is not participating in the freeroll practice.
	These flaggers must be available from a time that is 30 minutes before the
	freeroll practice is scheduled to start, until the practice is finished for
	that day. Each organization shall equip their flaggers with reflective vests
	and flags which must be used while they are acting as flaggers. All vests and
	flags must be approved by the Safety Chairman or anyone designated by that
	Chairman. The Sweepstakes Chairman, or anyone designated by that Chairman,
	shall determine when and where the flaggers must report for duty.

	Any organization that fails to provide the required number of properly equipped
	flaggers for any freeroll practice shall be fined the amount of \$20.00 for
	each missing or improperly equipped flagger. In addition, any organization
	failing to provide at least one properly equipped flagger for a freeroll
	practice shall not be permitted to participate in that freeroll practice.

	At the discretion of the Sweepstakes Chairman, some organizations may not be
	required to provide flaggers, in lieu of providing other services for freeroll
	practices.

\subsection{Barricades}

	Portable wooden barricades shall be placed at several locations near the buggy
	course while freeroll practices are in progress, to stop and/or redirect
	vehicular traffic when this traffic tries to approach the area of the buggy
	course. Warning signs which indicate that the road ahead is, or may be, closed,
	and that there will be flaggers ahead to stop and/or redirect traffic shall be
	placed near these barricades also.

	The barricades shall be put in place no later than 30 minutes before the
	freeroll practice is scheduled to begin. They shall be removed within 15
	minutes after the freeroll practice has ended. The Safety Chairman, or anyone
	designated by that Chairman, shall determine if enough barricades are available
	in order to have a freeroll practice. In order to provide the maximum amount of
	protection to the freeroll practice participants, barricades should be placed
	at the following locations:

	\begin{itemize}

		\item Margaret Morrison Street, at its intersection with Tech Street.

		\item Frew Street on the eastern side of its intersection with Tech
		Street.

		\item Schenley Drive, just east of its intersection with Tech Street.

		\item Circuit Road at its intersection with Schenley Drive, near the
		George Westinghouse Memorial Pond.

		\item Panther Hollow Road at its intersection with Schenley Drive, near
		the northern end of the Panther Hollow Bridge.

		\item Schenley Drive at the eastern end of the Schenley Bridge, near its
		intersection with Frew Street. 

		\item the Scaife Hall driveway at its intersection with Frew Street.

	\end{itemize}

	At the discretion of the Sweepstakes Chairman, the responsibility of obtaining,
	storing, and placing barricades in position for freeroll practices may be
	delegated to one organization. The organization charged with this
	responsibility shall not have to provide sweepers or flaggers for freeroll
	practices.

	If the organization responsible for the barricades fails to provide them or
	remove them for any freeroll practice, that organization shall be fined the
	amount of \$25.00.

\subsection{Warning Signs}

	Warning signs shall be placed at several locations near the buggy course while
	freeroll practices are in progress, to warn vehicular traffic when this traffic
	tries to approach the area of the buggy course. The signs shall indicate that
	the road ahead is, or may be, closed, and that there will be flaggers ahead to
	stop and/or redirect traffic. The signs shall be put in place no later than 30
	minutes before the freeroll practice is scheduled to begin. They shall be
	removed within 15 minutes after the freeroll practice has ended. The Safety
	Chairman, or anyone designated by that Chairman, shall determine if enough
	warning signs are in place in order to have a freeroll practice. In order to
	provide the maximum amount of protection to the freeroll practice participants,
	warning signs should be placed at all of the locations where barricades have
	been placed, plus the following locations:

	\begin{itemize}
		\item Margaret Morrison Street at its intersection with Forbes Avenue.
		\item Schenley Drive at its intersection with Forbes Avenue, near the clubhouse for the Schenley Park Golf Course.
		\item Circuit Road at its intersection with Serpentine Drive.
		\item Panther Hollow Road at the southern end of Panther Hollow Bridge.
		\item Schenley Drive between the Mary E. Schenley Memorial Fountain and the southwest corner of the Carnegie Museum building.
		\item the driveway to the rear of Hamburg Hall at its intersection with Forbes Avenue.
	\end{itemize}

	At the discretion of the Sweepstakes Chairman, the responsibility of obtaining,
	storing, and placing warning signs in position for freeroll practices may be
	delegated to one organization (usually the organization in charge of
	barricades). The organization charged with this responsibility shall not have
	to provide sweepers or flaggers for freeroll practices.

	If the organization responsible for the warning signs fails to provide them or
	remove them for any freeroll practice, that organization shall be fined the
	amount of \$25.00.

\subsection{Rollboard}

	At the discretion of the Sweepstakes Chairman, the responsibility of moving a
	portable chalkboard or dry-erase board to the sidewalk near the top of Hill 2,
	before the start of each freeroll practice, and returning it after the practice
	is over, may be delegated to one organization. This board shall be used to post
	the rolling order for freeroll practices.

\subsection{Drop Tests and Safety Equipment Check}

	Before each freeroll practice session starts, each buggy is required to take a
	drop brake test as follows:

	No buggy shall be permitted to participate in a freeroll practice unless it has
	successfully completed a drop brake test earlier that same day. Each buggy's
	performance in the drop brake test shall be recorded by the test administrator
	at the time that the buggy is tested.

	Any buggy that fails the drop brake test on each of two consecutive attempts
	before a freeroll practice session, shall be required to again successfully
	complete a braking capability test, with any of its drivers, before that buggy
	will be permitted to participate in any subsequent freeroll or push practice
	sessions.

	At the time that the drop brake tests are given, each participating buggy and
	driver may be checked by the test administrator to ensure that they have proper
	head protection, eye protection, hand protection, adequate field of vision, and
	safety harness.

	Drop brake tests will be administered by the Safety Chairman, or anyone
	designated by that Chairman, on the morning of each scheduled freeroll
	practice. The starting time for these tests shall be specified by the Safety
	Chairman, or anyone designated by that Chairman, sometime before the day of
	that freeroll practice.

\subsection{Rolling Order}

	The order in which the organizations who are participating in a freeroll
	practice session will be permitted to freeroll their buggies should usually be
	determined by the following procedure:

	Before the first scheduled freeroll practice of each school semester, the
	Sweepstakes Chairman shall compile a list of all of the organizations wishing
	to participate in freeroll practices. This list shall be in a random order
	which shall be determined by a lottery method chosen by the Chairman. The order
	of this list shall be the rolling order for the first freeroll practice session
	of that semester. For each subsequent freeroll practice session of that
	semester, the rotation of the rolling order shall remain the same and the
	organization that rolls first shall be the organization that was next in the
	order to roll at the previous freeroll practice session when that practice
	session ended.

	Changes to this rolling order procedure may be made for individual freeroll
	practices at the discretion of the Sweepstakes Chairman with the approval of
	the Sweepstakes Committee.

\subsection{Course Communications}

	Freeroll practices shall only be held when adequate radio communication
	equipment is available to provide voice communications around the buggy course
	for automobile traffic control, buggy traffic control, and emergency situation
	assistance. Radio communication equipment, and the personnel to operate it, are
	usually available through the Carnegie Mellon University Radio Club. Any
	personnel helping to provide radio communications should not be responsible for
	making decisions concerning what happens during a freeroll practice, but
	instead should be providing information to the Sweepstakes Chairman and his or
	her assistants, in order that they may make any necessary decisions.

\subsection{Traffic Control}

	Control of vehicular traffic on the buggy course during each freeroll practice
	session will be handled by the City of Pittsburgh police officers, usually from
	the Park Police Department, who are hired by the Sweepstakes Committee and
	Carnegie Mellon University to provide police protection during all freeroll
	practice sessions. These officers, with the assistance of the flaggers provided
	by the buggy racing organizations, will stop vehicular traffic from entering
	the buggy course while buggies are freerolling.

	If possible, the buggy course will be completely closed to vehicular traffic
	from the time that each freeroll practice session begins, until that session
	has ended for that day. If necessary, the officers will open the buggy course
	to traffic one or more times during the course of the freeroll practice session
	to relieve traffic buildup on the streets around the buggy course. In this
	event, adequate notice must be given to the Sweepstakes Chairman so that no
	buggies are permitted to freeroll while there is vehicular traffic on the buggy
	course. If the course is opened at any time during the practice session, all of
	the vehicles entering the course should be instructed not to park on the
	course. Freeroll practice will be continued when the Sweepstakes Chairman has
	determined that the course is once again closed and that it is clear of all
	vehicles.

\subsection{Rolling Procedure}

	Before and after each organization freerolls its buggies, the Sweepstakes
	Chairman, or anyone designated by that Chairman, will announce which
	organization is next in order to freeroll its buggies. The next organization in
	the rolling order shall be designated as being ``UP'', the second as being ``ON
	DECK'', and the third ss being ``IN THE HOLE''. When the Chairman determines
	that the course is clear of all vehicles and buggies, he or she will signal the
	``UP'' organization that it may roll its buggies. After this signal, that
	organization will have 15 seconds in which to get its first buggy started, (if
	that organization is running Hills 1 and 2 it will have 45 seconds from the
	signal to get its first buggy over Hill 2 and into the freeroll portion of the
	course).

	After an organization's first buggy is into the freeroll portion of the course,
	that organization will have an additional 30 seconds for each of its remaining
	buggies in order to get all of those buggies into the freeroll portion of the
	course. After an organization's last buggy is into the freeroll portion of the
	course, an additional 15 seconds will be allowed to get its follow car into the
	freeroll portion of the course. If an organization exceeds this total time
	limit, that organization will not be permitted to freeroll any of its buggies
	on its next turn in the rolling order, even if that turn isn't until a freeroll
	practice session at a later date.

	After an organization rolls its buggies at a freeroll practice session, the
	next organization in the rolling order will not be given the signal to start
	freerolling its buggies until the previous organization's follow car is past
	the chute.

\subsection{Follow Cars}

	During each freeroll practice session, each participating organization must
	provide a motor vehicle known as a follow car which will drive around the buggy
	course directly behind that organization's last buggy each time that
	organization freerolls its buggies. The follow car must be ready to leave when
	the organization receives the signal to start rolling its buggies from the
	Sweepstakes Chairman. The follow car must leave the Hill 2 lanes area of the
	course and be following the last buggy within 15 seconds of when that buggy is
	released into freeroll by its pusher. If this time limit is exceeded, the
	rolling organization will not be permitted to freeroll any of its buggies on
	its next turn in the rolling order, even if that turn isn't until a freeroll
	practice session at a later date.

	The follow car should stay approximately one hundred feet behind the last
	buggy. Several members of the freerolling organization should ride in the
	follow car so that they may observe the buggies from that vantage point and
	provide any assistance that may be necessary during that freeroll. In the event
	of an accident during a freeroll, the follow car should stop near the scene of
	the accident in order to render assistance to those involved in the accident.
	The members of the freerolling organization riding in the follow car MUST have
	any tools or devices that are necessary to quickly remove any of their drivers
	from their buggies. If an organization fails to have these tools or devices in
	the car following the buggies during any freeroll, that organization shall be
	fined the amount of \$15.00 and shall not be permitted to freeroll any of its
	buggies for the remainder of that day.

	Follow cars should not be stopped on the buggy course during freeroll practice
	sessions unless a buggy has stopped during its roll. Stopping a follow car on
	the course in order to talk to people or to allow people to enter or leave the
	car is strictly prohibited. Stopping a follow car while a freeroll practice is
	in progress creates a potential safety hazard and also causes delays to the
	buggies waiting to roll.

	If the vehicle used as the follow car is a pick-up truck and there are people
	riding in the back of that truck, its tail gate must always be closed while the
	vehicle is moving in order to reduce the chances of anyone falling out of it.

	At the discretion of the Sweepstakes Chairman, one or more organizations may be
	appointed to provide follow cars (with drivers) to follow the buggies of other
	organizations, either because they do not have access to a follow car vehicle
	during freeroll practices, or in order to reduce confusion and safety hazards
	caused by too many different follow cars during freeroll practices.
 
\subsection{Signal Flaggers}

	Each organization participating in freeroll practices must provide a signal
	flagger for each of its buggy drivers. These required signal flaggers shall be
	known as chute flaggers. Chute flaggers should provide a signal to the buggy
	drivers so that the drivers know when to start the right hand turn from
	Schenley Drive onto Frew Street. Chute flaggers should usually be positioned on
	the southern curb of Schenley Drive, just east of the intersection of Frew
	Street. Chute flaggers are not permitted to be on the street portion of
	Schenley Drive while any buggy is freerolling in that area unless they receive
	specific approval from the Safety Chairman. In general, if these flaggers need
	to position their flags more than an arms length away from the curb, their
	signal flags should be attached to extension poles so that they may hold these
	flags out over the street while still standing on the curb.

	Each organization's chute flagger must have a YELLOW stop flag with a large,
	black X across the entirety of its face so they may signal to their buggy drivers 
	that there is a hazard ahead. This will be accomplished by waving a YELLOW flag so 
	that the buggy driver can see it. Upon seeing this signal the driver must begin 
	slowing down and come to a stop in a controlled manner. NO organization may use a 
	yellow colored flag for any purpose other than to indicate a problem ahead.

	People providing course communications (usually the Carnegie Mellon
	University Radio Club) may also signal buggy drivers that there is a problem
	farther ahead. This will be accomplished by using a YELLOW flag in the same
	manner explained above.

	Each organization's chute flagger must have adequate experience relative to
	chute flagging. Each chute flagger should talk to the drivers that he or she
	will be flagging for, in order to determine where those drivers would like the
	flag to be placed. If possible each chute flagger should walk the buggy course
	with those drivers before the freeroll practice session starts each day that
	they will be flagging. If a chute flagger is considered to be acting in an
	unsafe or inexperienced manner, the Safety Chairman, or anyone designated by
	that Chairman, may require that flagger to be replaced with another more
	experienced flagger.

	If an organization fails to provide a chute flagger for any of its buggy
	drivers, that organization shall be fined the amount of \$10.00 each time one
	of its buggies attempts to make the turn from Schenley Drive onto Frew Street
	without a chute flagger.

	Each organization participating in freeroll practices may also provide
	additional signal flaggers for its drivers, in order to provide them with
	additional information about the buggy course. These flaggers may be located
	anywhere around the buggy course, such as just after Hill 2 or near the
	entrance to Phipps Conservatory. These flaggers are not permitted to stand
	anywhere on the course whenever any buggy is freerolling, unless they receive
	specific approval from the Safety Chairman.

\subsection{Accidents}

	A collision shall be defined as any incident in which a buggy comes in
	contact with another buggy or object, loses a wheel, rolls over, or there is
	reason to believe the driver may be injured. 

	An accident shall be considered any incident in which a collision occurs,
	or the buggy comes to an uncontrolled stop.

	Collisions may occur during freeroll practice sessions. If a collision does
	occur, and medical personnel are in attendance at that practice session, they
	should be dispatched to the scene of that accident as quickly as possible. In
	this case they will have the responsibility of determining the condition of any
	victims and of providing any needed first aid. If no medical personnel are
	immediately available, the responsibility of aiding any accident victims will
	be borne by either the Carnegie Mellon University Campus Police or the City of
	Pittsburgh Police who are in attendance.

	If no medical personnel are immediately available, the Safety Chairman
	and/or Sweepstakes Chairman shall be dispatched to the scene of the collision.
	Once at the scene the Safety Chairman and/or Sweepstakes Chairman will
	determine the proper course of action. after looking for any sign of apparent
	or suspected injury, they will determine if the driver should be removed from
	the buggy and whether medical assistance should be dispatched. Common factors
	in making this determination include but are not limited to: speed at the time
	of collision, disruption of barriers, and witness testimony. 

	Before any assistance arrives at the scene of an accident, common sense should
	be observed. The victim should not be moved or disturbed in any way unless
	there is some other danger in not doing so. If a buggy is involved in an
	accident and comes to rest in a position where it may be impacted by another
	buggy that is freerolling behind it, that buggy should be immediately moved, as
	quickly and as gently as possible, to a position of safety. If a buggy needs to
	be immediately moved from a dangerous position after an accident, ANYONE near
	it may do so, even if they are not members of that buggy's sponsoring
	organization. 

	In the event of a collision the buggy and its driver should not be moved or
	disturbed until available assistance arrives and assesses the situation. In
	general, a buggy driver should not be removed from a buggy until a responsible
	authority advises that it is wise to do so. In ABSOLUTELY NO CASE should an
	unconscious driver be removed from a buggy without the supervision of medical
	personnel, nor should the drivers helmet be removed.
	
	Any buggy involved in an accident should not be moved or tampered with in any 
	manner unrelated to rendering medical assistance to the driver. Once the driver 
	is safe, free, and clear of the buggy a blanket may be placed over the buggy for
	protection. However, the buggy may not be removed from the scene or otherwise 
	tampered with until after an appropriate assessment of the buggy can be made. 
	Only once permission is explicitly granted by the Safety Chairman, Chairman, or 
	Advisor may the buggy be removed for any purpose not medically related. 
	Inappropriate removal of the buggy will result in an immediate spot safety 
	failure, at the discretion of the Safety Chairman, Chairman, and Advisor. 
	Any movement and modification necessary to the rendering of medical assistance 
	must be permitted.

	Organizations are encouraged to position people near the chute portion of the
	buggy course during freeroll practices so that someone will be available to
	provide any needed assistance in the event that one of their buggies is
	involved in an accident.

	Any buggy that is involved in an accident during a freeroll practice session
	may not be used in any type of practice session again until that buggy's
	sponsoring organization has submitted an accident report and had that report
	approved as described in the Practice Sessions, General Procedures and Rules
	section of this document.

	In the event of a collision where EMS is present, the members of the
	sponsoring buggy's organization will cooperate fully with the EMS and allow
	the medical staff access to the driver.  Generally, EMS personnel will seek
	to communicate briefly with the driver to determine the condition of that
	driver. If any member of that organization hinders, or allows an alumni to
	hinder access to the driver by EMS personnel, that organization may be
	entirely disqualified from that year's races, at the collective discretion
	of the Sweepstakes Chairman, Assistant Sweepstakes Chairman, and Saftey
	chairman.

	Additionally, organizations and witnesses should cooperate with the Safety
	Chairman's documentation of the collision for Sweepstakes and University
	Records.

\subsection{Medical Personnel}

	If medical personnel are available during freeroll practices, such as off-duty
	paramedics from the City of Pittsburgh or members of an active Emergency
	Medical Service that might exist on the Carnegie Mellon University campus, they
	should be stationed near the chute portion of the buggy course, since that is
	the most likely area in which an accident might occur.

\subsection{Safety Checks}

	Each freeroll practice session will be observed by the Safety Chairman, and/or
	anyone designated by that Chairman. At any time during a freeroll practice
	session this observer may check for any or all of the following:

	\begin{itemize}

		\item Safety of any practicing buggy by performing a spot safety check.

		\item All required barricades and warning signs are in place.

		\item No-Parking signs are in place.

		\item Each organization has provided the correct number of
		properly equipped flaggers to help control vehicular traffic.

		\item Each practicing organization has an adequate number of
		people to properly attend to all of the buggies that they are using.

		\item All practicing buggy drivers are properly qualified to
		drive.

		\item An adequate number of hay bales are in place.

		\item Each practicing organization has a properly equipped chute flagger.

	\end{itemize}

\subsection{Drivers}

	The following procedures and rules shall apply to buggy drivers during freeroll
	practice sessions:
	\newline

	\noindent No driver shall be permitted to participate in a freeroll practice unless he or
	she complies with all of the following requirements:

	\begin{itemize}

		\item
		He or she has walked the buggy course earlier that same day. Each driver must
		check in with the Assistant Sweepstakes Chairman, or anyone designated by that
		Chairman, immediately before or after they walk the course. Drivers are only
		permitted to walk the course BEFORE a freeroll practice session starts.

		\item
		He or she has previously successfully completed a braking capability test in
		the buggy that he or she will be driving in that freeroll practice.

		\item
		He or she has previously successfully completed a field of vision test in the
		buggy that he or she will be driving in that freeroll practice.

		\item
		He or she has previously become familiar with the operation of the buggy that
		he or she will be driving in that freeroll practice by being pushed around (on
		sidewalks, in parking lots, etc.) in that same buggy.

		\item
		He or she has either attended all of the driver education meetings that have
		been held that school year, or has reviewed the content of those meetings with
		someone designated by the Sweepstakes Chairman.

	\end{itemize}
	
	Each driver is encouraged to complete a drop brake test before he or she
	participates in a freeroll practice session in the buggy that he or she will be
	driving at that session. This test does not have to be observed by the Safety
	Chairman provided that the buggy in question completed a drop brake test
	earlier that same day with some other driver. The purpose of this test is to
	ensure that the buggy's braking system is both properly adjusted for the
	current driver and is in proper working order.
	
\section{Push Practice Procedures and Rules}

\subsection{Time and Place}

	Street push practices shall only be held on Tech Street and Frew Street and
	only on the dates and at the times specified by the Sweepstakes Advisor.

	Street push practices will generally be held between the hours of 12:00
	midnight and 6:00 am, Mondays through Fridays, for approximately six weeks
	immediately preceding the date scheduled for the preliminary races each year.
	These days and times shall be determined by the Sweepstakes Advisor.

\subsection{Notification}

	The Sweepstakes Chairman shall notify the Carnegie Mellon University Campus
	Police Department when Push practice is scheduled to begin for the year. Each
	organization shall have a copy of the permits for push practice in case they
	are questioned by campus or city police. In addition, the campus community 
	itself must be made aware of push practices. A post should be made to 
	misc.market and a memo should be distributed to the faculty and staff so that 
	they are knowledgeable concerning when and where push practices will be in 
	progress.

	
	Several times during the first two weeks of street push practices,
	informational flyers shall be distributed to all of the motor vehicles parked
	along Tech Street and Frew Street, by placing these flyers under the windshield
	wipers of these vehicles. These informational flyers shall be provided by the
	Sweepstakes Chairman or the Sweepstakes Advisor, and they shall inform the
	drivers of the parked vehicles that street push practices are now in progress
	and that caution must be observed when driving in that area until the races are
	over. These flyers shall be distributed by members of one or more
	organizations, as assigned by the Sweepstakes Chairman.

\subsection{Barricades and Warning Signs}

	Portable wooden barricades with warning signs shall be placed at several
	locations near Tech Street and Frew Street while street push practices are in
	progress, to warn and/or stop vehicular traffic when this traffic tries to
	approach these streets. The signs shall indicate that the road is, or may be,
	closed, and that there will be flaggers to stop and/or redirect traffic. The
	barricades and signs shall be put in place immediately before any street push
	practices begin, by the participants of those practices. They shall be removed
	within 15 minutes of the end of those practices by those same participants.
	The last organization to finish practicing on that hill is required to put
	the barricades and signs away, regardless of who set them out.

	In order to provide the maximum amount of protection to the push practice
	participants, barricades should be placed at the following locations:

	\begin{itemize}

		\item Margaret Morrison Street, at its intersection with Tech Street.

		\item Frew Street on the eastern side of its intersection with Tech
		Street.

		\item Tech Street, just north of its intersection with Schenley Drive.

		\item Frew Street, just north of its intersection with Schenley Drive.

		\item Scaife Hall driveway at its intersection with Frew Street.

	\end{itemize}

	Any organization found to be conducting a push practice session without a
	sufficient number of barricades in place, shall be fined the amount of \$15.00.

\subsection{Flaggers}

	Each organization conducting a street or sidewalk push practice session must
	have a minimum of four flaggers present in order to control vehicular traffic
	on Tech Street and Frew Street. Each organization shall equip their flaggers
	with reflective vests and flags which must be used while they are acting as
	flaggers. All vests and flags must be approved by the Safety Chairman or anyone
	designated by that Chairman. In addition, all flaggers at night time practices
	are advised to use flashlights, preferably with illuminated orange extensions,
	in order to make themselves more visible to vehicular traffic. Flaggers are
	also advised to wear orange, yellow, or white construction type hard hats, so
	that they may appear to be more official-looking to vehicular traffic.

	When more than one organization is practicing on a hill or on adjacent hills,
	they may combine their flaggers in order to save manpower. However, if one
	organization leaves, any flaggers that leave with them must be replaced by the
	remaining organizations before they can continue practicing. In any event,
	flaggers from different organizations MUST cooperate with each other in order
	to control traffic. Flaggers should be positioned such that there is at least
	one flagger at each end, and one flagger in the middle, of each hill that is
	being used for practice. The flagger in the middle of the hill should be
	responsible for any vehicles that pull out of parking spaces along the hill.

	Any organization that fails to provide the required number of properly equipped
	flaggers during any push practice shall be fined the amount of \$15.00 for each
	missing or improperly equipped flagger, shall not be permitted to continue
	their push practice session that day, and shall not be permitted to participate
	in a push practice session on the next night that they are scheduled to
	practice.

\subsection{Traffic Control}

	The flaggers from each organization MUST communicate with each other concerning
	when the hills will be opened and closed to traffic. A hill should only be
	opened to traffic when all flaggers monitoring that hill have communicated to
	each other that it is safe for traffic to pass.

	If problems arise in stopping vehicular traffic during push practices, such as
	having a vehicle disregard the flaggers and drive onto a hill on which practice
	is in progress, the FIRST and most critical concern of the flaggers involved
	MUST be to warn those people who are practicing on that hill that a vehicle is
	approaching! The secondary concern of the flaggers should be to record the
	license number of the offending vehicle and report it to the Campus Police. IN
	NO CASE must any driver of a vehicle be physically assaulted or verbally abused
	by anyone participating in a push practice. Anyone found to have treated a
	driver or a vehicle disrespectfully, will be banned from any further
	participation in any Sweepstakes related activities for the remainder of that
	school year.

\subsection{Pushing Order}

	Each school year during the spring semester, before any street push practices
	are scheduled, the Sweepstakes Chairman, and/or anyone designated by that
	Chairman, shall assign days, times, and hills for each organization's street
	push practices. Each organization shall only be permitted to hold push
	practices on the street, on the days, at the times, and on the hills, that are
	assigned to them. The Sweepstakes Chairman shall determine the method by which
	these assignments shall be made. No more than four organizations shall be
	assigned to any one hill at any given time. No organization may practice on a
	hill they have not been assigned to unless they receive permission from all
	present organizations assigned to the hill. If an assigned organization
	arrives at a particular hill, all organizations not assigned to that hill 
	must ask permission to remain.

\subsection{Spot Safety Checks}

	Each push practice session will be observed by the Safety Chairman, and/or
	anyone designated by that Chairman, for at least the first hour of that
	session. At any time during a push practice session, this observer may check
	for any or all of the following:

	\begin{itemize}

		\item Check the safety of any practicing buggy by performing a spot safety
		check.

		\item Check that all required barricades and warning signs are in place.

		\item Check that each practicing organization has an adequate number of
		properly equipped flaggers to help control vehicular traffic.

		\item Check that no organization is practicing on any hill to which they
		are not assigned for that particular practice.

		\item Check that each practicing organization has an adequate number of
		people to properly attend to all of the buggies that they are using.  \item
		Check that each practicing organization has notified the Carnegie Mellon
		University Campus Police that they are currently conducting a push practice.

	\end{itemize}


