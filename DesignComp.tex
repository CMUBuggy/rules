\chapter{Design Competition}

\section{Schedule}

	The design competition shall be scheduled to be held on the day before the preliminary races are scheduled to be held.

\subsection{Time and Place}

	The Design Competition shall be held at a time and place determined by the Design Chairman and approved by the Sweepstakes Advisor. The Design Competition will generally be held between the hours of 9:00 am and 4:00 pm in the Wiegand Gymnasium. If that location is not available, the Design Competition may be held at another time and place, at the discretion of the Design Chairman and the Sweepstakes Advisor.

\section{Procedures}

	The Design Competition shall consist of two simultaneous events, the display of all of each organization's competing buggies to the public and the presentation, if desired, of one or two of each organization's competing buggies to a panel of judges for evaluation.

\subsection{Public Display}

	A public display of all of the buggies scheduled to compete in the Sweepstakes races shall take place in an indoor area that is large enough to accommodate them. Each organization must display ALL of its buggies scheduled to compete in the preliminary races at the Design Competition during the times specified by the Design Chairman. This display period should usually last between four and six hours.

\subsubsection{Display Area}

	Each organization will be assigned one area in which to display its buggies. The size and location of this area shall be determined by the Design Chairman. Organizations displaying a large number of buggies may be assigned a larger area than those with fewer buggies, at the discretion of the Design Chairman. Each organization's display area may be sectioned off from the display spectators by either a solid partition which rests on the floor and is less than two feet high, or by a barrier made of a single rope supported by stanchions at any height. Each organization may restrict the public from entering its sectioned off area or from touching any of its buggies. No organization may cover or otherwise obscure any of its buggies in order to prevent the public from seeing or photographing them while they are on display. 

\subsection{Presentation}

	Each participating organization may present a maximum of two of its buggies scheduled to compete in the preliminary races to the design judges for preliminary judging, and if necessary, final judging. The preliminary judging of individual buggies and the final judging of the six finalists shall occur during the public display of the competing buggies. The order in which individual buggies from all of the competing organizations will be presented to the judges for the preliminary judging shall be determined by a random lottery. This lottery shall be conducted using a method determined by the Design Chairman.

\subsubsection{Presentation For Preliminary Judging}

	The preliminary design competition judging shall be performed as follows: \newline
	
	Before any buggy is individually presented to the judges, the judges shall be permitted to view all of the buggies on display for approximately 30 minutes. During this time, no judge may ask any questions to any member of a competing organization about any of the buggies on display and no member of a competing organization may talk to or offer any information to any of the judges.

	After the judges have viewed all of the buggies on display, each buggy scheduled for presentation shall be privately presented to the judges. These presentations shall be accomplished as follows:
	
	\begin{itemize}

		\item
		Only one organization shall present its buggy (or buggies) to the judges at a time.

		\item
		Each organization shall be permitted to have a maximum of three representatives present at the presentation of that organization's buggy (or buggies) to the judges.

		\item
		Nobody but the judges and the representatives of the buggy's sponsoring organization may be present at the presentation.

		\item
		Each organization's representatives shall be given a maximum of ten minutes per buggy, to describe, demonstrate, or otherwise present that organization's buggy to the judges. One person, usually the Design Chairman, shall act as a timer and shall warn the representatives when they have only two minutes remaining, and shall tell them to stop when the time period is over. The timer must remain just outside of the room in which the presentations are given. The judges may not interrupt the representatives with questions or comments during their presentation but they may take notes in order to ask questions later.

		\item
		After the organization's representatives have finished their presentation, the judges may ask questions concerning each presentation or buggy, and may also examine each buggy. This period of questions and observation may last for a maximum of five minutes. The timer shall indicate when this period is over.

		\item
		After the question and observation period is over, the organization's representatives shall take that organization's buggy (or buggies) back to the public display area.


	\end{itemize}

\subsubsection{Presentation For Final Judging}

	The final design competition judging shall be performed as follows:
	\newline

	After the preliminary judging is complete, the finalists shall be re-evaluated by the judges. This reevaluation shall be accomplished as follows:

	\begin{itemize}
		\item The finalist buggies shall be brought back into the presence of the judges all at the same time.
		\item The judges shall be permitted to examine the buggies together for approximately 15 minutes.
		\item Each organization may have one representative per buggy present while any of its buggies are being
		re-evaluated by the judges.		
		\item The judges may not ask a representative of any buggy any questions during the re-evaluation.
	\end{itemize}

	After the judges have re-evaluated the finalists, the Design Competition shall be considered to be complete and all of the buggies competing in it may leave.

\section{Officials}

\subsection{Design Chairman}

	The Design Chairman shall be responsible for organizing and supervising all activities related to the Design Competition. The Design Chairman appoints all of the Design Judges with the approval of the Sweepstakes Advisor. Within two weeks after the Design Competition is completed, the Design Chairman shall compile and distribute the results of the competition to all participating organizations. During the actual Design Competition, the Design Chairman shall coordinate the efforts of all of the Design Judges and shall compile and record the results of all of the design judging using the forms included in Appendix I of this document.

\subsection{Design Judges}

	A panel of judges shall be appointed to evaluate the buggies presented by each organization during the Design Competition. The number of Design Judges appointed shall be determined by the Design Chairman.

\section{Design Rules and Regulations}
	
	The following rules and regulations shall govern the Design Competition:

	\begin{itemize}

		\item
		Each organization participating in the Sweepstakes races may enter a maximum of two buggies in the Design Competition.

		\item
		Each buggy entered in the Design Competition must be individually presented to the Design Judges for evaluation. Organizations entering two buggies may present them to the judges at the same time.

		\item
		Each organization participating in the Sweepstakes races must have all of its buggies that are scheduled to compete in the preliminary races on display at the Design Competition during the hours specified by the Design Chairman. If an organization does not have ALL of its competing buggies at the display area by the time specified by the Design Chairman, that organization shall be fined the amount of \$200.00 and any late buggies shall be disqualified from the Design Competition, if they were entered in that competition. If ANY buggy that is scheduled to compete in the preliminary races is greater than 30 minutes late in arriving at the display area, that buggy shall be disqualified from those races.

		\item
		Any organization found covering or otherwise obscuring any of its buggies in order to prevent the public from seeing or photographing them while they are on public display at the Design Competition shall have all of its entries disqualified from both the Design Competition and from all of the Sweepstakes races held that school year.

	\end{itemize}

\section{Judging Procedures}

\subsection{Preliminary Judging}

	The preliminary judging of buggies privately presented to the Design Judges shall be performed as follows:
	\newline

	The judges shall evaluate each buggy individually, based on the presentation by the representatives of the buggy's sponsoring organization and the results of the question and observation period immediately after that presentation.

	Each Design Judge shall compile a list of scores for each buggy presented based on their evaluation of that buggy. These scores shall be compiled using the Design Judging Evaluation form, which is included in Appendix I of this document.

	Each Design Judge shall add up their list of scores, in order to obtain one total score for each buggy evaluated and then submit that list and score to the Design Chairman for compilation.

	The Design Chairman shall check the totals of the scores of each Design Judge's list for each buggy evaluated and then compile a list of the total scores of all of the Design Judge's for that buggy. If the number of Design Judges is sufficiently high, the highest and the lowest scores may be discarded from the compiled list of scores. The Design Chairman shall total all of the scores remaining on the list in order to produce the preliminary score for each buggy evaluated.
	
	After all of the scores have been checked and compiled by the Design Chairman they shall be rechecked by another person designated by the Design Chairman. This designated person should usually be either the Sweepstakes Chairman, the Assistant Sweepstakes Chairman, or the Safety Chairman.

	The Design Chairman shall rank all of the evaluated buggies in order of their preliminary design judging scores. The buggy with the highest score shall be ranked first, the buggy with the second highest score shall be ranked second, and so on to the buggy with the lowest score, which shall be ranked last. The six highest ranked buggies shall be eligible for the final judging. In the event of ties, one or more extra buggies may be eligible for the final judging.

\subsection{Final Judging}

	The judging of the buggies eligible for the final judging, which are all presented to the Design Judges at the same time, shall be performed as follows:

	After the Design Judges have re-evaluated the finalists, each judge shall rank the buggies from first to last, with first being the highest ranking and last being the lowest ranking.

	The Design Chairman shall compile a list of the rankings of each Design Judge for the finalist buggies as follows:
	\newline

	\begin{itemize}

		\item
		The ranking of each Design Judge for each buggy shall be added up to produce a total ranking for each finalist buggy.

		\item
		The buggy with the lowest total ranking shall be placed first, the buggy with the second lowest total ranking shall be placed second, and so on until the buggy with the highest total ranking, which shall be placed last.

		\item
		If one or more buggies have the same total ranking, the tie shall be broken by using the preliminary scores for those buggies, such that a buggy with a higher preliminary score shall be placed higher than a buggy with a lower preliminary score. If the preliminary scores are also the same, the tie shall stand.

	\end{itemize}	

	After the Design Chairman has determined the rankings of the finalist buggies, the finalist's scores and rankings shall be checked by the same person designated to check the preliminary scores.

	The placements of the finalist buggies shall not be divulged until the awards ceremony after the Sweepstakes races.

\section{Eligibility For Design Awards}

	To be eligible for a Design Competition award a buggy must comply with all of the following requirements:
	\newline

	\begin{itemize}
		\item It must compete in and finish a preliminary or rerun Sweepstakes race without a design related failure. An entry may not be granted a rerun solely for the purpose of eligibility for a design award.
	\end{itemize}

\subsection{Eligibility Determination}

	If there is any question concerning the eligibility of a buggy for a Design Competition award, the final determination of that eligibility shall be made by the Design Chairman.


