\chapter{Common Raceday/Freeroll Rules and Procedures}
\label{ch:CommonRules}

\section{General Procedures and Rules}
	Freeroll events shall be known as the combination of freeroll practices 
	(sec. \ref{sec:FreerollRules} pg. \pageref{sec:FreerollRules}), and Sweepstakes races 
	(sec. \ref{ch:RaceRules} pg. \pageref{ch:RaceRules}). Push Practices 
	(sec. \ref{sec:PushPRules} pg. \pageref{sec:PushPRules}) shall 
	be governed by the following only where specifically mentioned, including 
	in sections \ref{ssec:SafetyBuggyDriver} and \ref{ssec:CleanUp}. Freeroll 
	events shall always be governed by the following:

\subsection{Sweepers}

	Each organization shall provide two sweepers for each freeroll event, to
	help clean debris from the buggy course. These sweepers must be provided even
	if an organization is not participating in that freeroll event. Each
	organization shall equip their sweepers with brooms (preferably large push type
	brooms) and/or shovels. Sweepers must also wear reflective vests so that they
	can be seen by any motorists.

	These sweepers MUST be available from a time that is two hours before the
	freeroll event is scheduled to start, until they are dismissed for the day,
	which should usually be when the freeroll event starts. However, some
	sweepers and brooms should be available during the entire freeroll event in
	case they are needed to clean up debris after an accident or some similar
	incident. The Sweepstakes Chairman, or anyone designated by that Chairman,
	shall determine when and where the sweepers must report for duty and when they
	can leave.

	Any organizations that fails to provide the required number of properly
	equipped sweepers shall be fined the amount of \$15.00 for freeroll practice or 
	\$30.00 on Raceday for each missing or improperly equipped sweeper. In addition, 
	any organization failing to provide at least one properly equipped sweeper for a
	freeroll practice shall not be permitted to participate in that freeroll practice,
	or one entry shall be withdrawn from that day of Sweepstakes races.

	At the discretion of the Sweepstakes Chairman, some organizations may not be
	required to provide sweepers, in lieu of providing other services for freeroll
	events.
	
\subsection{Flaggers}

	Each organization shall provide two flaggers for each scheduled day of
	Sweepstakes racing or practice, to help control vehicular traffic on the buggy course.
	These flaggers must be available from a time that is one hour before the events
	are scheduled to start, until the events are finished for that day. Each
	organization shall equip their flaggers with reflective vests and flags which
	must be used while they are acting as flaggers. All vests and flags must be
	approved by the Safety Chairman or anyone designated by that Chairman. The
	Sweepstakes Chairman, or anyone designated by that Chairman, shall determine
	when and where the flaggers must report for duty.

	Any organization that fails to provide the required number of properly equipped
	flaggers for any day of Sweepstakes racing shall be penalized by having one
	entry withdrawn, and the entry fee for that entry forfeited.

	Any organization that fails to provide the required number of properly equipped
	flaggers for any freeroll event shall be fined the amount of \$20.00 for
	each missing or improperly equipped flagger. In addition, any organization
	failing to provide at least one properly equipped flagger for a freeroll
	event shall not be permitted to participate in that freeroll event.

	At the discretion of the Sweepstakes Chairman, some organizations may not be
	required to provide flaggers, in lieu of providing other services for freeroll
	events.
	
	
\subsection{Signal Flaggers}

	Each organization participating in freeroll events must provide a signal
	flagger for each of its buggy drivers. These required signal flaggers shall be
	known as chute flaggers. Chute flaggers should provide a signal to the buggy
	drivers so that the drivers know when to start the right hand turn from
	Schenley Drive onto Frew Street. Chute flaggers should usually be positioned on
	the southern curb of Schenley Drive, just east of the intersection of Frew
	Street. Chute flaggers are not permitted to be on the street portion of
	Schenley Drive while any buggy is freerolling in that area unless they receive
	specific approval from the Safety Chairman. In general, if these flaggers need
	to position their flags more than an arms length away from the curb, their
	signal flags should be attached to extension poles so that they may hold these
	flags out over the street while still standing on the curb.

	Each organization's chute flagger must have a YELLOW stop flag with a large,
	black X across the entirety of its face so they may signal to their buggy drivers 
	that there is a hazard ahead. This will be accomplished by waving the YELLOW flag so 
	that the buggy driver can see it. Upon seeing this signal the driver must begin 
	slowing down and come to a stop in a controlled manner. NO organization may use a 
	yellow colored flag for any purpose other than to indicate a problem ahead.

	People providing course communications (usually the Carnegie Mellon
	University Radio Club) may also signal buggy drivers that there is a problem
	farther ahead. This will be accomplished by using a YELLOW flag in the same
	manner explained above.

	Each organization's chute flagger must have adequate experience relative to
	chute flagging. Each chute flagger should talk to the drivers that he or she
	will be flagging for, in order to determine where those drivers would like the
	flag to be placed. If possible each chute flagger should walk the buggy course
	with those drivers before the freeroll event starts each day that
	they will be flagging. If a chute flagger is considered to be acting in an
	unsafe or inexperienced manner, the Safety Chairman, or anyone designated by
	that Chairman, may require that flagger to be replaced with another more
	experienced flagger.

	If an organization fails to provide a chute flagger for any of its buggy
	drivers, that organization shall be fined the amount of \$10.00 for practices
	and \$25.00 on raceday each time one
	of its buggies attempts to make the turn from Schenley Drive onto Frew Street
	without a chute flagger.

	Each organization participating in freeroll events may also provide
	additional signal flaggers for its drivers, in order to provide them with
	additional information about the buggy course or about the race that they are
	competing in, such as where the other buggies in that heat are at that time.
	These flaggers may be located anywhere around the buggy course, such as just
	after Hill 2 or near the entrance to Phipps Conservatory. These flaggers are
	not permitted to stand anywhere on the course whenever any buggy is 
	freerolling, unless they receive specific approval from the Safety Chairman.
	
	
\subsection{Barricades}

	Portable wooden barricades shall be placed at several locations near the buggy 
	course while freeroll events are in progress, to stop and/or redirect
	vehicular traffic when this traffic tries to approach the area of the buggy
	course. Warning signs which indicate that the road ahead is, or may be, closed,
	and that there will be flaggers ahead to stop and/or redirect traffic shall be
	placed near these barricades also. The barricades shall be put in place no
	later than one hour before races are scheduled to start on each day of
	Sweepstakes racing, or 30 minutes before the scheduled start of freeroll 
	practice. They shall be removed within 15 minutes of either the end or the 
	cancellation of the freeroll event for that day. The
	Safety Chairman, or anyone designated by that Chairman, shall determine if
	enough barricades are in place in order to have a Sweepstakes race. In order to
	provide the maximum amount of protection to the Sweepstakes race participants,
	barricades should be placed at least at the following locations:

	\begin{itemize}

		\item Margaret Morrison Street, at its intersection with Tech Street.

		\item Frew Street on the eastern side of its intersection with Tech
		Street.

		\item Schenley Drive, just east of its intersection with Tech Street.

		\item Circuit Road at its intersection with Schenley Drive, near the
		George Westinghouse Memorial Pond.

		\item Panther Hollow Road at its intersection with Schenley Drive, near
		the northern end of the Panther Hollow Bridge.

		\item Schenley Drive at the eastern end of the Schenley Bridge, near its
		intersection with Frew Street. The placement of the barricades at this location
		should be performed carefully. Some space should be left on the northern side
		of the bridge so that if a buggy failed to make the turn onto Frew Street it
		would have adequate room to drive across the bridge in the right-hand lane.

		\item Scaife Hall driveway at its intersection with Frew Street.

	\end{itemize}

	At the discretion of the Sweepstakes Chairman, the responsibility of obtaining,
	storing, and placing barricades in position for freeroll events may be
	delegated to one organization. The organization charged with this
	responsibility may not have to provide sweepers, flaggers, or course marshals
	for each day of Sweepstakes racing, at the discretion of the Sweepstakes
	Chairman.

	If the organization responsible for the barricades fails to provide them or
	remove them for any freeroll event, that organization shall be fined
	the amount of \$25 for freeroll practice or \$50.00 on Raceday.

	
\subsection{Warning Signs}

	Warning signs shall be placed at several locations near the buggy course while
	freeroll events are in progress, to warn vehicular traffic when this traffic
	tries to approach the area of the buggy course. The signs shall indicate that
	the road ahead is, or may be, closed, and that there will be flaggers ahead to
	stop and/or redirect traffic. The signs shall be put in place no later than 1 hour
	minutes before the freeroll event is scheduled to begin. They shall be
	removed within 15 minutes after the freeroll event has ended. The Safety
	Chairman, or anyone designated by that Chairman, shall determine if enough
	warning signs are in place in order to have a freeroll event. In order to
	provide the maximum amount of protection to the freeroll event participants,
	warning signs should be placed at all of the locations where barricades have
	been placed, plus the following locations:

	\begin{itemize}
		\item Margaret Morrison Street at its intersection with Forbes Avenue.
		\item Schenley Drive at its intersection with Forbes Avenue, near the clubhouse for the Schenley Park Golf Course.
		\item Circuit Road at its intersection with Serpentine Drive.
		\item Panther Hollow Road at the southern end of Panther Hollow Bridge.
		\item Schenley Drive between the Mary E. Schenley Memorial Fountain and the southwest corner of the Carnegie Museum building.
		\item the driveway to the rear of Hamburg Hall at its intersection with Forbes Avenue.
	\end{itemize}

	At the discretion of the Sweepstakes Chairman, the responsibility of obtaining,
	storing, and placing warning signs in position for freeroll events may be
	delegated to one organization (usually the organization in charge of
	barricades). The organization charged with this responsibility shall not have
	to provide sweepers or flaggers for freeroll events.

	If the organization responsible for the warning signs fails to provide them or
	remove them, that organization shall be fined the amount of \$25.00 for any 
	freeroll practice or \$50.00 on Raceday.
	
	
\subsection{Hay Bales}

	Approximately 300 fall and an additional 300 spring hay bales should be 
	obtained for use during freeroll events, with two layers lining both sides 
	of the chute. Bales of hay, as opposed to bales of straw, are usually used 
	because they tend to hold up better and therefore can be used more times before 
	they start to fall apart. Freeroll events shall only be held when an adequate 
	number of hay bales are in place around the buggy course. The Safety Chairman, 
	or anyone designated by that Chairman, shall determine how many hay bales are 
	required in order to have a freeroll event, and where around the buggy course 
	those hay bales shall be placed, in order to provide the maximum amount of 
	protection to the freeroll event participants. Additional haybale requirements 
	for Sweepstakes races shall be determined by The Safety Chairman, or anyone 
	designated by that Chairman, and ensure two layers line both sides of the chute.

	At the discretion of the Sweepstakes Chairman, the responsibility of obtaining,
	storing, and placing hay bales in position for freeroll events may be
	delegated to one organization. The hay bales shall be put in place no later
	than one hour before the freeroll event is scheduled to start, and they
	should be in place before the sweepers clean that part of the course. They
	shall be removed within 30 minutes after the freeroll event has ended. The
	organization charged with this responsibility shall not have to provide
	sweepers or flaggers for freeroll events.

	If the organization responsible for the hay bales fails to provide them or
	remove them that organization shall be fined the amount of \$25.00 for 
	practice sessions and \$50.00 on Raceday.
	
	
\subsection{No-Parking Signs}

	Before each scheduled freeroll event, No-Parking signs shall be placed
	around the buggy course in order to prevent cars and other motor vehicles from
	parking there. The signs shall be obtained from the Police Department of the
	City of Pittsburgh, with the assistance of the Sweepstakes Advisor, if
	necessary. The signs shall be put in place around the buggy course as early as
	8:00 pm, and NO LATER THAN 11:00 pm, the night before each freeroll event is
	scheduled. They shall be removed from practices as the course is officially closed for
	freeroll event to begin each day, and NO LATER THAN 8:00 am, including dates
	on which free rolls have been officially cancelled due to inclement weather.

	At the discretion of the Sweepstakes Chairman, the responsibility of obtaining,
	storing, and placing No-Parking signs in position for freeroll events may be
	delegated to one organization. The organization charged with this
	responsibility shall not have to provide sweepers or flaggers for freeroll
	events.

	If the organization responsible for the No-Parking signs fails to provide them
	or remove them that organization shall be fined the
	amount of \$25.00 for any freeroll practice, and \$50.00 on Raceday

	
\subsection{Course Inspection and Official Notification}

	Approximately two and one half hours before the scheduled start of each
	freeroll practice and 30 minutes before the start of the first race on Raceday,
	the Sweepstakes Chairman, the Assistant Chairman, the Safety
	Chairman, and/or anyone designated by any of these Chairmen, shall inspect the
	buggy course (if necessary) and decide if a freeroll event can be held that
	day. After a decision has been made, the Carnegie Mellon University Campus
	Police Department, and the Police Department of the City of Pittsburgh shall be
	notified of that decision. If no RACES are to be held,
	representatives of all participating organizations and any other people
	involved with the running of the races shall be notified by the person or
	persons who made the decision.


\subsection{Course Communications}

	Freeroll events shall only be held when adequate radio communication
	equipment is available to provide voice communications around the buggy course
	for automobile traffic control, buggy traffic control, and emergency situation
	assistance. Radio communication equipment, and the personnel to operate it, are
	usually available through the Carnegie Mellon University Radio Club. Any
	personnel helping to provide radio communications should not be responsible for
	making decisions concerning what happens during a freeroll event, but
	instead should be providing information to the Sweepstakes Chairman and his or
	her assistants, in order that they may make any necessary decisions.
	
	
\subsection{Traffic Control}

	Control of vehicular traffic on the buggy course during each freeroll event
	will be handled by the City of Pittsburgh police officers, usually from
	the Park Police Department, who are hired by the Sweepstakes Committee and
	Carnegie Mellon University to provide police protection during all freeroll
	events. These officers, with the assistance of the flaggers provided
	by the buggy racing organizations, will stop vehicular traffic from entering
	the buggy course while buggies are freerolling.

	If possible, the buggy course will be completely closed to vehicular traffic
	from the time that each freeroll event begins, until that event
	has ended for that day. If necessary, the officers will open the buggy course
	to traffic one or more times during the course of the freeroll event
	to relieve traffic buildup on the streets around the buggy course. In this
	event, adequate notice must be given to the Sweepstakes Chairman so that no
	buggies are permitted to freeroll while there is vehicular traffic on the buggy
	course. If the course is opened at any time during the event, all of
	the vehicles entering the course should be instructed not to park on the
	course. The freeroll event will be continued when the Sweepstakes Chairman has
	determined that the course is once again closed and that it is clear of all
	vehicles.
	
	
\subsection{Permits}

	Practices Sessions and Raceday shall only be held with the approval of the City of
	Pittsburgh and the Department of Parks and Recreation. This approval shall be
	in the form of permits to use the public streets on campus and in Schenley
	Park, issued by both the City of Pittsburgh and the Department of Parks and Recreation.

	Applications for these permits should be made by the Sweepstakes Advisor, in
	cooperation with the Sweepstakes Chairman, at least six to eight weeks
	prior to the first scheduled Practice Session in the fall, 
	covering all dates in both the fall and the spring.
	
	
\subsection{Police}

	Freeroll events shall only be held with the protection and cooperation of
	both the Police Department of the City of Pittsburgh, and the Carnegie Mellon
	University Campus Police Department.

	Off-duty City of Pittsburgh police officers, usually from the Park Police
	Department, are hired by the Sweepstakes Committee and Carnegie Mellon
	University to provide police protection during all of the freeroll events.
	Arrangements to have these officers present during the events should be made by
	the Sweepstakes Chairman, in cooperation with the Sweepstakes Advisor, at the
	same time that the permits to use the streets are applied for.

	Usually a minimum of four officers are needed to provide protection during the
	events. They should be available during the entire time that the events are
	underway, and should report to the Sweepstakes Chairman or the Sweepstakes
	Advisor on each day of events, at least 30 minutes before the events are
	scheduled to begin that day. These officers should be stationed as follows:

	\begin{itemize}

		\item One on Schenley Drive near the clubhouse for the Schenley Park Golf
		Course.

		\item One on Circuit Road at its intersection with Schenley Drive, near the
		George Westinghouse Memorial Pond.

		\item One on Panther Hollow Road at its intersection with Schenley Drive,
		near the north end of the Panther Hollow Bridge.

		\item One on Schenley Drive at the eastern end of the Schenley Bridge, near
		its intersection with Frew Street. (This officer might alternatively be
		stationed at the western end of the Schenley Bridge.)

	\end{itemize}

	Carnegie Mellon University Campus Police should be available during the entire
	time that the events are underway in the event that their assistance is needed.
	Arrangements to have these officers present during the events should be made by
	the Sweepstakes Chairman, in cooperation with the Sweepstakes Advisor.
	
	
\subsection{Safety of Buggies with Drivers}
\label{ssec:SafetyBuggyDriver}
	No buggy that has a driver in it may be left unattended at ANY time. When any
	buggy is outdoors with a driver in it and it is not being used in a freeroll
	event, push practice, or a brake test, (i.e. it is not being pushed by a pusher, 
	is not rolling down the buggy course during a freeroll event, or is 
	not rolling freely during a braking capability or drop brake test), someone MUST 
	be within three feet of the buggy (preferably holding the pushbar), watching and
	attending it so that preventative action can be taken in the event that the
	buggy starts to move.
	
	
\subsection{Clean-Up}
\label{ssec:CleanUp}
	After each freeroll event or push practice has been completed, all debris 
	on the buggy course and on the sidewalks around the course must be cleaned up 
	and disposed of properly. Special care must be taken to ensure that all debris 
	left by any race participants is removed, such as empty food and beverage
	containers, duct tape, buggy preparation materials, hay that has fallen off hay
	bales, No-Parking signs that have been misplaced, etc.