\chapter{Buggy Construction and Performance Requirements}

	Each buggy shall have been designed and constructed by full-time,
	Activities Fee paying, undergraduate or graduate students of Carnegie Mellon
	University. These students shall also have been members of the buggy's
	sponsoring organization at the time that they designed and constructed the
	buggy. Last semester seniors shall be considered to be full-time students.

	Buggies may be bought and/or sold by one organization from or to another
	organization, provided that the Sweepstakes Chairman is notified of the
	Transition, and it takes place before heats and lanes are selected for the
	preliminary races (if that buggy is to be used in those races.)

\section{Braking System}

	Each buggy shall have a driver operated system or mechanism which is capable of
	stopping the rolling motion of that buggy. This system or mechanism shall be
	known as the buggy's braking system. Each buggy's braking system shall be
	capable of passing two separate braking performance tests, the braking
	capability test and the drop brake test. In addition, all braking systems must
	meet the following requirements:

	The brakes shall be self-resetting. This means that the brakes shall release
	their braking force whenever the driver stops actuating them.

	The brakes shall be capable of being actuated to full braking force, and then
	be completely released, at least three times in succession, without any
	maintenance.

	All fasteners and hardware used in the braking system shall be equipped with
	locking devices, such as lock nuts, lock washers, lock wires, cotter pins, etc.
	The use of anaerobic locking compounds (such as Loctite) is permissible, but is
	not sufficient. Tubing fittings used in hydraulic or pneumatic braking systems
	need not be equipped with locking devices, with the approval of the Safety
	Chairman, due to the inherent vibration resistance of these fittings alone.

\subsection{Braking Capability Test}

	Each buggy shall pass a braking capability test, at least once each school
	semester, with each person who will be driving it, before that person will be
	permitted to drive that buggy in any type of practice session at Carnegie
	Mellon University during that semester.

	The purpose of the braking capability test is to ensure that each buggy meets a
	minimum braking standard when moving in both the forward and rearward
	directions, while a particular person is driving it. The braking capability
	test shall be administered by the Safety Chairman, or anyone designated by that
	Chairman. The procedure for the braking capability test shall be as follows:

	The test shall take place on a flat, level, and smooth area, that is paved with
	either concrete or asphalt. If the area chosen is not completely level, the
	test shall be run in a direction such that the buggy being tested shall be
	moving toward the lower end of the area during the test. The sidewalk between
	Baker Hall and Doherty Hall shall be used if it is available and in acceptable
	condition. In this case any buggy being tested shall be moving from the Baker
	Hall end of the sidewalk to the Doherty Hall end of the sidewalk during the
	test. If this area is not available or acceptable, the sidewalk between the
	University Center and Purnell shall be used if it is available and in
	acceptable condition. In this case any buggy being tested shall be moving from
	the Purnell end of the sidewalk to the University Center end of the sidewalk
	during the test. If neither of these areas are available or acceptable, the
	Safety Chairman, or anyone designated by that Chairman, shall choose an
	alternate location for the test.

	The test shall be conducted as follows:

	The buggy shall be pushed until it is moving at a speed greater than or equal
	to 15 miles per hour (22 feet per second).

	After the pusher releases the buggy, the average speed of the buggy shall be
	measured while the buggy travels through a 50 foot long timing zone. The
	average speed of the buggy shall be determined by measuring the time required
	for the nose of the buggy to travel from the beginning to the end of the 50
	foot long timing zone. A stopwatch or other suitable timing device shall be
	used to make this measurement.

	When the nose of the buggy reaches the end of the 50 foot long timing zone, the
	driver shall be signaled to apply the buggy's brakes.

	After the buggy stops, the test administrator shall tell the driver to release
	and reapply the brakes two more times, while the administrator is pushing on
	the buggy's pushbar to verify that the brakes are being released and reapplied.

	The final portion of the brake capabilities test, shall be completed on hill 1.
	Each buggy driver combination will be placed on the steepest point of hill 1
	pointing uphill and must be able to hold its own weight using only the standard
	braking mechanism.

	Two members of the buggy's sponsoring organization shall be located in a
	position such that they will be able to stop the buggy in the event that the
	buggy's braking system fails to stop the buggy during the test.

	The organization whose buggy is being tested shall provide an adequate number
	of people to control pedestrian traffic in the test area and to properly attend
	to all of the buggies that they are using. If the braking capability test is
	performed between sunset and sunrise, any buggy that is tested must have all
	lights and reflectors that are required during night push practices.

	To successfully complete the test, the following requirements must be met:

	Nobody other than the driver may touch the buggy while the buggy is being
	tested.

	The average speed of the buggy while it is traveling through the 50 foot long
	timing zone must be determined to be 15 miles per hour (22 feet per second) or
	greater.

	The buggy must stop within the distance prescribed by the schedule that
	follows. This schedule lists the measured time required for the buggy to travel
	through the 50 foot long timing zone, the average speed of the buggy through
	that zone, and the allowable stopping distance for that speed. The stopping
	distance shall be measured from the end of the 50 foot long timing zone to the
	nose of the buggy, i.e. that part of the buggy that is farthest forward, after
	the buggy has stopped.

		\begin{tabular}{c c c}
			Measured Time & Average Speed & Distance \\
			2.25 sec. & 15.2 mph (22.2 fps) & 35 feet\\
			2.20 sec. & 15.5 mph (22.7 fps) & 35 feet\\
			2.15 sec. & 15.9 mph (23.3 fps) & 35 feet\\
			2.10 sec. & 16.2 mph (23.8 fps) & 40 feet\\
			2.05 sec. & 16.6 mph (24.4 fps) & 40 feet\\
			2.00 sec. & 17.0 mph (25.0 fps) & 45 feet\\
			1.95 sec. & 17.5 mph (25.6 fps) & 45 feet\\
			1.90 sec. & 17.9 mph (26.3 fps) & 45 feet\\
			1.85 sec. & 18.4 mph (27.0 fps) & 50 feet\\
			1.80 sec. & 18.9 mph (27.8 fps) & 50 feet\\
			1.75 sec. & 19.5 mph (28.6 fps) & 56 feet\\
			1.70 sec. & 20.1 mph (29.4 fps) & 62 feet\\
		\end{tabular}

	While the buggy is stopping, its centerline must not deviate more than 15
	degrees from the line that it was moving along, in the 50 foot long timing
	zone.

	After the buggy has stopped, it must remain in a stable position with no
	external assistance, and its driver must successfully release and reapply its
	brakes two additional times.

	The buggy's brakes must not be applied before the buggy reaches the end of the
	50 foot long timing zone.

	If a buggy fails to successfully complete the braking capability test during
	any four consecutive attempts on any one day, that buggy shall not be permitted
	to attempt the braking capability test again for a period of 24 hours.

	The buggy shall be in the same physical state as would be expected for regular
	free rolls practice. No tactics may be employed to aid in passing the drop test
	or slowing forward motion of the buggy that would not be used in an emergency
	condition on the course. Actions not allowed include but are not limited to:
	overtightening of the retaining mechanism on wheels, use of spacers or shims to
	cause additional friction, use of worn or dirty bearings, and objects dragging
	on the ground. Any organization's buggy caught violating this policy will be
	treated as if it failed a spot safety check

\subsection{Drop Brake Test}

	Each buggy is required to take a drop brake test before every freeroll
	practice, after it competes in any race, and whenever the Sweepstakes Chairman,
	the Assistant Sweepstakes Chairman, or the Safety Chairman request such a test.
	The purpose of the drop brake test is to demonstrate that the buggy's brakes
	are in proper working order and that the driver is able to actuate them. The
	drop brake test shall be administered by the Sweepstakes Chairman, Assistant
	Sweepstakes Chairman, Safety Chairman,or anyone designated by any of these
	Chairmen.

	The procedure for the drop brake test shall be as follows:

	The test shall take place somewhere on one of the sidewalks of Tech Street,
	along Hill 1 of the buggy course. The sidewalk directly in front of the middle
	entrance to Skibo Gymnasium shall be used if it is available and is in
	acceptable condition. If this area is not available or acceptable, the Safety
	Chairman, or anyone designated by that Chairman, shall choose an alternate
	location for the test.

	The test shall be conducted as follows:

	The buggy shall be placed at rest on the sidewalk facing toward the bottom of
	the hill, with its nose at the beginning of a marked 30 foot long zone.

	The buggy shall be released and allowed to roll down the hill without being
	inhibited, to the bottom end of the 30 foot long zone.

	When the buggy reaches the end of the 30 foot long zone, the driver shall be
	signaled to apply the buggy's brakes.

	After the buggy stops, the test administrator shall tell the driver to release
	and reapply the brakes once, while observing the movement of the buggy to
	verify that the brakes are being released and reapplied.

	A member of the buggy's sponsoring organization shall be located in a position
	such that he or she will be able to stop the buggy in the event that the
	buggy's braking system fails to stop the buggy during the test.

	The organization whose buggy is being tested shall provide an adequate number
	of people to control pedestrian traffic in the test area and to properly attend
	to all of the buggies that they are using. If the drop brake test is performed
	between sunset and sunrise, any buggy that is tested must have all lights and
	reflectors that are required during night push practices.

	To successfully complete the test, the following requirements must be met:

	Nobody other than the driver may touch the buggy during the test.

	The buggy must stop within 15 feet of the end of the 30 foot long zone through
	which it has been allowed to roll. This 15 foot distance shall be measured to
	the nose of the buggy, i.e. that part of the buggy that is farthest away from
	the end of the 30 foot long zone.

	After the buggy has stopped, its driver must successfully release and reapply
	its brakes once.

	The buggy's brakes must not be applied before the buggy reaches the end of the
	30 foot long zone through which it has been allowed to roll.

	The buggy shall be in the same physical state as would be expected for regular
	free rolls practice. No tactics may be employed to aid in passing the drop test
	or slowing forward motion of the buggy that would not be used in an emergency
	condition on the course. Actions not allowed include but are not limited to:
	overtightening of the retaining mechanism on wheels, use of spacers or shims to
	cause additional friction, use of worn or dirty bearings, and objects dragging
	on the ground. Any organization's buggy caught violating this policy will be
	treated as if it failed a spot safety check.

\section{Configuration}

	There shall be no restrictions on the configuration that a buggy may have with
	respect to the position of the driver, the type of braking system, the location
	of the braking system, the type of steering system, and the type of frame, as
	long as the buggy complies with all of the design, construction, and
	performance criteria specified in these Rules, Regulations, and Procedures.

\section{Driver's Personal Protection}

\subsection{Eye Protection}

	Drivers MUST wear approved goggles whenever they are driving a buggy. The
	goggles must have shatterproof lenses, and must provide side-shield protection
	to the eyes. Any goggles that are examined by the Safety Chairman and are
	deemed to be adequate, or that meet the requirements of ANSI Z87.1, are
	considered to be approved goggles. Only goggles having clear lenses may be used
	by buggy drivers when they are driving a buggy anytime after sunset and before
	sunrise.

\subsubsection{Fogging}
	Anti-fogging compounds or solutions shall be used to prevent goggles from fogging.

\subsection{Hand Protection}

	Drivers MUST wear leather gloves whenever they are driving a buggy. Gloves will
	help reduce the chances of cuts and abrasions in the event of an accident. The
	gloves should cover all parts of both hands, including the fingers, thumbs,
	palms, and the backs of the hands. Any gloves which do not cover all parts of
	both hands must be approved by the Safety Chairman.

\subsection{Head Protection}

	Drivers MUST wear an approved hard shell helmet whenever they are driving a
	buggy. Any helmet which carries the ANSI Z90.4 rating, the SNELL bicycle helmet
	rating, or any CSA hockey helmet rating (such as CAN3-Z262.2-M78, or Z262.1) is
	considered to be an approved helmet. The helmet must not restrict the movement
	of the driver or the driver's head while the buggy is being driven.

	Helmets MUST be secured (using duct tape or other means) so that they will not
	obstruct the driver's vision while the buggy is in motion. The method of
	securing the driver's helmet must be demonstrated during the safety inspection
	of the buggy.

\subsubsection{Modifications}
	
	Helmets may not be modified by cutting, removing or altering the lining,
	padding, shell, or chin strap in any way except as follows:

	The shell, lining, and padding of the helmet may be trimmed away in the upper
	front area of the helmet (above where the driver's forehead would be) in order
	to provide adequate vision for the driver. The section trimmed away shall be in
	the shape of a smooth arc no more than 1.5 inches high (at its greatest height)
	by 5.0 inches wide (at its widest point.) The edges of the trimmed away section
	of the helmet shall be smoothed, beveled, and/or padded in order to reduce the
	possibility of injury to the driver while the helmet is being worn. In no case
	may any stiffening ribs or supporting members of the helmet shell be removed or
	altered in any way. Any modifications made to a helmet must be approved by the
	Safety Chairman before the modified helmet can be used by a buggy driver.

	A helmet may also be modified, as described above, in the lower rear area
	(where the back of the driver's neck would be) in order to provide adequate
	head tilt for the driver. The requirements are the same as described above
	except that the maximum dimensions of the cut away area shall be 2.0 inches
	high by 5.0 inches wide.

\subsection{Safety Harness}

	Drivers MUST wear an approved safety harness whenever they are driving a buggy.
	Any harness that is examined and deemed to be acceptable by the Safety
	Chairman, or is supplied through the Sweepstakes Advisor, is considered to be
	an approved harness.

\subsubsection{Attachment}

	Every driver's safety harness must be attached to that driver's buggy at a
	minimum of three different points, whenever that person is driving that buggy.
	These attachments shall only be made to structural members of the buggy, and
	they shall be such that the movement of the driver is restricted in all
	directions (i.e. forward, rearward, side to side, and vertical)such that in the
	event of an impact to the buggy from any direction, the driver shall be
	retained inside, but not impact against, the driver's protective cage. All
	harness attachment points shall be approved by the Safety Chairman during the
	buggy's safety inspection.

	Safety harnesses must have the capability to permit quick and easy removal of
	the driver from the buggy in the event of an accident or other problem while
	the driver is secured in the buggy with the safety harness. This removal must
	be possible without the use of tools or other mechanical devices.

\subsubsection{Construction}

	The straps from which a safety harness is constructed must be at least 1.75
	inches wide. This is to ensure that the forces of a collision are distributed
	over a relatively large area of a driver's body. Standard automotive seat belt
	material is acceptable. With the approval of the Safety Chairman, the lower
	portion of a safety harness may consist of a commercially available rock
	climbing harness, even though its straps may be less than 1.75 inches wide.

\subsubsection{Additional Safety Harness Requirements}
	
	All harness clips must be d-ring carabineers and certified by the UIAA
	(International Mountaineering and Climbing Federation).

	All box stitching on the harness must be completed by professional stitchers,
	performed on a heavy duty industrial sewing machine, and deemed adequate by the
	Safety Chairman. In the event that the Safety Chairman does not believe the
	stitching was professionally done, the buggy organization’s Chairman must
	show a receipt indicating that a professional performed the stitching. In the
	event that the Chairman does not have the receipt the harness will be deemed
	unsafe and will not pass the safety.

	Harnesses can be bought from stores, provided that they comply with the rules
	listed in this section. Many rock climbing stores sell harnesses that comply to
	the rules.

\section{Driver's Protective Cage}

	Each buggy shall be constructed with a structurally sound cage which totally
	surrounds the driver. The purpose of this cage is to help protect the driver
	from front, side, rear, and rollover impacts in the event of an accident. The
	driver's protective cage shall meet the following minimum requirements:

	NO PART of the driver's body shall extend beyond the protective cage.

	The interior of the protective cage shall not have any sharp edges, pointed
	objects, or protruding objects, which could injure the driver in the event that
	the driver is pushed into or thrown against these edges or objects.

	None of the wheels of the buggy shall be considered to be part of the driver's
	protective cage.

	The protective cage shall be designed and constructed in such a way that injury
	to the driver will be minimized in the event that one or more of the buggy's
	wheels become detached from the buggy, while it is moving.

	The protective cage shall be designed and constructed in such a way that a 1.0
	inch wide by 72 inch long infinitely stiff bar, held in ANY position in a
	vertical plane perpendicular to the centerline of the buggy (this means with
	the bar's axis vertical, horizontal, or at any angle,) will not intrude into
	the front of the cage, nor contact any part of the driver's body, nor increase
	the force distribution on the driver's body, when pushed toward the buggy with
	a force of 1,000 pounds, while the buggy is restrained so that it cannot move.

	The protective cage shall be designed and constructed in such a way that a 2.0
	inch wide by 12 inch long infinitely stiff bar, held in ANY position (see
	preceding paragraph) in any vertical plane that is either perpendicular to or
	parallel to the centerline of the buggy, will not intrude into the sides or
	rear of the cage, nor contact any part of the driver's body, nor increase the
	force distribution on the driver's body, when pushed toward the buggy with a
	force of 500 pounds, while the buggy is restrained so that it cannot move.

	The protective cage shall be designed and constructed in such a way that it
	shall not collapse or deform such that any part of the driver's body would be
	contacted (by the cage or by the steel plate,) if a 2 ft. by 5 ft. by 1.00 inch
	thick steel plate was slowly lowered onto the top of the buggy. (This plate
	weighs approximately 408 pounds.)

	If a buggy has a pushbar that can change its position (this means that it is
	not rigidly and permanently attached to the buggy,) that pushbar shall NOT be
	considered to be a part of that buggy's protective cage. Any buggy with such a
	pushbar, must meet all of the requirements for the driver's protective cage
	with its pushbar in any of its possible positions. This is to ensure that the
	pushbar itself, will not cause an injury to the driver in the event of an
	accident.

	The protective cage shall be designed and constructed in such a way that no
	part of the driver's body shall be exposed to the road surface. The purpose of
	this requirement is to ensure that there is some substantial structure
	protecting the driver from below in the event that the bottom of the buggy
	comes in contact with another object, such as if the wheels came off and the
	buggy slid along the ground.

\section{Field of Vision}

	Each buggy shall pass a field of vision test, at least once each school
	semester, with each person who will be driving it, before that person will be
	permitted to drive that buggy in any type of practice session at Carnegie
	Mellon University during that semester. The purpose of the field of vision test
	is to ensure that each buggy provides a minimum field of vision to each person
	who will be driving it. The field of vision test shall be administered by the
	Safety Chairman, or anyone designated by that Chairman. The procedure for the
	field of vision test shall be as follows:

	The test shall take place at the same location as the braking capability test.

	The test shall be conducted as follows:

	The driver to be tested shall be placed inside the buggy with all of the
	driver's personal protection equipment in place as it would be if the driver
	were going to drive the buggy, i.e. wearing a properly secured helmet, a safety
	harness, goggles, and gloves.

	The buggy shall be aligned in such a way that a 90 degree angle, whose origin
	is located at the point where the centerline of the buggy and the buggy's front
	axle intersect, and whose bisector is coincident with the centerline of the
	buggy, can be determined.

	Cards with identifiable symbols on them shall be placed at various locations
	throughout the 90 degree angle. The tester may also hold up between 0 and 10
	fingers.

	The driver inside the buggy shall be asked if he or she can identify the
	symbols on the cards, or how many fingers the tester is holding up, and
	features and objects on the horizon beyond the cards.

	To successfully complete the test, the following requirements must be met:

	The driver inside the buggy must be able to identify the symbol on a cards, or
	the number of fingers the tester is holding up, placed at ground level at a
	location that is 25 feet in front of the buggy's front axle and along a line
	formed by the centerline of the buggy.

	The driver inside the buggy must be able to identify the symbols on cards, or
	the number of fingers the tester is holding up, placed at ground level at
	locations that are 25 feet from the point at which the front axle of the buggy
	and the centerline of the buggy intersect and are along lines that are at 45
	degree angles to, and to either side of, the centerline of the buggy, and which
	intersect the centerline of the buggy at the same point at which the buggy's
	front axle intersects it.

	The driver inside the buggy must be able to identify features and objects that
	are located on the horizon along a line formed by the centerline of the buggy.

	The driver inside the buggy must be able to identify features and objects that
	are located on the horizon along lines that are at 45 degree angles to, and to
	either side of, the centerline of the buggy, and which intersect the centerline
	of the buggy at the same point at which the buggy's front axle intersects it.

\section{Pushbar}

	Each buggy shall have a structure which its pushers can push on, in order to
	propel that buggy forward. This structure shall be known as a pushbar. Each
	pushbar shall have a structurally sound cross-piece at its end, which shall be
	known as the pushbar handle. The pushbar handle will usually be perpendicular
	to the pushbar and, to ensure safety, should have the recommended, but not
	required, dimensions of 8 inches long by 1 inch in diameter.

	If any part of a buggy's pushbar moves from the position that it is normally in
	when its pushers are propelling that buggy forward, that part or parts of that
	pushbar may only move to a position that is inside the driver's protective
	cage, or to a position that is directly above the outline of the driver's
	protective cage when viewed from above the buggy. During any such movement, no
	part of the pushbar may extend below or behind the normal pushing position of
	the pushbar unless it is also within or above the outline of the driver's
	protective cage at the same time.

\section{Restrictions}

	No buggy may have any device whose purpose is to store energy independent of
	the speed of the buggy (such as a geared flywheel) or whose purpose is to
	increase the forward speed of that buggy by adding kinetic energy to that
	buggy.

	No buggy may have any means of internal propulsion.

	No buggy may have any type of protrusion (such as a sharp and/or pointed
	structure) extending from that buggy which could increase the likelihood of
	injury to the driver of another buggy in the event of an accident. The
	nose/tail of a buggy should be parabolic and not come to a point. If it were a
	curve, the derivative of any point on that curve should be defined across the
	whole length. This prohibits nose and tails that come to a point in three or
	two dimensions (i.e. a true point or an edge.) An allowed exception to this
	would be flat or cut-off tails, though the flat face must form an angle between
	45 and 90 degrees to the ground. The minimum radius of curvature permitted for
	any nose or tail is .75. Final discretion as to what might be considered overly
	pointy is up to the discretion of the Safety Chairman.

	No buggy may be designed such that it would intentionally have less than three
	wheels in contact with the ground whenever it is on the buggy course.

\section{Shell}

	Each buggy shall have a means to securely attach its shell, hatch, or cover, if
	that part of the buggy is removable. The adequacy of any attachment method
	shall be determined by the Safety Chairman, during the buggy's safety
	inspection.

\section{Size}

	No buggy shall have at any time any dimension, parallel to the centerline of
	the buggy, greater than 15 feet including that buggy's pushbar.

	No buggy shall have at any time any dimension, perpendicular to the centerline
	of the buggy, greater than 6 feet.

\section{Steering System}

	Each buggy shall have a driver operated system or mechanism that shall control
	and determine the direction in which that buggy will move, whenever that buggy
	is in motion. This system or mechanism shall be known as that buggy's steering
	system. Each buggy's steering system shall meet the following minimum
	requirements:

	All fasteners and hardware (i.e. nuts, bolts, collars, etc.) that are used to
	attach or to secure any part of a buggy's steering system, shall be equipped
	with locking devices such as lock nuts, lock washers, lock wires, cotter pins,
	etc. The use of anaerobic locking compounds (such as Loctite) is permissible,
	but is not sufficient.

	Each buggy's steering system shall operate in a smooth manner. The steering
	system shall not bind or behave in an erratic way during use, such that it
	might cause the driver to lose control of the buggy.

	The steering system of each buggy shall be designed such that it can be
	operated in a stable manner, i.e. the steering mechanism should usually tend to
	return to a neutral (straight ahead) position when no steering inputs are
	received from the driver.

	The steering system shall be designed such that the driver will be able to
	control the buggy under all normal driving conditions.

\section{Wheels}

	All fasteners and hardware (i.e. nuts, bolts, collars, etc.) that are used to
	secure a buggy's wheels to that buggy shall be equipped with locking devices
	such as lock nuts, lock washers, lock wires, cotter pins, etc. The use of
	anaerobic locking compounds (such as Loctite) is permissible,but is not
	sufficient.

\subsection{Restrictions}

	No buggy may use any official soap box derby race wheels if those wheels are
	constructed of any type of plastic material. Experience has shown that these
	wheels, while adequate for straight line soap box derby racing, are
	structurally deficient in buggy racing applications.

\section{Windscreens}

	Each buggy shall have a windscreen in front of the driver's face in order to
	help protect the driver from wind, dust, flying objects, and debris. Each
	windscreen shall be made of clear polycarbonate plastic. (The common trade name
	for polycarbonate is LEXAN, which is a registered trademark of the General
	Electric Company.) The minimum nominal thickness for each windscreen shall be
	0.062 inches.

	Each buggy's windscreen shall be mechanically attached to structural members of
	that buggy such that in the event of an impact to that windscreen, the
	windscreen will fail before the windscreen's supporting members will fail. Each
	windscreen shall be secured along more than two of its edges with mechanical
	fasteners (such as bolts, rivets, etc.) that are approved by the Safety
	Chairman. Attachment of the windscreen to the buggy with duct tape or other
	types of tape is permissible, but is not sufficient.

	The use of tinted or mirrored windshields on buggies during after dark practice
	sessions must be approved by the Safety Chairman.

\subsection{Interior Windscreens}

	Any buggy which has a wheel located in front of the driver, shall also have a
	protective shield between that wheel and the driver's face. This shield shall
	be made of clear polycarbonate plastic, which has a minimum nominal thickness
	of 0.062 inches. (The common trade name for polycarbonate is LEXAN, which is a
	registered trademark of the General Electric Company.) It shall extend from the
	bottom of the buggy to the top of the buggy, and shall be at least 4.0 inches
	wide. It shall be attached to a structural member of the buggy with mechanical
	fasteners (such as bolts, rivets, etc.)and its attachment method shall be
	approved by the Safety Chairman during the buggy's safety inspection.

	The purpose of this shield is to reduce the possibility of debris being thrown
	into the driver's eyes and/or face by the wheel. This debris could come from
	either the road surface or from the wheel itself, in the event that the tire
	fails and breaks into pieces.

\subsection{Fogging}

	Anti-fogging compounds or solutions shall be used to prevent windscreens
	from fogging.


