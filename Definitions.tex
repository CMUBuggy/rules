\chapter{Definitions}

Terms listed in this section, when used in other parts of these Rules, Regulations and Procedures, shall be defined as follows:

\begin{description}
	
	\item[Alternate Team:] 
	a group of six people, consisting of five pushers and one driver,
	designated as possible substitutes for any of the members of an entry's
	team. 

	\item[Assistant Head Judge:]
	a person appointed by the Sweepstakes Chairman to assist the Head Judge
	during the Sweepstakes races. The Assistant Head Judge usually observes the
	races from the follow car.

	\item[Assistant Sweepstakes Chairman:]
	a person elected by the Buggy Chairmen to aid the Sweepstakes Chairman in
	organizing and supervising all Sweepstakes activities during the entire school
	year, and to act in the place of the Sweepstakes Chairman in his or her
	absence.

	\item[Braking Capability Test:]
	a test of a buggy's braking system which demonstrates a minimum performance
	standard of that system while a particular person is driving that buggy.

	\item[Buggy:]
	any vehicle which complies with all of the criteria listed in the Buggy
	Construction and Performance Requirements section of these Rules, Regulations,
	and Procedures.

	\item[Buggy Book Chairman:]
	a person appointed by the Sweepstakes Chairman to organize and supervise
	all activities required to compile, design, edit, publish, and distribute the
	Buggy Book which describes the Sweepstakes Competitions.

	\item[Buggy Chairman:]
	a person who is in charge of an organization's buggy program and who is
	empowered to represent that organization at all official Sweepstakes functions.
	The office of Buggy Chairman for each organization may be jointly held by two
	or more people. All Buggy Chairmen serve as members of the Sweepstakes
	Committee.

	\item[Centerline of the Buggy:]
	an imaginary line on a buggy that would pass through the nose of the driver
	of that buggy and would be aligned in the direction in which that buggy would
	travel if that buggy was not turning to the right or the left.

	\item[Chute:]
	the section of the freeroll portion of the buggy course (near the
	southwestern end of Frew Street at its intersection with Schenley Drive) where
	buggies make the sharp right-hand turn from Schenley Drive onto Frew Street.

	\item[Chute Flagger:]
	a person required by the Sweepstakes Rules, Regulations and Procedures to
	be located on the southern curb of Schenley Drive, just east of its
	intersection with Frew Street, to provide a signal (usually with a flag)to
	buggy drivers so that those drivers know when to start the right-hand turn from
	Schenley Drive onto Frew Street.

	\item[Course:]
	the public streets of the Carnegie Mellon University campus and of Schenley
	Park on which the Sweepstakes races are held. The course starts on Margaret
	Morrison Street near its intersection with Tech Street and ends on Frew Street
	near the eastern end of the Graduate School of Industrial Administration
	building.

	\item[Course Judge:]
	a person designated to watch the races from some position around the buggy
	course in order to help the Head Judge in evaluating any mishaps that might
	occur during the races. Course judges are usually provided by the organizations
	participating in the Sweepstakes races or be selected from among the University
	faculty and staff by the Sweepstakes Advisor and the Sweepstakes Committee

	\item[Course Marshal:]
	a person provided by an organization participating in the Sweepstakes races
	to help control the race spectators on or near the buggy course during the
	races.

	\item[Design Chairman:]
	a person (male or female) appointed by the Sweepstakes Chairman to organize
	and supervise all Design Competition activities during the entire school year.

	\item[Driver:]
	a person who travels with a buggy and controls that buggy's motions by
	operating its steering and braking systems.

	\item[Drop Brake Test:]
	a test of a buggy's braking system which demonstrates that a buggy's brakes
	are currently in proper working order and that the driver is able to actuate
	them.

	\item[Entry:]
	one buggy, one team, and one alternate team.

	\item[Entry Withdrawal:]
	the act of removing an entry from the Sweepstakes competition after it has
	selected a heat and lane for the preliminary races.

	\item[Executive Committee:]
	a committee composed to the elected sweepstakes officials: the Sweepstakes
	Chair, the Assistant Sweepstakes Chair, and the Saftey Chair.	

	\item[Flagger:]
	a person provided by an organization participating in the Sweepstakes races
	to help control vehicular traffic on the buggy course during freeroll
	practices, push practices, or the races.

	\item[Follow Car:]
	a vehicle which follows the last buggy in a Sweepstakes race heat or of a
	group of an organization's buggies on the buggy course during a freeroll
	practice, so that its occupants may observe the buggies while they are on the
	course and assist drivers in the case of an emergency.

	\item[Freeroll:]
	that portion of the buggy course over which a buggy is not being pushed by
	a pusher, usually starting just over the top of Hill 2 and ending between the
	chute and the Hill 3-4 Transition Zone.

	\item[Freeroll Practice:]
	an authorized time period during which the buggy course is closed to
	vehicular traffic and participating organizations are permitted to use that
	entire course for driver and pusher practice and training .

	\item[Fulltime Student:]
	a person who is currently enrolled as an undergraduate at Carnegie Mellon
	University and is carrying a course load of at least 36 units,or who is a last
	semester student carrying a course load that is at least the minimum needed for
	graduation after that semester, or who is a graduate student who has paid the
	activities fee.

	\item[Head Judge:]
	a person appointed by the Sweepstakes Chairman to observe the Sweepstakes
	races (usually from the lead car), to hear and rule on all protests and
	appeals, to render final decisions on any disputes concerning the rules
	governing the Sweepstakes races, and to supervise and coordinate the activities
	of all of the course judges.

	\item[Head Timer:]
	a person appointed by the Sweepstakes Advisor to time entries in the
	Sweepstakes races and to coordinate the efforts of all of the other race
	timers.

	\item[Hills:]
	that portion of the buggy course over which a buggy is being pushed by a
	pusher, usually encompassing the areas between the starting line and just over
	the top of Hill 2 and also between the chute and the finish line.

	\item[Lead Car:]
	a vehicle which drives in front of the leading buggy in each heat of the
	Sweepstakes races so that its occupants may observe the races, especially to
	view the first place finisher of each heat.

	\item[Nose of the Buggy:]
	that part of a buggy that is farthest forward relative to the direction in
	which that buggy was last traveling. This farthest forward point could be a
	point on the body of the buggy, a point on the pushbar, a point on a leading
	wheel, or a point on any other part of the buggy.

	\item[Organization:]
	a club, sorority, fraternity, or other group of currently enrolled Carnegie
	Mellon University students that is officially recognized by the Student Senate
	of Carnegie Mellon University.

	\item[Pass Test:]
	a test of a buggy and driver which demonstrates that that driver has
	certain minimum driving skills while driving that buggy. This test requires
	that the driver and buggy under test pass another freerolling buggy during a
	freeroll practice session.

	\item[Pushbar:]
	a structure attached to a buggy which a person can push on in order to
	propel that buggy forward.

	\item[Pusher:]
	a person who propels a buggy forward along one of the five hills of the
	buggy course by pushing on that buggy's pushbar.

	\item[Push Team:]
	a group of five people who are members of an entry's team and who push that
	entry's buggy during a Sweepstakes race.

	\item[Push Practice:]
	an authorized time period during which the uphill portions (Tech and Frew
	Streets) of the buggy course can be closed to vehicular traffic and
	participating organizations are permitted to use those portions of the course
	for pusher and driver practice and training.

	\item[Saftey Chairman:]
	a person elected by the Sweepstakes Committee to conduct the safety
	inspections required of all participating buggies, to conduct Spot Safety
	Checks of buggies participating in both freeroll practices and push practices,
	to help in the driver education program, to observe pass tests, to enforce all
	safety related requirements at all freeroll practices and push practices, and
	to evaluate and rule on, any safety related issues that arise relative to any
	Sweepstakes activities the entire school year.

	\item[Safety Inspection:]
	an examination of a buggy by the Safety Chairman to determine if that buggy
	conforms with all of the applicable Buggy Construction and Performance
	Requirements specified in these Rules, Regulations, and Procedures.

	\item[Shell:]
	the entire outer structure or covering of a buggy which determines that
	buggy's aerodynamic characteristics. A buggy's shell may or may not be a part
	of that buggy's driver's protective cage.

	\item[Signal Flagger:]
	a person positioned along the buggy course who, using a flag or other
	device, signals a buggy driver in order to help that driver in negotiating the
	course.
	
	\item[Spin Test:]
	The buggy shall be placed on a surface or held such that all wheels are able to 
	spin freely. All wheels shall be spun by hand and then the brake exercised to 
	ensure it operates properly, and no tactics are employed to aid in passing the 
	drop test or slowing forward motion of the buggy that would not be used in an 
	emergency condition on the course. Correct operation of some designs may not 
	affect the wheels during this test. Any wheels that have been stopped upon 
	exercising the brake should be spun again to ensure their continued free 
	motion.

	\item[Spot Safety Check:]
	an examination of any buggy and its driver participating in any type of
	buggy practice session at Carnegie Mellon University by the Sweepstakes
	Chairman, the Assistant Sweepstakes Chairman, the Safety Chairman, or anyone
	designated by the Sweepstakes Advisor in order to determine if that buggy and
	its driver are currently in compliance with all of the applicable rules,
	regulations, and procedures specified in this document.

	\item[Starter:]
	a person appointed by the Sweepstakes Chairman to officially start each
	heat of the Sweepstakes races using a traditional starting gun, and to act as a
	course judge for as much of the buggy course as he or she can see from the
	starting location.

	\item[Sweeper:]
	a person provided by an organization participating in the Sweepstakes races
	to help clean debris from the buggy course during freeroll practices, push
	practices, or the races.

	\item[Sweepstakes:]
	the official term used to designate the Carnegie Mellon University buggy
	races.

	\item[Sweepstakes Advisor:]
	a person employed by the Department of Student Activities within the
	Division of Student Affairs of Carnegie Mellon University who has the overall
	responsibility for directing and supervising all student activities at Carnegie
	Mellon University, including Spring Carnival and the Sweepstakes Competitions.

	\item[Sweepstakes Chairman:]
	a person elected by the Buggy Chairmen to organize and supervise all
	Sweepstakes activities during the entire school year, including the Sweepstakes
	races, the Design Competition, the Buggy Book, all freeroll practice sessions,
	all push practice sessions, the Sweepstakes awards presentations, and all
	preparation activities for any of these events. After his or her election, the
	Sweepstakes Chairman presides over the Sweepstakes Committee.

	\item[Sweepstakes Committee:]
	the governing body of the Sweepstakes Competitions which is presided over
	by the Sweepstakes Chairman and is composed of the Sweepstakes Chairman, the
	Assistant Sweepstakes Chairman, the Safety Chairman, the Design Chairman, the
	Buggy Book Chairman, and all of the Buggy Chairman from all of the
	organizations participating in the Sweepstakes Competitions.

	\item[Team:]
	a group of six people, consisting of one driver and five pushers, all of
	whom are currently fulltime Activities Fee paying undergraduate or graduate
	students of Carnegie Mellon University.

	\item[Timer:]
	a person appointed by the Sweepstakes Advisor to measure the time required
	for a buggy to travel from the starting line to the finish line of the buggy
	course during a Sweepstakes race. While performing their duties,timers are
	under the direction of the Head Timer.

	\item[Transition:]
	the procedure whereby one pusher finishes pushing a buggy and the next
	pusher in sequence starts to push that same buggy.

	\item[Transition Zone:]
	an area on the buggy course that is 45 feet long and as wide as the street
	or lane on which it is located in which two different pushers are permitted to
	touch a buggy, separately or simultaneously.

\end{description}

