\chapter{Staff, Organization, and Procedures}

\section{Administrative Responsibilities}

\subsection{Personnel}

\subsubsection{Sweepstakes Advisor}

	The Sweepstakes Advisor is employed by the office of Student Activities within
	the Division of Student Affairs of Carnegie Mellon University. The Sweepstakes
	Advisor has the overall responsibility for directing and supervising all
	Sweepstakes related activities.	

	As such, the Sweepstakes Advisor has responsibility and accountability for all Sweepstakes
	related matters. The Sweepstakes Advisor should work closely with the Sweepstakes Chairman
	and his or her assistants, and with the Sweepstakes Committee, providing
	direction, advice, and assistance whenever it is needed. In this position of
	ultimate responsibility, the Sweepstakes Advisor may, at his or her discretion,
	effect changes in the organizational structure, the rules, the regulations, or
	the procedures governing the Sweepstakes Competition.

\subsubsection{Sweepstakes Chairman}

	The Sweepstakes Chairman shall be elected by the Buggy Chairmen of all of the
	organizations participating in the Sweepstakes Competition. The Sweepstakes
	Chairman may NOT also be a Buggy Chairman for any organization.

	The duties of the Sweepstakes Chairman are to organize and supervise all
	Sweepstakes activities during the entire school year. These activities include
	the Sweepstakes races, the Design Competition, the Buggy Book, all freeroll
	practice sessions, all push practice sessions, the Sweepstakes awards
	presentations, and all preparation activities for any of these events.

	The Sweepstakes Chairman may appoint one or more people as assistants to help
	the Sweepstakes Chairman, the Assistant Sweepstakes Chairman, the Safety
	Chairman, the Design Chairman, and the Buggy Book Chairman in the execution of
	their duties. In the case of the Safety Chairman, an assistant shall only be
	named in his/her absence, and only after consulting the Safety Chairman. Any
	person who is to stand in place of the Safety Chairman for any period of time,
	must first be approved by a majority vote of the organizational Buggy Chairmen.
	In extreme emergencies, where the Safety Chairman is not present and no person
	has been approved by a majority vote to act as Safety Chairman, the Sweepstakes
	Chairman or the Assistant Chairman shall act in the Safety Chairman's place by
	default.

\subsubsection{Assistant Sweepstakes Chairman}

	The Assistant Sweepstakes Chairman (Assistant Chairman) shall be elected by the
	Buggy Chairmen of all of the organizations participating in the Sweepstakes
	Competition. The Assistant Sweepstakes Chairman may NOT also be a Buggy
	Chairman for any organization.

	The duties of the Assistant Sweepstakes Chairman are to aid the Sweepstakes
	Chairman in organizing and supervising all Sweepstakes activities during the
	entire school year, and to act in the place of the Sweepstakes Chairman in his
	or her absence.

\subsubsection{Safety Chairman}

	The Safety Chairman shall be elected by the Buggy Chairmen of all of the
	organizations participating in the Sweepstakes Competition. The Safety Chairman
	may NOT also be a Buggy Chairman for any organization.

	The duties of the Safety Chairman are to conduct the safety inspections
	required of all participating buggies, to conduct Spot Safety Checks of buggies
	participating in both freeroll practices and push practices, to observe pass
	tests, to enforce all safety related requirements at all freeroll practices and
	push practices, and to evaluate and rule on, any safety related issues that
	arise relative to any Sweepstakes activities during the entire school year.

\subsubsection{Design Chairman}

	The Design Chairman shall be appointed by the Sweepstakes Chairman. The Design
	Chairman may NOT also be a Buggy Chairman for any organization.

	The duties of the Design Chairman are to organize and supervise all Design
	Competition activities during the entire school year. These activities include
	the appointment of judges, the selection of a time and place for the
	competition, the arrangements for publicity, the supervision of both the public
	display of the buggies and the presentation of the buggies to the judges, the
	compilation of the results of the judging and the presentation of the awards
	for the competition. The Design Chairman may select one or more people to
	assist him or her in the execution of his or her duties. The Sweepstakes
	Chairman must approve any people selected as assistants by the Design Chairman.

\subsubsection{Buggy Book Chairman}

	The Buggy Book Chairman shall be appointed by the Sweepstakes Chairman. The
	Buggy Book Chairman may NOT also be a Buggy Chairman for any organization.

	The duties of the Buggy Book Chairman are to organize and supervise all
	activities required to compile, design, edit, publish, and distribute the Buggy
	Book which describes the Sweepstakes Competitions. The Buggy Book Chairman
	shall also be responsible for the design, ordering and distribution of the
	official Sweepstakes wear such as t-shirts and hats. The Buggy Book Chairman
	may select one or more people to assist him or her in the execution of his or
	her duties. The Sweepstakes Chairman must approve any people selected as
	assistants by the Buggy Book Chairman.

\subsubsection{Buggy Chairman}

	Each organization participating in the Sweepstakes Competitions shall designate
	one or more people as its representatives for all official Sweepstakes
	functions. These representatives shall be known as Buggy Chairmen and they
	shall serve as members of the Sweepstakes Committee.

\subsection{Committees}

\subsubsection{Sweepstakes Advisory Council}

	At the discretion of the Sweepstakes Advisor and/or the Dean of Student
	Affairs, a Sweepstakes Advisory Council may be appointed. The purpose of this
	council is to provide advice, assistance, and guidance to the Sweepstakes
	Advisor relative to his or her interactions with the Sweepstakes Committee.
	This council should be composed of any Carnegie Mellon University alumni and/or
	staff who can provide the experience and knowledge needed by the Dean of
	Student Affairs, the Sweepstakes Advisor, and the Sweepstakes Committee. In
	addition, the elected members of the Sweepstakes Committee (the Sweepstakes
	Chairman, the Assistant Sweepstakes Chairman, and the Safety Chairman)and three
	Buggy Chairmen (elected by the Sweepstakes Committee) also serve as members of
	the Sweepstakes Advisory Council. The Sweepstakes Advisor serves as the
	Chairman of the Sweepstakes Advisory Council.

\subsubsection{Sweepstakes Committee}

	The Sweepstakes Committee shall be the governing body of the Sweepstakes
	Competitions. It shall consist of the Sweepstakes Chairman, the Assistant
	Sweepstakes Chairman, the Safety Chairman, the Design Chairman, the Buggy Book
	Chairman, and all of the Buggy Chairman from all of the organizations
	participating in the Sweepstakes Competitions. The Sweepstakes Committee shall
	be presided over by the Sweepstakes Chairman.

\section{Functional Organization}

\subsection{Chain of Command}

	By permitting Sweepstakes Competitions to take place on its campus, Carnegie
	Mellon University accepts responsibility for those competitions and is thus
	authorized to take any actions necessary to ensure that those competitions are
	conducted in a manner that is as orderly and as safe as is possible. This
	responsibility is administered by the University through the Division of
	Student Affairs, the Dean of Student Affairs, the Department of Student
	Activities, and most directly, through the Sweepstakes Advisor.

	At the student level, the responsibility for actually organizing, conducting,
	and supervising all activities related to the Sweepstakes Competitions is
	administered by all of the members of the Sweepstakes Committee, and most
	directly, by the Sweepstakes Chairman. The Sweepstakes Chairman has the
	responsibility of ensuring that all activities related to the Sweepstakes
	Competitions are conducted according to the rules, regulations, and procedures
	that have been approved by the University, through its representatives.

	The link between the University and the student participants of the Sweepstakes
	Competitions is accomplished through the Sweepstakes Advisor and the
	Sweepstakes Chairman. The Sweepstakes Committee is directly accountable to the
	University through this link. During the actual organization and execution of
	the Sweepstakes Competitions, the Sweepstakes Chairman must work closely with
	the Sweepstakes Advisor in order to assure the University that the competitions
	are being conducted in an acceptable manner. In addition, the Sweepstakes
	Advisor can provide much advice, direction, and assistance to the Sweepstakes
	Chairman and other members of the Sweepstakes Committee concerning procedural
	and organizational aspects of the Sweepstakes Competitions.

	When desired, the Sweepstakes Advisor, the Sweepstakes Chairman, and any of the
	members of the Sweepstakes Committee may seek opinions, advice, and assistance
	from the members of the Sweepstakes Advisory Council concerning any aspect of
	Sweepstakes related activities.

\section{Governing Procedures}

\subsection{Governing Body}

	The Sweepstakes Competitions shall be governed by the Sweepstakes Committee,
	which is presided over by the Sweepstakes Chairman. The committee shall govern
	by the democratic process. The only members of the committee eligible to vote
	shall be the Buggy Chairmen of all of the participating organizations. Each
	participating organization is entitled to only one vote on any issue voted on
	by the committee (i.e. only one Buggy Chairman from each organization is
	eligible to vote on any single issue). In the event of a tie vote, the
	Sweepstakes Chairman shall cast the deciding vote.

	All binding decisions shall be made by a majority vote of the eligible voting
	members of the committee who are present, with one vote per organization. NO
	issue may be voted on unless two-thirds of the competing Buggy organizations,
	who are eligible to vote, are represented.

\subsection{Meetings}

	Meetings of the Sweepstakes Committee shall be scheduled by the Sweepstakes
	Chairman whenever that Chairman considers that a meeting is necessary, or when
	three or more organizations request such a meeting. Sweepstakes Committee
	meetings should usually be held once a week during the weeks that freeroll
	practice sessions are scheduled, and somewhat less often at other times of the
	school year, as needed.

	As a rule of thumb, meetings should be consistently scheduled each week at the
	same time and on the same day (usually every Monday evening at 10pm) to make it
	reasonable for organizations to regularly attend. Any organization failing to
	have a representative of that organization in attendance at any Sweepstakes
	Committee meeting for which at least 24 hours notice has been given, shall be
	fined the amount of \$15.00.

\subsection{Initial Organizational Meeting}

	At the beginning of each school year an organizational meeting of the
	Sweepstakes Committee shall be held. The purpose of this meeting is to elect a
	Sweepstakes Chairman, an Assistant Sweepstakes Chairman, and a Safety Chairman
	for that school year. This meeting should be held before September 15th each
	school year. The organization that sponsored the winning entry in the previous
	school year's Sweepstakes races shall be responsible for scheduling,
	publicizing, organizing, and supervising this meeting.

\subsection{Amendments to the Rules}

	Amendments to these Sweepstakes Rules, Regulations, and Procedures may be made
	in the following ways:

	\begin{itemize}

		\item
		The Dean of Student Affairs and/or the Sweepstakes Advisor may, at their
		discretion, direct that any changes which they specify, be made to these
		Sweepstakes Rules, Regulations, and Procedures, at any time.

		\item
		Through the Sweepstakes Chairman, the Sweepstakes Committee may, by a majority
		vote of a quorum, propose changes to these Sweepstakes Rules, Regulations, and
		Procedures, at any time by submitting, in writing, those proposed changes to
		the Sweepstakes Advisor. Any proposed changes shall not become effective until,
		and unless, they are approved in writing by the Sweepstakes Advisor.

	\end{itemize}

	The Sweepstakes Advisor must either approve or reject any proposed changes within two
	weeks of the date on which those changes are submitted. If any proposed changes
	are submitted to the Sweepstakes Advisor on a date that is less than two weeks before the
	preliminary races are held that school year, the Sweepstakes Advisor is not required to
	act on those proposed changes. If the Sweepstakes Advisor chooses not to act on those
	proposed changes, they may be resubmitted the following school year after they
	have again been voted on by the Sweepstakes Committee. If the Sweepstakes Advisor
	considers that more than two weeks is required to evaluate any proposed
	changes, he or she may extend the two week response period,provided that the
	Sweepstakes Committee is informed of this extension and of its expected
	duration.

\section{Disciplinary Actions}

\subsection{Fines}

	Organizations may be fined by the Sweepstakes Advisor, the Sweepstakes Chairman
	or the Safety Chairman for violations of any of the rules and regulations
	specified in this document. The dollar amount of these fines is usually called
	out in the rule or regulation invoking the fine.

	Other fines, not listed within these rules, may be specified and imposed by the
	Sweepstakes Advisor or the Sweepstakes Chairman in response to continued
	negligence on the part of organizations or any need that may arise. In such a
	case, the conditions for and the dollar amounts of those other fines must be
	made known to the Sweepstakes Committee members before any of those fines are
	imposed.

	A summary of all of the fines that are specified in detail elsewhere in these
	Rules, Regulations, and Procedures follows:

	\begin{itemize}

		\item Failure to be represented at a Sweepstakes Committee meeting: \$15.00
		per meeting

		\item Failure to pass a spot safety check: \$25.00 for the first occurrence
		and \$50.00 for any additional occurrences

		\item Failure to provide required lights and reflectors for after dark
		practices (both at push practices and freerolls): \$20.00 per buggy per
		occurrence

		\item Failure to provide or remove No-Parking signs for any freeroll
		practice: \$25.00 per occurrence

		\item Failure to provide properly equipped sweepers for any freeroll
		practice: \$15.00 per sweeper

		\item Failure to provide or remove hay bales for any freeroll practice:
		\$25.00 per occurrence

		\item Failure to provide properly equipped flaggers for any freeroll
		practice: \$20.00 per flagger

		\item Failure to provide or remove barricades for any freeroll practice:
		\$25.00 per occurrence

		\item Failure to provide or remove warning signs for any freeroll practice:
		\$25.00 per occurrence

		\item Failure to have the equipment needed to remove any of its drivers
		from its buggies in the car following the buggies during any freeroll practice:
		\$15.00 per occurrence

		\item Failure to provide a chute flagger for any buggy driver at any
		freeroll practice: \$10.00 per occurrence

		\item Failure of a driver to stop at a yellow stop flag: \$30.00 per
		occurrence

		\item Failure to provide sufficient barricades for a push practice: \$15.00
		per occurrence

		\item Failure to provide properly equipped flaggers for any push practice:
		\$15.00 per flagger

		\item Failure to paint lane and zone markings: \$25.00 per occurrence

		\item Failure to provide people and vehicles for course watch duty the
		night before any day of Sweepstakes racing: \$25.00 per person or vehicle

		\item Failure to provide or remove No-Parking signs for any day of
		Sweepstakes racing: \$50.00 per occurrence

		\item Failure to provide or remove hay bales for any day of Sweepstakes
		racing: \$50.00 per occurrence

		\item Failure to provide or remove barricades for any day of Sweepstakes
		racing: \$50.00 per occurrence

		\item Failure to provide or remove warning signs for any day of Sweepstakes
		racing: \$50.00 per occurrence

		\item Failure to provide or remove crowd control barriers for any day of
		Sweepstakes racing: \$50.00 per occurrence

		\item Failure to provide properly equipped course marshals for any day of
		Sweepstakes racing: \$25.00 per course marshal

		\item Failure to have the equipment needed to remove any of its drivers
		from its buggies in the car following the buggies during any Sweepstakes race:
		\$25.00 per occurrence

		\item Failure to provide a chute flagger for any buggy driver during any
		Sweepstakes race: \$25.00 per occurrence

		\item Having unsafe or dangerous conditions in a buggy preparation area
		before or during any Sweepstakes races: \$100.00 per occurrence

	\end{itemize}

\subsection{Payments}

	Fines should be subtracted from an organization's safety deposit. If an
	organization accrues more than \$50 in fines, they must pay an additional \$100
	deposit before they can participate in any Sweepstakes practice sessions or
	races (as stated under Deposits, section 3.5, in these rules). Additional
	deposits for fines can be made by check, payable to Carnegie Mellon University
	- Sweepstakes, or be transferred from an organization's campus account into the
	Sweepstakes account.

\subsection{Penalties}

	Organizations may be penalized for violations of any of the rules and
	regulations specified in these Rules, Regulations, and Procedures by the
	Sweepstakes Advisor or the Sweepstakes Chairman. If the rule or regulation that
	was violated specifies that the offending organization shall have one of its
	entries withdrawn, that withdrawal shall be accomplished as follows:

	\begin{itemize}

		\item
		If the infraction is attributable to an individual buggy that is scheduled to
		race that day or that school year, that buggy and its entry shall be withdrawn
		from that scheduled day of racing.

		\item
		If the infraction is attributable to an individual buggy that is not scheduled
		to race that day or that school year, one of the buggy's sponsoring
		organization's entries shall be withdrawn from the next day of racing that it
		would have participated in, and the sponsoring organization shall select the
		entry to be withdrawn. (This applies even if an entry must be withdrawn from
		the first day of racing of the following school year.)

		\item
		If the infraction is attributable to some part of an organization but not an
		individual buggy, one of that organization's entries which has not yet run
		shall be withdrawn from the races that same day, or from that organization's
		next scheduled day of racing, and the offending organization shall select the
		entry to be withdrawn. (This applies even if an entry must be withdrawn from
		the first day of racing of the following school year.)

		\item
		If an offending organization is permitted to select the entry to be withdrawn,
		and that organization has entries in both the men's races and the women's
		races, the selection shall be made from the entries in the men's races for that
		organization's first infraction, from the entries in the women's races for the
		second infraction, and so on, alternating for any additional infractions.

	\end{itemize}

\subsection{Appeals}

	Any organization or individual receiving a fine or penalty on any day except a
	day on which races are scheduled, may appeal that fine or penalty to the
	Sweepstakes Advisor. Final determination of any appeals shall be made by the
	Sweepstakes Advisor.

\section{Deposits}

	Each participating organization is required to submit a security deposit of
	\$300 to the Sweepstakes Advisor or the Sweepstakes Chairman before the first
	scheduled freeroll practice during the fall semester of the school year.

	The first \$100.00 of this deposit will be used to cover any fines that an
	organization may incur during the course of the school year. The remaining
	\$200.00 will be used to cover any extraordinary expenses incurred due to any
	incidents that may occur during push practices or freeroll practices which may
	require compensation to people not participating in Sweepstakes, such as damage
	to cars driving on the buggy course or damage to other private property. This
	\$200.00 will also be used to cover any similar expenses which are not
	attributable to any single organization.

	If an organization's deposit for fines (\$100) is reduced to less than \$50.00,
	an additional \$100.00 deposit must be submitted by that organization before
	they will be permitted to participate in any subsequent freeroll practices or
	push practices. If an organization's deposit for extraordinary expenses is
	reduced to less than \$100.00, an additional \$200.00 deposit must be submitted
	by that organization before they will be permitted to participate in any
	subsequent freeroll practices or push practices.

	By the day before the preliminary races are scheduled, each organization shall
	submit, if necessary, an additional security deposit such that their total
	security deposit at that time, shall be at least \$200.00. If anorganization's
	deposit is less than \$200.00 on the day that the races are scheduled, they
	will not be permitted to participate in the races.

	After all Sweepstakes activities have concluded for the year, each organization
	will be refunded its deposit balance. An organization shall receive the full
	amount they paid in deposits over the year, minus fines. No organization shall
	be reimbursed any more than the dollar amount they paid in deposits over the
	year. Deposit funds shall be made either by check or account credit, depending
	on the method of original payment.

\subsection{Payments}

	Deposits should be made by check, payable to Carnegie Mellon University -
	Sweepstakes, or by a money transfer from an organization's campus account into
	the Sweepstakes account.

\section{Unsportsman-like Conduct}

	Any organization may be disqualified from this and/or next year's races if
	their behavior (or the behavior of anyone associated with their organization)
	is deemed unsportsman-like by the Sweepstakes Advisor/Executive Committee.


