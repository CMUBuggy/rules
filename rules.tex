
\documentclass[openany]{book}
\title{Carnegie Mellon Sweepstakes Rules and Regulations}
\author{}
\renewcommand{\chaptername}{Part}
\begin{document}
\maketitle
\tableofcontents
\newpage

\chapter{Purpose}

The purpose of these Rules, Regulations, and Procedures is as follows:
\begin{itemize}
	\item To establish a framework of rules under which the Sweepstakes races and Design Competition can be contested.
	\item To establish minimum safety and performance standards with which buggies must comply to be used in the Sweepstakes race competition.
	\item To establish safety procedures to be used by Sweepstakes participants both during practice sessions and during the race competitions.
	\item To document typical operating procedures for Sweepstakes practice sessions, the Sweepstakes race competitions, and the Design Competition.
\end{itemize}

\chapter{Definitions}

Terms listed in this section, when used in other parts of these Rules, Regulations and Procedures, shall be defined as follows:\\
\\
\textbf{Alternate Team}
\begin{quote}
	a group of six people, consisting of five pushers and one driver, designated as possible substitutes for any of the members of an entry's team.
\end{quote}

\textbf{Assistant Head Judge}
\begin{quote}
	a person appointed by the Sweepstakes Chairman to assist the Head Judge during the Sweepstakes races. The Assistant Head Judge usually observes the races from the follow car.
\end{quote}

\textbf{Assistant Sweepstakes Chairman}
\begin{quote}
	a person elected by the Buggy Chairmen to aid the Sweepstakes Chairman in organizing and supervising all Sweepstakes activities during the entire school year, and to act in the place of the Sweepstakes Chairman in his or her absence.
\end{quote}

\textbf{Braking Capability Test}
\begin{quote}
	a test of a buggy's braking system which demonstrates a minimum performance standard of that system while a particular person is driving that buggy.
\end{quote}

\textbf{Buggy}
\begin{quote}
	any vehicle which complies with all of the criteria listed in the Buggy Construction and Performance Requirements section of these Rules, Regulations, and Procedures.
\end{quote}

\textbf{Buggy Book Chairman}
\begin{quote}
	a person appointed by the Sweepstakes Chairman to organize and supervise all activities required to compile, design, edit, publish, and distribute the Buggy Book which describes the Sweepstakes Competitions.
\end{quote}

\textbf{Buggy Chairman}
\begin{quote}
	a person who is in charge of an organization's buggy program and who is empowered to represent that organization at all official Sweepstakes functions. The office of Buggy Chairman for each organization may be jointly held by two or more people. All Buggy Chairmen serve as members of the Sweepstakes Committee.
\end{quote}

\textbf{Centerline of The Buggy}
\begin{quote}
	an imaginary line on a buggy that would pass through the nose of the driver of that buggy and would be aligned in the direction in which that buggy would travel if that buggy was not turning to the right or the left.
\end{quote}

\textbf{Chute}
\begin{quote}
	the section of the freeroll portion of the buggy course (near the southwestern end of Frew Street at its intersection with Schenley Drive) where buggies make the sharp right-hand turn from Schenley Drive onto Frew Street.
\end{quote}

\textbf{Chute Flagger}
\begin{quote}
	a person required by the Sweepstakes Rules, Regulations and Procedures to be located on the southern curb of Schenley Drive, just east of its intersection with Frew Street, to provide a signal (usually with a flag)to buggy drivers so that those drivers know when to start the right-hand turn from Schenley Drive onto Frew Street.
\end{quote}

\textbf{Course}
\begin{quote}
	the public streets of the Carnegie Mellon University campus and of Schenley Park on which the Sweepstakes races are held. The course starts on Margaret Morrison Street near its intersection with Tech Street and ends on Frew Street near the eastern end of the Graduate School of Industrial Administration building.
\end{quote}

\textbf{Course Judge}
\begin{quote}
	a person designated to watch the races from some position around the buggy course in order to help the Head Judge in evaluating any mishaps that might occur during the races. Course judges are usually provided by the organizations participating in the Sweepstakes races or be selected from among the University faculty and staff by the Sweepstakes Advisor and the Sweepstakes Committee
\end{quote}

\textbf{Course Marshal}
\begin{quote}
	a person provided by an organization participating in the Sweepstakes races to help control the race spectators on or near the buggy course during the races.
\end{quote}

\textbf{Design Chairman}
\begin{quote}
	a person (male or female) appointed by the Sweepstakes Chairman to organize and supervise all Design Competition activities during the entire school year.
\end{quote}

\textbf{Driver}
\begin{quote}
	a person who travels with a buggy and controls that buggy's motions by operating its steering and braking systems.
\end{quote}

\textbf{Drop Brake Test}
\begin{quote}
	a test of a buggy's braking system which demonstrates that a buggy's brakes are currently in proper working order and that the driver is able to actuate them.
\end{quote}

\textbf{Entry}
\begin{quote}
	one buggy, one team, and one alternate team.
\end{quote}

\textbf{Entry Withdrawal}
\begin{quote}
	the act of removing an entry from the Sweepstakes competition after it has selected a heat and lane for the preliminary races.
\end{quote}

\textbf{Flagger}
\begin{quote}
	a person provided by an organization participating in the Sweepstakes races to help control vehicular traffic on the buggy course during freeroll practices, push practices, or the races.
\end{quote}

\textbf{Follow Car}
\begin{quote}
	a vehicle which follows the last buggy in a Sweepstakes race heat or of a group of an organization's buggies on the buggy course during a freeroll practice, so that its occupants may observe the buggies while they are on the course and assist drivers in the case of an emergency.
\end{quote}

\textbf{Freeroll}
\begin{quote}
	that portion of the buggy course over which a buggy is not being pushed by a pusher, usually starting just over the top of Hill 2 and ending between the chute and the Hill 3-4 Transition Zone.
\end{quote}

\textbf{Freeroll Practice}
\begin{quote}
	an authorized time period during which the buggy course is closed to vehicular traffic and participating organizations are permitted to use that entire course for driver and pusher practice and training .
\end{quote}

\textbf{Fulltime Student}
\begin{quote}
	a person who is currently enrolled as an undergraduate at Carnegie Mellon University and is carrying a course load of at least 36 units,or who is a last semester student carrying a course load that is at least the minimum needed for graduation after that semester, or who is a graduate student who has paid the activities fee.
\end{quote}

\textbf{Head Judge}
\begin{quote}
	a person appointed by the Sweepstakes Chairman to observe the Sweepstakes races (usually from the lead car), to hear and rule on all protests and appeals, to render final decisions on any disputes concerning the rules governing the Sweepstakes races, and to supervise and coordinate the activities of all of the course judges.
\end{quote}

\textbf{Head Timer}
\begin{quote}
	a person appointed by the Sweepstakes Advisor to time entries in the Sweepstakes races and to coordinate the efforts of all of the other race timers.
\end{quote}

\textbf{Hills}
\begin{quote}
	that portion of the buggy course over which a buggy is being pushed by a pusher, usually encompassing the areas between the starting line and just over the top of Hill 2 and also between the chute and the finish line.
\end{quote}

\textbf{Lead Car}
\begin{quote}
	a vehicle which drives in front of the leading buggy in each heat of the Sweepstakes races so that its occupants may observe the races, especially to view the first place finisher of each heat.
\end{quote}

\textbf{Nose of The Buggy}
\begin{quote}
	that part of a buggy that is farthest forward relative to the direction in which that buggy was last traveling. This farthest forward point could be a point on the body of the buggy, a point on the pushbar, a point on a leading wheel, or a point on any other part of the buggy.
\end{quote}

\textbf{Organization}
\begin{quote}
	a club, sorority, fraternity, or other group of currently enrolled Carnegie Mellon University students that is officially recognized by the Student Senate of Carnegie Mellon University.
\end{quote}

\textbf{Pass Test}
\begin{quote}
	a test of a buggy and driver which demonstrates that that driver has certain minimum driving skills while driving that buggy. This test requires that the driver and buggy under test pass another freerolling buggy during a freeroll practice session.
\end{quote}

\textbf{Pushbar}
\begin{quote}
	a structure attached to a buggy which a person can push on in order to propel that buggy forward.
\end{quote}

\textbf{Pusher}
\begin{quote}
	a person who propels a buggy forward along one of the five hills of the buggy course by pushing on that buggy's pushbar.
\end{quote}

\textbf{Push Team}
\begin{quote}
	a group of five people who are members of an entry's team and who push that entry's buggy during a Sweepstakes race.
\end{quote}

\textbf{Push Practice}
\begin{quote}
	an authorized time period during which the uphill portions (Tech and Frew Streets) of the buggy course can be closed to vehicular traffic and participating organizations are permitted to use those portions of the course for pusher and driver practice and training.
\end{quote}

\textbf{Safety Chairman}
\begin{quote}
	a person elected by the Sweepstakes Committee to conduct the safety inspections required of all participating buggies, to conduct Spot Safety Checks of buggies participating in both freeroll practices and push practices, to help in the driver education program, to observe pass tests, to enforce all safety related requirements at all freeroll practices and push practices, and to evaluate and rule on, any safety related issues that arise relative to any Sweepstakes activities the entire school year.
\end{quote}

\textbf{Safety Inspection}
\begin{quote}
	an examination of a buggy by the Safety Chairman to determine if that buggy conforms with all of the applicable Buggy Construction and Performance Requirements specified in these Rules, Regulations, and Procedures.
\end{quote}

\textbf{Shell}
\begin{quote}
	the entire outer structure or covering of a buggy which determines that buggy's aerodynamic characteristics. A buggy's shell may or may not be a part of that buggy's driver's protective cage.
\end{quote}

\textbf{Signal Flagger}
\begin{quote}
	a person positioned along the buggy course who, using a flag or other device, signals a buggy driver in order to help that driver in negotiating the course.
\end{quote}

\textbf{Spot Safety Check}
\begin{quote}
	an examination of any buggy and its driver participating in any type of buggy practice session at Carnegie Mellon University by the Sweepstakes Chairman, the Assistant Sweepstakes Chairman, the Safety Chairman, or anyone designated by the Sweepstakes Advisor in order to determine if that buggy and its driver are currently in compliance with all of the applicable rules, regulations, and procedures specified in this document.
\end{quote}

\textbf{Starter}
\begin{quote}
	a person appointed by the Sweepstakes Chairman to officially start each heat of the Sweepstakes races using a traditional starting gun, and to act as a course judge for as much of the buggy course as he or she can see from the starting location.
\end{quote}

\textbf{Sweeper}
\begin{quote}
	a person provided by an organization participating in the Sweepstakes races to help clean debris from the buggy course during freeroll practices, push practices, or the races.
\end{quote}

\textbf{Sweepstakes}
\begin{quote}
	the official term used to designate the Carnegie Mellon University buggy races.
\end{quote}

\textbf{Sweepstakes Advisor}
\begin{quote}
	a person employed by the Department of Student Activities within the Division of Student Affairs of Carnegie Mellon University who has the overall responsibility for directing and supervising all student activities at Carnegie Mellon University, including Spring Carnival and the Sweepstakes Competitions.
\end{quote}

\textbf{Sweepstakes Chairman}
\begin{quote}
	a person elected by the Buggy Chairmen to organize and supervise all Sweepstakes activities during the entire school year, including the Sweepstakes races, the Design Competition, the Buggy Book, all freeroll practice sessions, all push practice sessions, the Sweepstakes awards presentations, and all preparation activities for any of these events. After his or her election, the Sweepstakes Chairman presides over the Sweepstakes Committee.
\end{quote}

\textbf{Sweepstakes Committee}
\begin{quote}
	the governing body of the Sweepstakes Competitions which is presided over by the Sweepstakes Chairman and is composed of the Sweepstakes Chairman, the Assistant Sweepstakes Chairman, the Safety Chairman, the Design Chairman, the Buggy Book Chairman, and all of the Buggy Chairman from all of the organizations participating in the Sweepstakes Competitions.
\end{quote}

\textbf{Team}
\begin{quote}
	a group of six people, consisting of one driver and five pushers, all of whom are currently fulltime Activities Fee paying undergraduate or graduate students of Carnegie Mellon University.
\end{quote}

\textbf{Timer}
\begin{quote}
	a person appointed by the Sweepstakes Advisor to measure the time required for a buggy to travel from the starting line to the finish line of the buggy course during a Sweepstakes race. While performing their duties,timers are under the direction of the Head Timer.
\end{quote}

\textbf{Transition}
\begin{quote}
	the procedure whereby one pusher finishes pushing a buggy and the next pusher in sequence starts to push that same buggy.
\end{quote}

\textbf{Transition Zone}
\begin{quote}
	an area on the buggy course that is 45 feet long and as wide as the street or lane on which it is located in which two different pushers are permitted to touch a buggy, separately or simultaneously.
\end{quote}

\chapter{Staff, Organization, and Procedures}

\section{Administrative Responsibilities}

\subsection{Personnel}

\subsubsection{Sweepstakes Advisor}

The Sweepstakes Advisor is employed by the office of Student Activities within the Division of Student Affairs of Carnegie Mellon University. The Sweepstakes Advisor has the overall responsibility for directing and supervising all student activities at Carnegie Mellon University, including Spring Carnival and the Sweepstakes competitions.

As such, the Director has responsibility and accountability for all Sweepstakes related matters. The Director should work closely with the Sweepstakes Chairman and his or her assistants, and with the Sweepstakes Committee, providing direction, advice, and assistance whenever it is needed. In this position of ultimate responsibility, the Sweepstakes Advisor may, at his or her discretion, effect changes in the organizational structure, the rules, the regulations, or the procedures governing the Sweepstakes Competition.

\subsubsection{Sweepstakes Chairman}

The Sweepstakes Chairman shall be elected by the Buggy Chairmen of all of the organizations participating in the Sweepstakes Competition. The Sweepstakes Chairman may NOT also be a Buggy Chairman for any organization.

The duties of the Sweepstakes Chairman are to organize and supervise all Sweepstakes activities during the entire school year. These activities include the Sweepstakes races, the Design Competition, the Buggy Book, all freeroll practice sessions, all push practice sessions, the Sweepstakes awards presentations, and all preparation activities for any of these events.

The Sweepstakes Chairman may appoint one or more people as assistants to help the Sweepstakes Chairman, the Assistant Sweepstakes Chairman, the Safety Chairman, the Design Chairman, and the Buggy Book Chairman in the execution of their duties. In the case of the Safety Chairman, an assistant shall only be named in his/her absence, and only after consulting the Safety Chairman. Any person who is to stand in place of the Safety Chairman for any period of time, must first be approved by a majority vote of the organizational Buggy Chairmen. In extreme emergencies, where the Safety Chairman is not present and no person has been approved by a majority vote to act as Safety Chairman, the Sweepstakes Chairman or the Assistant Chairman shall act in the Safety Chairman's place by default.

\subsubsection{Assistant Sweepstakes Chairman}
The Assistant Sweepstakes Chairman (Assistant Chairman) shall be elected by the Buggy Chairmen of all of the organizations participating in the Sweepstakes Competition. The Assistant Sweepstakes Chairman may NOT also be a Buggy Chairman for any organization.

The duties of the Assistant Sweepstakes Chairman are to aid the Sweepstakes Chairman in organizing and supervising all Sweepstakes activities during the entire school year, and to act in the place of the Sweepstakes Chairman in his or her absence.

\subsubsection{Safety Chairman}
The Safety Chairman shall be elected by the Buggy Chairmen of all of the organizations participating in the Sweepstakes Competition. The Safety Chairman may NOT also be a Buggy Chairman for any organization.

The duties of the Safety Chairman are to conduct the safety inspections required of all participating buggies, to conduct Spot Safety Checks of buggies participating in both freeroll practices and push practices, to observe pass tests, to enforce all safety related requirements at all freeroll practices and push practices, and to evaluate and rule on, any safety related issues that arise relative to any Sweepstakes activities during the entire school year.

\subsubsection{Design Chairman}
The Design Chairman shall be appointed by the Sweepstakes Chairman. The Design Chairman may NOT also be a Buggy Chairman for any organization.

The duties of the Design Chairman are to organize and supervise all Design Competition activities during the entire school year. These activities include the appointment of judges, the selection of a time and place for the competition, the arrangements for publicity, the supervision of both the public display of the buggies and the presentation of the buggies to the judges, the compilation of the results of the judging and the presentation of the awards for the competition. The Design Chairman may select one or more people to assist him or her in the execution of his or her duties. The Sweepstakes Chairman must approve any people selected as assistants by the Design Chairman.

\subsubsection{Buggy Book Chairman}
The Buggy Book Chairman shall be appointed by the Sweepstakes Chairman. The Buggy Book Chairman may NOT also be a Buggy Chairman for any organization.

The duties of the Buggy Book Chairman are to organize and supervise all activities required to compile, design, edit, publish, and distribute the Buggy Book which describes the Sweepstakes Competitions. The Buggy Book Chairman shall also be responsible for the design, ordering and distribution of the official Sweepstakes wear such as t-shirts and hats. The Buggy Book Chairman may select one or more people to assist him or her in the execution of his or her duties. The Sweepstakes Chairman must approve any people selected as assistants by the Buggy Book Chairman.

\subsubsection{Buggy Chairman}
Each organization participating in the Sweepstakes Competitions shall designate one or more people as its representatives for all official Sweepstakes functions. These representatives shall be known as Buggy Chairmen and they shall serve as members of the Sweepstakes Committee.

\subsection{Committees}

\textbf{Sweepstakes Advisory Council}\\
At the discretion of the Sweepstakes Advisor and/or the Dean of Student Affairs, a Sweepstakes Advisory Council may be appointed. The purpose of this council is to provide advice, assistance, and guidance to the Sweepstakes Advisor relative to his or her interactions with the Sweepstakes Committee. This council should be composed of any Carnegie Mellon University alumni and/or staff who can provide the experience and knowledge needed by the Dean of Student Affairs, the Sweepstakes Advisor, and the Sweepstakes Committee. In addition, the elected members of the Sweepstakes Committee (the Sweepstakes Chairman, the Assistant Sweepstakes Chairman, and the Safety Chairman)and three Buggy Chairmen (elected by the Sweepstakes Committee) also serve as members of the Sweepstakes Advisory Council. The Sweepstakes Advisor serves as the Chairman of the Sweepstakes Advisory Council.

\textbf{Sweepstakes Committee}\\
The Sweepstakes Committee shall be the governing body of the Sweepstakes Competitions. It shall consist of the Sweepstakes Chairman, the Assistant Sweepstakes Chairman, the Safety Chairman, the Design Chairman, the Buggy Book Chairman, and all of the Buggy Chairman from all of the organizations participating in the Sweepstakes Competitions. The Sweepstakes Committee shall be presided over by the Sweepstakes Chairman.

\section{Functional Organization}

\subsection{Chain of Command}

By permitting Sweepstakes Competitions to take place on its campus, Carnegie Mellon University accepts responsibility for those competitions and is thus authorized to take any actions necessary to ensure that those competitions are conducted in a manner that is as orderly and as safe as is possible. This responsibility is administered by the University through the Division of Student Affairs, the Dean of Student Affairs, the Department of Student Activities, and most directly, through the Sweepstakes Advisor.

At the student level, the responsibility for actually organizing, conducting, and supervising all activities related to the Sweepstakes Competitions is administered by all of the members of the Sweepstakes Committee, and most directly, by the Sweepstakes Chairman. The Sweepstakes Chairman has the responsibility of ensuring that all activities related to the Sweepstakes Competitions are conducted according to the rules, regulations, and procedures that have been approved by the University, through its representatives.

The link between the University and the student participants of the Sweepstakes Competitions is accomplished through the Sweepstakes Advisor and the Sweepstakes Chairman. The Sweepstakes Committee is directly accountable to the University through this link. During the actual organization and execution of the Sweepstakes Competitions, the Sweepstakes Chairman must work closely with the Sweepstakes Advisor in order to assure the University that the competitions are being conducted in an acceptable manner. In addition, the Sweepstakes Advisor can provide much advice, direction, and assistance to the Sweepstakes Chairman and other members of the Sweepstakes Committee concerning procedural and organizational aspects of the Sweepstakes Competitions.

When desired, the Sweepstakes Advisor, the Sweepstakes Chairman, and any of the members of the Sweepstakes Committee may seek opinions, advice, and assistance from the members of the Sweepstakes Advisory Council concerning any aspect of Sweepstakes related activities.

\section{Governing Procedures}

\subsection{Governing Body}

The Sweepstakes Competitions shall be governed by the Sweepstakes Committee, which is presided over by the Sweepstakes Chairman. The committee shall govern by the democratic process. The only members of the committee eligible to vote shall be the Buggy Chairmen of all of the participating organizations. Each participating organization is entitled to only one vote on any issue voted on by the committee (i.e. only one Buggy Chairman from each organization is eligible to vote on any single issue). In the event of a tie vote, the Sweepstakes Chairman shall cast the deciding vote.

All binding decisions shall be made by a majority vote of the eligible voting members of the committee who are present, with one vote per organization. NO issue may be voted on unless two-thirds of the competing Buggy organizations, who are eligible to vote, are represented.

\subsection{Meetings}

Meetings of the Sweepstakes Committee shall be scheduled by the Sweepstakes Chairman whenever that Chairman considers that a meeting is necessary, or when three or more organizations request such a meeting. Sweepstakes Committee meetings should usually be held once a week during the weeks that freeroll practice sessions are scheduled, and somewhat less often at other times of the school year, as needed.

As a rule of thumb, meetings should be consistently scheduled each week at the same time and on the same day (usually every Monday evening at 10pm) to make it reasonable for organizations to regularly attend. Any organization failing to have a representative of that organization in attendance at any Sweepstakes Committee meeting for which at least 24 hours notice has been given, shall be fined the amount of \$15.00.

\subsection{Initial Organizational Meeting}

At the beginning of each school year an organizational meeting of the Sweepstakes Committee shall be held. The purpose of this meeting is to elect a Sweepstakes Chairman, an Assistant Sweepstakes Chairman, and a Safety Chairman for that school year. This meeting should be held before September 15th each school year. The organization that sponsored the winning entry in the previous school year's Sweepstakes races shall be responsible for scheduling, publicizing, organizing, and supervising this meeting.

\subsection{Amendments To The Rules}

Amendments to these Sweepstakes Rules, Regulations, and Procedures may be made in the following ways:

The Dean of Student Affairs and/or the Sweepstakes Advisor may, at their discretion, direct that any changes which they specify, be made to these Sweepstakes Rules, Regulations, and Procedures, at any time.

Through the Sweepstakes Chairman, the Sweepstakes Committee may, by a majority vote of a quorum, propose changes to these Sweepstakes Rules, Regulations, and Procedures, at any time by submitting, in writing, those proposed changes to the Sweepstakes Advisor. Any proposed changes shall not become effective until, and unless, they are approved in writing by the Sweepstakes Advisor.

The Director must either approve or reject any proposed changes within two weeks of the date on which those changes are submitted. If any proposed changes are submitted to the Director on a date that is less than two weeks before the preliminary races are held that school year, the Director is not required to act on those proposed changes. If the Director chooses not to act on those proposed changes, they may be resubmitted the following school year after they have again been voted on by the Sweepstakes Committee. If the Director considers that more than two weeks is required to evaluate any proposed changes, he or she may extend the two week response period,provided that the Sweepstakes Committee is informed of this extension and of its expected duration.

\section{Disciplinary Actions}

\subsection{Fines}

Organizations may be fined by the Sweepstakes Advisor, the Sweepstakes Chairman or the Safety Chairman for violations of any of the rules and regulations specified in this document. The dollar amount of these fines is usually called out in the rule or regulation invoking the fine.

Other fines, not listed within these rules, may be specified and imposed by the Sweepstakes Advisor or the Sweepstakes Chairman in response to continued negligence on the part of organizations or any need that may arise. In such a case, the conditions for and the dollar amounts of those other fines must be made known to the Sweepstakes Committee members before any of those fines are imposed.

A summary of all of the fines that are specified in detail elsewhere in these Rules, Regulations, and Procedures follows:
\begin{itemize}
	\item Failure to be represented at a Sweepstakes Committee meeting: \$15.00 per meeting
	\item Failure to pass a spot safety check: \$25.00 for the first occurrence and \$50.00 for any additional occurrences
	\item Failure to provide required lights and reflectors for after dark practices (both at push practices and freerolls): \$20.00 per buggy per occurrence
	\item Failure to provide or remove No-Parking signs for any freeroll practice: \$25.00 per occurrence
	\item Failure to provide properly equipped sweepers for any freeroll practice: \$15.00 per sweeper
	\item Failure to provide or remove hay bales for any freeroll practice: \$25.00 per occurrence
	\item Failure to provide properly equipped flaggers for any freeroll practice: \$20.00 per flagger
	\item Failure to provide or remove barricades for any freeroll practice: \$25.00 per occurrence
	\item Failure to provide or remove warning signs for any freeroll practice: \$25.00 per occurrence
	\item Failure to have the equipment needed to remove any of its drivers from its buggies in the car following the buggies during any freeroll practice: \$15.00 per occurrence
	\item Failure to provide a chute flagger for any buggy driver at any freeroll practice: \$10.00 per occurrence
	\item Failure of a driver to stop at a yellow stop flag: \$30.00 per occurrence
	\item Failure to provide sufficient barricades for a push practice: \$15.00 per occurrence
	\item Failure to provide properly equipped flaggers for any push practice: \$15.00 per flagger
	\item Failure to paint lane and zone markings: \$25.00 per occurrence
	\item Failure to provide people and vehicles for course watch duty the night before any day of Sweepstakes racing: \$25.00 per person or vehicle
	\item Failure to provide or remove No-Parking signs for any day of Sweepstakes racing: \$50.00 per occurrence
	\item Failure to provide or remove hay bales for any day of Sweepstakes racing: \$50.00 per occurrence
	\item Failure to provide or remove barricades for any day of Sweepstakes racing: \$50.00 per occurrence
	\item Failure to provide or remove warning signs for any day of Sweepstakes racing: \$50.00 per occurrence
	\item Failure to provide or remove crowd control barriers for any day of Sweepstakes racing: \$50.00 per occurrence
	\item Failure to provide properly equipped course marshals for any day of Sweepstakes racing: \$25.00 per course marshal
	\item Failure to have the equipment needed to remove any of its drivers from its buggies in the car following the buggies during any Sweepstakes race: \$25.00 per occurrence
	\item Failure to provide a chute flagger for any buggy driver during any Sweepstakes race: \$25.00 per occurrence
	\item Having unsafe or dangerous conditions in a buggy preparation area before or during any Sweepstakes races: \$100.00 per occurrence
\end{itemize}

\subsection{Payments}

Fines should be subtracted from an organization's safety deposit. If an organization accrues more than \$50 in fines, they must pay an additional \$100 deposit before they can participate in any Sweepstakes practice sessions or races (as stated under Deposits, section 3.5, in these rules). Additional deposits for fines can be made by check, payable to Carnegie Mellon University - Sweepstakes, or be transferred from an organization's campus account into the Sweepstakes account.

\subsection{Penalties}
Organizations may be penalized for violations of any of the rules and regulations specified in these Rules, Regulations, and Procedures by the Sweepstakes Advisor or the Sweepstakes Chairman. If the rule or regulation that was violated specifies that the offending organization shall have one of its entries withdrawn, that withdrawal shall be accomplished as follows:

If the infraction is attributable to an individual buggy that is scheduled to race that day or that school year, that buggy and its entry shall be withdrawn from that scheduled day of racing.

If the infraction is attributable to an individual buggy that is not scheduled to race that day or that school year, one of the buggy's sponsoring organization's entries shall be withdrawn from the next day of racing that it would have participated in, and the sponsoring organization shall select the entry to be withdrawn. (This applies even if an entry must be withdrawn from the first day of racing of the following school year.)

If the infraction is attributable to some part of an organization but not an individual buggy, one of that organization's entries which has not yet run shall be withdrawn from the races that same day, or from that organization's next scheduled day of racing, and the offending organization shall select the entry to be withdrawn. (This applies even if an entry must be withdrawn from the first day of racing of the following school year.)

If an offending organization is permitted to select the entry to be withdrawn, and that organization has entries in both the men's races and the women's races, the selection shall be made from the entries in the men's races for that organization's first infraction, from the entries in the women's races for the second infraction, and so on, alternating for any additional infractions.

\subsection{Appeals}
Any organization or individual receiving a fine or penalty on any day except a day on which races are scheduled, may appeal that fine or penalty to the Sweepstakes Advisor. Final determination of any appeals shall be made by the Director.

\section{Deposits}
Each participating organization is required to submit a security deposit of \$300 to the Sweepstakes Advisor or the Sweepstakes Chairman before the first scheduled freeroll practice during the fall semester of the school year.

The first \$100.00 of this deposit will be used to cover any fines that an organization may incur during the course of the school year. The remaining \$200.00 will be used to cover any extraordinary expenses incurred due to any incidents that may occur during push practices or freeroll practices which may require compensation to people not participating in Sweepstakes, such as damage to cars driving on the buggy course or damage to other private property. This \$200.00 will also be used to cover any similar expenses which are not attributable to any single organization.

If an organization's deposit for fines (\$100) is reduced to less than \$50.00, an additional \$100.00 deposit must be submitted by that organization before they will be permitted to participate in any subsequent freeroll practices or push practices. If an organization's deposit for extraordinary expenses is reduced to less than \$100.00, an additional \$200.00 deposit must be submitted by that organization before they will be permitted to participate in any subsequent freeroll practices or push practices.

By the day before the preliminary races are scheduled, each organization shall submit, if necessary, an additional security deposit such that their total security deposit at that time, shall be at least \$200.00. If anorganization's deposit is less than \$200.00 on the day that the races are scheduled, they will not be permitted to participate in the races.

After all Sweepstakes activities have concluded for the year, each organization will be refunded its deposit balance. An organization shall receive the full amount they paid in deposits over the year, minus fines. No organization shall be reimbursed any more than the dollar amount they paid in deposits over the year. Deposit funds shall be made either by check or account credit, depending on the method of original payment.

\subsection{Payments}
Deposits should be made by check, payable to Carnegie Mellon University - Sweepstakes, or by a money transfer from an organization's campus account into the Sweepstakes account.

\section{Unsportsman-like Conduct}
Any organization may be disqualified from this and/or next year's races if their behavior (or the behavior of anyone associated with their organization) is deemed unsportsman-like by the Sweepstakes Advisor/Executive Committee.

\chapter{Entry Requirements and Procedures}

\section{Race Competition Classes}
There shall be two classes of race competition. These classes shall be known as the men's races and as the women's races. Entries in the men's races will be distinguished by having only men as pushers, and entries in the women's races will be distinguished by having only women as pushers. The drivers of the buggies in either of the two classes of competition, may be men or women.

\section{Race Entry Requirements}

\subsection{Entry}
An entry shall consist of one buggy, one team, and one alternate team. No pusher may compete on more than one entry on any single day of competition. No driver may compete on more than one entry in each class of race competition (one men's race and one women's race) on any single day of competition. No buggy may be used by more than one entry in each class of race competition (one men's race and one women's race) on any single day of competition. All alternate teams are also bound by these conditions.

Each organization must submit to the Sweepstakes Chairman, or anyone designated by that Chairman, a list for each of its entries. The list must include the entry's buggy, all of the members of the entry's team, all of the members of the entry's alternate team All entry lists must be submitted by noon on the day before the preliminary races are scheduled to be held.  In entering members, Sweepstakes will abide by the CMU Statement of Assurance (www.cmu.edu/policies/documents/SoA.html).

\subsection{Team}
Each team shall consist of one driver and five pushers. Each member of a team must be a currently enrolled, Activities Fee paying, full-time student of Carnegie Mellon University and must be a member of the organization entering the buggy. In the case of Greek organizations, members are limited to pledges and initiated members of that chapter and must be, or have been, listed on an IFC membership roster or Panhellenic roster in the case of a sorority. The driver for any team may be any eligible student.

Fraternities may have female push teams. These women do not need to be on the IFC roster of said fraternity. Sororities may have male push teams. These men do not need to be on the Panhellenic roster of said sorority.

Last semester seniors shall be considered to be full-time students(regardless of credits carried) and must pay the Activities Fee for the semester in which they participate in buggy. Any graduate student may participate in buggy for any organization (Greek or independent) given they are full-time students of Carnegie Mellon University and have paid the applicable Activities Fee for graduate students. All potential team members of any organization must necessarily have been named previously on the general membership roster submitted by the organization.

\subsection{Alternate Team}
Each entry may have one alternate team consisting of five pushers and one driver. Each member of an alternate team must comply with all the same requirements that an entry's team members must comply with. No person may be a member of more than one team and one alternate team in each class of competition. A member of an entry's alternate team may substitute for any of that entry's team members.

Any team member who is substituted for may not compete on any other entry's team on the same day unless he or she is also acting as a substitute for a member of a different team that day. If any team member of an organization is substituted for, at least one other team member from that organization must not compete that same day. For example, if a ``B'' team member substitutes for an ``A'' team member, a ``C'' team member may substitute for the ``B'' team member even though that person is still competing that same day provided that the ``A'' team member does not compete that day. In this way several substitutions can take place in order to cover an injury to a member of any team, but a member of one team can't switch places with a member of another team.

\subsection{Membership Roster}
Each organization must submit a membership roster to the Sweepstakes Chairman or anyone designated by that Chairman. This roster must be submitted at least three weeks prior to the date scheduled for the preliminary races. This roster must include the names and student identification numbers of all members of that organization who may wish to be members of any of that organization's teams or alternate teams in the Sweepstakes competition. A person listed on this roster does not necessarily need to then participate in Sweepstakes Races. This roster cannot be changed after it is submitted.

\subsection{Design Competition}
Each organization entering the race competition must also participate in the display portion of the Design Competition.

\section{Number of Entries}

\subsection{Men's Races}

Each participating organization may have a maximum of four entries in the men's races, but no more than one more entry than that organization had during the previous year's races unless they obtain express consent by a majority vote of the Buggy Chairmen and approval of the Sweepstakes officials.

\subsection{Women's Races}

Each participating organization may have a maximum of four entries in the women's races, but no more than one more entry than that organization had during the previous year's races unless they obtain express consent by a majority vote of the Buggy Chairmen and approval of the Sweepstakes officials.

\subsection{Design Competition}

Each participating organization must display all of its buggies scheduled to compete in the Sweepstakes races at the display portion of the Design Competition. In addition, each organization may enter a maximum of two buggies in the presentation portion of the Design Competition.

\section{Fees}

Checks submitted for the payment of fees should be made payable to Carnegie Mellon University. Payments may also be made by a funds transfer from an organization's campus account into the Sweepstakes account.

\subsection{Race Entry Fees}

Race entry fees must be paid to the Sweepstakes Advisor, or anyone designated by the Sweepstakes Advisor, at least one day before the day that heats are selected for the preliminary races unless otherwise specified by the Director or the Sweepstakes Chairman.

\begin{itemize}
	\item First Entry \$35.00
	\item Second Entry \$25.00
	\item Third Entry \$15.00
	\item Fourth Entry \$15.00
\end{itemize}

\subsection{Design Entry Fees}

There shall be no entry fees for the design competition.

\chapter{Course Description}

The race shall be held on the public streets of the Carnegie Mellon University campus and of Schenley Park. The course starts near the intersection of Margaret Morrison and Tech Streets and proceeds south along Tech Street, past the intersection of Tech and Frew Streets, and onto Schenley Drive. The race shall then proceed west along Schenley Drive (passing the George Westinghouse Memorial Pond, the Panther Hollow bridge, and through the area between the Edward Manning Bigelow monument and Phipps Conservatory), turns toward the northeast onto Frew Street, and then east along Frew Street to the finish line.

Maps showing the race course and all of its boundaries and zones can be obtained from the Sweepstakes Advisor through the Sweepstakes Chairman.

\section{Lines}

The zones and lanes of the course shall be delineated by lines painted onto the course. These lines shall be not less than 2 inches nor more than 6 inches in width and shall be of a color that is easily visible(usually white.) All measurements made to these lines shall be made to their centerlines, except for the starting and finishing lines. The location of the starting line shall be measured to the edge that is farthest away from Hill 1 and the location of the finishing line shall be made to the edge that is closest to Hill 5.

\section{Lanes}

There shall be three lanes painted on the course from the starting line to a line that is approximately 58 feet (as measured along the curve of the curb) east of the centerline of the gravel footpath which runs from the southern edge of Schenley Drive to the northern edge of Circuit Road. (As a second reference point, this line is also approximately 49 feet east of the centerline of the eastern-most cross country course marker pole on the southern side of Schenley Drive.)

The lanes shall be 7 feet wide. A single line, perpendicular to Schenley Drive, shall mark the ends of the lanes. The lane farthest to the west shall be designated as Lane 1. The lane adjacent to and to the east of Lane 1 shall be designated as Lane 2. The lane adjacent to and to the east of Lane 2 shall be designated as Lane 3 (ordered 3, 2, 1, from left to right). The western (right) edge of Lane 1 shall be parallel to and 6 feet 6 inches away from the curb on the western (right) side of Tech Street. The eastern (left) edge of Lane 1 shall also be the western (right) edge of Lane 2. The eastern (left) edge of Lane 2 shall also be the western (right) edge of Lane 3.

The western (right) edge of Lane 1 shall be 8 feet to the south of the northern curb of Schenley Drive. The western (right) edge of Lane 1 shall be 4 feet to the east (left) of the western (right) curb of Tech Street at its closest approach to that curb, which should be approximately halfway between Frew Street and Schenley Drive.

\section{Zones}

The course shall be divided into five pushing zones and three transition zones. Each transition zone shall be 45 feet in length. The course shall start at a line known as the starting line, and shall end at a line known as the finish line.

\subsection{Starting Line}

The starting line for each of the three lanes shall be located as follows:

The starting line for Lane 1 shall be perpendicular to Tech Street and 12 feet 4 inches north of a line formed by extending the northern-most face of Skibo Gymnasium.

The starting line for Lane 2 shall be parallel to, and 6 feet to the south of, the starting line for Lane 1.

The starting line for Lane 3 shall be parallel to, and 12 feet to the south of, the starting line for Lane 1.

\subsection{Hill 1}

The first pushing zone shall be known as Hill 1. It shall extend from the starting lines to lines perpendicular to Tech Street and approximately 410 feet south of the starting lines. The end lines for the first pushing zone in each of the three lanes shall be located as follows:

The end line in Lane 1 shall be perpendicular to Tech Street and 32 feet 3 inches north of a line formed by extending the southern-most face of Skibo Gymnasium.

The end line in Lane 2 shall be parallel to, and 6 feet to the south of, the end line in Lane 1.

The end line in Lane 3 shall be parallel to, and 12 feet to the south of, the end line in Lane 1.

\subsection{Hill 1-2 Transition Zone}

The first transition zone shall be known as the Hill 1-2 Transition Zone. It shall start at the end lines for the first pushing zone and shall extend to lines that are parallel to, and 45 feet south of the end lines for the first pushing zone.

\subsection{Hill 2}

The second pushing zone shall be known as Hill 2. It shall start at the end lines for the first transition zone and shall extend from those lines and proceed along the course to the area between the Edward Manning Bigelow monument and the sidewalk in front of Phipps Conservatory. The second pushing zone shall end at a line that is perpendicular to Schenley Drive and which passes through the center of the Edward Manning Bigelow monument. The end of the second pushing zone shall not be marked by a line painted on the course.

\subsection{Hill 3}

The third pushing zone shall be known as Hill 3. It shall start at the end of the second pushing zone and shall extend from that location and proceed along the course (northwest along Schenley Drive and north and east along Frew Street) to a line that is perpendicular to Frew Street and located 60 feet east of the centerline of the western-most doorway,that is on the southern side of Porter Hall.

\subsection{Hill 3-4 Transition Zone}

The second transition zone shall be known as the Hill 3-4 Transition Zone. It shall start at the end line for the third pushing zone and shall extend to a line that is parallel to, and 45 feet east of the end line for the third pushing zone.

\subsection{Hill 4}

The fourth pushing zone shall be known as Hill 4. It shall start at the end line of the second transition zone and shall extend from that line and proceed along the course (east along Frew Street) to a line that is perpendicular to Frew Street and located 60 feet west of a line formed by extending the eastern-most face of Baker Hall.

\subsection{Hill 4-5 Transition Zone}

The third transition zone shall be known as the Hill 4-5 Transition Zone. It shall start at the end line for the fourth pushing zone and shall extend to a line that is parallel to, and 45 feet east of the end line for the fourth pushing zone.

\subsection{Hill 5}

The fifth pushing zone shall be known as Hill 5. It shall start at the end line of the third transition zone and shall extend from that line and proceed along the course (east along Frew Street) to the finish line.

\subsection{Finish Line}

The course shall end at a line known as the finish line. The finish line shall be located on Frew Street. It shall be perpendicular to Frew Street and located 5 feet 6 inches west of the eastern-most face of the brick wall surrounding the window wells on the southern side of the Graduate School of Industrial Administration (GSIA) building. As a second reference, this line is 19 feet 6 inches west of the first corner of the GSIA building that is east of the window well wall.

\chapter{Buggy Construction and Performance Requirements}

Each buggy shall have been designed and constructed by full-time, Activities Fee paying, undergraduate or graduate students of Carnegie Mellon University. These students shall also have been members of the buggy's sponsoring organization at the time that they designed and constructed the buggy. Last semester seniors shall be considered to be full-time students.

Buggies may be bought and/or sold by one organization from or to another organization, provided that the Sweepstakes Chairman is notified of the Transition, and it takes place before heats and lanes are selected for the preliminary races (if that buggy is to be used in those races.)

\section{Braking System}

Each buggy shall have a driver operated system or mechanism which is capable of stopping the rolling motion of that buggy. This system or mechanism shall be known as the buggy's braking system. Each buggy's braking system shall be capable of passing two separate braking performance tests, the braking capability test and the drop brake test. In addition, all braking systems must meet the following requirements:

The brakes shall be self-resetting. This means that the brakes shall release their braking force whenever the driver stops actuating them.

The brakes shall be capable of being actuated to full braking force, and then be completely released, at least three times in succession, without any maintenance.

All fasteners and hardware used in the braking system shall be equipped with locking devices, such as lock nuts, lock washers, lock wires, cotter pins, etc. The use of anaerobic locking compounds (such as Loctite) is permissible, but is not sufficient. Tubing fittings used in hydraulic or pneumatic braking systems need not be equipped with locking devices, with the approval of the Safety Chairman, due to the inherent vibration resistance of these fittings alone.

\subsection{Braking Capability Test}

Each buggy shall pass a braking capability test, at least once each school semester, with each person who will be driving it, before that person will be permitted to drive that buggy in any type of practice session at Carnegie Mellon University during that semester.

The purpose of the braking capability test is to ensure that each buggy meets a minimum braking standard when moving in both the forward and rearward directions, while a particular person is driving it. The braking capability test shall be administered by the Safety Chairman, or anyone designated by that Chairman. The procedure for the braking capability test shall be as follows:

The test shall take place on a flat, level, and smooth area, that is paved with either concrete or asphalt. If the area chosen is not completely level, the test shall be run in a direction such that the buggy being tested shall be moving toward the lower end of the area during the test. The sidewalk between Baker Hall and Doherty Hall shall be used if it is available and in acceptable condition. In this case any buggy being tested shall be moving from the Baker Hall end of the sidewalk to the Doherty Hall end of the sidewalk during the test. If this area is not available or acceptable, the sidewalk between the University Center and Purnell shall be used if it is available and in acceptable condition. In this case any buggy being tested shall be moving from the Purnell end of the sidewalk to the University Center end of the sidewalk during the test. If neither of these areas are available or acceptable, the Safety Chairman, or anyone designated by that Chairman, shall choose an alternate location for the test.

The test shall be conducted as follows:

The buggy shall be pushed until it is moving at a speed greater than or equal to 15 miles per hour (22 feet per second).

After the pusher releases the buggy, the average speed of the buggy shall be measured while the buggy travels through a 50 foot long timing zone. The average speed of the buggy shall be determined by measuring the time required for the nose of the buggy to travel from the beginning to the end of the 50 foot long timing zone. A stopwatch or other suitable timing device shall be used to make this measurement.

When the nose of the buggy reaches the end of the 50 foot long timing zone, the driver shall be signaled to apply the buggy's brakes.

After the buggy stops, the test administrator shall tell the driver to release and reapply the brakes two more times, while the administrator is pushing on the buggy's pushbar to verify that the brakes are being released and reapplied.

The final portion of the brake capabilities test, shall be completed on hill 1. Each buggy driver combination will be placed on the steepest point of hill 1 pointing uphill and must be able to hold its own weight using only the standard braking mechanism.

Two members of the buggy's sponsoring organization shall be located in a position such that they will be able to stop the buggy in the event that the buggy's braking system fails to stop the buggy during the test.

The organization whose buggy is being tested shall provide an adequate number of people to control pedestrian traffic in the test area and to properly attend to all of the buggies that they are using. If the braking capability test is performed between sunset and sunrise, any buggy that is tested must have all lights and reflectors that are required during night push practices.

To successfully complete the test, the following requirements must be met:

Nobody other than the driver may touch the buggy while the buggy is being tested.

The average speed of the buggy while it is traveling through the 50 foot long timing zone must be determined to be 15 miles per hour (22 feet per second) or greater.

The buggy must stop within the distance prescribed by the schedule that follows. This schedule lists the measured time required for the buggy to travel through the 50 foot long timing zone, the average speed of the buggy through that zone, and the allowable stopping distance for that speed. The stopping distance shall be measured from the end of the 50 foot long timing zone to the nose of the buggy, i.e. that part of the buggy that is farthest forward, after the buggy has stopped.

\begin{tabular}{c c c}
	Measured Time & Average Speed & Distance \\
	2.25 sec. & 15.2 mph (22.2 fps) & 35 feet\\
	2.20 sec. & 15.5 mph (22.7 fps) & 35 feet\\
	2.15 sec. & 15.9 mph (23.3 fps) & 35 feet\\
	2.10 sec. & 16.2 mph (23.8 fps) & 40 feet\\
	2.05 sec. & 16.6 mph (24.4 fps) & 40 feet\\
	2.00 sec. & 17.0 mph (25.0 fps) & 45 feet\\
	1.95 sec. & 17.5 mph (25.6 fps) & 45 feet\\
	1.90 sec. & 17.9 mph (26.3 fps) & 45 feet\\
	1.85 sec. & 18.4 mph (27.0 fps) & 50 feet\\
	1.80 sec. & 18.9 mph (27.8 fps) & 50 feet\\
	1.75 sec. & 19.5 mph (28.6 fps) & 56 feet\\
	1.70 sec. & 20.1 mph (29.4 fps) & 62 feet\\
\end{tabular}

While the buggy is stopping, its centerline must not deviate more than 15 degrees from the line that it was moving along, in the 50 foot long timing zone.

After the buggy has stopped, it must remain in a stable position with no external assistance, and its driver must successfully release and reapply its brakes two additional times.

The buggy's brakes must not be applied before the buggy reaches the end of the 50 foot long timing zone.

If a buggy fails to successfully complete the braking capability test during any four consecutive attempts on any one day, that buggy shall not be permitted to attempt the braking capability test again for a period of 24 hours.

The buggy shall be in the same physical state as would be expected for regular free rolls practice. No tactics may be employed to aid in passing the drop test or slowing forward motion of the buggy that would not be used in an emergency condition on the course. Actions not allowed include but are not limited to: overtightening of the retaining mechanism on wheels, use of spacers or shims to cause additional friction, use of worn or dirty bearings, and objects dragging on the ground. Any organization's buggy caught violating this policy will be treated as if it failed a spot safety check

\subsection{Drop Brake Test}

Each buggy is required to take a drop brake test before every freeroll practice, after it competes in any race, and whenever the Sweepstakes Chairman, the Assistant Sweepstakes Chairman, or the Safety Chairman request such a test. The purpose of the drop brake test is to demonstrate that the buggy's brakes are in proper working order and that the driver is able to actuate them. The drop brake test shall be administered by the Sweepstakes Chairman, Assistant Sweepstakes Chairman, Safety Chairman,or anyone designated by any of these Chairmen.

The procedure for the drop brake test shall be as follows:

The test shall take place somewhere on one of the sidewalks of Tech Street, along Hill 1 of the buggy course. The sidewalk directly in front of the middle entrance to Skibo Gymnasium shall be used if it is available and is in acceptable condition. If this area is not available or acceptable, the Safety Chairman, or anyone designated by that Chairman, shall choose an alternate location for the test.

The test shall be conducted as follows:

The buggy shall be placed at rest on the sidewalk facing toward the bottom of the hill, with its nose at the beginning of a marked 30 foot long zone.

The buggy shall be released and allowed to roll down the hill without being inhibited, to the bottom end of the 30 foot long zone.

When the buggy reaches the end of the 30 foot long zone, the driver shall be signaled to apply the buggy's brakes.

After the buggy stops, the test administrator shall tell the driver to release and reapply the brakes once, while observing the movement of the buggy to verify that the brakes are being released and reapplied.

A member of the buggy's sponsoring organization shall be located in a position such that he or she will be able to stop the buggy in the event that the buggy's braking system fails to stop the buggy during the test.

The organization whose buggy is being tested shall provide an adequate number of people to control pedestrian traffic in the test area and to properly attend to all of the buggies that they are using. If the drop brake test is performed between sunset and sunrise, any buggy that is tested must have all lights and reflectors that are required during night push practices.

To successfully complete the test, the following requirements must be met:

Nobody other than the driver may touch the buggy during the test.

The buggy must stop within 15 feet of the end of the 30 foot long zone through which it has been allowed to roll. This 15 foot distance shall be measured to the nose of the buggy, i.e. that part of the buggy that is farthest away from the end of the 30 foot long zone.

After the buggy has stopped, its driver must successfully release and reapply its brakes once.

The buggy's brakes must not be applied before the buggy reaches the end of the 30 foot long zone through which it has been allowed to roll.

The buggy shall be in the same physical state as would be expected for regular free rolls practice. No tactics may be employed to aid in passing the drop test or slowing forward motion of the buggy that would not be used in an emergency condition on the course. Actions not allowed include but are not limited to: overtightening of the retaining mechanism on wheels, use of spacers or shims to cause additional friction, use of worn or dirty bearings, and objects dragging on the ground. Any organization's buggy caught violating this policy will be treated as if it failed a spot safety check.

\section{Configuration}

There shall be no restrictions on the configuration that a buggy may have with respect to the position of the driver, the type of braking system, the location of the braking system, the type of steering system, and the type of frame, as long as the buggy complies with all of the design, construction, and performance criteria specified in these Rules, Regulations, and Procedures.

\section{Driver's Personal Protection}

\subsection{Eye Protection}

Drivers MUST wear approved goggles whenever they are driving a buggy. The goggles must have shatterproof lenses, and must provide side-shield protection to the eyes. Any goggles that are examined by the Safety Chairman and are deemed to be adequate, or that meet the requirements of ANSI Z87.1, are considered to be approved goggles. Only goggles having clear lenses may be used by buggy drivers when they are driving a buggy anytime after sunset and before sunrise.

\subsubsection{Fogging}
Anti-fogging compounds or solutions shall be used to prevent goggles from fogging.

\subsection{Hand Protection}

Drivers MUST wear leather gloves whenever they are driving a buggy. Gloves will help reduce the chances of cuts and abrasions in the event of an accident. The gloves should cover all parts of both hands, including the fingers, thumbs, palms, and the backs of the hands. Any gloves which do not cover all parts of both hands must be approved by the Safety Chairman.

\subsection{Head Protection}

Drivers MUST wear an approved hard shell helmet whenever they are driving a buggy. Any helmet which carries the ANSI Z90.4 rating, the SNELL bicycle helmet rating, or any CSA hockey helmet rating (such as CAN3-Z262.2-M78, or Z262.1) is considered to be an approved helmet. The helmet must not restrict the movement of the driver or the driver's head while the buggy is being driven.

Helmets MUST be secured (using duct tape or other means) so that they will not obstruct the driver's vision while the buggy is in motion. The method of securing the driver's helmet must be demonstrated during the safety inspection of the buggy.

\subsubsection{Modifications}
Helmets may not be modified by cutting, removing or altering the lining, padding, shell, or chin strap in any way except as follows:

The shell, lining, and padding of the helmet may be trimmed away in the upper front area of the helmet (above where the driver's forehead would be) in order to provide adequate vision for the driver. The section trimmed away shall be in the shape of a smooth arc no more than 1.5 inches high (at its greatest height) by 5.0 inches wide (at its widest point.) The edges of the trimmed away section of the helmet shall be smoothed, beveled, and/or padded in order to reduce the possibility of injury to the driver while the helmet is being worn. In no case may any stiffening ribs or supporting members of the helmet shell be removed or altered in any way. Any modifications made to a helmet must be approved by the Safety Chairman before the modified helmet can be used by a buggy driver.

A helmet may also be modified, as described above, in the lower rear area (where the back of the driver's neck would be) in order to provide adequate head tilt for the driver. The requirements are the same as described above except that the maximum dimensions of the cut away area shall be 2.0 inches high by 5.0 inches wide.

\subsection{Safety Harness}

Drivers MUST wear an approved safety harness whenever they are driving a buggy. Any harness that is examined and deemed to be acceptable by the Safety Chairman, or is supplied through the Sweepstakes Advisor, is considered to be an approved harness.

\subsubsection{Attachment}

Every driver's safety harness must be attached to that driver's buggy at a minimum of three different points, whenever that person is driving that buggy. These attachments shall only be made to structural members of the buggy, and they shall be such that the movement of the driver is restricted in all directions (i.e. forward, rearward, side to side, and vertical)such that in the event of an impact to the buggy from any direction, the driver shall be retained inside, but not impact against, the driver's protective cage. All harness attachment points shall be approved by the Safety Chairman during the buggy's safety inspection.

Safety harnesses must have the capability to permit quick and easy removal of the driver from the buggy in the event of an accident or other problem while the driver is secured in the buggy with the safety harness. This removal must be possible without the use of tools or other mechanical devices.

\subsubsection{Construction}

The straps from which a safety harness is constructed must be at least 1.75 inches wide. This is to ensure that the forces of a collision are distributed over a relatively large area of a driver's body. Standard automotive seat belt material is acceptable. With the approval of the Safety Chairman, the lower portion of a safety harness may consist of a commercially available rock climbing harness, even though its straps may be less than 1.75 inches wide.

\subsubsection{Additional Safety Harness Requirements}
All harness clips must be d-ring carabineers and certified by the UIAA (International Mountaineering and Climbing Federation).

All box stitching on the harness must be completed by professional stitchers, performed on a heavy duty industrial sewing machine, and deemed adequate by the Safety Chairman. In the event that the Safety Chairman does not believe the stitching was professionally done, the buggy organization’s Chairman must show a receipt indicating that a professional performed the stitching. In the event that the Chairman does not have the receipt the harness will be deemed unsafe and will not pass the safety.

Harnesses can be bought from stores, provided that they comply with the rules listed in this section. Many rock climbing stores sell harnesses that comply to the rules.

\section{Driver's Protective Cage}

Each buggy shall be constructed with a structurally sound cage which totally surrounds the driver. The purpose of this cage is to help protect the driver from front, side, rear, and rollover impacts in the event of an accident. The driver's protective cage shall meet the following minimum requirements:

NO PART of the driver's body shall extend beyond the protective cage.

The interior of the protective cage shall not have any sharp edges, pointed objects, or protruding objects, which could injure the driver in the event that the driver is pushed into or thrown against these edges or objects.

None of the wheels of the buggy shall be considered to be part of the driver's protective cage.

The protective cage shall be designed and constructed in such a way that injury to the driver will be minimized in the event that one or more of the buggy's wheels become detached from the buggy, while it is moving.

The protective cage shall be designed and constructed in such a way that a 1.0 inch wide by 72 inch long infinitely stiff bar, held in ANY position in a vertical plane perpendicular to the centerline of the buggy (this means with the bar's axis vertical, horizontal, or at any angle,) will not intrude into the front of the cage, nor contact any part of the driver's body, nor increase the force distribution on the driver's body, when pushed toward the buggy with a force of 1,000 pounds, while the buggy is restrained so that it cannot move.

The protective cage shall be designed and constructed in such a way that a 2.0 inch wide by 12 inch long infinitely stiff bar, held in ANY position (see preceding paragraph) in any vertical plane that is either perpendicular to or parallel to the centerline of the buggy, will not intrude into the sides or rear of the cage, nor contact any part of the driver's body, nor increase the force distribution on the driver's body, when pushed toward the buggy with a force of 500 pounds, while the buggy is restrained so that it cannot move.

The protective cage shall be designed and constructed in such a way that it shall not collapse or deform such that any part of the driver's body would be contacted (by the cage or by the steel plate,) if a 2 ft. by 5 ft. by 1.00 inch thick steel plate was slowly lowered onto the top of the buggy. (This plate weighs approximately 408 pounds.)

If a buggy has a pushbar that can change its position (this means that it is not rigidly and permanently attached to the buggy,) that pushbar shall NOT be considered to be a part of that buggy's protective cage. Any buggy with such a pushbar, must meet all of the requirements for the driver's protective cage with its pushbar in any of its possible positions. This is to ensure that the pushbar itself, will not cause an injury to the driver in the event of an accident.

The protective cage shall be designed and constructed in such a way that no part of the driver's body shall be exposed to the road surface. The purpose of this requirement is to ensure that there is some substantial structure protecting the driver from below in the event that the bottom of the buggy comes in contact with another object, such as if the wheels came off and the buggy slid along the ground.

\section{Field of Vision}

Each buggy shall pass a field of vision test, at least once each school semester, with each person who will be driving it, before that person will be permitted to drive that buggy in any type of practice session at Carnegie Mellon University during that semester. The purpose of the field of vision test is to ensure that each buggy provides a minimum field of vision to each person who will be driving it. The field of vision test shall be administered by the Safety Chairman, or anyone designated by that Chairman. The procedure for the field of vision test shall be as follows:

The test shall take place at the same location as the braking capability test.

The test shall be conducted as follows:

The driver to be tested shall be placed inside the buggy with all of the driver's personal protection equipment in place as it would be if the driver were going to drive the buggy, i.e. wearing a properly secured helmet, a safety harness, goggles, and gloves.

The buggy shall be aligned in such a way that a 90 degree angle, whose origin is located at the point where the centerline of the buggy and the buggy's front axle intersect, and whose bisector is coincident with the centerline of the buggy, can be determined.

Cards with identifiable symbols on them shall be placed at various locations throughout the 90 degree angle. The tester may also hold up between 0 and 10 fingers.

The driver inside the buggy shall be asked if he or she can identify the symbols on the cards, or how many fingers the tester is holding up, and features and objects on the horizon beyond the cards.

To successfully complete the test, the following requirements must be met:

The driver inside the buggy must be able to identify the symbol on a cards, or the number of fingers the tester is holding up, placed at ground level at a location that is 25 feet in front of the buggy's front axle and along a line formed by the centerline of the buggy.

The driver inside the buggy must be able to identify the symbols on cards, or the number of fingers the tester is holding up, placed at ground level at locations that are 25 feet from the point at which the front axle of the buggy and the centerline of the buggy intersect and are along lines that are at 45 degree angles to, and to either side of, the centerline of the buggy, and which intersect the centerline of the buggy at the same point at which the buggy's front axle intersects it.

The driver inside the buggy must be able to identify features and objects that are located on the horizon along a line formed by the centerline of the buggy.

The driver inside the buggy must be able to identify features and objects that are located on the horizon along lines that are at 45 degree angles to, and to either side of, the centerline of the buggy, and which intersect the centerline of the buggy at the same point at which the buggy's front axle intersects it.

\section{Pushbar}

Each buggy shall have a structure which its pushers can push on, in order to propel that buggy forward. This structure shall be known as a pushbar. Each pushbar shall have a structurally sound cross-piece at its end, which shall be known as the pushbar handle. The pushbar handle will usually be perpendicular to the pushbar and, to ensure safety, should have the recommended, but not required, dimensions of 8 inches long by 1 inch in diameter.

If any part of a buggy's pushbar moves from the position that it is normally in when its pushers are propelling that buggy forward, that part or parts of that pushbar may only move to a position that is inside the driver's protective cage, or to a position that is directly above the outline of the driver's protective cage when viewed from above the buggy. During any such movement, no part of the pushbar may extend below or behind the normal pushing position of the pushbar unless it is also within or above the outline of the driver's protective cage at the same time.

\section{Restrictions}

No buggy may have any device whose purpose is to store energy independent of the speed of the buggy (such as a geared flywheel) or whose purpose is to increase the forward speed of that buggy by adding kinetic energy to that buggy.

No buggy may have any means of internal propulsion.

No buggy may have any type of protrusion (such as a sharp and/or pointed structure) extending from that buggy which could increase the likelihood of injury to the driver of another buggy in the event of an accident. The nose/tail of a buggy should be parabolic and not come to a point. If it were a curve, the derivative of any point on that curve should be defined across the whole length. This prohibits nose and tails that come to a point in three or two dimensions (i.e. a true point or an edge.) An allowed exception to this would be flat or cut-off tails, though the flat face must form an angle between 45 and 90 degrees to the ground. The minimum radius of curvature permitted for any nose or tail is .75. Final discretion as to what might be considered overly pointy is up to the discretion of the Safety Chairman.

No buggy may be designed such that it would intentionally have less than three wheels in contact with the ground whenever it is on the buggy course.

\section{Shell}

Each buggy shall have a means to securely attach its shell, hatch, or cover, if that part of the buggy is removable. The adequacy of any attachment method shall be determined by the Safety Chairman, during the buggy's safety inspection.

\section{Size}

No buggy shall have at any time any dimension, parallel to the centerline of the buggy, greater than 15 feet including that buggy's pushbar.

No buggy shall have at any time any dimension, perpendicular to the centerline of the buggy, greater than 6 feet.

\section{Steering System}

Each buggy shall have a driver operated system or mechanism that shall control and determine the direction in which that buggy will move, whenever that buggy is in motion. This system or mechanism shall be known as that buggy's steering system. Each buggy's steering system shall meet the following minimum requirements:

All fasteners and hardware (i.e. nuts, bolts, collars, etc.) that are used to attach or to secure any part of a buggy's steering system, shall be equipped with locking devices such as lock nuts, lock washers, lock wires, cotter pins, etc. The use of anaerobic locking compounds (such as Loctite) is permissible, but is not sufficient.

Each buggy's steering system shall operate in a smooth manner. The steering system shall not bind or behave in an erratic way during use, such that it might cause the driver to lose control of the buggy.

The steering system of each buggy shall be designed such that it can be operated in a stable manner, i.e. the steering mechanism should usually tend to return to a neutral (straight ahead) position when no steering inputs are received from the driver.

The steering system shall be designed such that the driver will be able to control the buggy under all normal driving conditions.

\section{Wheels}

All fasteners and hardware (i.e. nuts, bolts, collars, etc.) that are used to secure a buggy's wheels to that buggy shall be equipped with locking devices such as lock nuts, lock washers, lock wires, cotter pins, etc. The use of anaerobic locking compounds (such as Loctite) is permissible,but is not sufficient.

\subsection{Restrictions}

No buggy may use any official soap box derby race wheels if those wheels are constructed of any type of plastic material. Experience has shown that these wheels, while adequate for straight line soap box derby racing, are structurally deficient in buggy racing applications.

\section{Windscreens}

Each buggy shall have a windscreen in front of the driver's face in order to help protect the driver from wind, dust, flying objects, and debris. Each windscreen shall be made of clear polycarbonate plastic. (The common trade name for polycarbonate is LEXAN, which is a registered trademark of the General Electric Company.) The minimum nominal thickness for each windscreen shall be 0.062 inches.

Each buggy's windscreen shall be mechanically attached to structural members of that buggy such that in the event of an impact to that windscreen, the windscreen will fail before the windscreen's supporting members will fail. Each windscreen shall be secured along more than two of its edges with mechanical fasteners (such as bolts, rivets, etc.) that are approved by the Safety Chairman. Attachment of the windscreen to the buggy with duct tape or other types of tape is permissible, but is not sufficient.

The use of tinted or mirrored windshields on buggies during after dark practice sessions must be approved by the Safety Chairman.

\subsection{Interior Windscreens}

Any buggy which has a wheel located in front of the driver, shall also have a protective shield between that wheel and the driver's face. This shield shall be made of clear polycarbonate plastic, which has a minimum nominal thickness of 0.062 inches. (The common trade name for polycarbonate is LEXAN, which is a registered trademark of the General Electric Company.) It shall extend from the bottom of the buggy to the top of the buggy, and shall be at least 4.0 inches wide. It shall be attached to a structural member of the buggy with mechanical fasteners (such as bolts, rivets, etc.)and its attachment method shall be approved by the Safety Chairman during the buggy's safety inspection.

The purpose of this shield is to reduce the possibility of debris being thrown into the driver's eyes and/or face by the wheel. This debris could come from either the road surface or from the wheel itself, in the event that the tire fails and breaks into pieces.

\subsection{Fogging}

Anti-fogging compounds or solutions shall be used to prevent windscreens from fogging.

\chapter{General Safety Requirements and Procedures}

\section{Buggy Safety}

\subsection{Safety Inspection Requirements}

The requirements for the safety inspection of buggies shall be as follows:

Each buggy must successfully complete a safety inspection each school semester, before it shall be permitted to participate in any freeroll practice, push practice, or Sweepstakes race during that semester at Carnegie Mellon University. The purpose of this safety inspection is to determine if the buggy being inspected complies with all of the Buggy Construction and Performance Requirements specified by these Rules, Regulations, and Procedures. Each buggy's sponsoring organization has the burden of proof to demonstrate that that buggy meets or exceeds ALL of these requirements.

There shall be no limit to the number of times that a buggy may try to successfully complete a safety inspection.

Safety inspections shall be conducted by the Safety Chairman or anyone designated by the Sweepstakes Advisor.

Any buggy that is involved in any accident during a freeroll practice or push practice, may be required to again successfully complete a safety inspection, if the Sweepstakes Chairman or the Safety Chairman, so requests, before that buggy shall be permitted to participate in any future practice sessions at Carnegie Mellon University.

Any buggy that is modified in any way which might affect the safety of that buggy after it has successfully completed a safety inspection, must successfully complete an additional safety inspection before it shall be permitted to participate in any freeroll practice, push practice, or Sweepstakes race at Carnegie Mellon University. Any buggy found to be participating in any type of practice session or race competition after a safety related modification, and before this additional safety inspection, shall be penalized by being banned from any further participation in any type of practice sessions and by being disqualified from all race competition for that entire school year.

Modifications considered to affect the safety of a buggy include, but are not limited to, the following:
\begin{itemize}
	\item Changing the configuration of the braking system.
	\item Changing the driver's personal protection equipment, such as modifying or changing the helmet, or changing the attachment method or construction of the safety harness.
	\item Changing the driver's protective cage.
	\item Any change that could affect the driver's field of vision, such as a change to the shell or windscreen.
	\item Changing the configuration of the steering system.
	\item Changing the diameter of any wheel or tire by more than 0.5 inches.
	\item Changing the position of any wheel relative to the frame or the other wheels.
	\item Changing the tire on any wheel from a solid tire to a pneumatic tire or vice versa.
	\item Changing the attachment method for any wheel.
	\item Adding or removing an aerodynamic covering or shroud to any wheel.
	\item Changing the attachment method or transparency of any windscreen.
\end{itemize}

Any buggy that varies its configuration while it is on the buggy course, must be capable of successfully completing a safety inspection, in any of its different configurations. These variations can include, but are not limited to, the following:
\begin{itemize}
	\item Changing any wheel position relative to the frame, the other wheels, or the center of mass of the buggy.
	\item Changing the number of wheels in contact with the ground.
	\item Changing the position of the driver relative to the frame of the buggy.
	\item Changing the position of any part of the pushbar relative to the frame of the buggy.
\end{itemize}

\subsection{Safety Inspection Procedures}

Each safety inspection shall consist of two parts, the design inspection and the performance demonstration.

\subsubsection{Design Inspection}

The design inspection shall be a demonstration to the inspector that the buggy meets or exceeds all of the buggy construction requirements specified by these Rules, Regulations, and Procedures.

Each design inspection shall be conducted as follows:

A representative of the organization desiring to have a buggy design inspected shall contact the Safety Chairman and make an appointment for that inspection.

At the appointed time and place, the Safety Chairman shall perform a design inspection of the organization's buggy. This design inspection shall be performed in accordance with the Design Inspection Checklist provided and kept by the Safety Chairman. A copy of the Design Inspection Checklist may be obtained from the Sweepstakes Advisor through the Safety Chairman.

Representatives of each buggy's sponsoring organization shall demonstrate at that buggy's design inspection, that the buggy complies with all of the construction criteria specified in the Buggy Construction and Performance Requirements section of these Rules, Regulations, and Procedures. This demonstration can be either theoretical or experimental, i.e. the design features of a buggy may be shown by engineering analysis, drawings, and calculations, or they may be shown by an actual demonstration of those features, such as actually showing the inspector the buggy and its hardware and how it operates. In either case the buggy under inspection must be presented to the inspector for actual viewing.

Successful completion of a prior design inspection by a buggy shall NOT be considered as a contributing factor in evaluating a buggy during a design inspection. Each buggy MUST comply with all of the CURRENT construction requirements in order to successfully complete a design inspection.

A driver that fits in each buggy being design inspected shall be present during the inspection in order to demonstrate the adequacy of the driver's personal protection features of the buggy, and the fit of the driver inside the driver's protective cage.

The inspector shall permanently record on the Design Inspection Checklist any and all information pertinent to the condition of the buggy being inspected. The inspector shall also record some identifiable feature or marking that is on, or is a part of, a permanent structure of that buggy. This feature or marking will be used during any subsequent safety inspections, spot safety checks, braking capability tests, or field of vision tests to uniquely identify that buggy. This information shall be kept on file by the Sweepstakes Advisor for future reference, as long as that particular buggy is active in Sweepstakes practice sessions or races. This information shall be used by present and/or future Safety Chairmen and design inspectors to evaluate changes made to a buggy after its initial design inspection.

\subsubsection{Performance Demonstration}
The performance demonstration shall be a demonstration to the inspector that the buggy can successfully complete the field of vision test, the braking capability test, and the drop brake test specified in the Buggy Construction and Performance Requirements section of these Rules, Regulations, and Procedures. It shall also be an on-going demonstration to the Safety Chairman, of the buggy's actual performance during both freeroll and push practices.

The performance demonstration portion of the safety inspection shall consist of the following:

Successful completion of the braking capability test specified in the Buggy Construction and Performance Requirements section of these Rules, Regulations, and Procedures. The braking capability test shall be conducted at a time and placed arranged by appointment with the Safety Chairman or anyone designated by that Chairman.

Successful completion of the field of vision test specified in the Buggy Construction and Performance Requirements section of these Rules,Regulations, and Procedures. The field of vision test can usually be performed at the same time and place as the braking capability test.

Successful completion, when required, of the drop brake test specified in the Buggy Construction and Performance Requirements section of these Rules, Regulations, and Procedures. The drop brake test shall be conducted whenever required before freeroll practices, after races, or during spot safety checks. The Safety Chairman may require an additional design inspection and braking capability test if a buggy frequently fails the drop brake test on its first attempt of the day.

An on-going demonstration that the buggy's steering and control systems can be operated in a stable manner. The stability and performance of a buggy's steering and control systems and of that buggy's driver shall be observed by the Safety Chairman, and/or anyone designated by that Chairman, whenever that buggy is participating in any type of practice session at Carnegie Mellon University.

If the Safety Chairman determines that a buggy and/or its driver is not stable, that buggy and/or driver shall not be permitted to participate in any subsequent freeroll practices or push practices until the cause of that instability has been eliminated. The Safety Chairman may require an additional design inspection or may ban a buggy from any future practices or races if that buggy consistently exhibits erratic control or performance at any practice session.

An on-going demonstration of the buggy's static and dynamic lateral stability during freeroll and push practices (i.e. the buggy must demonstrate by its performance during practices that it is laterally stable while it is both at rest and in motion, and that it can be used during practice sessions without tipping over, rolling over, or having one or more wheels lift off the ground while it is cornering).

The static and dynamic lateral stability of a buggy shall be observed by the Safety Chairman, and/or anyone designated by that Chairman, whenever that buggy is participating in any type of practice session at Carnegie Mellon University. If the Safety Chairman determines that the static or dynamic lateral stability of a buggy is not adequate (due to tipping over, rolling over, or having one or more wheels lift off the ground when cornering), that buggy shall not be permitted to participate in any subsequent freeroll practices or push practices until the cause of that instability has been eliminated. The Safety Chairman may require an additional design inspection, or may ban a buggy from any future practices or races if that buggy continues to exhibit erratic static or dynamic lateral stability at any practice session.

\subsection{Spot Safety Checks}

At any time during any type of buggy practice session at Carnegie Mellon University, the Sweepstakes Chairman, Assistant Chairman, Safety Chairman, or anyone designated by the Sweepstakes Advisor, may perform a spot safety check on any buggy participating in that practice session. A spot safety check shall consist of any or all of the following:
\begin{itemize}
	\item Performance of a drop brake test to determine if the buggy's braking system is working properly .
	\item A check of the buggy driver's required head protection, including the manner in which the driver's helmet is secured in order to prevent it from obscuring the driver's vision.
	\item A check of the buggy driver's required safety harness.
	\item A check of the buggy driver's required eye protection, including goggles and how they are secured, windscreen, and where required, an interior windscreen.
	\item A check of the buggy driver's required gloves.
	\item A check of the buggy's lights and reflectors, when they are required, such as during after dark drop brake tests and push practices.
	\item A check of the general condition of the buggy to ensure that it is in as good, or better condition, in regards to safety related items, as it was during its safety inspection, and that no modifications have been made to the buggy since its inspection which could reduce its safety.
\end{itemize}

Any buggy found to be out of compliance with any applicable safety rules and regulations during a spot safety check shall not be permitted to participate in any type of practice session for the remainder of that day and that buggy's sponsoring organization shall be fined the amount of \$25.00 for the first occurrence and \$50.00 for any additional occurrences by that organization during that school year. If a buggy is found to be out of compliance with any applicable safety rules and regulations during a spot safety check on more than one occasion during any one school year, that buggy shall not be permitted to participate in any type of practice session or Sweepstakes race for the remainder of that school year.

\section{Driver Safety}

\subsection{Education Program}

Each buggy driver MUST participate in a driver education program before they will be permitted to drive a buggy during any type of buggy practice session at Carnegie Mellon University. The driver education program shall consist of the following:

Before any freeroll practice takes place during the fall semester of each school year, a meeting of all of the people who want to be buggy drivers shall be held. This meeting shall be conducted by the Sweepstakes Chairman, the Assistant Chairman, the Safety Chairman, and/or anyone designated by the Sweepstakes Chairman. Buggy Chairmen from individual organizations shall not be permitted to attend the first at least a portion (usually the first hour) of this meeting unless they are also buggy drivers.

Topics that shall be discussed at this meeting shall include, but not be limited to, the following:
\begin{itemize}
	\item Potential dangers of buggy driving.
	\item Rules, Regulations, and Procedures relative to driving a buggy.
	\item Driver's personal safety and protective equipment.
	\item Driver training methods.
	\item Driving techniques for both normal and emergency situations.
	\item Use of signal flags, including the required chute flaggers and the yellow colored emergency flag. 
\end{itemize}

In addition to the discussions at this meeting, each potential driver shall be given a written summary of the topics covered during this meeting.

Topics that shall be discussed at the spring semester driver's education meeting shall include, but not be limited to, all of the same topics covered during the fall semester buggy driver's education meeting.

Approximately one week before the preliminary races are scheduled each school year, a meeting of all of the people who are to drive buggies during those races shall be held. This meeting shall be conducted by the Sweepstakes Chairman, the Assistant Chairman, the Safety Chairman, and/or anyone designated by the Sweepstakes Chairman. Buggy Chairmen from individual organizations will usually, but not always, be permitted to attend all of this meeting. Topics that shall be discussed at this meeting shall include, but not be limited to, a review of all of the topics covered during the previous buggy driver's education meetings, plus a detailed discussion and review of race day procedures and regulations, and driving techniques that can be employed during the races.

Additional driver education meetings may be held at the discretion of the Sweepstakes Advisor, the Sweepstakes Chairman, or the Safety Chairman.

Attendance at ALL of the required driver education meetings is MANDATORY for anyone who wants to drive a buggy at Carnegie Mellon University. If a person who wants to be a buggy driver fails to attend any of the required meetings, he or she must submit a request (either written or verbal) for a review of the content of each missed meeting to the Sweepstakes Chairman. This request must state the reason that each meeting in question was missed. The Sweepstakes Chairman shall then require that person to review the content of each missed meeting with someone that he or she designates.

\subsection{Liability Waiver}

Each potential buggy driver must sign a Waiver of liability form (this is usually done at the buggy driver's education meeting). No person shall be permitted to drive a buggy at any type of practice session or Sweepstakes race at Carnegie Mellon University, until they have signed this form. These waiver forms shall be kept on file by the Sweepstakes Advisor. A copy of this waiver form can be obtained from the Sweepstakes Advisor through the Safety Chairman. If the buggy driver is less than 18 years of age, that driver's parent or legal guardian must also sign that waiver form.

\subsection{Pass Test}

Each buggy driver must successfully complete a pass test during a spring semester freeroll practice session in each buggy that he or she will be driving during the Sweepstakes races before he or she will be permitted to compete in those races. The purpose of this pass test is to demonstrate that each driver has certain minimum driving skills while driving the buggy that they will be racing in. The procedure for the pass test shall be as follows:

The test shall take place during a freeroll practice session while the driver and buggy being tested are freerolling on the buggy course. Arrangements to have the test observed by a qualified observer must be made by the organization taking the test with the Safety Chairman, or anyone designated by that Chairman, prior to the test.

The test shall be conducted as follows:
\begin{itemize}
	\item A buggy and driver not taking the test shall be allowed to freeroll down the buggy course. If the organization being tested has more than one buggy, it must use one of its own buggies as the buggy being passed. If it only has one buggy, it may use a buggy from another organization, but that buggy may only be a buggy that is already qualified for the races, i.e. that buggy must have already been driven down the buggy course at least five times and have successfully completed its own pass test.
	\item The buggy and driver taking the test shall be allowed to freeroll down the buggy course after the first buggy, at a speed that is greater than that of the first buggy, so that the buggy and driver taking the test may eventually overtake the first buggy.
	\item The buggy and driver taking the test shall pass the first buggy and driver somewhere between the George Westinghouse Memorial Pond and the service driveway that extends eastward onto Flagstaff Hill from Frew Street, approximately halfway between Schenley Drive and Scaife Hall.
	\item The test shall be observed by the Safety Chairman, or anyone designated by that Chairman. This observer shall ride in a follow car immediately behind the two buggies participating in the test.
\end{itemize}

To successfully complete the test, the following requirements must be met:
\begin{itemize}
	\item The test observer must agree that the pass was adequate, based on that observer's judgment and experience.
	\item The buggy being passed must be moving at a reasonable speed, as determined by the test observer.
	\item The pass must take place somewhere between the George Westinghouse Memorial Pond and the service driveway that extends eastward onto Flagstaff Hill from Frew Street, approximately halfway between Schenley Drive and Scaife Hall.
	\item The pass must demonstrate the skill of the passing driver (i.e. the passing buggy must come up behind the slower buggy, smoothly pull out to one side of it, pass it, and then smoothly pull over in front of the passed buggy). The passing buggy must appear to be in control at all times during the pass and must not exhibit any instability attributable to either the driver or the buggy. The driver of the passing buggy must leave adequate clearance when passing the slower buggy and must return to his or her desired course smoothly after the pass has been completed. The manner in which the pass is made must demonstrate to the observer that the passing driver is able to adequately judge the position of his or her buggy relative to the buggy being passed.
\end{itemize}

\chapter{Practice Sessions}

\section{General Procedures and Rules}

Buggy freeroll and push practice sessions shall be governed by the following general procedures and rules:

No buggy freeroll or street push practice sessions of any type shall be held at Carnegie Mellon University at any time without the approval and authority of the following:
\begin{itemize}
	\item The Police Department of the City of Pittsburgh.
	\item The Department of Parks and Recreation of the City of Pittsburgh.
	\item The Carnegie Mellon University Campus Police.
	\item The Sweepstakes Advisor.
	\item The Sweepstakes Chairman.
	\item The Safety Chairman.
\end{itemize}

Buggy practice sessions shall only be held at those times and places authorized by the Sweepstakes Advisor.

All persons participating in any activity directly or indirectly related to the Sweepstakes Competition at Carnegie Mellon University are considered to be representatives of Carnegie Mellon University. As such, these persons are expected and required to act in a responsible and orderly manner whenever they are involved in any such activity, especially when they are interacting with any segment of the public at large, the Pittsburgh Police, or the Carnegie Mellon University Campus Police.

Sweepstakes races can only be held with the cooperation and approval of the City of Pittsburgh and the people of Pittsburgh. Any irresponsible action on the part of anyone involved with the Sweepstakes Competition could jeopardize this cooperation and approval. Any individual or organization found to be acting in an irresponsible manner while participating in any activity directly or indirectly related to the Sweepstakes Competition at Carnegie Mellon University may be barred from any further participation in that activity by the Dean of Student Affairs or the Sweepstakes Advisor.

Any organizations found to be conducting or otherwise participating in any unauthorized freeroll practice or push practice on the public streets in Schenley Park and/or on the Carnegie Mellon University campus shall be penalized by being barred from ALL Sweepstakes activities (including both practices and competitions) for a period of one year from the date of the occurrence of that violation.

\subsection{Buggies}
The following procedures and rules shall apply to buggies during practice sessions:

No buggy shall be permitted to participate in any freeroll practice or push practice session at Carnegie Mellon University unless it meets all of the following requirements:
\begin{itemize}
	\item It has successfully completed a safety inspection.
	\item It has successfully completed a braking capability test, with the driver who will be driving it during that practice session.
	\item It has successfully completed a field of vision test, with the driver who will be driving it during that practice session.
	\item The driver who will be driving it during that practice session has become familiar with the operation of that buggy by being pushed around(on sidewalks, in parking lots, etc.) in that same buggy.
\end{itemize}

Any organization that permits any of its buggies to participate in any freeroll practice or push practice in any semester, before that buggy has successfully completed a safety inspection that same semester shall be penalized by having one entry withdrawn, as described in the ``DISCIPLINARY ACTIONS'' section of this document. In addition, the entry fee for that entry shall be forfeited.

Any organization that fails to provide all of the required lights and reflectors, in proper working order, for any of its buggies that are being used between sunset and sunrise shall be fined the amount of \$20.00 for each buggy that is not properly equipped.

Any buggy that is involved in any type of accident during any type of practice session at Carnegie Mellon University may not be used in any type of practice session again until that buggy's sponsoring organization has submitted an accident report to the Safety Chairman, or anyone designated by that Chairman, and had that report approved by the Sweepstakes Chairman or the Safety Chairman. In any case this report must be submitted within 48 hours of the time that the accident occurred. The form to be used for this accident report can be obtained from the Sweepstakes Advisor through the Safety Chairman. Accident reports will not be approved unless the Sweepstakes Chairman or the Safety Chairman consider that the problem that caused the accident has been corrected and that a similar accident is not likely to occur again. The Sweepstakes Chairman or the Safety Chairman may require any buggy that has been involved in an accident to again successfully complete a safety inspection before that buggy is permitted to participate in any future practice session.

A buggy shall be considered to have been involved in an accident if any of the following have occurred:
\begin{itemize}
	\item That buggy leaves the prescribed buggy course while it is freerolling.
	\item That buggy contacts any border of the buggy course, such as a hay-bale or a curb, while it is freerolling.
	\item That buggy contacts any material obstruction, such as another buggy, a person, a bicycle, a motor vehicle, etc.
	\item The driver of a freerolling buggy loses control of that buggy to the extent that the buggy comes to a stop before the end of the freeroll portion of the buggy course.
	\item Any part of a buggy, such as the shell, a hatch, a cover, etc., falls off that buggy while that buggy is freerolling.
\end{itemize}

\subsection{Safety}

\subsubsection{Buggies With Drivers}

No buggy that has a driver in it may be left unattended at ANY time. When any buggy is outdoors with a driver in it and it is not being used in a practice session or a brake test, (i.e. it is not being pushed by a pusher, is not rolling down the buggy course during a freeroll practice session, or is not rolling freely during a braking capability or drop brake test), someone MUST be within three feet of the buggy (preferably holding the pushbar), watching and attending it so that preventative action can be taken in the event that the buggy starts to move.

\subsubsection{Safety Lighting}

Each buggy shall have an operating light attached to it at a location that is within 12 inches of the highest point of that buggy (usually the pushbar or the top of the buggy shell), whenever that buggy is being used during any type of practice session that takes place at Carnegie Mellon University between sunset and sunrise. This requirement is applicable to push practices, activities before freeroll practices, drop brake tests, braking capability tests, and whenever drivers are being familiarized with their buggies by being pushed around on sidewalks, in parking lots, etc. The light shall be white in color and it shall be aimed approximately parallel to the ground and in the forward direction. The light shall have sufficient brightness such that it is visible from a distance of 550 feet, which is the approximate distance from the finish line to the end of the Hill 4-5 Transition zone.

Each buggy being used between sunset and sunrise shall also have some type of reflective material that is at least four square inches in area attached to the rear of that buggy in order to reflect the headlights of vehicles that approach it from behind. This material may be white, red, orange, or yellow in color and it must be approved by the Safety Chairman.

Each organization is encouraged to provide additional safety lights for each of its buggies, such as red lights aimed toward the rear, flashing red or orange lights anywhere on the buggies, and flashing strobe type lights on the tops of the buggies.

\subsubsection{Pushers}
All pushers are encouraged to wear reflective vests or other clothing which has some type of reflective material on it whenever they are participating in a night time push practice, so that they will be more visible to vehicular traffic while they are pushing a buggy.

\section{Freeroll Practice Procedures and Rules}

\subsection{Time and Place}

Freeroll practices shall only be held on the buggy course and only on the dates and at the times specified by the Sweepstakes Advisor. Freeroll practices will generally be held between the hours of 6:00 am and 9:00 am on Saturday and Sunday mornings, starting as soon as there is enough light to roll safely, and ending sometime between 8:30 am and 9:00 am, depending on vehicular traffic volume and other extenuating circumstances. Fall freeroll practices will usually be scheduled for two to six weekends sometime between late September and late November each school year. Spring freeroll practices will usually be scheduled for the six weekends immediately preceding the weekend scheduled for the races.

\subsection{Permits}

Freeroll practices shall only be held with the approval of the City of Pittsburgh and the Department of Parks and Recreation. This approval shall be in the form of permits to use the public streets on campus and in Schenley Park, issued by both the City of Pittsburgh and by the Department of Parks and Recreation.

Applications for these permits should be made by the Sweepstakes Advisor, in cooperation with the Sweepstakes Chairman, approximately six to eight weeks prior to the first scheduled freeroll practice in both the fall and in the spring.

\subsection{Police}

Freeroll practices shall only be held with the protection and cooperation of both the Police Department of the City of Pittsburgh, and the Carnegie Mellon University Campus Police Department.

Off-duty City of Pittsburgh police officers, usually from the Park Police Department, are hired by the Sweepstakes Committee and Carnegie Mellon University to provide police protection during all freeroll practice sessions. Arrangements to have these officers present during freeroll practices should be made by the Sweepstakes Chairman, in cooperation with the Sweepstakes Advisor, at the same time that the permits to use the streets are applied for.

Usually a minimum of four officers are needed to provide protection at a freeroll practice. They should be available for the entire freeroll practice, and should report to the Sweepstakes Chairman at least 30 minutes before the practice is scheduled to begin that day. These officers should be stationed as follows:
\begin{itemize}
	\item One on Schenley Drive near the clubhouse for the Schenley Park Golf Course.
	\item One on Circuit Road at its intersection with Schenley Drive, near the George Westinghouse Memorial Pond.
	\item One on Panther Hollow Road at its intersection with Schenley Drive, near the north end of the Panther Hollow Bridge.
	\item One on Schenley Drive at the eastern end of the Schenley Bridge, near its intersection with Frew Street. (This officer might alternatively be stationed at the western end of the Schenley Bridge.)
\end{itemize}

\subsection{No-Parking Signs}

Before each scheduled freeroll practice, No-Parking signs shall be placed around the buggy course in order to prevent cars and other motor vehicles from parking there. The signs shall be obtained from the Police Department of the City of Pittsburgh, with the assistance of the Sweepstakes Advisor, if necessary. The signs shall be put in place around the buggy course as early as 8:00 pm, and NO LATER THAN 11:00 pm, the night before each freeroll practice is scheduled. They shall be removed as soon as the course is officially closed for freeroll practice to begin each day, and NO LATER THAN 8:00 am, including dates on which free rolls have been officially cancelled due to inclement weather.

At the discretion of the Sweepstakes Chairman, the responsibility of obtaining, storing, and placing No-Parking signs in position for freeroll practices may be delegated to one organization. The organization charged with this responsibility shall not have to provide sweepers or flaggers for freeroll practices.

If the organization responsible for the No-Parking signs fails to provide them or remove them for any freeroll practice, that organization shall be fined the amount of \$25.00.

\subsection{Course Inspection and Official Notification}

Approximately two and one half hours before the scheduled start of each freeroll practice, the Sweepstakes Chairman, the Assistant Chairman, the Safety Chairman, and/or anyone designated by any of these Chairmen, shall inspect the buggy course (if necessary) and decide if a freeroll practice can be held that day. After a decision has been made, the Carnegie Mellon University Campus Police Department, and the Police Department of the City of Pittsburgh shall be notified of that decision.

Starting two hours before the scheduled start of a freeroll practice, all of the participating organizations may call the Campus Police Department in order to find out if a freeroll practice is to be held that day.

\subsection{Sweepers}

Each organization shall provide two sweepers for each freeroll practice, to help clean debris from the buggy course. These sweepers must be provided even if an organization is not participating in that freeroll practice. Each organization shall equip their sweepers with brooms (preferably large push type brooms) and/or shovels. Sweepers must also wear reflective vests so that they can be seen by any motorists.

These sweepers MUST be available from a time that is two hours before the freeroll practice is scheduled to start, until they are dismissed for the day, which should usually be when the freeroll practice starts. However, some sweepers and brooms should be available during the entire freeroll practice in case they are needed to clean up debris after an accident or some similar incident. The Sweepstakes Chairman, or anyone designated by that Chairman, shall determine when and where the sweepers must report for duty and when they can leave.

Any organizations that fails to provide the required number of properly equipped sweepers for any freeroll practice shall be fined the amount of \$15.00 for each missing or improperly equipped sweeper. In addition, any organization failing to provide at least one properly equipped sweeper for a freeroll practice shall not be permitted to participate in that freeroll practice.

At the discretion of the Sweepstakes Chairman, some organizations may not be required to provide sweepers, in lieu of providing other services for freeroll practices.

\subsection{Hay Bales}

Freeroll practices shall only be held when an adequate number of hay bales are in place around the buggy course. The Safety Chairman, or anyone designated by that Chairman, shall determine how many hay bales are required in order to have a freeroll practice, and where around the buggy course those hay bales shall be placed, in order to provide the maximum amount of protection to the freeroll practice participants.

Bales of hay, as opposed to bales of straw, are usually used because they tend to hold up better and therefore can be used more times before they start to fall apart. Approximately 80 hay bales should be obtained for use during fall freeroll practices. An additional 100 hay bales should be obtained before the start of freeroll practices in the spring. The haybales should be placed along both curbs of the southwestern end of Frew Street where it intersects with Schenley Drive.

At the discretion of the Sweepstakes Chairman, the responsibility of obtaining, storing, and placing hay bales in position for freeroll practices may be delegated to one organization. The hay bales shall be put in place no later than one hour before the freeroll practice is scheduled to start, and they should be in place before the sweepers clean that part of the course. They shall be removed within 30 minutes after the freeroll practice has ended. The organization charged with this responsibility shall not have to provide sweepers or flaggers for freeroll practices.

If the organization responsible for the hay bales fails to provide them or remove them for any freeroll practice, that organization shall be fined the amount of \$25.00.

\subsection{Flaggers}

Each organization shall provide two flaggers for each freeroll practice, to help control vehicular traffic on the buggy course. These flaggers must be provided even if an organization is not participating in the freeroll practice. These flaggers must be available from a time that is 30 minutes before the freeroll practice is scheduled to start, until the practice is finished for that day. Each organization shall equip their flaggers with reflective vests and flags which must be used while they are acting as flaggers. All vests and flags must be approved by the Safety Chairman or anyone designated by that Chairman. The Sweepstakes Chairman, or anyone designated by that Chairman, shall determine when and where the flaggers must report for duty.

Any organization that fails to provide the required number of properly equipped flaggers for any freeroll practice shall be fined the amount of \$20.00 for each missing or improperly equipped flagger. In addition, any organization failing to provide at least one properly equipped flagger for a freeroll practice shall not be permitted to participate in that freeroll practice.

At the discretion of the Sweepstakes Chairman, some organizations may not be required to provide flaggers, in lieu of providing other services for freeroll practices.

\subsection{Barricades}

Portable wooden barricades shall be placed at several locations near the buggy course while freeroll practices are in progress, to stop and/or redirect vehicular traffic when this traffic tries to approach the area of the buggy course. Warning signs which indicate that the road ahead is, or may be, closed, and that there will be flaggers ahead to stop and/or redirect traffic shall be placed near these barricades also.

The barricades shall be put in place no later than 30 minutes before the freeroll practice is scheduled to begin. They shall be removed within 15 minutes after the freeroll practice has ended. The Safety Chairman, or anyone designated by that Chairman, shall determine if enough barricades are available in order to have a freeroll practice. In order to provide the maximum amount of protection to the freeroll practice participants, barricades should be placed at the following locations:
\begin{itemize}
	\item On Margaret Morrison Street, at its intersection with Tech Street.
	\item On Frew Street on the eastern side of its intersection with Tech Street.
	\item On Schenley Drive, just east of its intersection with Tech Street.
	\item On Circuit Road at its intersection with Schenley Drive, near the George Westinghouse Memorial Pond.
	\item On Panther Hollow Road at its intersection with Schenley Drive, near the northern end of the Panther Hollow Bridge.
	\item On Schenley Drive at the eastern end of the Schenley Bridge, near its intersection with Frew Street. The placement of the barricades at this location should be performed carefully. Some space should be left on the northern side of the bridge so that if a buggy failed to make the turn onto Frew Street it would have adequate room to drive across the bridge in the right-hand lane.
	\item On the Scaife Hall driveway at its intersection with Frew Street.
\end{itemize}

At the discretion of the Sweepstakes Chairman, the responsibility of obtaining, storing, and placing barricades in position for freeroll practices may be delegated to one organization. The organization charged with this responsibility shall not have to provide sweepers or flaggers for freeroll practices.

If the organization responsible for the barricades fails to provide them or remove them for any freeroll practice, that organization shall be fined the amount of \$25.00.

\subsection{Warning Signs}

Warning signs shall be placed at several locations near the buggy course while freeroll practices are in progress, to warn vehicular traffic when this traffic tries to approach the area of the buggy course. The signs shall indicate that the road ahead is, or may be, closed, and that there will be flaggers ahead to stop and/or redirect traffic. The signs shall be put in place no later than 30 minutes before the freeroll practice is scheduled to begin. They shall be removed within 15 minutes after the freeroll practice has ended. The Safety Chairman, or anyone designated by that Chairman, shall determine if enough warning signs are in place in order to have a freeroll practice. In order to provide the maximum amount of protection to the freeroll practice participants, warning signs should be placed at all of the locations where barricades have been placed, plus the following locations:
\begin{itemize}
	\item On Margaret Morrison Street at its intersection with Forbes Avenue.
	\item On Schenley Drive at its intersection with Forbes Avenue, near the clubhouse for the Schenley Park Golf Course.
	\item On Circuit Road at its intersection with Serpentine Drive.
	\item On Panther Hollow Road at the southern end of Panther Hollow Bridge.
	\item On Schenley Drive between the Mary E. Schenley Memorial Fountain and the southwest corner of the Carnegie Museum building.
	\item On the driveway to the rear of Hamburg Hall at its intersection with Forbes Avenue.
\end{itemize}

At the discretion of the Sweepstakes Chairman, the responsibility of obtaining, storing, and placing warning signs in position for freeroll practices may be delegated to one organization (usually the organization in charge of barricades). The organization charged with this responsibility shall not have to provide sweepers or flaggers for freeroll practices.

If the organization responsible for the warning signs fails to provide them or remove them for any freeroll practice, that organization shall be fined the amount of \$25.00.

\subsection{Rollboard}

At the discretion of the Sweepstakes Chairman, the responsibility of moving a portable chalkboard or dry-erase board to the sidewalk near the top of Hill 2, before the start of each freeroll practice, and returning it after the practice is over, may be delegated to one organization. This board shall be used to post the rolling order for freeroll practices.

\subsection{Drop Tests and Safety Equipment Check}

Before each freeroll practice session starts, each buggy is required to take a drop brake test as follows:

No buggy shall be permitted to participate in a freeroll practice unless it has successfully completed a drop brake test earlier that same day. Each buggy's performance in the drop brake test shall be recorded by the test administrator at the time that the buggy is tested.

Any buggy that fails the drop brake test on each of two consecutive attempts before a freeroll practice session, shall be required to again successfully complete a braking capability test, with any of its drivers, before that buggy will be permitted to participate in any subsequent freeroll or push practice sessions.

At the time that the drop brake tests are given, each participating buggy and driver may be checked by the test administrator to ensure that they have proper head protection, eye protection, hand protection, adequate field of vision, and safety harness.

Drop brake tests will be administered by the Safety Chairman, or anyone designated by that Chairman, on the morning of each scheduled freeroll practice. The starting time for these tests shall be specified by the Safety Chairman, or anyone designated by that Chairman, sometime before the day of that freeroll practice.

\subsection{Rolling Order}

The order in which the organizations who are participating in a freeroll practice session will be permitted to freeroll their buggies should usually be determined by the following procedure:

Before the first scheduled freeroll practice of each school semester, the Sweepstakes Chairman shall compile a list of all of the organizations wishing to participate in freeroll practices. This list shall be in a random order which shall be determined by a lottery method chosen by the Chairman. The order of this list shall be the rolling order for the first freeroll practice session of that semester. For each subsequent freeroll practice session of that semester, the rotation of the rolling order shall remain the same and the organization that rolls first shall be the organization that was next in the order to roll at the previous freeroll practice session when that practice session ended.

Changes to this rolling order procedure may be made for individual freeroll practices at the discretion of the Sweepstakes Chairman with the approval of the Sweepstakes Committee.

\subsection{Course Communications}

Freeroll practices shall only be held when adequate radio communication equipment is available to provide voice communications around the buggy course for automobile traffic control, buggy traffic control, and emergency situation assistance. Radio communication equipment, and the personnel to operate it, are usually available through the Carnegie Mellon University Radio Club. Any personnel helping to provide radio communications should not be responsible for making decisions concerning what happens during a freeroll practice, but instead should be providing information to the Sweepstakes Chairman and his or her assistants, in order that they may make any necessary decisions.

\subsection{Traffic Control}

Control of vehicular traffic on the buggy course during each freeroll practice session will be handled by the City of Pittsburgh police officers, usually from the Park Police Department, who are hired by the Sweepstakes Committee and Carnegie Mellon University to provide police protection during all freeroll practice sessions. These officers, with the assistance of the flaggers provided by the buggy racing organizations, will stop vehicular traffic from entering the buggy course while buggies are freerolling.

If possible, the buggy course will be completely closed to vehicular traffic from the time that each freeroll practice session begins, until that session has ended for that day. If necessary, the officers will open the buggy course to traffic one or more times during the course of the freeroll practice session to relieve traffic buildup on the streets around the buggy course. In this event, adequate notice must be given to the Sweepstakes Chairman so that no buggies are permitted to freeroll while there is vehicular traffic on the buggy course. If the course is opened at any time during the practice session, all of the vehicles entering the course should be instructed not to park on the course. Freeroll practice will be continued when the Sweepstakes Chairman has determined that the course is once again closed and that it is clear of all vehicles.

\subsection{Starting Procedure}

Before and after each organization freerolls its buggies, the Sweepstakes Chairman, or anyone designated by that Chairman, will announce which organization is next in order to freeroll its buggies. The next organization in the rolling order shall be designated as being ``UP'', the second organization in order shall be designated as being ``ON DECK'', and the third organization in order shall be designated as being ``IN THE HOLE''. When the Chairman determines that the course is clear of all vehicles and buggies, he or she will signal the ``UP'' organization that it may roll its buggies. After this signal, that organization will have 15 seconds in which to get its first buggy started, (if that organization is running Hills 1 and 2 it will have 45 seconds from the signal to get its first buggy over Hill 2 and into the freeroll portion of the course).

After an organization's first buggy is into the freeroll portion of the course, that organization will have an additional 30 seconds for each of its remaining buggies in order to get all of those buggies into the freeroll portion of the course. After an organization's last buggy is into the freeroll portion of the course, an additional 15 seconds will be allowed to get its follow car into the freeroll portion of the course. If an organization exceeds this total time limit, that organization will not be permitted to freeroll any of its buggies on its next turn in the rolling order, even if that turn isn't until a freeroll practice session at a later date.

After an organization rolls its buggies at a freeroll practice session, the next organization in the rolling order will not be given the signal to start freerolling its buggies until the previous organization's follow car is past the chute.

\subsection{Follow Cars}

During each freeroll practice session, each participating organization must provide a motor vehicle known as a follow car which will drive around the buggy course directly behind that organization's last buggy each time that organization freerolls its buggies. The follow car must be ready to leave when the organization receives the signal to start rolling its buggies from the Sweepstakes Chairman. The follow car must leave the Hill 2 lanes area of the course and be following the last buggy within 15 seconds of when that buggy is released into freeroll by its pusher. If this time limit is exceeded, the rolling organization will not be permitted to freeroll any of its buggies on its next turn in the rolling order, even if that turn isn't until a freeroll practice session at a later date.

The follow car should stay approximately one hundred feet behind the last buggy. Several members of the freerolling organization should ride in the follow car so that they may observe the buggies from that vantage point and provide any assistance that may be necessary during that freeroll. In the event of an accident during a freeroll, the follow car should stop near the scene of the accident in order to render assistance to those involved in the accident. The members of the freerolling organization riding in the follow car MUST have any tools or devices that are necessary to quickly remove any of their drivers from their buggies. If an organization fails to have these tools or devices in the car following the buggies during any freeroll, that organization shall be fined the amount of \$15.00 and shall not be permitted to freeroll any of its buggies for the remainder of that day.

Follow cars should not be stopped on the buggy course during freeroll practice sessions unless a buggy has stopped during its roll. Stopping a follow car on the course in order to talk to people or to allow people to enter or leave the car is strictly prohibited. Stopping a follow car while a freeroll practice is in progress creates a potential safety hazard and also causes delays to the buggies waiting to roll.

If the vehicle used as the follow car is a pick-up truck and there are people riding in the back of that truck, its tail gate must always be closed while the vehicle is moving in order to reduce the chances of anyone falling out of it.

At the discretion of the Sweepstakes Chairman, one or more organizations may be appointed to provide follow cars (with drivers) to follow the buggies of other organizations, either because they do not have access to a follow car vehicle during freeroll practices, or in order to reduce confusion and safety hazards caused by too many different follow cars during freeroll practices. The organization(s) charged with this responsibility may not have to provide sweepers or flaggers for freeroll practices.

\subsection{Signal Flaggers}

Each organization participating in freeroll practices must provide a signal flagger for each of its buggy drivers. These required signal flaggers shall be known as chute flaggers. Chute flaggers should provide a signal to the buggy drivers so that the drivers know when to start the right hand turn from Schenley Drive onto Frew Street. Chute flaggers should usually be positioned on the southern curb of Schenley Drive, just east of the intersection of Frew Street. Chute flaggers are not permitted to be on the street portion of Schenley Drive while any buggy is freerolling in that area unless they receive specific approval from the Safety Chairman. In general, if these flaggers need to position their flags more than an arms length away from the curb, their signal flags should be attached to extension poles so that they may hold these flags out over the street while still standing on the curb.

Each organization's chute flagger must be able to provide an alternate signal to their buggy drivers (such as waving the flag in a particular manner or using different colored flags) which will indicate to those drivers that there is a problem farther ahead and that the driver should start slowing down in a controlled manner.

People providing course communications (usually the Carnegie Mellon University Radio Club) may also signal buggy drivers that there is a problem farther ahead. This will be accomplished by waving a YELLOW flag so that the buggy driver can see it. This yellow flag indicates that the buggy driver should start slowing down in a controlled manner so that he or she may stop as soon as it is possible to do so safely. NO organization may use a yellow colored flag for any purpose other than to indicate a problem ahead.

Each organization's chute flagger must have adequate experience relative to chute flagging. Each chute flagger should talk to the drivers that he or she will be flagging for, in order to determine where those drivers would like the flag to be placed. If possible each chute flagger should walk the buggy course with those drivers before the freeroll practice session starts each day that they will be flagging. If a chute flagger is considered to be acting in an unsafe or inexperienced manner, the Safety Chairman, or anyone designated by that Chairman, may require that flagger to be replaced with another more experienced flagger.

If an organization fails to provide a chute flagger for any of its buggy drivers, that organization shall be fined the amount of \$10.00 each time one of its buggies attempts to make the turn from Schenley Drive onto Frew Street without a chute flagger.

Each organization participating in freeroll practices may also provide additional signal flaggers for its drivers, in order to provide them with additional information about the buggy course. These flaggers may be located anywhere around the buggy course, such as just after Hill 2 or near the entrance to Phipps Conservatory. These flaggers are not permitted to stand anywhere on the course whenever any buggy is freerolling, unless they receive specific approval from the Safety Chairman.

\subsection{Accidents}

Accidents may occur during freeroll practice sessions. If an accident does occur, and medical personnel are in attendance at that practice session, they should be dispatched to the scene of that accident as quickly as possible. In this case they will have the responsibility of determining the condition of any victims and of providing any needed first aid. If no medical personnel are immediately available, the responsibility of aiding any accident victims will be borne by either the Carnegie Mellon University Campus Police or the City of Pittsburgh Police who are in attendance.

Before any assistance arrives at the scene of an accident, common sense should be observed. The victim should not be moved or disturbed in any way unless there is some other danger in not doing so. If a buggy is involved in an accident and comes to rest in a position where it might be impacted by another buggy that is freerolling behind it, that buggy should be immediately moved, as quickly and as gently as possible, to a position of safety. If a buggy needs to be immediately moved from a dangerous position after an accident, ANYONE near it may do so, even if they are not members of that buggy's sponsoring organization. If there is no immediate danger however, the buggy and its driver should not be moved or disturbed until available assistance arrives and assesses the situation. In general, a buggy driver should not be removed from a buggy until a responsible authority advises that it is wise to do so. In ABSOLUTELY NO CASE should an unconscious driver be removed from a buggy without the supervision of medical personnel.

Organizations are encouraged to position people near the chute portion of the buggy course during freeroll practices so that someone will be available to provide any needed assistance in the event that one of their buggies is involved in an accident.

Any buggy that is involved in an accident during a freeroll practice session may not be used in any type of practice session again until that buggy's sponsoring organization has submitted an accident report and had that report approved as described in the Practice Sessions, General Procedures and Rules section of this document.

\subsection{Medical Personnel}

If medical personnel are available during freeroll practices, such as off-duty paramedics from the City of Pittsburgh or members of an active Emergency Medical Service that might exist on the Carnegie Mellon University campus, they should be stationed near the chute portion of the buggy course, since that is the most likely area in which an accident might occur.

\subsection{Cancellation}

Any freeroll practice session may be canceled at any time by the Sweepstakes Chairman, or anyone designated by that Chairman, due to inclement weather, inadequate police protection, inadequate communications, lack of medical personnel, vehicles parked on the course or on the sidewalks around the course, or any other condition which might endanger the participants or spectators of that practice session.

\subsection{Clean-Up}

After each freeroll practice session has been completed, all debris on the buggy course and on the sidewalks around the course must be cleaned up and disposed of properly. Special care must be taken to ensure that all debris left by freeroll practice participants is removed, such as empty food and beverage containers, duct tape, buggy preparation materials, hay that has fallen off hay bales, No-Parking signs that have been misplaced, etc.

\subsection{Safety Checks}

Each freeroll practice session will be observed by the Safety Chairman, and/or anyone designated by that Chairman. At any time during a freeroll practice session this observer may check for any or all of the following:
\begin{itemize}
	\item Check the safety of any practicing buggy by performing a spot safety check.
	\item Check that all required barricades and warning signs are in place.
	\item Check that No-Parking signs are in place.
	\item Check that each organization has provided the proper number of properly equipped flaggers to help control vehicular traffic.
	\item Check that each practicing organization has an adequate number of people to properly attend to all of the buggies that they are using.
	\item Check that all practicing buggy drivers are properly qualified to drive.
	\item Check that an adequate number of hay bales are in place.
	\item Check that each practicing organization has a properly equipped chute flagger.
\end{itemize}

\subsection{Drivers}

The following procedures and rules shall apply to buggy drivers during freeroll practice sessions:

No driver shall be permitted to participate in a freeroll practice unless he or she complies with all of the following requirements:

He or she has walked the buggy course earlier that same day. Each driver must check in with the Assistant Sweepstakes Chairman, or anyone designated by that Chairman, immediately before or after they walk the course. Drivers are only permitted to walk the course BEFORE a freeroll practice session starts.

He or she has previously successfully completed a braking capability test in the buggy that he or she will be driving in that freeroll practice.

He or she has previously successfully completed a field of vision test in the buggy that he or she will be driving in that freeroll practice.

He or she has previously become familiar with the operation of the buggy that he or she will be driving in that freeroll practice by being pushed around (on sidewalks, in parking lots, etc.) in that same buggy.

He or she has either attended all of the driver education meetings that have been held that school year, or has reviewed the content of those meetings with someone designated by the Sweepstakes Chairman.

Each driver is encouraged to complete a drop brake test before he or she participates in a freeroll practice session in the buggy that he or she will be driving at that session. This test does not have to be observed by the Safety Chairman provided that the buggy in question completed a drop brake test earlier that same day with some other driver. The purpose of this test is to ensure that the buggy's braking system is both properly adjusted for the current driver and is in proper working order.

\section{Push Practice Procedures and Rules}

\subsection{Time and Place}

Street push practices shall only be held on Tech Street and Frew Street and only on the dates and at the times specified by the Sweepstakes Advisor.

Street push practices will generally be held between the hours of 12:00 midnight and 6:00 am, Mondays through Fridays, for approximately six weeks immediately preceding the date scheduled for the preliminary races each year. These days and times shall be determined by the Sweepstakes Advisor.

Sidewalk push practices, when permitted, shall only be held during daylight hours, of any day of the week, of any week that school is in session. Sidewalk push practices shall only be held with the permission of the Sweepstakes Chairman or anyone designated by that Chairman.

\subsection{Permits}

Street push practices shall only be held with the approval of the City of Pittsburgh and the Department of Parks and Recreation. This approval shall be in the form of permits to use the streets on campus and in the park, issued by both the City of Pittsburgh and the Department of Parks and Recreation.

Applications for these permits should be made by the Sweepstakes Advisor, in cooperation with the Sweepstakes Chairman, approximately six to eight weeks prior to the first scheduled push practice in the spring.

\subsection{Police}

Push practices shall only be held with the authorization of both the Police Department of the City of Pittsburgh, and the Carnegie Mellon University Campus Police Department.

\subsection{Informational Flyers}

Several times during the first two weeks of street push practices, informational flyers shall be distributed to all of the motor vehicles parked along Tech Street and Frew Street, by placing these flyers under the windshield wipers of these vehicles. These informational flyers shall be provided by the Sweepstakes Chairman or the Sweepstakes Advisor, and they shall inform the drivers of the parked vehicles that street push practices are now in progress and that caution must be observed when driving in that area until the races are over. These flyers shall be distributed by members of one or more organizations, as assigned by the Sweepstakes Chairman.

Informational flyers should also be distributed (by mail or hand delivery) to any residence or business in the area that might be affected by push practices. Places like apartment buildings near campus, pizza delivery services, etc. should all be notified about push practices. In addition, the campus community itself must be made aware of push practices. A notice should be placed in the school newspaper and a memo should be distributed to the faculty and staff so that they are knowledgeable concerning when and where push practices will be in progress.

\subsection{Official Notification}

The sweepstakes chair shall notify the Carnegie Mellon University Campus Police Department when Push practice is scheduled to begin for the year. Each organization shall notify the Carnegie Mellon University Campus Police Department, and obtain their authorization, before they start any push practice session, either on the streets or on the sidewalks. As part of this notification, each organization shall inform the Campus Police Department where they will be practicing, and how long they expect to be there, and shall have a copy of the permits for push practice in case they are questioned by campus or city police.

\subsection{Barricades and Warning Signs}

Portable wooden barricades with warning signs shall be placed at several locations near Tech Street and Frew Street while street push practices are in progress, to warn and/or stop vehicular traffic when this traffic tries to approach these streets. The signs shall indicate that the road is, or may be, closed, and that there will be flaggers to stop and/or redirect traffic. The barricades and signs shall be put in place immediately before any street push practices begin, by the participants of those practices. They shall be removed within 15 minutes of the end of those practices by those same participants. Generally, the first organization starting to practice on any particular hill puts the barricades out, and the last organization to finish practicing on that hill puts them away.

In order to provide the maximum amount of protection to the push practice participants, barricades should be placed at the following locations:
\begin{itemize}
	\item On Margaret Morrison Street, at its intersection with Tech Street.
	\item On Frew Street on the eastern side of its intersection with Tech Street.
	\item On Tech Street, just north of its intersection with Schenley Drive.
	\item On Frew Street, just north of its intersection with Schenley Drive.
	\item On the Scaife Hall driveway at its intersection with Frew Street.
\end{itemize}

Any organization found to be conducting a push practice session without a sufficient number of barricades in place, shall be fined the amount of \$15.00.

\subsection{Flaggers}

Each organization conducting a street or sidewalk push practice session must have a minimum of four flaggers present in order to control vehicular traffic on Tech Street and Frew Street. Each organization shall equip their flaggers with reflective vests and flags which must be used while they are acting as flaggers. All vests and flags must be approved by the Safety Chairman or anyone designated by that Chairman. In addition, all flaggers at night time practices are advised to use flashlights, preferably with illuminated orange extensions, in order to make themselves more visible to vehicular traffic. Flaggers are also advised to wear orange, yellow, or white construction type hard hats, so that they may appear to be more official-looking to vehicular traffic.

When more than one organization is practicing on a hill or on adjacent hills, they may combine their flaggers in order to save manpower. However, if one organization leaves, any flaggers that leave with them must be replaced by the remaining organizations before they can continue practicing. In any event, flaggers from different organizations MUST cooperate with each other in order to control traffic. Flaggers should be positioned such that there is at least one flagger at each end, and one flagger in the middle, of each hill that is being used for practice. The flagger in the middle of the hill should be responsible for any vehicles that pull out of parking spaces along the hill.

Any organization that fails to provide the required number of properly equipped flaggers during any push practice shall be fined the amount of \$15.00 for each missing or improperly equipped flagger, shall not be permitted to continue their push practice session that day, and shall not be permitted to participate in a push practice session on the next night that they are scheduled to practice.

\subsection{Traffic Control}

The flaggers from each organization MUST communicate with each other concerning when the hills will be opened and closed to traffic. A hill should only be opened to traffic when all flaggers monitoring that hill have communicated to each other that it is safe for traffic to pass.

If problems arise in stopping vehicular traffic during push practices, such as having a vehicle disregard the flaggers and drive onto a hill on which practice is in progress, the FIRST and most critical concern of the flaggers involved MUST be to warn those people who are practicing on that hill that a vehicle is approaching! The secondary concern of the flaggers should be to record the license number of the offending vehicle and report it to the Campus Police. IN NO CASE must any driver of a vehicle be physically assaulted or verbally abused by anyone participating in a push practice. Anyone found to have treated a driver or a vehicle disrespectfully, will be banned from any further participation in any Sweepstakes related activities for the remainder of that school year.

\subsection{Pushing Order}

Each school year during the spring semester, before any street push practices are scheduled, the Sweepstakes Chairman, and/or anyone designated by that Chairman, shall assign days, times, and hills for each organization's street push practices. Each organization shall only be permitted to hold push practices on the street, on the days, at the times, and on the hills, that are assigned to them. The Sweepstakes Chairman shall determine the method by which these assignments shall be made. No more than four organizations shall be assigned to any one hill at any given time. No organization may practice running more than one hill at a time with the same buggy until after 2:00 am on any night that street push practices are in progress.

\subsection{Cancellation}

Any push practice session may be canceled at any time by the Sweepstakes Chairman, or anyone designated by that Chairman, due to inclement weather, inadequate police protection, inadequate communications, lack of medical personnel, vehicles obstructing the course or any other condition which might endanger the participants or spectators of that practice session.

\subsection{Clean-Up}

After each organization has finished its push practice session, all debris on the buggy course and on the sidewalks around the course must be cleaned up and disposed of properly. Special care must be taken to ensure that all debris left by push practice participants is removed, such as empty food and beverage containers, duct tape, buggy preparation materials, etc. In addition, all barricades and warning signs must be returned to their storage areas by the last organization to use them. The responsibility of removing and properly storing the barricades and warning signs for each hill belongs to the last organization to finish practicing on that hill.

\subsection{Spot Safety Checks}

Each push practice session will be observed by the Safety Chairman, and/or anyone designated by that Chairman, for at least the first hour of that session. At any time during a push practice session, this observer may check for any or all of the following:
\begin{itemize}
	\item Check the safety of any practicing buggy by performing a spot safety check.
	\item Check that all required barricades and warning signs are in place.
	\item Check that informational flyers have been distributed, if so required.
	\item Check that each practicing organization has an adequate number of properly equipped flaggers to help control vehicular traffic.
	\item Check that no organization is practicing on any hill to which they are not assigned for that particular practice.
	\item Check that each practicing organization has an adequate number of people to properly attend to all of the buggies that they are using.
	\item Check that each practicing organization has notified the Carnegie Mellon University Campus Police that they are currently conducting a push practice.
\end{itemize}

\chapter{Race Rules, Regulations, and Procedures}

\section{Sweepstakes Race Schedule}

The Sweepstakes races shall be scheduled to be held at or near the same time that the Carnegie Mellon University Spring Carnival is held each school year. The races will usually be scheduled for two consecutive days.

\subsection{First Day of Racing}

The preliminary races shall be held on the first day of Sweepstakes Racing, with the women's preliminary races being held first, followed by the men's preliminary races.

\subsection{Second Day of Racing}

The alumni/exhibition races, the rerun races, and the finals races shall be held on the second day of Sweepstakes racing. If alumni/exhibition races are to be run, they shall be scheduled first, followed by the women's rerun races (if necessary), the men's rerun races (if necessary), the women's finals races, and the men's finals races, in that order. The order in which these races are scheduled may be changed at the discretion of the Sweepstakes Chairman or the Sweepstakes Advisor, based on weather forecasts and/or time constraints.

\subsection{Cancellation}

Either or both days of Sweepstakes racing may be canceled by the Dean of Student Affairs, the Sweepstakes Advisor, or the Sweepstakes Chairman, due to inclement weather, inadequate police protection, inadequate communications, lack of medical personnel, vehicles parked on the course or on the sidewalks around the course, or any other condition which might endanger the contestants or spectators of the races.

If the preliminary races are canceled on the day that they were originally scheduled to take place (Friday), they shall be rescheduled for the day that the finals races were originally scheduled to take place (Saturday), and no finals races shall be held that year.

If the preliminary races are run as originally scheduled (on Friday) and the finals races are canceled on the day that they were originally scheduled to take place (Saturday), no finals races shall be held that year.

If the preliminary races are canceled on the day that they were originally scheduled to take place (Friday), and then are canceled again on the day for which they were rescheduled (Saturday), the preliminary races may be rescheduled again and held on the Sunday of Spring Carnival or a later date, at the discretion of the Sweepstakes Advisor.

\section{Race Day Procedures}

\subsection{Time and Place}

Sweepstakes races shall only be held on the buggy course and only on the dates and at the times specified by the Sweepstakes Advisor. Sweepstakes races will generally be held between the hours of 6:00 am and 2:00 pm on the Friday and Saturday of Spring Carnival weekend. Sweepstakes races may be held on other days and at other times at the discretion of the Sweepstakes Advisor.

\subsection{Permits}

Sweepstakes races shall only be held with the approval of the City of Pittsburgh and the Department of Parks and Recreation. This approval shall be in the form of permits to use the public streets on campus and in Schenley Park, issued by both the City of Pittsburgh and by the Department of Parks and Recreation.

Applications for these permits should be made by the, Sweepstakes Advisor in cooperation with the Sweepstakes Chairman, at the same time that applications for permits for spring freeroll practices are made.

\subsection{Police}

Sweepstakes races shall only be held with the protection and cooperation of both the Police Department of the City of Pittsburgh, and the Carnegie Mellon University Campus Police Department.

Off-duty City of Pittsburgh police officers, usually from the Park Police Department, are hired by the Sweepstakes Committee and Carnegie Mellon University to provide police protection during all of the Sweepstakes races. Arrangements to have these officers present during the races should be made by the Sweepstakes Chairman, in cooperation with the Sweepstakes Advisor, at the same time that the permits to use the streets are applied for.

Usually a minimum of four officers are needed to provide protection during the races. They should be available during the entire time that the races are underway, and should report to the Sweepstakes Chairman or the Sweepstakes Advisor on each day of racing, at least 30 minutes before the races are scheduled to begin that day. These officers should be stationed as follows:
\begin{itemize}
	\item One on Schenley Drive near the clubhouse for the Schenley Park Golf Course.
	\item One on Circuit Road at its intersection with Schenley Drive, near the George Westinghouse Memorial Pond.
	\item One on Panther Hollow Road at its intersection with Schenley Drive, near the north end of the Panther Hollow Bridge.
	\item One on Schenley Drive at the eastern end of the Schenley Bridge, near its intersection with Frew Street. (This officer might alternatively be stationed at the western end of the Schenley Bridge.)
\end{itemize}

Carnegie Mellon University Campus Police should be available during the entire time that the races are underway in the event that their assistance is needed. Arrangements to have these officers present during the races should be made by the Sweepstakes Chairman, in cooperation with the Sweepstakes Advisor.

\subsection{Lane and Zone Markings}

Lines delineating the lanes on Hills 1 and 2, the starting line, the finish line, and the beginnings and ends of the three transition zones must be painted on the buggy course. At the discretion of the Sweepstakes Chairman, the responsibility of painting these lane and zone markings on the buggy course for the Sweepstakes races may be delegated to one organization. These lines should normally be painted by a date that is at least two weeks before the date scheduled for the preliminary races. Paint and the striper to paint the lines should be obtained through the Sweepstakes Advisor.

If the organization responsible for painting lane and zone markings fails to provide them without adequate reason, such as bad weather, before the last scheduled freeroll practice before the races, that organization shall be fined the amount of \$25.00.

\subsection{Buggy Preparation Areas}

Each organization participating in the Sweepstakes races shall be permitted to select an area near the starting line of the buggy course, in which they may prepare their buggies prior to the races. These areas may be occupied by trucks or any other non-permanent enclosures. The order of selection of these areas shall be the same as the seeding order for the preliminary races. The selection of these areas should be made at least four weeks prior to the date scheduled for the preliminary races. The selection process shall be supervised by the Sweepstakes Chairman, or anyone designated by that Chairman. The Sweepstakes chairman will assign spaces to organizations at least 2 weeks prior to the date schedule for the preliminary races. No organization may use or attempt to use any area that has been selected by another organization on any day of Sweepstakes racing.

\subsection{Electrical Power}

Electrical power may be made available on each day of Sweepstakes racing, to each organization participating in the races. If available, this electrical power will be located near each organization's buggy preparation area, and will nominally be 110 volt alternating current, with a maximum current capacity of 20 amperes. (This will provide a maximum nominal power of 2,200 watts.) Two standard 110 VAC electrical outlets will be provided to each participating organization. Each organization will be responsible for providing adequate grounded cabling to transfer this power from the location of the outlets, to their individual buggy preparation areas.

When available, the electrical power near the buggy preparation areas will be provided by the Physical Plant Department of Carnegie Mellon University. Arrangements to provide this power should be made with the Physical Plant Department by the Sweepstakes Chairman, through the office of the Sweepstakes Advisor, at least six weeks before the races are scheduled to take place.

\subsection{Finish Line}

A platform several feet high should be placed on the northern sidewalk of Frew Street, directly in line with the finish line of the buggy course. This platform will provide a vantage point for the Sweepstakes race timers, so that they have a better view of the finish line of the race.

A flatbed truck is usually used as the platform at the finish line. This truck can usually be obtained from the Physical Plant Department of Carnegie Mellon University. Arrangements to borrow this truck should be made by the Sweepstakes Chairman, approximately six weeks before the preliminary races are scheduled to take place, through the office of the Sweepstakes Advisor.

\subsection{Course Watch}

The night before each scheduled day of Sweepstakes racing, the Sweepstakes Chairman, or anyone designated by that Chairman, shall assign representatives of each participating organization to watch various parts of the buggy course in order to prevent motor vehicles from parking on any part of the course. Each participating organization must provide two people to watch the course and a motor vehicle for those people to ride in. The length of time that each person must watch the course and the area that they must watch shall be determined by the Sweepstakes Chairman.

Any organization that fails to provide the required number of people and vehicles for course watch duty on the night before any day of Sweepstakes racing, shall be fined the amount of \$25.00 for each missing person or vehicle.

\subsection{No-Parking Signs}

Before each scheduled day of Sweepstakes racing, No-Parking signs shall be placed around the buggy course in order to prevent cars and other motor vehicles from parking there. The signs shall be obtained from the Police Department of the City of Pittsburgh, with the assistance of the Sweepstakes Advisor, if necessary. The signs shall be put in place around the buggy course as early as 8:00 pm, but NO LATER THAN 11:00 pm, the night before each day of Sweepstakes racing is scheduled. They shall be removed at the time that the course is officially closed to vehicular traffic for each day of races, or the cancellation of the races for that day, unless otherwise specified by the Sweepstakes Chairman or the Sweepstakes Advisor.

At the discretion of the Sweepstakes Chairman, the responsibility of obtaining, storing, and placing No-Parking signs in position for each scheduled day of Sweepstakes racing may be delegated to one organization. The organization charged with this responsibility may not have to provide sweepers, flaggers, or course marshals for each day of Sweepstakes racing, at the discretion of the Sweepstakes Chairman.

If the organization responsible for the No-Parking signs fails to provide them or remove them for any day of Sweepstakes racing, that organization shall be fined the amount of \$50.00.

\subsection{Course Inspection and Official Notification}

Approximately 30 minutes before the scheduled start of the first race on each scheduled day of Sweepstakes racing, the Sweepstakes Chairman and the Safety Chairman shall inspect the buggy course (if necessary) and decide if Sweepstakes races can be safely held that day. If no races are to be held, representatives of all participating organizations and any other people involved with the running of the races shall be notified by the person or persons who made the decision.

\subsection{Sweepers}

Each organization shall provide two sweepers for each scheduled day of Sweepstakes racing, to help clean debris from the buggy course. Each organization shall equip their sweepers with brooms (preferably large push type brooms) and/or shovels. These sweepers must be available from a time that is four hours before the races are scheduled to start, until the races are finished for that day. The Sweepstakes Chairman, or anyone designated by that Chairman, shall determine when and where the sweepers must report for duty.

Any organization that fails to provide the required number of properly equipped sweepers for any day of Sweepstakes racing shall be penalized by having one entry withdrawn, and the entry fee for that entry forfeited.

At the discretion of the Sweepstakes Chairman, some organizations may not be required to provide sweepers, in lieu of providing other services on each day of Sweepstakes racing.

\subsection{Hay Bales}

Sweepstakes races shall only be held when an adequate number of hay bales are in place around the buggy course. The Safety Chairman, or anyone designated by that Chairman, shall determine how many hay bales are required in order to have a Sweepstakes race, and where around the buggy course those hay bales shall be placed, in order to provide the maximum amount of protection to the Sweepstakes race participants.

Bales of hay, as opposed to bales of straw, are usually used because they tend to hold up better and therefore can be used more times before they start to fall apart. Approximately 120 hay bales (in addition to those already obtained for freeroll practices) should be obtained for use during Sweepstakes races. The hay bales should be placed along both curbs of the western end of Frew Street where it intersects with Schenley Drive. For races, bales should be set up as they are for normal freeroll practices, except that the curbs should be lined two bales deep for added cushion.

At the discretion of the Sweepstakes Chairman, the responsibility of obtaining, storing, and placing hay bales in position for each day of Sweepstakes racing may be delegated to one organization. The hay bales shall be put in place no later than two hours before races are scheduled to start on each day of Sweepstakes racing, and they should be in place before the sweepers clean that part of the course. They shall be removed within 45 minutes of either the end of the races for that day, or the cancellation of the races for that day. The organization charged with this responsibility may not have to provide sweepers, flaggers, or course marshals for each day of Sweepstakes racing, at the discretion of the Sweepstakes Chairman.

If the organization responsible for the hay bales fails to provide them or remove them for any day of Sweepstakes racing, that organization shall be fined the amount of \$50.00.

\subsection{Flaggers}

Each organization shall provide two flaggers for each scheduled day of Sweepstakes racing, to help control vehicular traffic on the buggy course. These flaggers must be available from a time that is one hour before the races are scheduled to start, until the races are finished for that day. Each organization shall equip their flaggers with reflective vests and flags which must be used while they are acting as flaggers. All vests and flags must be approved by the Safety Chairman or anyone designated by that Chairman. The Sweepstakes Chairman, or anyone designated by that Chairman, shall determine when and where the flaggers must report for duty.

Any organization that fails to provide the required number of properly equipped flaggers for any day of Sweepstakes racing shall be penalized by having one entry withdrawn, and the entry fee for that entry forfeited.

At the discretion of the Sweepstakes Chairman, some organizations may not be required to provide flaggers, in lieu of providing other services on each day of Sweepstakes racing.

\subsection{Barricades}

Portable wooden barricades shall be placed at several locations near the buggy course while Sweepstakes races are in progress, to stop and/or redirect vehicular traffic when this traffic tries to approach the area of the buggy course. Warning signs which indicate that the road ahead is, or may be, closed, and that there will be flaggers ahead to stop and/or redirect traffic shall be placed near these barricades also. The barricades shall be put in place no later than one hour before races are scheduled to start on each day of Sweepstakes racing. They shall be removed within 15 minutes of either the end of the races for that day, or the cancellation of the races for that day. The Safety Chairman, or anyone designated by that Chairman, shall determine if enough barricades are in place in order to have a Sweepstakes race. In order to provide the maximum amount of protection to the Sweepstakes race participants, barricades should be placed at least at the following locations:
\begin{itemize}
	\item On Margaret Morrison Street, at its intersection with Tech Street.
	\item On Frew Street on the eastern side of its intersection with Tech Street.
	\item On Schenley Drive, just east of its intersection with Tech Street.
	\item On Circuit Road at its intersection with Schenley Drive, near the George Westinghouse Memorial Pond.
	\item On Panther Hollow Road at its intersection with Schenley Drive, near the northern end of the Panther Hollow Bridge.
	\item On Schenley Drive at the eastern end of the Schenley Bridge, near its intersection with Frew Street. The placement of the barricades at this location should be performed carefully. Some space should be left on the northern side of the bridge so that if a buggy failed to make the turn onto Frew Street it would have adequate room to drive across the bridge in the right-hand lane.
	\item On the Scaife Hall driveway at its intersection with Frew Street.
\end{itemize}

At the discretion of the Sweepstakes Chairman, the responsibility of obtaining, storing, and placing barricades in position for Sweepstakes races may be delegated to one organization. The organization charged with this responsibility may not have to provide sweepers, flaggers, or course marshals for each day of Sweepstakes racing, at the discretion of the Sweepstakes Chairman.

If the organization responsible for the barricades fails to provide them or remove them for any day of Sweepstakes racing, that organization shall be fined the amount of \$50.00.

\subsection{Warning Signs}

Warning signs shall be placed at several locations near the buggy course while Sweepstakes races are in progress, to warn vehicular traffic when this traffic tries to approach the area of the buggy course. The signs shall indicate that the road ahead is, or may be, closed, and that there will be flaggers ahead to stop and/or redirect traffic. The warning signs shall be put in place no later than one hour before races are scheduled to start on each day of Sweepstakes racing. They shall be removed within 15 minutes of either the end of the races for that day, or the cancellation of the races for that day. The Safety Chairman, or anyone designated by that Chairman, shall determine if enough warning signs are in place in order to have a Sweepstakes race. In order to provide the maximum amount of protection to the Sweepstakes race participants, warning signs should be placed at all of the locations where barricades have been placed, plus the following locations:
\begin{itemize}
	\item On Margaret Morrison Street at its intersection with Forbes Avenue.
	\item On Schenley Drive at its intersection with Forbes Avenue, near the clubhouse for the Schenley Park Golf Course.
	\item On Circuit Road at its intersection with Serpentine Drive.
	\item On Panther Hollow Road at the southern end of Panther Hollow Bridge.
	\item On Schenley Drive between the Mary E. Schenley Memorial Fountain and the southwest comer of the Carnegie Museum building.
	\item On the driveway to the rear of Hamburg Hall at its intersection with Forbes Avenue.
\end{itemize}

At the discretion of the Sweepstakes Chairman, the responsibility of obtaining, storing, and placing warning signs in position for Sweepstakes races may be delegated to one organization. The organization charged with this responsibility may not have to provide sweepers, flaggers, or course marshals for each day of Sweepstakes racing, at the discretion of the Sweepstakes Chairman.

If the organization responsible for the warning signs fails to provide them or remove them for any day of Sweepstakes racing, that organization shall be fined the amount of \$50.00.

\subsection{Crowd Control Barriers}

Crowd control barriers should be used to restrain race spectators located along Hill 2 and Hill 5. These barriers can be ropes strung between stanchions, or even held by course marshals. The barriers should be placed along Hill 2 on each side of the buggy course, from the Hill 1-2 Transition Zone to the end of the lanes, and along Hill 5 on only the northern side of the course, from a point that is approximately 200 feet from the finish line to the finish line. On Hill 2 the barriers should be placed about 3 feet outside of the outermost edges of the lanes, and on Hill 5 they should be placed even with the northern curb of Frew Street. The barriers shall be put in place no later than 30 minutes before races are scheduled to start on each day of Sweepstakes racing. They shall be removed within 15 minutes of either the end of the races for that day, or the cancellation of the races for that day.

At the discretion of the Sweepstakes Chairman, the responsibility of obtaining, storing, and placing crowd control barriers in position for Sweepstakes races may be delegated to the same organization responsible for placing warning signs in position for the Sweepstakes races. The organization charged with this responsibility may not have to provide sweepers, flaggers, or course marshals for each day of Sweepstakes racing, at the discretion of the Sweepstakes Chairman.

If the organization responsible for the crowd control barriers fails to provide them or remove them for any day of Sweepstakes racing, that organization shall be fined the amount of \$50.00.

\subsection{Course Marshals}

Each organization shall provide at least two course marshals for each scheduled day of Sweepstakes racing, to help control the race spectators on or near the buggy course. These course marshals must be available from a time that is one hour before the races are scheduled to start, until the races are finished for that day. Each organization shall equip at least two of the course marshals which it provides with reflective vests and flags which must be used while they are acting as marshals. Additional marshals must be equipped with some sort of brightly colored distinguishing attire, such as vests, hats, arm-bands, etc. so that they may be easily recognized by the race participants and spectators. The Sweepstakes Chairman, or anyone designated by that Chairman, shall determine when and where the course marshals must report for duty.

Any organization that fails to provide the required number of properly equipped course marshals for any day of Sweepstakes racing, shall be fined the amount of \$25.00 for each missing or improperly equipped course marshal.

At the discretion of the Sweepstakes Chairman, organizations charged with other responsibilities, such as hay bales, no-parking signs, barricades, warning signs, etc., may not have to provide course marshals for each day of Sweepstakes racing.

\subsection{Course Communications}

Sweepstakes races shall only be held when adequate radio communication equipment is available to provide voice communications around the buggy course for automobile traffic control, buggy traffic control, and emergency situation assistance. Radio communication equipment, and the personnel to operate it, are usually available through the Carnegie Mellon University Radio Club. Any personnel helping to provide radio communications should not be responsible for making decisions concerning what happens during the Sweepstakes races, but instead should be providing information to the Sweepstakes Chairman and his or her assistants, in order that they may make any necessary decisions.

\subsection{Traffic Control}
Control of vehicular traffic on the buggy course on each day of Sweepstakes racing will be handled by the City of Pittsburgh police officers, usually from the Park Police Department, who are hired by the Sweepstakes Committee and Carnegie Mellon University to provide police protection during all of the Sweepstakes races. These officers, with the assistance of the flaggers provided by the participating organizations, will stop vehicular traffic from entering the buggy course while the races are in progress. If possible, the buggy course will be completely closed to vehicular traffic from the time that the races begin each day, until all of the races are finished for that day. If necessary, the officers will open the buggy course to traffic one or more times during the course of the day's races to relieve traffic buildup on the streets around the buggy course. In this event, adequate notice must be given to the Sweepstakes Chairman and the official Starter of the races so that no race heat is started while there is vehicular traffic on the buggy course.

\subsection{Lead Car}

During each Sweepstakes race a motor vehicle known as the lead car shall drive around the buggy course ahead of the leading buggy in that race. The lead car should stay several hundred feet in front of the leading buggy at all times so as not to interfere with that buggy during the race. The Head Judge and the Sweepstakes Chairman shall ride in the rear portion of the lead car so that they may observe each race from that vantage point. Other people involved with the operation of the Sweepstakes races may ride in the lead car at the discretion of the Sweepstakes Chairman. The Sweepstakes Chairman shall be responsible for giving verbal instructions to the driver of the lead car during each race, so that the car may be kept at a safe and consistent (from race to race) distance in front of the leading buggy in that race.

The vehicle used as the lead car should be either a convertible top car or some sort of pick-up truck, in order to afford the best view of the race possible to the officials riding in it. If this vehicle is a pick-up truck, its tail-gate must always be closed while the vehicle is moving in order to reduce the chances of anyone falling out of it. The vehicle used as the lead car shall be obtained by the Sweepstakes Chairman with the assistance of the Sweepstakes Advisor. The driver of the lead car shall be determined by the Sweepstakes Advisor.

In the event of an accident during a heat, the lead car should continue along the buggy course, as long as there is at least one buggy still continuing with the race. If all of the buggies in the heat come to a stop, the lead car should stop near the leading buggy in order to render any necessary assistance.

\subsection{Follow Car}

During each Sweepstakes race a motor vehicle known as the follow car shall drive around the buggy course behind the trailing buggy in that race. The follow car should stay approximately one hundred feet behind the trailing buggy at all times so as not to interfere with that buggy during the race. The Assistant Head Judge and one representative of each entry in the heat underway shall ride in the follow car so that they may observe that heat from that vantage point and provide any assistance that may be necessary during that heat. Other people involved with the Sweepstakes races may ride in the follow car at the discretion of the Sweepstakes Chairman.

The vehicle used as the follow car should be either a convertible top car or some sort of pick-up truck, in order to afford the best view of the race possible to the people riding in it. If this vehicle is a pickup truck, its tail gate must always be closed while the vehicle is moving in order to reduce the chances of anyone falling out of it. It is also advisable to have a framework type of structure extending up from the sides of the pick-up truck bed, in order to help prevent anyone from falling out over the sides of the truck bed. The vehicle used as the follow car shall be obtained by the Sweepstakes Chairman with the assistance of the Sweepstakes Advisor. The driver of the follow car shall be determined by the Sweepstakes Advisor.

In the event of an accident during a heat, the follow car should be stopped near the scene of the accident in order that the people in the follow car may render assistance to those involved in the accident. The members of the racing organizations riding in the follow car MUST have any tools or devices that are necessary to quickly remove any of their drivers from their buggies. If an organization fails to have these tools or devices in the car following the buggies during any race, that organization shall be fined the amount of \$25.00 and the buggy in question shall be disqualified from that race.

\subsection{Signal Flaggers}

Each organization participating in the Sweepstakes races must provide a signal flagger for each of its entry's buggy drivers. These required signal flaggers shall be known as chute flaggers. Chute flaggers should provide a signal to the buggy drivers so that the drivers know when to start the right hand turn from Schenley Drive onto Frew Street. Chute flaggers should usually be positioned on the southern curb of Schenley Drive, just east of the intersection of Frew Street. Chute flaggers are not permitted to be on the street portion of Schenley Drive during any Sweepstakes race, unless they receive specific approval from the Safety Chairman. In general, if these flaggers need to position their flags more than an arms length away from the curb, their signal flags should be attached to extension poles so that they may hold these flags out over the street while still standing on the curb.

Each organization's chute flagger must be able to provide an alternate signal to their buggy drivers (such as waving the flag in a particular manner or using different colored flags) which will indicate to those drivers that there is a problem farther ahead and that the driver should start slowing down in a controlled manner.

People providing course communications (usually the Carnegie Mellon University Radio Club) may also signal buggy drivers that there is a problem farther ahead. This will be accomplished by waving a YELLOW flag so that the buggy driver can see it. This yellow flag indicates that the buggy driver should start slowing down in a controlled manner so that he or she may stop as soon as it is possible to do so safely. No organization may use a yellow colored flag for any purpose other than to indicate a problem ahead.

Each organization's chute flagger must have adequate experience relative to chute flagging. Each chute flagger should talk to the drivers that he or she will be flagging for, in order to determine where those drivers would like the flag to be placed. If possible each chute flagger should walk the buggy course with those drivers before the Sweepstakes races start each day that they will be flagging.

If a chute flagger is considered to be acting in an unsafe or inexperienced manner, the Safety Chairman, or anyone designated by that Chairman, may require that flagger to be replaced with another more experienced flagger.

If an organization fails to provide a chute flagger for any of its buggy drivers during any Sweepstakes race, that organization shall be fined the amount of \$25.00 each time one of its buggies attempts to make the turn from Schenley Drive onto Frew Street without a chute flagger.

Each organization participating in the Sweepstakes races may also provide additional signal flaggers for its drivers, in order to provide them with additional information about the buggy course or about the race that they are competing in, such as where the other buggies in that heat are at that time. These flaggers may be located anywhere around the buggy course,such as just after Hill 2 or near the entrance to Phipps Conservatory. These flaggers are not permitted to stand anywhere on the course whenever a race is in progress, unless they receive specific approval from the Safety Chairman.

Each organization is encouraged to provide these additional flaggers to help their drivers during their heats. They are also encouraged to discuss flagging strategies with the other entries in their heats, so as not to confuse each others drivers.

\subsection{Accidents}

If an accident occurs during the Sweepstakes races any medical personnel in attendance will be dispatched to the scene of that accident as quickly as possible. They alone will have the responsibility of determining the condition of any victims and of providing any needed first aid.

Before the medical personnel arrive at the scene of an accident, common sense should be observed. The victim should not be moved or disturbed in any way unless there is a greater danger in not doing so. If a buggy is involved in an accident and comes to rest in a position where it might be impacted by another buggy in the same heat, that buggy should be immediately moved, as quickly and as gently as possible, to a position of safety.

If a buggy needs to be immediately moved from a dangerous position after an accident, anyone near it may do so, even if they are not members of the buggy's sponsoring organization. If there is no immediate danger however, the buggy and its driver should not be moved or disturbed until the medical personnel arrive and assess the situation. In general, a buggy driver should not be removed from a buggy until medical personnel advise that it is wise to do so. In ABSOLUTELY ABSOLUTELY NO CASE should an unconscious driver be removed from a buggy without the supervision of medical personnel.

\subsection{Medical Personnel}

Medical personnel should be available at all times while the Sweepstakes races are underway. off duty paramedics from the City of Pittsburgh can usually be obtained to provide any necessary medical assistance for each day of racing. Arrangements to have paramedics available during the races should be made by the Sweepstakes Chairman, approximately six weeks before the preliminary races are scheduled to take place, through the office of the Sweepstakes Advisor.

If an active Emergency Medical Service exists on the Carnegie Mellon University campus, members of that organization may also be obtained to help provide medical assistance during the races. Arrangements to have members of a campus Emergency Medical Service available during the races should be made by the Sweepstakes Chairman, approximately six weeks before the preliminary races are scheduled to take place, with the assistance of the Sweepstakes Advisor.

\subsection{Clean-Up}

After each day of Sweepstakes racing has been completed, all debris on the buggy course and on the sidewalks around the course must be cleaned up and disposed of properly. Special care must be taken to ensure that all debris left by any race participants is removed, such as empty food and beverage containers, duct tape, buggy preparation materials, hay that has fallen off hay bales, No-Parking signs that have been misplaced, etc.

\subsection{officials}

All of the Sweepstakes race officials should be given some sort of distinguishing attire to wear during the races, so that they may be easily recognized by the race participants and race spectators. This attire should be brightly colored so that it is readily visible. It may consist of armbands, hats, shirts, vests, or any other type of distinguishing attire. This attire shall be provided by the Sweepstakes Chairman in conjunction with the Sweepstakes Advisor.

The following Sweepstakes officials shall wear this distinguishing attire: the Sweepstakes Advisor, the Sweepstakes Chairman, the Assistant Chairman, the Safety Chairman, all assistants to the above people, the Starter,the Head Judge, the Assistant Head Judge, all course judges, all timers,and all course marshals.

\subsection{Starter}

A person shall be appointed by the Sweepstakes Chairman, with the approval of the Sweepstakes Advisor, to be the official Starter of all of the Sweepstakes races. The Starter shall be located near the starting line of the buggy course before and during all of the races, usually just to the right of Lane 1 when looking south from the starting line. The Starter shall announce the time remaining until the start of each heat before each heat and shall announce any delays or holds in the countdown to each heat. The Starter shall use a traditional starting gun, or other similar device,to indicate the start of each heat. Two starting guns should always be available in the event that the first one used to start a heat misfires.

The Starter shall also act as a judge for the race by observing the race as far up Hills 1 and 2 as he or she can see it, watching for fouls by or interference between the competitors.

\subsection{Judges}

Judges shall be utilized to observe the Sweepstakes races and to point out and rule on possible violations of the rules and regulations by the race participants.

\subsubsection{Head Judge}

A person shall be appointed by the Sweepstakes Chairman, with the approval of the Sweepstakes Advisor, to be the Head Judge of the Sweepstakes races. The duties of the Head Judge shall be as follows:

The Head Judge shall observe each heat while riding in the rear of the lead car, watching for fouls by or interference between the entries.

The Head Judge shall hear all protests and appeals by any entry or organization and gather all information available from other judges, officials,or race participants pertinent to those protests and appeals. The Head Judge may use a portable tape recorder in order to record verbal protests and appeals, for later review and consideration.

The Head Judge shall render the final decisions concerning all protests and appeals based on any information that he or she has gathered relative to those protests and appeals, and on his or her interpretation of these Rules, Regulations, and Procedures.

Approximately 30 minutes before the Sweepstakes races begin each day, the Head Judge shall assign the available course judges to various positions around the buggy course such that all portions of the races are observable by at least one course judge. The Head Judge should instruct the course judges concerning what to look for during the races and should provide each course judge with a copy of the race rules and regulations which are pertinent to what they will be observing. The recommended positions for the course judges around the buggy course are as follows:
\begin{itemize}
	\item One judge on Hill 1 on the Lane 3 side of the course, halfway between the starting line and the Hill 1-2 Transition Zone.
	\item Two judges on Tech Street at the Hill 1-2 Transition Zone, one on each side of the course.
	\item Two judges on Schenley Drive just before the end of the lanes, one on each side of the course.
	\item One judge on Schenley Drive near the Westinghouse Memorial Pond.
	\item One judge on Schenley Drive near the end of the first transition on the buggy course.
	\item One judge on Schenley Drive near the Panther Hollow Bridge.
	\item One judge on Schenley Drive in front of the entrance to Phipps Conservatory, near the end of the second transition on the buggy course.
	\item One judge up on the base of the Edward Manning Bigelow monument in the middle of Schenley Drive near Phipps Conservatory.
	\item One judge on Schenley Drive near the chute flaggers at the beginning of the turn onto Frew Street .
	\item Two judges at the intersection of Schenley Drive and Frew Street, one on each side of the course.
	\item One judge on Frew Street, halfway between Schenley Drive and Scaife Hall.
	\item One judge on Frew Street near the driveway beside Scaife Hall.
	\item One judge on Frew Street, halfway between the driveway beside Scaife Hall and the Hill 3-4 Transition Zone.
	\item Two judges on Frew Street at the Hill 3-4 Transition Zone, one on each side of the course.
	\item One judge on Frew Street, halfway between the Hill 3-4 Transition Zone and the Hill 4-5 Transition Zone.
	\item Two judges on Frew Street at the Hill 4-5 Transition Zone, one on each side of the course.
	\item One judge on Frew Street, halfway between the Hill 4-5 Transition Zone and the finish line.
	\item Two judges on Frew Street at the finish line, one on each side of the course.
\end{itemize}

\subsubsection{Assistant Head Judge}

A person shall be appointed by the Sweepstakes Chairman, with the approval of the Sweepstakes Advisor, to be the Assistant Head Judge of the Sweepstakes races. The duties of the Assistant Head Judge shall be as follows:
\begin{itemize}
	\item The Assistant Head Judge shall observe each race while riding in the follow car, watching for fouls by or interference between the competitors.
	\item The Assistant Head Judge shall ensure that only authorized people are permitted to ride in the follow car during each of the Sweepstakes races.
	\item The Assistant Head Judge shall assist the Head Judge with any of his or her duties, when so requested.
\end{itemize}

\subsubsection{Course Judges}

Each organization participating in the Sweepstakes races must provide two course judges for each day of Sweepstakes racing. These judges should be alumni of the organizations providing them. Course judges may also be members of the Carnegie Mellon University faculty and staff, as selected by Sweepstakes Advisor.

The course judges must report to the Head Judge approximately 30 minutes before the races are scheduled to begin on each day of Sweepstakes racing, so that they can be assigned to positions around the buggy course from where they should observe the Sweepstakes races. No course judge shall be permitted to provide information or comments on any heat in which an entry from the organization which provided that judge is competing. The duties of each course judge shall be as follows:
\begin{itemize}
	\item Each course judge shall watch each Sweepstakes race from the location assigned to that course judge by the Head Judge.
	\item Each course judge shall watch for fouls by or interference between the competitors in each of the Sweepstakes races that he or she observes.
	\item Each course judge shall provide to the Head Judge, when so requested, any and all information that they may have, that might be pertinent to any alleged fouls or incidences of interference that may have occurred during any of the Sweepstakes races that they observed.
\end{itemize}

\subsection{Timers}

Timers shall be utilized to measure the time required for each entry's buggy to travel from the starting line to the finish line of the buggy course during the Sweepstakes races. The timers shall be appointed by the Sweepstakes Advisor. A minimum of two timers shall be required for each entry in a heat. The timers shall use stop watches or other suitable measuring devices to determine the time taken by each buggy to travel the buggy course. Sweepstakes Advisor shall appoint one timer as the Head Timer. The Head Timer will coordinate the efforts of all of the other timers.

The timers should be located on the northern sidewalk of Frew Street, directly in line with the finish line of the buggy course. They should be positioned on a platform several feet above the sidewalk so that they have a better view of the finish line of the race.

The timers will usually use the sound of the starter's gun, as broadcast over the Carnegie Mellon University radio station (WRCT), as a signal to start their timing devices at the beginning of a race. If the radio station is not broadcasting the start of the race, another method of starting the timing devices must be devised.

The timers shall stop their timing devices when the nose of the buggy that they are timing reaches the finish line of the buggy course. The finishing time for each entry shall be the average of all of the times determined by the timers for that entry. All finishing times shall be announced by the Head Timer, whether or not they are subsequently invalidated because of a disqualification.

The Head Timer will be responsible for recording official and/or unofficial finishing times for all entries in all Sweepstakes races that take place on each day of racing. These times shall be recorded on Sweepstakes Race Timing Forms. Along with the finishing times, notations concerning disqualifications, accidents, protests, appeals, brake tests, etc. for each entry shall also be recorded on the Sweepstakes Race Timing Forms by the Head Timer. At the end of each day of racing the Head Timer shall present all of the timing forms for that day's races to the Sweepstakes Chairman.

At the discretion of the Sweepstakes Advisor, alternate methods of automatic or semiautomatic timing of the race entrants may be employed, provided that these timing methods can be shown to be as accurate or more accurate than the usual timing method. If any automatic or semi-automatic timing methods are used, the manual timing method described above should also be used as a back-up system

\section{Race Schedule}

\subsection{Starting Times}

The time between the start of one scheduled heat and the start of the next scheduled heat shall usually be 8 minutes during the preliminary races, the alumni/exhibition races, and the rerun races, and 15 minutes during the finals races. The actual time intervals between heats shall be determined by the Sweepstakes Chairman, based on the number of heats to be run and on the amount of time available in which to run them. The actual time intervals to be used for the different races shall be announced to the participating organizations by the Sweepstakes Chairman sometime before the first scheduled day of racing.

The intervals between heats shall be timed by the Starter. Before each scheduled heat starts the Starter shall announce the time remaining to the start of that heat. These announcements shall usually be made when there are 10 minutes (finals races only), five minutes, two minutes, one minute, 30 seconds, and 15 seconds remaining before the start of the next scheduled heat. The last 10 seconds before the start of each heat shall be counted off by the Starter, in a manner such that all competitors located near the starting line are able to hear that count. After the Starter's countdown reaches the count of ``ONE,'' the final warnings to the competitors before the heat starts shall be -- ``READY, -- SET,'' after which the Starter shall fire the starting gun to start the heat.

Each entry must be ready to start its scheduled heat on time. No requests for extensions to delay the scheduled start of any heat shall be granted. Any entry not in position at the starting line when the Starter indicates that there are five seconds left in the countdown until the start of that entry's heat, will not be permitted to start that heat and will not be eligible for a rerun.

\subsection{Delays and Holds}

Delays and holds shall be handled as follows:

If a delay or hold of the countdown to the start of a heat occurs immediately after the end of the previous heat, when the delay or hold is over, the countdown to the start of that next heat shall resume at a time that is between four and seven minutes before the start of that next heat if that heat is a preliminary race, an alumni/exhibition race, or a rerun race. For a finals race, the countdown to the next heat shall resume in between seven and 12 minutes.

If a delay or hold of the countdown to the start of a heat during the preliminary races, the alumni/exhibition races, or the rerun races occurs after there is less than five minutes remaining to the start of that heat, the countdown may be held for up to one minute, and then resumed at the time at which it was held. If the delay or hold is longer than one minute, when that delay or hold is over, the countdown to the start of that next heat shall resume at a time that is between four and seven minutes before the start of that next heat.

If a delay or hold of the countdown to the start of a heat during the finals races occurs after there is less than eight minutes remaining to the start of that heat, the countdown may be held for up to one minute, and then resumed at the time at which it was held. If the delay or hold is longer than one minute, when that delay or hold is over, the countdown to the start of that next heat shall resume at a time that is between seven and 12 minutes before the start of that next heat.

Whenever a delay or hold occurs, the Starter will announce that the countdown has stopped and how much time will remain when the count resumes.

Whenever a delay or hold ends, the Starter will announce that the countdown has resumed and how much time remains until the start of the next heat.

\subsection{Starting Positions}

Starting positions in the preliminary races, rerun races, and finals races in each class of competition shall be determined separately, but by the same method.

\subsection{Preliminary Races}

\subsubsection{Heats}

The preliminary races shall be run in heats, with a maximum of three entries in each heat. The number of heats in each class of competition shall be determined by dividing the total number of entries by three and rounding up to the nearest whole number.

\subsection{Seeding}

1. Entries for each organization shall be identified according to their letter designations, not by the buggy used by each entry. An organization’s “A” entry should be its fastest entry, its “B” entry should be its next fastest entry, and so on.

2. The order of seeding shall be based on the weighted average of an entry’s last three year’s finishing times in the preliminary races, with the time for three years ago being multiplied by one, the time from two years ago being multiplied by two, and the time from last year being multiplied by three. These three numbers would be added together and then divided by six to determine that entry’s seeding time. For example, for race day 2002 the entry’s race day 2001 preliminary time would be multiplied by three and added to the entry’s race day 2000 preliminary time multiplied by two, and then finally added to the entry’s race day 1999 preliminary time. This total would then be divided by six to obtain that entry’s seeding time. The entry with the fastest seeding time will then be placed in the last heat; the entry with the second fastest seeding time will then be placed in the second last heat. This process will be repeated until every heat has one team in it, then the process will continue by going back to the last heat of the day and assigning another team to that heat. By continuing this process each heat will end up with a minimum of two teams and a maximum of three teams. The next step in the seeding process will be the lane selection phase. For every heat the team with the best seeding time will choose their lane first, then the team with the second best seeding time will choose one of the two available lanes, and finally the team with the third best seeding time will be assigned to the remaining available lane.

3. The finishing times used for seeding purposes need not be official times. If an entry finishes the race, but is subsequently disqualified (for example by failing the brake test), its finishing time shall still be used for seeding purposes. In addition if an entry does not finish its preliminary race, but is subsequently granted an official or unofficial reroll, the time from that reroll shall be used for seeding purposes. If an entry does not finish its preliminary race and does not have an official or unofficial preliminary time for one or more of the three most recent years, that year shall be thrown out of that entry’s seeding time. That entry’s seeding time will be computed, using the method stated above, however the year that the entry did not finish will not be entered into the average, and the total time will be divided by the appropriate number.

4. If for any reason the Sweepstakes Executive Committee feels that a heat contains an unsafe grouping of buggies, the Sweepstakes Executive Committee will discuss the matter with the University’s staff Sweepstakes Advisor, and at their approval make the changes necessary to provide a safe race.

\subsubsection{Heat Selection}

The purpose of the heat selection procedure is to help ensure that each heat of the preliminary races has entries that are as far apart, with respect to probable finishing times, as is possible. This should help to reduce the chances of an accident during any of these heats. After all of the entries in each class of competition have been seeded, they shall select heats for the preliminary races using the following procedure:

Heat selections shall be made at a meeting of the Sweepstakes Committee. The heat selection procedure shall be supervised by the Sweepstakes Chairman, or anyone designated by that Chairman. This supervisor shall maintain order during the selection process and shall keep records of all of the selections that are made. If any entered organization is not represented at this meeting, the supervisor shall be empowered to make any required selections for the unrepresented organization.

The seeded entries shall be divided into three groups. If the number of preliminary heats scheduled is $n$, then the first group will have $n$ entries, starting with the highest seeded entry and ending with the $n^{th}$ seeded entry. The second group will also have $n$ entries, starting with the $(n+1)^{th}$ seeded entry and ending with the $(n+n)^{th}$ seeded entry. The third group will have all of the remaining seeded entries.

Representatives of each of the entries in the first group shall select the heats that their entries shall compete in using the following procedure: a) The highest seeded entry in the first group shall select the heat that it wants to compete in during the preliminary races. b) The next highest seeded entry in the first group shall select the heat that it wants to compete in during the preliminary races, with the provision that it cannot select a heat which already has one or more entries from the first group in it,unless all of those other first group entries grant it permission to do so.) The preceding step shall be repeated until all of the seeded entries in the first group have selected heats.

Representatives of each of the entries in the second group shall select the heats that their entries shall compete in using the following procedure: a) The highest seeded entry in the second group shall select the heat that it wants to compete in during the preliminary races. b) The next highest seeded entry in the second group shall select the heat that it wants to compete in during the preliminary races, with the provision that it cannot select a heat which already has one or more entries from the second group in it, unless all of those other second group entries grant it permission to do so. c) The preceding step shall be repeated until all of the seeded entries in the second group have selected heats.

Representatives of each of the entries in the third group shall select the heats that their entries shall compete in using the following procedure:

The highest seeded entry in the third group shall select the heat that it wants to compete in during the preliminary races.

The next highest seeded entry in the third group shall select the heat that it wants to compete in during the preliminary races.

The preceding step shall be repeated until all of the seeded entries in the third group have selected heats.

Heat selections shall be made by a date that is no later than the second day of Truck Weekend (the weekend just prior to the date scheduled for the preliminary races). Within two days of the day that heats and lanes are selected, the Sweepstakes Chairman shall distribute a schedule of all of the heat and lane selections for the preliminary races, to each of the participating organizations.

\subsubsection{Lane Selection}

After heats have been selected for all entries, the lane that each entry will occupy on Hills I and 2 shall be chosen. The entry seeded first shall choose first. The entry seeded second shall choose second. The entry seeded third shall choose third, and so on until all of the entries have chosen lanes. No entry may choose a lane that is already occupied by a higher seeded entry.

Lane selection shall take place at the same meeting at which heats are selected.

\subsubsection{Heat and Lane Switching}

Any entry may switch their heat and/or lane selections with any other entry provided that they comply with the following:

All of the entries in each heat in which a change is to take place must approve the change.

The Sweepstakes Chairman and the Safety Chairman must approve the change.

The change must be made within 24 hours of the meeting at which the heats and lanes were selected.

\subsubsection{Entry Withdrawals}

If an entry withdraws from the Sweepstakes competition before the preliminary races begin, the heat to which that entry was assigned will not have three entries competing in it. If more than one heat has less than three entries competing in it, the Sweepstakes Chairman may reassign entries in heats with less than three entries to different heats if that reassignment will reduce the total number of heats in the preliminary races. A reassignment may only be made if the Sweepstakes Chairman and the Safety Chairman determine that that reassignment would not increase the likelihood of an accident in any heat, and if the reassignment does not cause any reassigned entry to start the race in a lane other than the one that they were originally scheduled to be in, unless that entry agrees to the different lane assignment. No entry of an organization may be reassigned to a heat in a class of race competition that is less than two heats away from any heat in that same class of race competition in which another of that organization's entries is scheduled to compete.

Any organization withdrawing an entry shall forfeit all entry fees for that entry. If an organization withdraws an entry after heats and lanes have been selected for the preliminary races, they shall only be permitted to enter as many entries in the following year's competition as they had competing in the year that they made the late withdrawal. Each organization shall always be permitted to have at least one entry regardless of the number of late withdrawals they had during the previous year's competition.

Exceptions to this rule may be made if an entry is withdrawn involuntarily, such as if the entry's buggy is damaged beyond repair during a practice session. However, the evidence that the withdrawal was involuntary must be very strong and must be agreed upon by the Sweepstakes Chairman, the Assistant Sweepstakes Chairman, and the Safety Chairman. Failure of an entry's driver to successfully complete a passing test or to complete the minimum required number of rolls down the buggy course during freeroll practice sessions shall NOT be considered as valid reasons for the involuntary withdrawal of that entry. Failure of an entry's buggy to complete the minimum required number of rolls down the buggy course during freeroll practice sessions shall NOT be considered as a valid reason for the involuntary withdrawal of that entry.

\subsection{Alumni/Exhibition Races}

Alumni/exhibition races are not considered to be official Sweepstakes races. If alumni/exhibition races are held, all safety related rules, regulations, and requirements applicable to the Sweepstakes races shall also apply to the alumni/exhibition races. This includes the requirement for a drop brake test for each participating buggy after each heat of the alumni/exhibition races, and all of the requirements for drivers. (This also includes the requirement that all drivers be currently enrolled, Activities Fee paying,full-time students of Carnegie Mellon University.) Graduate students may participate in alumni/exhibition races.

\subsubsection{Starting Positions}

If alumni races are scheduled, heat and lane assignments shall be made by the Sweepstakes Chairman, or anyone designated by that Chairman. The procedure used to assign heats and lanes for the alumni/exhibition races shall be determined by the Sweepstakes Chairman.

\subsection{Rerun Races}
Any rerun races that are held shall be run in heats, with a maximum of three entries in each heat. The minimum number of rerun heats in each class of competition shall be determined by dividing the number of rerun entries by three and rounding up to the nearest whole number.

\subsubsection{Heat Assignments}

Heat assignments for rerun races shall be made by the Sweepstakes Chairman and the Safety Chairman. The purpose of the heat assignments is to help ensure that each heat of the rerun races has entries that are as far apart with respect to probable finishing times, as is possible, in order to reduce the chances of an accident during any of the rerun heats. The seeding order used to assign heats in the preliminary races shall be used as a guideline in assigning entries to heats for rerun races. Rerun race heats may be run with less than three entries each, if the Sweepstakes Chairman and the Safety Chairman, determine that the safety of those heats might be substantially increased by doing so. The method of assigning entries to heats for the rerun races shall be determined by the Sweepstakes Chairman and the Safety Chairman. Heat assignments shall be completed as soon as is practical after the preliminary races have ended.

No rerun heat shall have more than one entry from any single organization in it.

\subsubsection{Lane Selection}
After all rerun entries have been assigned to a heat, the lane that each entry will occupy on Hills 1 and 2 shall be chosen. The entry seeded highest before the preliminary heats shall choose first. The entry seeded next highest before the preliminary heats shall choose second, and so on until all of the rerun entries have chosen lanes. No rerun entry may choose a lane that is already occupied by a higher seeded rerun entry.

Lane selection shall be completed as soon as is practical after heat assignments have been made.

\subsection{Finals Races}

\subsubsection{Rankings}

After the preliminary races are finished, and before any rerun races have taken place, all of the entries that finished a preliminary race and were not disqualified or granted a rerun, shall be ranked according to their finishing times in those preliminary races for each class of competition. The entry with the fastest time shall be ranked first, the entry with the second fastest time shall be ranked second, and so on to the entry with the slowest time, which shall be ranked last.

The six highest ranked entries from the women's preliminary races shall be eligible to compete in the women's finals races.

The ten highest ranked entries from the men's preliminary races shall be eligible to compete in the men's finals races.

In the event of ties, one or more extra entries may be eligible to compete in the women's or men's finals races. The rankings of tied entries shall be determined by the Sweepstakes Chairman, or anyone designated by that Chairman.

\subsubsection{Heats}

The finals races shall be run in heats, with a maximum of two entries in each heat. The number of finals heats in each class of competition shall be determined by dividing the number of entries eligible to compete in the finals races by two and rounding up to the nearest whole number.

\subsubsection{Heat Assignment}

The entries eligible to compete in the finals races will be assigned to heats for those races, by the Sweepstakes Chairman, or anyone designated by that Chairman. The purpose of the heat assignments is to help ensure that each heat of the finals races has entries that are as far apart with respect to probable finishing times as is possible, in order to reduce the chance of an accident during any of the finals heats. Heat assignments shall be completed as soon as is practical after the preliminary races have ended.

Heat assignments for each class of competition shall be made as follows:

The entry ranked first after the preliminary races shall be assigned to the last finals heat, the entry ranked second shall be assigned to the next to last finals heat, the entry ranked third shall be assigned to the third to last finals heat, and so on, until one entry has been assigned to each finals heat.

After all of the finals heats have one entry assigned to each of them, the highest ranked entry of the remaining entries shall be assigned to the last finals heat, the next highest ranked of the remaining entries shall be assigned to the next to last finals heat, and so on, until all of the entries eligible to compete in the finals races have been assigned to finals heats.

In the event that this procedure assigns more than one entry from any organization into the same finals heat, the heat assignments shall not be changed.

\subsubsection{Lane Selection}

After each entry eligible to compete in the finals races has been assigned to a finals heat, the lane that each entry will occupy on Hills 1 and 2 shall be chosen. Lanes for each finals heat shall be selected by each entry, with the entry having the higher ranking after the preliminary races choosing first. No finals entry may choose a lane that is already occupied by a higher ranked finals entry.

Lane selection shall be completed as soon as is practical after heat assignments have been made.

\section{Race Rules, Regulations, and Requirements}

If the Head Judge determines that any entry has violated any of the Race Rules, Regulations, and Requirements listed in this section of this document, that entry shall be disqualified from the race competition. The Head Judge does not require a formal protest from a competitor in order to disqualify an entry from the races. Any infraction of the race rules observed by the Head Judge, or brought to his or her attention by any of the other judges or officials, can be grounds for disqualifying or penalizing the offending entry.

\subsection{Regulations and Requirements}

\subsubsection{Drivers}

No driver shall be permitted to participate in a Sweepstakes race unless he or she satisfies all of the following requirements:

He or she has walked the buggy course earlier that same day with the drivers of the other buggies that are scheduled to be in his or her heat. Each driver must notify the Assistant Sweepstakes Chairman, or anyone designated by that Chairman, immediately before or after he or she walks the course. If a driver is scheduled to race in more than one heat on the same day, that driver must walk the course once for each of the heats that he or she is scheduled to race in, with the other drivers of each of those heats. Drivers are only permitted to walk the course BEFORE any races are held on any day of Sweepstakes racing.

He or she has previously successfully completed a braking capability test in the buggy that he or she will be driving in that Sweepstakes race.

He or she has previously successfully completed a field of vision test in the buggy that he or she will be driving in that Sweepstakes race.

If a driver raced in a Sweepstakes race the previous year (an ``OLD'' driver), he or she must have a minimum total of ten freerolls for the school year in which they are racing. A minimum of five of those ten freerolls must be completed during the Spring semester in which the race is to take place, with one being the pass test.

If a driver did not race in a Sweepstakes race during the previous school year (a ``NEW'' driver), he or she must have a minimum total of fifteen freerolls. A minimum of seven of those fifteen freerolls must be completed during the Spring semester in which the race is to take place, with one being a pass test.

All drivers, new and old, must have a minimum total of ten freerolls in any buggy they will be driving during races. (i.e. If an organization brings out t new buggy in the Spring semester, its raceday driver will still need a minimum of ten freerolls in that buggy in order to qualify to drive that buggy during races.)

In the case of very special circumstances, such as a long duration of inclement weather where no freeroll practices are held, an organization may seek permission to race a driver and buggy that do not meet these specific requirements. In such cases, the organization must submit an appeal in writing to the Sweepstakes Committee, explaining that the failure to meet the qualifying standards was through no fault of the driver of the organization. The Safety Chairman, with counsel of the Sweepstakes Chairman and Assistant Chairman, will evaluate the appeal and either approve or reject it. If an appeal is approved by the Safety Chairman, it must then be passed on the the Buggy Chairmen to be approved by a majority vote.

He or she has successfully completed a pass test, during a freeroll practice session that occurred during the same school semester in which that race is scheduled, in the buggy that he or she will be driving in that race.

He or she has demonstrated by his or her performance at both freeroll and push practices that he or she is capable of driving a buggy in a safe and controlled manner.

\subsubsection{Buggies}

No buggy shall be permitted to participate in a Sweepstakes race unless it satisfies all of the following requirements:

It has successfully completed a safety inspection and been driven down the buggy course at least once, in any configuration that it will be in, during that race.

It has successfully completed a braking capability test with the driver who will be driving it during that race.

It has successfully completed a field of vision test with the driver who will be driving it during that race.

It has previously been driven down the buggy course at least five times during freeroll practice sessions.

It has been the passing buggy in a successfully completed pass test, during a freeroll practice session that occurred during the same school semester in which that race is scheduled.

\subsection{Rules}

\subsubsection{General}

No organization may use or attempt to use any buggy preparation area that has been assigned to another organization on any day of Sweepstakes racing.

No organization's entry may use or attempt to use any lane that has been assigned to another organization's entry in any Sweepstakes race heat.

Each organization may allow only one of its representatives to ride in the follow car during each race in which one of its entries is competing.

If any member of any entry's team is substituted for by a member of that entry's alternate team, that substitution must be declared to the Sweepstakes Chairman before the start of the heat preceding the heat in which that entry is scheduled to compete.

\subsubsection{Entries}

In each class of competition, each organization's entries must compete in the heats designated for those entries during the heat selection process as follows:

No organization may permit its fastest entry to compete in any heat of the Sweepstakes races other than the heat designated as the heat for that organization's ``A'' entry.

No organization may permit its second fastest entry to compete in any heat of the Sweepstakes races other than the heat designated as the heat for that organization's ``B'' entry.

No organization may permit its third fastest entry to compete in any heat of the Sweepstakes races other than the heat designated as the heat for that organization's ``C'' entry.

No organization may permit its fourth fastest entry to compete in any heat of the Sweepstakes races other than the heat designated as the heat for that organization's ``D'' entry.

Any entry which competes in a heat that is not the heat designated for that entry shall be disqualified from the race competition. Determination of whether or not an entry has violated this rule shall be made collectively by the Sweepstakes Advisor, the Sweepstakes Chairman, the Assistant Sweepstakes Chairman, the Safety Chairman, the Head Judge, the Assistant Head Judge, and anyone else designated by either the Sweepstakes Advisor or the Sweepstakes Chairman.

\subsubsection{Pushers}

No pusher may use any type of mechanical device with moving parts (such as roller skates or a skate board) which could cause that pusher to travel faster, while that pusher is pushing a buggy during a Sweepstakes race.

\subsubsection{Buggies}

The combined weight of a buggy and its driver may not change while that buggy is competing in any Sweepstakes race. Weight loss during a race shall be permitted only if the Head Judge rules that the weight loss was unintentional, the weight loss was not caused by a design failure of the buggy, and the weight loss did not interfere with any of the other entries in that heat.

The loss of any part of a buggy's shell, hatch, or cover while that buggy is competing in any Sweepstakes race shall be considered to be a design failure of that buggy, and will result in the disqualification of that buggy and its entry from the race competition, even if that loss did not interfere with any of the other entries in that heat.

The dimensions of a buggy, not including those of that buggy's pushbar, shall not change while that buggy is competing in a heat of the Sweepstakes races.

No buggy may have any changes made to it between the end of the Design Competition and the conclusion of any race that it competes in except for changes to its wheels, tires, bearings, and windscreens. Damaged parts of a buggy may be changed after the Design Competition and before a race,provided that the Safety Chairman is informed of the change, is permitted to view the damaged parts, and is given a detailed account of how the parts were damaged.

Each entry's driver must ride in or on that entry's buggy during the entire time that entry is competing in a Sweepstakes race heat. No other person is permitted in or on that buggy at that same time.

Each entry's buggy must take a drop brake test immediately after any race in which that entry has competed, whether that buggy completed that race or not. Failure to successfully complete this test will result in disqualification of that entry from the race in which it just competed and will make that entry ineligible for a rerun unless the Safety Chairman and the Head Judge determine that the buggy failed the drop brake test because of damage sustained due to interference that occurred during the race. Any buggy which fails the drop brake test after a race but is subsequently granted a rerun, must successfully complete a drop brake test BEFORE it will be permitted to take that rerun. These drop brake tests shall be administered by the Safety Chairman, or by someone designated by that Chairman. Each buggy shall only be given one chance to pass the drop brake test after a race unless any of the following occur:

The drop brake test is not properly administered, for example, if the test administrator tells the driver to apply the brakes too soon or too late. In this event, the test shall be administered again in the proper manner.

The driver unintentionally applies the brakes too soon. In this event the driver shall be given one and only one more chance to successfully complete the test. In the unlikely event that the buggy does not roll after it is initially released, or if it does not roll a distance of 30 feet after it is initially released, the test administrator shall pull the buggy in a forward direction, using a spring scale, or other suitable force measuring device, while the driver is actuating the buggy's brakes, in order to determine how much braking force the buggy's brakes are able to exert in that direction. The force used to pull the buggy shall be applied parallel to, and as close to the ground as is practicable. If the buggy's brakes are able to exert a minimum braking force of 25 pounds in the forward direction, and the driver can release and reapply that buggy's brakes once, that buggy shall be considered to have passed the drop brake test.

Any entry's buggy that is involved in any type of accident (such as spinning out, having parts fall off while it is rolling, hitting another buggy, being hit by another buggy, hitting another object, etc.) during any Sweepstakes race will not be permitted to race again until that entry's sponsoring organization submits, and has approved, an accident report concerning that accident. Accident reports must be submitted to the Safety Chairman or the Sweepstakes Chairman. Accident reports can only be approved by the Sweepstakes Chairman or the Safety Chairman. The form to be used for these reports can be obtained from the Sweepstakes Advisor through the Safety Chairman.

Any device that allows the driver in the buggy communication with individuals outside of the buggy and provides information to a driver that otherwise could not be known by the driver are illegal to use on Raceday rolls or when performing a pass test. These devices may be used during freeroll and push practice; however, use of these prohibited devices on raceday or when more than one buggy is rolling at the same time will result in immediate disqualification. Examples of the devices are headsets, walkie-talkies,telemetry systems, and video display units.

\subsubsection{Starting}

When the Starter signals the start of an entry's heat, that entry's buggy must comply with all of the following:
\begin{itemize}
	\item The nose of the buggy must be at or behind the starting line.
	\item The buggy must not be moving in the forward direction.
	\item The forward motion of the buggy must not be restrained in any way.
	\item When the Starter signals the start of an entry's heat, that entry's Hill 1 pusher must comply with all of the following:
	\item He or she must be touching that entry's buggy.
	\item He or she must have both of his or her feet on the ground.
	\item He or she must not be moving in the forward direction.
	\item He or she must not be using starting blocks, or any similar device.
\end{itemize}

After the starter announces that there are five seconds remaining to the start of a heat, nobody except the drivers and Hill 1 pushers of the entries in that heat, and the starter, may be within five feet of any of the buggies in that heat, until that heat begins.

If the nose of an entry's buggy in any heat is beyond the starting line when the Starter signals the start of that heat, that entry shall be considered to have false started. If an entry false starts more than two times at the start of any single heat, that entry shall be disqualified from that heat, shall not be permitted to run in that heat, and shall not be eligible for a rerun.

If an entry in a heat false starts, all of the buggies in that heat shall be brought back to the starting line and that heat shall be restarted. The Starter shall restart the countdown at his or her discretion. If the time required to restart the heat is greater than 1 minute and any of the entries in that heat which did not cause the false start so request,the start of that heat may be delayed until the time scheduled for the start of the heat after the current heat, and all of the remaining heats shall be moved back for the amount of time required for one heat to take place. If the delayed heat is the last scheduled heat of the day, it shall be rescheduled for a time that is between 5 and 12 minutes later than its originally scheduled starting time.

For example, if Heat 5 is delayed because of a false start, it shall be rescheduled for the time at which Heat 7 was originally scheduled to take place. Heat 6 would be run at its originally scheduled time. No heat would be run at the time originally scheduled for Heat 5. Heat 7 would be run at the time originally scheduled for Heat 8 and all subsequent heats would be moved back to the time originally scheduled for the heat after them. If Heat 5 was the last scheduled heat of the day, no heat would be run at the time originally scheduled for Heat 5 and Heat 5 would be run at a time that was between 5 and 12 minutes later than its original starting time, depending on how much time is still available for racing.

If the Starter's gun fails to fire when the Starter tries to signal the start of a heat, all of the buggies in that heat shall be brought back to the starting line and that heat shall be restarted. The Starter shall restart the countdown at his or her discretion. If the time required to restart the heat is greater than 1 minute and any of the entries in that heat so request, the start of that heat may be delayed until the time scheduled for the start of the heat after the current heat, just as might happen if a false start had occurred.

\subsubsection{Lanes}

A buggy is considered to be in a lane when no portion of that buggy extends beyond or above the innermost edges of the lines delineating that lane.

The judges shall disqualify an entry from the heat that it is competing in if all of the wheels of that entry's buggy are simultaneously out of its assigned lane at any point between the starting line and the end of that lane, even if that entry doesn't interfere with any other entry.

If some part of, but not all of, an entry's buggy is out of its assigned lane at any point between the starting line and the end of that lane,that entry shall be disqualified from the heat that it is competing in if it interferes with another entry in that heat, or if one of the other entries in that heat protests the lane violation (even if no interference is claimed.)

A pusher is considered to be in a lane when the greatest part of that pusher's body is on or above that lane and no part of that pusher is touching anything that is not considered to be in that lane also, except for the buggy which that pusher is pushing.

If some part of, but not all of, one of an entry's pushers is out of that entry's assigned lane at any point between the starting line and the end of that lane, that entry shall be disqualified from the heat that it is competing in if that pusher interferes with any other entry in that heat, and if the entry that was interfered with protests that interference.

\subsubsection{Pushing}

The position of an entry's pusher along the buggy course is determined by the location of the forwardmost part of the pusher's forwardmost foot, i.e. the foot that is farthest along the course from the starting line.

For example, if a pusher's forwardmost foot is on Hill 1, that pusher is considered to be on Hill 1. If a pusher's forwardmost foot is in the Hill 1-2 Transition zone, that pusher is considered to be in that Transition zone even if that pusher's rearmost foot is still on Hill 1.

Each entry's Hill 1 pusher may only touch that entry's buggy while he or she is either on Hill 1 or in the Hill 1-2 Transition Zone.

Each entry's Hill 2 pusher may only touch that entry's buggy while he or she is either in the Hill 1-2 Transition Zone or on Hill 2.

Each entry's Hill 3 pusher may only touch that entry's buggy while he or she is either on Hill 3 or in the Hill 3-4 Transition Zone.

Each entry's Hill 4 pusher may only touch that entry's buggy while he or she is either in the Hill 3-4 Transition Zone, on Hill 4, or in the Hill 4-5 Transition Zone.

Each entry's Hill 5 pusher may only touch that entry's buggy while he or she is in the Hill 4-5 Transition Zone, on Hill 5, or crossing the finish line.

Each entry's Hill 5 pusher must be in contact with that entry's buggy when the nose of that buggy crosses the finish line at the end of that entry's heat.

No organization may cause anyone or anything to pace a pusher of any of that organization's entries while that pusher is competing in a race. A person or device which is not the next pusher in sequence and which is moving near a buggy and moving in the same direction in which that buggy is moving, while in view of the pusher currently pushing that buggy, is considered to be a pacer.

While an entry's pusher is pushing that entry's buggy and after that pusher has finished pushing that entry's buggy, that pusher is entitled to run, walk, stand, or otherwise be, anywhere in the path which that entry's buggy has followed. A buggy's path is considered to be a lane that is 6 feet wide which is centered on the centerline of that buggy.

\subsubsection{Driving}

Each entry's buggy in each heat may not bump into or otherwise contact any other entry's buggy in that same heat at any time between the start and finish of that heat.

If during a heat, one entry's buggy passes another entry's buggy, the pass shall be considered to be complete whenever the rearmost part of the passing buggy (or that buggy's pusher, if it is being pushed at the time) is beyond the nose of the buggy being passed.

If during a heat, one entry's buggy (and pusher, if the buggy is being pushed at that time) tries to pass another entry's buggy (and pusher),the passing buggy (and pusher) has the primary responsibility of ensuring that the pass is completed without contact or any other type of foul between the buggies (or pushers). If contact or some other type of foul occurs between the buggies (or pushers) during an attempted pass, the judges shall determine which buggy (or pusher), if any, is at fault.

If during a heat, two entries' buggies (and pushers, if the buggies are being pushed at the time) are traveling beside each other and neither is clearly passing the other, both buggies (and pushers) have the responsibility of ensuring that no contact or other type of foul occurs between the buggies (and pushers). If contact or some other type of foul occurs between the buggies (or pushers), the judges shall determine which buggy (or pusher), if any, is at fault.

If during a heat, an entry's driver stops that entry's buggy because that driver considered that an accident was about to take place, that entry shall be eligible for a rerun provided that the following conditions are met:

The Head Judge determines that an accident was about to take place.

The accident that was about to occur was not due to any failure or foul on the part of the buggy that stopped.

The buggy that stopped is taken to the drop brake test area and given a drop brake test as soon as possible after the heat is finished.

If during a heat, an entry's buggy or pusher is interfered with or fouled, that buggy or pusher is not required to stop at that point in the race, in order to be eligible for a rerun.

\subsection{Protests and Appeals}

A protest may be filed by one entry against another entry if the protesting entry considers that it was fouled or interfered with by the protested entry during a Sweepstakes race.

A protest may be filed by any participating organization against any entry if the protesting organization considers that the protested entry has not complied with all of the rules and requirements of the Sweepstakes races.

An appeal may be filed by an entry if that entry considers that it was interfered with while it was competing in a Sweepstakes race by someone or something other than any of the other entries in that race.

Protests and appeals shall be filed as follows:

The Buggy Chairman of the protesting or appealing organization shall verbally inform the Sweepstakes Chairman and/or the Head Judge that he or she wishes to file a protest or appeal on behalf of his or her entry or organization. This verbal notification must be accomplished before the start of the next heat after the protested or appealed heat. If the protested or appealed heat is the last heat of the day, the notification must be accomplished within ten minutes after the end of that heat.

The Buggy Chairman of the protesting or appealing organization shall verbally describe the reason for the protest or appeal and any information pertinent to that protest or appeal to the Head Judge of the Sweepstakes races. If possible, protests and appeals should also be submitted in written form, using the same form that is used for accident reports, and which can be obtained from the Safety Chairman.

The Head Judge will confer with the Sweepstakes Chairman, the Assistant Sweepstakes Chairman, the Safety Chairman, the follow car judge, any of the course judges that could have witnessed the alleged incident, any of the drivers or pushers that were involved in the incident, and/or anyone else who might be able to provide information pertinent to the alleged incident, in order to determine as many facts about the incident as possible. (If video tapes of the incident are available, they may be used as an additional source of information. )

After reviewing all of the available information the Head Judge will render a decision concerning the protest or appeal and will inform the Sweepstakes Chairman of that decision. This decision should normally be made before the start of the second heat after the protested or appealed heat, but it may be delayed until all of the races are finished for that day at the discretion of the Head Judge, if circumstances warrant such a delay.

The Sweepstakes Chairman will confer with all of the affected entries and inform them of the judges' decision.

The Sweepstakes Chairman will announce the decision of the judges to the remaining race participants and to the public at large.

If a protest is filed due to an alleged foul or interference during a race by another entry, that protest shall be dispositioned as follows:

If the protest is disallowed, the results of the protested heat shall stand as they were.

If the protest is allowed and the protesting entry was interfered with or fouled, that entry will be granted a rerun.

If the protest is allowed, the entry which caused the interference or foul will be disqualified from the protested heat.

If the Head Judge rules that some type of interference or foul did occur but that no violation of the rules occurred, the protesting entry will be granted a rerun.

If a protest is filed due to an alleged violation of the rules and requirements of the Sweepstakes races, that protest shall be dispositioned as follows:

If the protest is disallowed, the protested entry shall not be penalized in any way.

If the protest is allowed the protested entry will be disqualified from the heat in which it competed.

If an appeal is filed due to an alleged interference during a race by someone or something other than any of the other entries in that race, that appeal shall be dispositioned as follows:

If the appeal is disallowed, the results of the appealed heat shall stand as they were.

If the appeal is allowed the appealing entry shall be granted a rerun.

\subsection{Reruns}

An entry may be granted a rerun by the Head Judge if all of the following occur:

The entry is either interfered with or fouled while it is competing in a Sweepstakes race.

The entry files a protest or an appeal in accordance with the procedure for filing protests and appeals .

The Head Judge rules that the protesting or appealing entry was interfered with or fouled.

Any entry which is granted a rerun has the option of taking the rerun or of allowing its results in the protested or appealed heat to stand as they are. If the entry chooses to take the rerun, the results of the rerun race shall then be that entry's official results for the protested or appealed heat.

If an entry is granted and accepts a rerun because of an incident that occurs during the preliminary races, that entry's rerun race shall be scheduled to take place on the same day that the finals races are scheduled to take place.

If an entry is granted and accepts a rerun because of an incident that occurs during the rerun races, that entry's rerun race shall be scheduled to take place after all of the women's finals races are finished, but before the men's finals races are started.

If an entry is granted and accepts a rerun because of an incident that occurs during the finals races, that entry's rerun race shall be scheduled to take place after all of the finals races are finished. If that entry competed in the last scheduled finals race, its rerun race shall be scheduled to start between seven and 12 minutes after the end of that last finals race.

If an entry is granted a rerun and wishes to accept that rerun but is not able to do so because of damage to its buggy or some other reason beyond its control, that entry's finishing time for its rerun race shall be its fastest time from any race that it completed during that Sweepstakes competition.

\subsection{Safety}

\subsubsection{Buggies With Drivers}

No buggy that has a driver in it may be left unattended at ANY time during the Sweepstakes races. When any buggy is outdoors with a driver in it and it is not being used in a race or a drop brake test, it MUST have someone attending it by holding onto the buggy's pushbar.

\subsubsection{Buggies}

In order to ensure that all of the buggies competing in the Sweepstakes races are in compliance with all applicable safety rules, regulations, and requirements, any buggy may be given a spot safety check by the Sweepstakes Chairman, the Assistant Sweepstakes Chairman, the Safety Chairman, or anyone designated by the Sweepstakes Advisor, immediately after that buggy has finished competing in a race and before that buggy's driver has been removed from that buggy. Any buggy found to be out of compliance with any of the applicable safety rules, regulations, or requirements during one of these spot safety checks shall be disqualified from all of the men's races, the women's races, and the design competition for that school year.

The following spot safety checks are required to be administered:

During all of the men's preliminary and rerun races, any entry which finishes its race with a time that is faster than the tenth fastest time recorded in the men's preliminary and rerun races of the previous year's Sweepstakes competition, shall have its buggy inspected before the driver is removed from that buggy after that race.

During all of the women's preliminary and rerun races, any entry which finishes its race with a time that is faster than the sixth fastest time recorded in the women's preliminary and rerun races of the previous year's Sweepstakes competition, shall have its buggy inspected before the driver is removed from that buggy after that race.

Any entry which competes in a finals race shall have its buggy inspected before the driver is removed from that buggy after that finals race.

\subsubsection{Buggy Preparation Areas}

In order to reduce the possibility of accidents or injuries in the areas in which the buggies are prepared for the races, these areas shall be randomly inspected both before and during the Sweepstakes races by fire marshals, the Safety Chairman, or anyone else designated by the Dean of Student Affairs or the Sweepstakes Advisor. Any organization found to have unsafe or dangerous conditions in their buggy preparation area before or during the Sweepstakes races by any of these safety inspectors shall be fined the amount of \$100.00 AND shall have ALL of their buggies immediately disqualified from all of the men's races, the women's races,and the design competition for that school year. In addition, any organization found to have ANY combustible liquids and/or ANY source of open flame in or near their buggy preparation area before or during the Sweepstakes races shall not be permitted to participate in ANY activity related to the Sweepstakes Competition for a period of 15 months from the date on which the violation is discovered. The following safety requirements apply AS A MINIMUM, to all buggy preparation areas both during and immediately before all Sweepstakes races:

NO quantity of ANY combustible liquid is permitted in or near the buggy preparation areas. (The ONLY exceptions to this requirement shall be quantities of lubricating fluid not greater in volume than two fluid ounces used to lubricate wheel bearings and the working fluids contained in motor vehicles and needed for their proper operation.)

No source of open flame is permitted in or near the buggy preparation areas.

All electrical apparatus and power cables MUST be grounded.

Each organization is required to have a minimum of two fire extinguishers in their buggy preparation area. Each of these fire extinguishers must contain a minimum of ten pounds of extinguishing agent. These fire extinguishers must be rated such that Class A, Class B, and Class C fires can all be extinguished. Class A fires consist of ordinary combustible materials,such as wood, paper, textiles, etc., and they require cooling and quenching. Class B fires consist of flammable liquids and greases, such as gasoline,hexane, oils, paints, etc., and they require smothering. Class C fires consist of electrical equipment, such as motors, switches, cables, etc.,and they require a nonconducting agent capable of extinguishing a fire in materials that might be present. All fire extinguishers must be adequately charged and fully operational.

NO alcoholic beverages are permitted in or near the buggy preparation areas. Alcoholic beverages to be used in post race celebrations may be stored near the buggy preparation areas, such as in the cab section of a truck, provided that none of them are opened or consumed until AFTER all of the races for that day have been completed.

No member of an organization who is in that organization's buggy preparation area may be intoxicated or under the influence of any type of mind altering substance.

If any organization uses any type of enclosed or semi-enclosed structure as a buggy preparation area, such as a truck, a tent, a free standing building, etc., the following requirements shall apply to that structure:

Smoking shall NOT be permitted inside that structure.

The inside of that structure shall NOT be sealed off from the outside of that structure by any type of solid door or cover at ANY time that ANY person is inside that structure. Trucks must have their rear doors COMPLETELY open whenever there are people inside the rear section of those trucks. Curtains or other types of non-sealing partitions are permissible in order to prevent people outside the structure from looking inside the structure.

Whenever a buggy preparation area is completely enclosed and there are people inside that area, it is recommended that fresh air be circulated through that area with some type of powered ventilating fan or blower.

\section{Final Standings}

Final standings in each class of competition shall be determined separately, but by the same method. If all of the preliminary races in a class of competition are not completed, no final standings shall be declared.

\subsection{One Day Races}

If only one day of racing is completed in a class of competition, such that all of the preliminary races in that class of competition are completed but not all of the rerun races and finals races in that class of competition are completed, the final standings of the Sweepstakes races in that class of competition shall be determined as follows:

If no entries were granted reruns, all of the entries that completed a preliminary race and were not disqualified shall be ranked in order of finishing time. The entry with the fastest finishing time shall be awarded first place, the entry with the second fastest finishing time shall be awarded second place, and so on, to the entry with the slowest finishing time in the preliminary races.

If one or more entries were granted reruns and all of the granted rerun races were completed, the finishing times from the rerun races shall be considered to be the preliminary race times for the rerun entries, and the final standings shall be determined in the same manner as if no reruns had been granted.

If one or more entries were granted reruns but not all of the granted rerun races were completed, the final standings shall be determined in the same manner as if no reruns had been granted, i.e. only the actual finishing times from the preliminary races shall be used to determine the final standings.

\subsection{Two Day Races}

If both days of racing are completed in a class of competition, such that all of the preliminary races, rerun races, and finals races are completed in that class of competition, the final standings of the Sweepstakes races in that class of competition shall be determined as follows:

If no entries were granted reruns, all of the entries that completed a finals race and were not disqualified shall be ranked in order of finishing time in those finals races. The entry with the fastest finishing time in the finals races shall be awarded first place, the entry with the second fastest finishing time in the finals races shall be awarded second place, and so on, to the entry with the slowest finishing time in the finals races. If any entries were disqualified in the finals races, they shall be ranked below all of the entries that were not disqualified in the finals races and the ranking among those disqualified entries shall be the same as their relative ranking after the preliminary races.

If one or more entries were granted reruns and any rerun races that took place were held on the same day that the finals races were held, the finishing times from the rerun races shall be considered to be finishing times for the finals races, just as if no entries had been granted reruns. If any entries were disqualified in the rerun or finals races, they shall be ranked below all of the entries that were not disqualified in the rerun or finals races and the ranking among those disqualified entries shall be the same as their relative ranking after the preliminary races.

\chapter{Design Competition}

\section{Schedule}

The design competition shall be scheduled to be held on the day before the preliminary races are scheduled to be held.

\subsection{Time and Place}

The Design Competition shall be held at a time and place determined by the Design Chairman and approved by the Sweepstakes Advisor. The Design Competition will generally be held between the hours of 9:00 am and 4:00 pm in the Wiegand Gymnasium. If that location is not available, the Design Competition may be held at another time and place, at the discretion of the Design Chairman and the Sweepstakes Advisor.

\section{Procedures}

The Design Competition shall consist of two simultaneous events, the display of all of each organization's competing buggies to the public and the presentation, if desired, of one or two of each organization's competing buggies to a panel of judges for evaluation.

\subsection{Public Display}

A public display of all of the buggies scheduled to compete in the Sweepstakes races shall take place in an indoor area that is large enough to accommodate them. Each organization must display ALL of its buggies scheduled to compete in the preliminary races at the Design Competition during the times specified by the Design Chairman. This display period should usually last between four and six hours.

\subsubsection{Display Area}

Each organization will be assigned one area in which to display its buggies. The size and location of this area shall be determined by the Design Chairman. Organizations displaying a large number of buggies may be assigned a larger area than those with fewer buggies, at the discretion of the Design Chairman. Each organization's display area may be sectioned off from the display spectators by either a solid partition which rests on the floor and is less than two feet high, or by a barrier made of a single rope supported by stanchions at any height. Each organization may restrict the public from entering its sectioned off area or from touching any of its buggies. No organization may cover or otherwise obscure any of its buggies in order to prevent the public from seeing or photographing them while they are on display.

\subsection{Presentation}

Each participating organization may present a maximum of two of its buggies scheduled to compete in the preliminary races to the design judges for preliminary judging, and if necessary, final judging. The preliminary judging of individual buggies and the final judging of the six finalists shall occur during the public display of the competing buggies. The order in which individual buggies from all of the competing organizations will be presented to the judges for the preliminary judging shall be determined by a random lottery. This lottery shall be conducted using a method determined by the Design Chairman.

\subsubsection{Presentation For Preliminary Judging}

The preliminary design competition judging shall be performed as follows:

Before any buggy is individually presented to the judges, the judges shall be permitted to view all of the buggies on display for approximately 30 minutes. During this time, no judge may ask any questions to any member of a competing organization about any of the buggies on display and no member of a competing organization may talk to or offer any information to any of the judges.

After the judges have viewed all of the buggies on display, each buggy scheduled for presentation shall be privately presented to the judges. These presentations shall be accomplished as follows:

Only one organization shall present its buggy (or buggies) to the judges at a time.

Each organization shall be permitted to have a maximum of three representatives present at the presentation of that organization's buggy (or buggies)to the judges.

Nobody but the judges and the representatives of the buggy's sponsoring organization may be present at the presentation.

Each organization's representatives shall be given a maximum of ten minutes per buggy, to describe, demonstrate, or otherwise present that organization's buggy to the judges. One person, usually the Design Chairman, shall act as a timer and shall warn the representatives when they have only two minutes remaining, and shall tell them to stop when the time period is over. The timer must remain just outside of the room in which the presentations are given. The judges may not interrupt the representatives with questions or comments during their presentation but they may take notes in order to ask questions later.

After the organization's representatives have finished their presentation, the judges may ask questions concerning each presentation or buggy, and may also examine each buggy. This period of questions and observation may last for a maximum of five minutes. The timer shall indicate when this period is over.

After the question and observation period is over, the organization's representatives shall take that organization's buggy (or buggies) back to the public display area.

\subsubsection{Presentation For Final Judging}

The final design competition judging shall be performed as follows:

After the preliminary judging is complete, the finalists shall be re-evaluated by the judges. This re-evaluation shall be accomplished as follows:

The finalist buggies shall be brought back into the presence of the judges all at the same time.

The judges shall be permitted to examine the buggies together for approximately 15 minutes.

Each organization may have one representative per buggy present while any of its buggies are being re-evaluated by the judges.

The judges may not ask a representative of any buggy any questions during the re-evaluation.

After the judges have re-evaluated the finalists, the Design Competition shall be considered to be complete and all of the buggies competing in it may leave.

\section{officials}

\subsection{Design Chairman}

The Design Chairman shall be responsible for organizing and supervising all activities related to the Design Competition. The Design Chairman appoints all of the Design Judges with the approval of the Sweepstakes Advisor. Within two weeks after the Design Competition is completed, the Design Chairman shall compile and distribute the results of the competition to all participating organizations. During the actual Design Competition, the Design Chairman shall coordinate the efforts of all of the Design Judges and shall compile and record the results of all of the design judging using the forms included in Appendix I of this document.

\subsection{Design Judges}

A panel of judges shall be appointed to evaluate the buggies presented by each organization during the Design Competition. The number of Design Judges appointed shall be determined by the Design Chairman.

10.4 Design Rules and Regulations
The following rules and regulations shall govern the Design Competition:

Each organization participating in the Sweepstakes races may enter a maximum of two buggies in the Design Competition.

Each buggy entered in the Design Competition must be individually presented to the Design Judges for evaluation. Organizations entering two buggies may present them to the judges at the same time.

Each organization participating in the Sweepstakes races must have all of its buggies that are scheduled to compete in the preliminary races on display at the Design Competition during the hours specified by the Design Chairman. If an organization does not have ALL of its competing buggies at the display area by the time specified by the Design Chairman,that organization shall be fined the amount of \$200.00 and any late buggies shall be disqualified from the Design Competition, if they were entered in that competition. If ANY buggy that is scheduled to compete in the preliminary races is greater than 30 minutes late in arriving at the display area,that buggy shall be disqualified from those races.

Any organization found covering or otherwise obscuring any of its buggies in order to prevent the public from seeing or photographing them while they are on public display at the Design Competition shall have all of its entries disqualified from both the Design Competition and from all of the Sweepstakes races held that school year.

\section{Judging Procedures}

\subsection{Preliminary Judging}

The preliminary judging of buggies privately presented to the Design Judges shall be performed as follows:

The judges shall evaluate each buggy individually, based on the presentation by the representatives of the buggy's sponsoring organization and the results of the question and observation period immediately after that presentation.

Each Design Judge shall compile a list of scores for each buggy presented based on his or her evaluation of that buggy. These scores shall be compiled using the Design Judging Evaluation form, which is included in Appendix I of this document.

Each Design Judge shall add up his or her list of scores, in order to obtain one total score for each buggy evaluated and then submit that list and score to the Design Chairman for compilation.

The Design Chairman shall check the totals of the scores of each Design Judge's list for each buggy evaluated and then compile a list of the total scores of all of the Design Judge's for that buggy. If the number of Design Judges is sufficiently high, the highest and the lowest scores may be discarded from the compiled list of scores. The Design Chairman shall total all of the scores remaining on the list in order to produce the preliminary score for each buggy evaluated.

After all of the scores have been checked and compiled by the Design Chairman they shall be rechecked by another person designated by the Design Chairman. This designated person should usually be either the Sweepstakes Chairman, the Assistant Sweepstakes Chairman, or the Safety Chairman.

The Design Chairman shall rank all of the evaluated buggies in order of their preliminary design judging scores. The buggy with the highest score shall be ranked first, the buggy with the second highest score shall be ranked second, and so on to the buggy with the lowest score, which shall be ranked last. The six highest ranked buggies shall be eligible for the final judging. In the event of ties, one or more extra buggies may be eligible for the final judging.

\subsection{Final Judging}

The judging of the buggies eligible for the final judging, which are all presented to the Design Judges at the same time, shall be performed as follows:

After the Design Judges have re-evaluated the finalists, each judge shall rank the buggies from first to last, with first being the highest ranking and last being the lowest ranking.

The Design Chairman shall compile a list of the rankings of each Design Judge for the finalist buggies as follows:

The ranking of each Design Judge for each buggy shall be added up to produce a total ranking for each finalist buggy.

The buggy with the lowest total ranking shall be placed first, the buggy with the second lowest total ranking shall be placed second, and so on until the buggy with the highest total ranking, which shall be placed last.

If one or more buggies have the same total ranking, the tie shall be broken by using the preliminary scores for those buggies, such that a buggy with a higher preliminary score shall be placed higher than a buggy with a lower preliminary score. If the preliminary scores are also the same, the tie shall stand.

After the Design Chairman has determined the rankings of the finalist buggies, the finalist's scores and rankings shall be checked by the same person designated to check the preliminary scores.

The placements of the finalist buggies shall not be divulged until the awards ceremony after the Sweepstakes races.

\section{Eligibility For Design Awards}

To be eligible for a Design Competition award a buggy must comply with all of the following requirements:

It must compete in and finish a preliminary or rerun Sweepstakes race without a design related failure. An entry may not be granted a rerun solely for the purpose of eligibility for a design award.

If it competed only in the men's races, its finishing time from its preliminary or rerun race in the men's races must be among the top third of all of the entries that received official finishing times in the preliminary and rerun races in the men's races. If it competed only in the women's races,its finishing time from its preliminary or rerun race in the women's races must be among the top third of all of the entries that received official finishing times in the preliminary and rerun races in the women's races. If it competed in both the men's and women's races, it must finish in the top third of at least one of these two classes of competition .

If the total number of official finishing times in a class of competition is not evenly divisible by three, the number shall be rounded up to the next whole number evenly divisible by three in order to determine the top third.

\subsection{Eligibility Determination}

If there is any question concerning the eligibility of a buggy fora Design Competition award, the final determination of that eligibility shall be made by the Design Chairman.

\chapter{Awards}

Awards to the competitors of the Sweepstakes Competitions may be made at an awards ceremony sometime after all of the competitions have ended. This ceremony is usually held in the evening of the same day on which the finals races are scheduled to take place. All recipients of awards will be determined by the rules herein, and independently of the rules of the Spring Carnival Committee.

\section{Races}

Awards for the Sweepstakes races should be presented by the Sweepstakes Chairman.

\subsection{Men's Races}

The entries placing first, second, third, fourth, fifth, and sixth in the final standings of the men's races shall receive awards. The nature of these awards shall be determined by the Sweepstakes Chairman, or anyone designated by that Chairman.

\subsection{Women's Races}

The entries placing first, second, and third in the final standings of the women's races shall receive awards. The nature of these awards shall be determined by the Sweepstakes Chairman, or anyone designated by that Chairman.

\section{Design Competition}

The three buggies placing the highest in the final judging of the Design Competition and meeting all of the eligibility requirements for design awards shall receive awards. If there are not three buggies that meet all of these requirements, the scores from the preliminary judging of the Design Competition may be used to select additional buggies to receive awards, at the discretion of the Design Chairman. The nature of these awards shall be determined by the Design Chairman, or anyone designated by that Chairman.
\end{document}
