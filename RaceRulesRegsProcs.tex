\chapter{Race Rules, Regulations, and Procedures}
\label{ch:RaceRules}

\section{Common Rules and Procedures}
	Some rules and procedures are common to freeroll practice and Raceday.
	Please see Race and Freeroll Common Rules and Procedures, in sec. \ref{ch:CommonRules} pg \pageref{ch:CommonRules}.

\section{Sweepstakes Race Schedule}

	The Sweepstakes races shall be scheduled to be held at or near the same time
	that the Carnegie Mellon University Spring Carnival is held each school year.
	The races will usually be scheduled for two consecutive days.

\subsection{First Day of Racing}

	The preliminary races shall be held on the first day of Sweepstakes Racing,
	with the women's preliminary races being held first, followed by the men's
	preliminary races.

\subsection{Second Day of Racing}

	The alumni/exhibition races, the rerun races, and the finals races shall be
	held on the second day of Sweepstakes racing. If alumni/exhibition races are to
	be run, they shall be scheduled first, followed by the women's rerun races (if
	necessary), the men's rerun races (if necessary), the women's finals races, and
	the men's finals races, in that order. The order in which these races are
	scheduled may be changed at the discretion of the Sweepstakes Chairman or the
	Sweepstakes Advisor, based on weather forecasts and/or time constraints.

\subsection{Cancellation}

	Either or both days of Sweepstakes racing may be canceled by the Dean of
	Student Affairs, the Sweepstakes Advisor, or the Sweepstakes Chairman, due to
	inclement weather, inadequate police protection, inadequate communications,
	lack of medical personnel, vehicles parked on the course or on the sidewalks
	around the course, or any other condition which might endanger the contestants
	or spectators of the races.

	If the preliminary races are canceled on the day that they were originally
	scheduled to take place (Friday), they shall be rescheduled for the day that
	the finals races were originally scheduled to take place (Saturday), and no
	finals races shall be held that year.

	If the preliminary races are run as originally scheduled (on Friday) and the
	finals races are canceled on the day that they were originally scheduled to
	take place (Saturday), no finals races shall be held that year.

	If the preliminary races are canceled on the day that they were originally
	scheduled to take place (Friday), and then are canceled again on the day for
	which they were rescheduled (Saturday), the preliminary races may be
	rescheduled again and held on the Sunday of Spring Carnival or a later date, at
	the discretion of the Sweepstakes Advisor.

\section{Race Day Procedures}

\subsection{Time and Place}

	Sweepstakes races shall only be held on the buggy course and only on the dates
	and at the times specified by the Sweepstakes Advisor. Sweepstakes races will
	generally be held between the hours of 6:00 am and 2:00 pm on the Friday and
	Saturday of Spring Carnival weekend. Sweepstakes races may be held on other
	days and at other times at the discretion of the Sweepstakes Advisor.

\subsection{Lane and Zone Markings}
\label{subsec:Lanes}

	Lines delineating the lanes on Hills 1 and 2, the starting line, the finish
	line, and the beginnings and ends of the three transition zones must be painted
	on the buggy course. At the discretion of the Sweepstakes Chairman, the
	responsibility of painting these lane and zone markings on the buggy course for
	the Sweepstakes races may be delegated to one organization. These lines should
	normally be painted by a date that is at least two weeks before the date
	scheduled for the preliminary races. Paint and the striper to paint the lines
	should be obtained through the Sweepstakes Advisor.

	If the organization responsible for painting lane and zone markings fails to
	provide them without adequate reason, such as bad weather, before the last
	scheduled freeroll practice before the races, that organization shall be fined
	the amount of \$25.00.

\subsection{Buggy Preparation Areas}

	Each organization participating in the Sweepstakes races shall be permitted to
	select an area near the starting line of the buggy course, in which they may
	prepare their buggies prior to the races. These areas may be occupied by trucks
	or any other non-permanent enclosures. The order of selection of these areas
	shall be the same as the seeding order for the preliminary races. The selection
	of these areas should be made at least four weeks prior to the date scheduled
	for the preliminary races. The selection process shall be supervised by the
	Sweepstakes Chairman, or anyone designated by that Chairman. The Sweepstakes
	chairman will assign spaces to organizations at least 2 weeks prior to the date
	schedule for the preliminary races. No organization may use or attempt to use
	any area that has been selected by another organization on any day of
	Sweepstakes racing.

\subsection{Electrical Power}

	Electrical power may be made available on each day of Sweepstakes racing, to
	each organization participating in the races. If available, this electrical
	power will be located near each organization's buggy preparation area, and will
	nominally be 110 volt alternating current, with a maximum current capacity of
	20 amperes. (This will provide a maximum nominal power of 2,200 watts.) Two
	standard 110 VAC electrical outlets will be provided to each participating
	organization. Each organization will be responsible for providing adequate
	grounded cabling to transfer this power from the location of the outlets, to
	their individual buggy preparation areas.

	When available, the electrical power near the buggy preparation areas will be
	provided by the Physical Plant Department of Carnegie Mellon University.
	Arrangements to provide this power should be made with the Physical Plant
	Department by the Sweepstakes Chairman, through the office of the Sweepstakes
	Advisor, at least six weeks before the races are scheduled to take place.

\subsection{Course Watch}
\label{subsec:Course Watch}

	The night before each scheduled day of Sweepstakes racing, the Sweepstakes
	Chairman, or anyone designated by that Chairman, shall assign representatives
	of each participating organization to watch various parts of the buggy course
	in order to prevent motor vehicles from parking on any part of the course. Each
	participating organization must provide two people to watch the course and a
	motor vehicle for those people to ride in. The length of time that each person
	must watch the course and the area that they must watch shall be determined by
	the Sweepstakes Chairman.

	Any organization that fails to provide the required number of people and
	vehicles for course watch duty on the night before any day of Sweepstakes
	racing, shall be fined the amount of \$25.00 for each missing person or
	vehicle.

\subsection{Crowd Control Barriers}
\label{subsec:Crowd Control}

	Crowd control barriers should be used to restrain race spectators located along
	Hill 2 and Hill 5. These barriers can be ropes strung between stanchions, or
	even held by course marshals. The barriers should be placed along Hill 2 on
	each side of the buggy course, from the Hill 1-2 Transition Zone to the end of
	the lanes, and along Hill 5 on only the northern side of the course, from a
	point that is approximately 200 feet from the finish line to the finish line.
	On Hill 2 the barriers should be placed about 3 feet outside of the outermost
	edges of the lanes, and on Hill 5 they should be placed even with the northern
	curb of Frew Street. The barriers shall be put in place no later than 30
	minutes before races are scheduled to start on each day of Sweepstakes racing.
	They shall be removed within 15 minutes of either the end of the races for that
	day, or the cancellation of the races for that day.

	At the discretion of the Sweepstakes Chairman, the responsibility of obtaining,
	storing, and placing crowd control barriers in position for Sweepstakes races
	may be delegated to the same organization responsible for placing warning signs
	in position for the Sweepstakes races. The organization charged with this
	responsibility may not have to provide sweepers, barricaders, or course marshals
	for each day of Sweepstakes racing, at the discretion of the Sweepstakes
	Chairman.

	If the organization responsible for the crowd control barriers fails to provide
	them or remove them for any day of Sweepstakes racing, that organization shall
	be fined the amount of \$50.00.

\subsection{Course Marshals}
\label{subsec:Course Marshals}

	Each organization shall provide at least two course marshals for each scheduled
	day of Sweepstakes racing, to help control the race spectators on or near the
	buggy course. These course marshals must be available from a time that is one
	hour before the races are scheduled to start, until the races are finished for
	that day. Each organization shall equip at least two of the course marshals
	which it provides with reflective vests and flags which must be used while they
	are acting as marshals. Additional marshals must be equipped with some sort of
	brightly colored distinguishing attire, such as vests, hats, arm-bands, etc. so
	that they may be easily recognized by the race participants and spectators. The
	Sweepstakes Chairman, or anyone designated by that Chairman, shall determine
	when and where the course marshals must report for duty.

	Any organization that fails to provide the required number of properly equipped
	course marshals for any day of Sweepstakes racing, shall be fined the amount of
	\$25.00 for each missing or improperly equipped course marshal.

	At the discretion of the Sweepstakes Chairman, organizations charged with other
	responsibilities, such as hay bales, no-parking signs, barricades, warning
	signs, etc., may not have to provide course marshals for each day of
	Sweepstakes racing.


\subsection{Vehicles}

	During each Sweepstakes race two motor vehicles known individually as the lead
	car and the follow car shall drive around the buggy course. A third large
	flatbed or similar truck shall be positioned at the finish line parallel to the
	direction of the course for use by timers, and for storage of relevant Sweepstakes
	equipment during the races.

	The lead truck and follow truck should be either a convertible top car or
	some sort of pick-up truck, in order to afford the best view of the race
	possible to the people riding in it. If this vehicle is a pickup truck, its
	tail gate must always be closed while the vehicle is moving in order to reduce
	the chances of anyone falling out of it. It is also advisable to have straps or
	a framework type of structure extending up from the sides of the pick-up truck
	bed, in order to help prevent anyone from falling out over the sides of the
	truck bed.

	Each vehicle moving along the course shall have a dedicated spotter who is responsible
	for giving verbal instructions to the driver of the vehicle during each race, so that
	the car may be kept at a safe and consistent (from race to race) distance from any
	buggies in that race.

	All vehicles shall be obtained by the Sweepstakes Chairman with the assistance of
	the Sweepstakes Advisor. The driver of each vehicle shall be determined by the
	Sweepstakes Advisor.


\subsubsection{Finish Line Truck}

	A platform several feet high should be placed on the northern sidewalk of Frew
	Street, directly in line with the finish line of the buggy course. This
	platform will provide a vantage point for the Sweepstakes race timers, so that
	they have a better view of the finish line of the race.

	A flatbed truck is usually used as the platform at the finish line. This truck
	can usually be obtained from the Physical Plant Department of Carnegie Mellon
	University. Arrangements to borrow this truck should be made by the Sweepstakes
	Chairman, approximately six weeks before the preliminary races are scheduled to
	take place, through the office of the Sweepstakes Advisor.


\subsubsection{Lead Car}

	The lead car should stay several hundred feet in front of the leading buggy at all times so
	as not to interfere with that buggy during the race. The following individuals may occupy
	the Lead Car:

	\begin{itemize}

		\item Lead Car Driver

		\item Lead Car Spotter next to the Lead Car Driver

		\item Sweepstakes Chairman in the rear to observe each race

		\item Head Judge in the rear to observe each race

		\item Lead Car Judge in the rear to assist with video recording

		\item Other people involved with the Sweepstakes races may ride in the follow
		car at the discretion of the Sweepstakes Chairman.

	\end{itemize}

	At the point when the leading buggy passes Porter hall on the course, the Lead Car shall
	accelerate ahead of the race to clear the course of spectators and allow the Head Judge,
	Lead Car Judge, and Sweepstakes Chairman to exit the vehicle. The Head Judge and Lead Car
	Judge are then responsible for recording and Judging the buggy on Hill 5 as well as when
	it crosses the Finish Line.

	In the event of an accident during a heat, the lead car should continue along
	the buggy course, as long as there is at least one buggy still continuing with
	the race. If all of the buggies in the heat come to a stop, the lead car should
	stop near the leading buggy in order to render any necessary assistance.


\subsubsection{Follow Car}
\label{subsubsec:Follow Car}

	The follow car should stay approximately one hundred feet behind the trailing buggy
	at all times so as not to interfere with that buggy during the race.
	The following individuals may occupy the Follow Car:

	\begin{itemize}

		\item Follow Car Driver

		\item Follow Car Spotter

		\item Safety Chairman

		\item Assistant Head Judge

		\item Follow Car Judge to assist with video recording

		\item One representative of each entry in the heat underway to observe
		that heat and provide any necessary assistance

		\item Other people involved with the Sweepstakes races may ride in the follow
		car at the discretion of the Sweepstakes Chairman.

	\end{itemize}

	In the event of an accident during a heat, the follow car should be stopped
	near the scene of the accident in order that the people in the follow car may
	render assistance to those involved in the accident. The members of the racing
	organizations riding in the follow car MUST have any tools or devices that are
	necessary to quickly remove any of their drivers from their buggies. If an
	organization fails to have these tools or devices in the car following the
	buggies during any race, that organization shall be fined the amount of \$25.00
	and the buggy in question shall be disqualified from that race.


\subsection{Medical Personnel}

	Medical personnel should be available at all times while the Sweepstakes races
	are underway. Off duty paramedics from the City of Pittsburgh can usually be
	obtained to provide any necessary medical assistance for each day of racing.
	Arrangements to have paramedics available during the races should be made by
	the Sweepstakes Chairman, approximately six weeks before the preliminary races
	are scheduled to take place, through the office of the Sweepstakes Advisor.

	If an active Emergency Medical Service exists on the Carnegie Mellon University
	campus, members of that organization may also be obtained to help provide
	medical assistance during the races. Arrangements to have members of a campus
	Emergency Medical Service available during the races should be made by the
	Sweepstakes Chairman, approximately six weeks before the preliminary races are
	scheduled to take place, with the assistance of the Sweepstakes Advisor.

\subsection{Officials}

	All of the Sweepstakes race officials should be given some sort of
	distinguishing attire to wear during the races, so that they may be easily
	recognized by the race participants and race spectators. This attire should be
	brightly colored so that it is readily visible. It may consist of armbands,
	hats, shirts, vests, or any other type of distinguishing attire. This attire
	shall be provided by the Sweepstakes Chairman in conjunction with the
	Sweepstakes Advisor.

	The following Sweepstakes officials shall wear this distinguishing attire: the
	Sweepstakes Advisor, the Executive Committee, all assistants to the above people,
	all judges, all timers, all course marshals, and the Starter.

\subsection{Starter}

	A person shall be appointed by the Sweepstakes Chairman, with the approval of
	the Sweepstakes Advisor, to be the official Starter of all of the Sweepstakes
	races. The Starter shall be located near the starting line of the buggy course
	before and during all of the races, usually just to the right of Lane 1 when
	looking south from the starting line. The Starter shall announce the time
	remaining until the start of each heat before each heat and shall announce any
	delays or holds in the countdown to each heat. The Starter shall use a
	traditional starting gun, or other similar device, to indicate the start of
	each heat. Two starting guns should always be available in the event that the
	first one used to start a heat misfires.

	The Starter shall also act as a judge for the race by observing the race as far
	up Hills 1 and 2 as he or she can see it, watching for fouls by or interference
	between the competitors.

\subsection{Judges}

	Judges shall be utilized to observe the Sweepstakes races and to point out and
	rule on possible violations of the rules and regulations by the race
	participants.

\subsubsection{Head Judge}

	A person shall be appointed by the Sweepstakes Chairman, with the approval of
	the Sweepstakes Advisor, to be the Head Judge of the Sweepstakes races. The
	duties of the Head Judge shall be as follows:

	Approximately 30 minutes before the Sweepstakes races begin each day, the Head
	Judge shall assign the available course judges to various positions around the
	buggy course. The Head Judge should instruct the course judges concerning what
	to look for during the races and should provide each course judge with a copy
	of the race rules and regulations which are pertinent to what they will be
	observing. The recommended positions for the course judges around the buggy
	course are as follows:

	\begin{itemize}

		\item Starting line on the Lane 3 side of the course

		\item Hill 1-2 Transition Zone

		\item Schenley Drive with the chute flaggers

		\item Two judges at the chute, one on each side of the course

		\item Hill 3-4 Transition Zone

		\item Hill 4-5 Transition Zone

		\item Lead Car, strictly to record footage for the Head Judge

		\item Follow Car, strictly to record footage for the Assistant Head Judge

	\end{itemize}

	The Head Judge shall observe each heat while riding in the rear of the lead
	car, watching for fouls by or interference between the entries.

	The Head Judge shall hear all protests and appeals by any entry or organization
	and gather all information available from other judges, officials, or race
	participants pertinent to those protests and appeals. The Head Judge may use a
	portable tape recorder in order to record verbal protests and appeals, for
	later review and consideration.

	The Head Judge shall render the final decisions concerning all protests and
	appeals based on any information that he or she has gathered relative to those
	protests and appeals, and on his or her interpretation of these Rules,
	Regulations, and Procedures.


\subsubsection{Assistant Head Judge}

	The Safety Chairman or another individual shall be appointed by the Sweepstakes
	Chairman, with the approval of the Sweepstakes Advisor, to be the Assistant Head
	Judge of the Sweepstakes races. The duties of the Assistant Head Judge shall be
	as follows:

	\begin{itemize}

		\item Observe each race while riding in the follow car, watching for fouls
		by or interference between the competitors.

		\item Ensure that only authorized people are permitted to ride in the follow
		car during each of the Sweepstakes races.

		\item Assist the Head Judge with any of his or her duties, when so requested.

	\end{itemize}

\subsubsection{Course Judges}

	Each organization participating in the Sweepstakes races must provide two
	course judges for each day of Sweepstakes racing. These judges should be
	alumni of the organizations providing them. No course judge shall be
	permitted to provide information or comments on any heat in which an
	entry from the organization which provided that judge is competing.
	Course judges may also be members of the Carnegie Mellon University
	faculty and staff, as selected by Sweepstakes Advisor.

	The course judges must report to the Head Judge approximately 30 minutes before
	the races are scheduled to begin on each day of Sweepstakes racing, so that
	they can be assigned to positions around the buggy course from where they
	should observe the Sweepstakes races. The duties of each course
	judge shall be as follows:

	\begin{itemize}

		\item Watch each Sweepstakes race from the location
		assigned to that course judge by the Head Judge.

		\item Watch for any fouls by or interference between the competitors in
		each of the Sweepstakes races that he or she observes, and record them in
		a notebook provided by the Sweepstakes Chairman with a listing of the heats.

		\item Record the races for tape review after the day of Sweepstakes races ends,
		using video equipment provided by the Sweepstakes Chairman.

		\item Each course judge shall provide to the Head Judge, when so requested,
		any and all information that they may have, that might be pertinent to any
		alleged fouls or incidences of interference that may have occurred during any
		of the Sweepstakes races that they observed.

	\end{itemize}

\subsection{Timers}

	Timers shall be utilized to measure the time required for each entry's buggy to
	travel from the starting line to the finish line of the buggy course during the
	Sweepstakes races. The timers shall be appointed by the Sweepstakes Advisor. A
	minimum of two timers shall be required for each entry in a heat. The timers
	shall use stop watches or other suitable measuring devices to determine the
	time taken by each buggy to travel the buggy course. Sweepstakes Advisor shall
	appoint one timer as the Head Timer. The Head Timer will coordinate the efforts
	of all of the other timers.

	The timers should be located on the northern sidewalk of Frew Street, directly
	in line with the finish line of the buggy course. They should be positioned on
	a platform several feet above the sidewalk so that they have a better view of
	the finish line of the race.

	The timers will usually use the sound of the starter's gun, as broadcast over
	the Carnegie Mellon University radio station (WRCT), as a signal to start their
	timing devices at the beginning of a race. If the radio station is not
	broadcasting the start of the race, another method of starting the timing
	devices must be devised.

	The timers shall stop their timing devices when the nose of the buggy that they
	are timing reaches the finish line of the buggy course. The finishing time for
	each entry shall be the average of all of the times determined by the timers
	for that entry. All finishing times shall be announced by the Head Timer,
	whether or not they are subsequently invalidated because of a disqualification.

	The Head Timer will be responsible for recording official and/or unofficial
	finishing times for all entries in all Sweepstakes races that take place on
	each day of racing. These times shall be recorded on Sweepstakes Race Timing
	Forms. Along with the finishing times, notations concerning disqualifications,
	accidents, protests, appeals, brake tests, etc. for each entry shall also be
	recorded on the Sweepstakes Race Timing Forms by the Head Timer. At the end of
	each day of racing the Head Timer shall present all of the timing forms for
	that day's races to the Sweepstakes Chairman.

	At the discretion of the Sweepstakes Advisor, alternate methods of automatic or
	semiautomatic timing of the race entrants may be employed, provided that these
	timing methods can be shown to be as accurate or more accurate than the usual
	timing method. If any automatic or semi-automatic timing methods are used, the
	manual timing method described above should also be used as a back-up system.

\section{Race Schedule}

\subsection{Starting Times}

	The time between the start of one scheduled heat and the start of the next
	scheduled heat shall usually be 8 minutes during the preliminary races, the
	alumni/exhibition races, and the rerun races, and 15 minutes during the finals
	races. The actual time intervals between heats shall be determined by the
	Sweepstakes Chairman, based on the number of heats to be run and on the amount
	of time available in which to run them. The actual time intervals to be used
	for the different races shall be announced to the participating organizations
	by the Sweepstakes Chairman sometime before the first scheduled day of racing.

	The intervals between heats shall be timed by the Starter. Before each
	scheduled heat starts the Starter shall announce the time remaining to the
	start of that heat. These announcements shall usually be made when there are 10
	minutes (finals races only), five minutes, two minutes, one minute, 30 seconds,
	and 15 seconds remaining before the start of the next scheduled heat. The last
	10 seconds before the start of each heat shall be counted off by the Starter,
	in a manner such that all competitors located near the starting line are able
	to hear that count. After the Starter's countdown reaches the count of ``ONE,''
	the final warnings to the competitors before the heat starts shall be --
	``READY, -- SET,'' after which the Starter shall fire the starting gun to start
	the heat.

	Each entry must be ready to start its scheduled heat on time. No requests for
	extensions to delay the scheduled start of any heat shall be granted. Any entry
	not in position at the starting line when the Starter indicates that there are
	five seconds left in the countdown until the start of that entry's heat, will
	not be permitted to start that heat and will not be eligible for a rerun.

\subsection{Delays and Holds}

	Delays and holds shall be handled as follows:
	\newline

	If a delay or hold of the countdown to the start of a heat occurs immediately
	after the end of the previous heat, when the delay or hold is over, the
	countdown to the start of that next heat shall resume at a time that is between
	four and seven minutes before the start of that next heat if that heat is a
	preliminary race, an alumni/exhibition race, or a rerun race. For a finals
	race, the countdown to the next heat shall resume in between seven and 12
	minutes.

	If a delay or hold of the countdown to the start of a heat during the
	preliminary races, the alumni/exhibition races, or the rerun races occurs after
	there is less than five minutes remaining to the start of that heat, the
	countdown may be held for up to one minute, and then resumed at the time at
	which it was held. If the delay or hold is longer than one minute, when that
	delay or hold is over, the countdown to the start of that next heat shall
	resume at a time that is between four and seven minutes before the start of
	that next heat.

	If a delay or hold of the countdown to the start of a heat during the finals
	races occurs after there is less than eight minutes remaining to the start of
	that heat, the countdown may be held for up to one minute, and then resumed at
	the time at which it was held. If the delay or hold is longer than one minute,
	when that delay or hold is over, the countdown to the start of that next heat
	shall resume at a time that is between seven and 12 minutes before the start of
	that next heat.

	Whenever a delay or hold both begins and ends the Starter will announce that the
	countdown has stopped or resumed, in addition to the remaining time on the clock.

\subsection{Starting Positions}

	Starting positions in the preliminary races, rerun races, and finals races in
	each class of competition shall be determined separately, but by the same
	method.

\subsection{Preliminary Races}

\subsubsection{Heats}

	The preliminary races shall be run in heats, with a maximum of three entries in
	each heat. The number of heats in each class of competition shall be determined
	by dividing the total number of entries by three and rounding up to the nearest
	whole number.

\subsubsection{Seeding}

	\begin{enumerate}

		\item Entries for each organization shall be identified according to
		their letter designations, not by the buggy used by each entry. An
		organization's A entry should be its fastest entry, its B entry should be its
		next fastest entry, and so on.

		\item The order of seeding shall be based on the weighted average of an
		entry's last three years finishing times in the preliminary races, with the
		time for three years ago being multiplied by one, the time from two years ago
		being multiplied by two, and the time from last year being multiplied by three.
		These three numbers would be added together and then divided by six to
		determine that entry's seeding time. For example, for race day 2002 the entry's
		race day 2001 preliminary time would be multiplied by three and added to the
		entry's race day 2000 preliminary time multiplied by two, and then finally
		added to the entry's race day 1999 preliminary time. This total would then be
		divided by six to obtain that entry's seeding time. The entry with the fastest
		seeding time will then be placed in the last heat; the entry with the second
		fastest seeding time will then be placed in the second to last heat. This
		process will be repeated until every heat has one team in it, then the process
		will continue by going back to the last heat of the day and assigning another
		team to that heat. By continuing this process each heat will end up with a
		minimum of two teams and a maximum of three teams. The next step in the seeding
		process will be the lane selection phase. For every heat the team with the best
		seeding time will choose their lane first, then the team with the second best
		seeding time will choose one of the two available lanes, and finally the team
		with the third best seeding time will be assigned to the remaining available
		lane.

		\item The finishing times used for seeding purposes need not be
		official times. If an entry finishes the race, but is subsequently disqualified
		(for example by failing the brake test), its finishing time shall still be used
		for seeding purposes. In addition if an entry does not finish its preliminary
		race, but is subsequently granted an official or unofficial reroll, the time
		from that reroll shall be used for seeding purposes. If an entry does not
		finish its preliminary race and does not have an official or unofficial
		preliminary time for one or more of the three most recent years, that year
		shall be thrown out of that entry's seeding time. That entry's seeding time
		will be computed, using the method stated above, however the year that the
		entry did not finish will not be entered into the average, and the total time
		will be divided by the appropriate number.

		\item If for any reason the Sweepstakes Executive Committee feels that
		a heat contains an unsafe grouping of buggies, the Sweepstakes Executive
		Committee will discuss the matter with the University's staff Sweepstakes
		Advisor, and at their approval make the changes necessary to provide a safe
		race.

	\end{enumerate}

\subsubsection{Heat Selection}

	The purpose of the heat selection procedure is to help ensure that each heat of
	the preliminary races has entries that are as far apart, with respect to
	probable finishing times, as is possible. This should help to reduce the
	chances of an accident during any of these heats. After all of the entries in
	each class of competition have been seeded, they shall select heats for the
	preliminary races using the following procedure:

	Heat selections shall be made at a meeting of the Sweepstakes Committee. The
	heat selection procedure shall be supervised by the Sweepstakes Chairman, or
	anyone designated by that Chairman. This supervisor shall maintain order during
	the selection process and shall keep records of all of the selections that are
	made. If any entered organization is not represented at this meeting, the
	supervisor shall be empowered to make any required selections for the
	unrepresented organization.

	The seeded entries shall be divided into three groups. If the number of
	preliminary heats scheduled is $n$, then the first group will have $n$ entries,
	starting with the highest seeded entry and ending with the $n^{th}$ seeded
	entry. The second group will also have $n$ entries, starting with the
	$(n+1)^{th}$ seeded entry and ending with the $(n+n)^{th}$ seeded entry. The
	third group will have all of the remaining seeded entries.

	Representatives of each of the entries in the first group shall select the
	heats that their entries shall compete in using the following procedure:

	\begin{enumerate}

		\item
		The highest seeded entry in the first group shall select the heat that it
		wants to compete in during the preliminary races.

		\item
		The next highest seeded entry in the first group shall select the heat that
		it wants to compete in during the preliminary races, with the provision that it
		cannot select a heat which already has one or more entries from the first group
		in it, unless all of those other first group entries grant it permission to do
		so.)

		\item
		The preceding step shall be repeated until all of the seeded entries in the
		first group have selected heats.

	\end{enumerate}

	Representatives of each of the entries in the second group shall select the
	heats that their entries shall compete in using the following procedure:

	\begin{enumerate}

		\item
		The highest seeded entry in the second group shall select the heat that it
		wants to compete in during the preliminary races.

		\item
		The next highest seeded entry in the second group shall select the heat that
		it wants to compete in during the preliminary races, with the provision that it
		cannot select a heat which already has one or more entries from the second
		group in it, unless all of those other second group entries grant it permission
		to do so.

		\item
		The preceding step shall be repeated until all of the seeded entries in the
		second group have selected heats.

	\end{enumerate}

	Representatives of each of the entries in the third group shall select the
	heats that their entries shall compete in using the following procedure:

	\begin{enumerate}

		\item
		The highest seeded entry in the third group shall select the heat that it wants
		to compete in during the preliminary races.

		\item
		The next highest seeded entry in the third group shall select the heat that it
		wants to compete in during the preliminary races.

		\item
		The preceding step shall be repeated until all of the seeded entries in the
		third group have selected heats.

	\end{enumerate}

	Heat selections shall be made by a date that is no later than the second day of
	Truck Weekend (the weekend just prior to the date scheduled for the preliminary
	races). Within two days of the day that heats and lanes are selected, the
	Sweepstakes Chairman shall distribute a schedule of all of the heat and lane
	selections for the preliminary races, to each of the participating
	organizations.

\subsubsection{Lane Selection}

	After heats have been selected for all entries, the lane that each entry will
	occupy on Hills 1 and 2 shall be chosen. The entry seeded first shall choose
	first. The entry seeded second shall choose second. The entry seeded third
	shall choose third, and so on until all of the entries have chosen lanes. No
	entry may choose a lane that is already occupied by a higher seeded entry.

	Lane selection shall take place at the same meeting at which heats are
	selected.

\subsubsection{Heat and Lane Switching}

	Any entry may switch their heat and/or lane selections with any other entry
	provided that they comply with the following:

	\begin{itemize}
		\item All of the entries in each heat in which a change
		is to take place must approve the change.
		\item The Sweepstakes Chairman and the Safety Chairman
		must approve the change.
		\item The change must be made within 24 hours of the
		meeting at which the heats and lanes were selected.
	\end{itemize}

\subsubsection{Entry Withdrawals}

	If an entry withdraws from the Sweepstakes competition before the preliminary
	races begin, the heat to which that entry was assigned will not have three
	entries competing in it. If more than one heat has less than three entries
	competing in it, the Sweepstakes Chairman may reassign entries in heats with
	less than three entries to different heats if that reassignment will reduce the
	total number of heats in the preliminary races. A reassignment may only be made
	if the Sweepstakes Chairman and the Safety Chairman determine that that
	reassignment would not increase the likelihood of an accident in any heat, and
	if the reassignment does not cause any reassigned entry to start the race in a
	lane other than the one that they were originally scheduled to be in, unless
	that entry agrees to the different lane assignment. No entry of an organization
	may be reassigned to a heat in a class of race competition that is less than
	two heats away from any heat in that same class of race competition in which
	another of that organization's entries is scheduled to compete.

	Any organization withdrawing an entry shall forfeit all entry fees for that
	entry. If an organization withdraws an entry after heats and lanes have been
	selected for the preliminary races, they shall only be permitted to enter as
	many entries in the following year's competition as they had competing in the
	year that they made the late withdrawal. Each organization shall always be
	permitted to have at least one entry regardless of the number of late
	withdrawals they had during the previous year's competition.

	Exceptions to this rule may be made if an entry is withdrawn involuntarily,
	such as if the entry's buggy is damaged beyond repair during a practice
	session. However, the evidence that the withdrawal was involuntary must be very
	strong and must be agreed upon by the Sweepstakes Chairman, the Assistant
	Sweepstakes Chairman, and the Safety Chairman. Failure of an entry's driver to
	successfully complete a passing test or to complete the minimum required number
	of rolls down the buggy course during freeroll practice sessions shall NOT be
	considered as valid reasons for the involuntary withdrawal of that entry.
	Failure of an entry's buggy to complete the minimum required number of rolls
	down the buggy course during freeroll practice sessions shall NOT be considered
	as a valid reason for the involuntary withdrawal of that entry.

\subsection{Alumni/Exhibition Races}

	Alumni/exhibition races are not considered to be official Sweepstakes races. If
	alumni/exhibition races are held, all safety related rules, regulations, and
	requirements applicable to the Sweepstakes races shall also apply to the
	alumni/exhibition races. This includes the requirement for a drop brake test
	for each participating buggy after each heat of the alumni/exhibition races,
	and all of the requirements for drivers. (This also includes the requirement
	that all drivers be currently enrolled, Activities Fee paying, full-time
	students of Carnegie Mellon University.) Graduate students may participate in
	alumni/exhibition races.

\subsubsection{Starting Positions}

	If alumni races are scheduled, heat and lane assignments shall be made by the
	Sweepstakes Chairman, or anyone designated by that Chairman. The procedure used
	to assign heats and lanes for the alumni/exhibition races shall be determined
	by the Sweepstakes Chairman.

\subsection{Rerun Races}

	Any rerun races that are held shall be run in heats, with a maximum of three
	entries in each heat. The minimum number of rerun heats in each class of
	competition shall be determined by dividing the number of rerun entries by
	three and rounding up to the nearest whole number.

\subsubsection{Heat Assignments}

	Heat assignments for rerun races shall be made by the Sweepstakes Chairman and
	the Safety Chairman. The purpose of the heat assignments is to help ensure that
	each heat of the rerun races has entries that are as far apart with respect to
	probable finishing times, as is possible, in order to reduce the chances of an
	accident during any of the rerun heats. The seeding order used to assign heats
	in the preliminary races shall be used as a guideline in assigning entries to
	heats for rerun races. Rerun race heats may be run with less than three entries
	each, if the Sweepstakes Chairman and the Safety Chairman, determine that the
	safety of those heats might be substantially increased by doing so. The method
	of assigning entries to heats for the rerun races shall be determined by the
	Sweepstakes Chairman and the Safety Chairman. Heat assignments shall be
	completed as soon as is practical after the preliminary races have ended.

	No rerun heat shall have more than one entry from any single organization in
	it.

\subsubsection{Lane Selection}

	After all rerun entries have been assigned to a heat, the lane that each entry
	will occupy on Hills 1 and 2 shall be chosen. The entry seeded highest before
	the preliminary heats shall choose first. The entry seeded next highest before
	the preliminary heats shall choose second, and so on until all of the rerun
	entries have chosen lanes. No rerun entry may choose a lane that is already
	occupied by a higher seeded rerun entry.

	Lane selection shall be completed as soon as is practical after heat
	assignments have been made.

\subsection{Finals Races}

\subsubsection{Rankings}

	After the preliminary races are finished, and before any rerun races have taken
	place, all of the entries that finished a preliminary race and were not
	disqualified or granted a rerun, shall be ranked according to their finishing
	times in those preliminary races for each class of competition. The entry with
	the fastest time shall be ranked first, the entry with the second fastest time
	shall be ranked second, and so on to the entry with the slowest time, which
	shall be ranked last.

	The eight highest ranked entries from the women's preliminary races shall be
	eligible to compete in the women's finals races.

	The ten highest ranked entries from the men's preliminary races shall be
	eligible to compete in the men's finals races.

	In the event of ties, one or more extra entries may be eligible to compete in
	the women's or men's finals races. The rankings of tied entries shall be
	determined by the Sweepstakes Chairman, or anyone designated by that Chairman.

\subsubsection{Heats}

	The finals races shall be run in heats, with a maximum of two entries in each
	heat. The number of finals heats in each class of competition shall be
	determined by dividing the number of entries eligible to compete in the finals
	races by two and rounding up to the nearest whole number.

\subsubsection{Heat Assignment}

	The entries eligible to compete in the finals races will be assigned to heats
	for those races, by the Sweepstakes Chairman, or anyone designated by that
	Chairman. The purpose of the heat assignments is to help ensure that each heat
	of the finals races has entries that are as far apart with respect to probable
	finishing times as is possible, in order to reduce the chance of an accident
	during any of the finals heats. Heat assignments shall be completed as soon as
	is practical after the preliminary races have ended.
	\newline

	\noindent Heat assignments for each class of competition shall be made as follows:

	\begin{itemize}

		\item
		The entry ranked first after the preliminary races shall be assigned to the
		last finals heat, the entry ranked second shall be assigned to the next to last
		finals heat, the entry ranked third shall be assigned to the third to last
		finals heat, and so on, until one entry has been assigned to each finals heat.

		\item
		After all of the finals heats have one entry assigned to each of them, the
		highest ranked entry of the remaining entries shall be assigned to the last
		finals heat, the next highest ranked of the remaining entries shall be assigned
		to the next to last finals heat, and so on, until all of the entries eligible
		to compete in the finals races have been assigned to finals heats.

		\item
		In the event that this procedure assigns more than one entry from any
		organization into the same finals heat, the heat assignments shall not be
		changed.

	\end{itemize}

\subsubsection{Lane Selection}

	After each entry eligible to compete in the finals races has been assigned to a
	finals heat, the lane that each entry will occupy on Hills 1 and 2 shall be
	chosen. Lanes for each finals heat shall be selected by each entry, with the
	entry having the higher ranking after the preliminary races choosing first. No
	finals entry may choose a lane that is already occupied by a higher ranked
	finals entry.

	Lane selection shall be completed as soon as is practical after heat
	assignments have been made.

\section{Race Rules, Regulations, and Requirements}

	If the Head Judge determines that any entry has violated any of the Race Rules,
	Regulations, and Requirements listed in this section of this document, that
	entry shall be disqualified from the race competition. The Head Judge does not
	require a formal protest from a competitor in order to disqualify an entry from
	the races. Any infraction of the race rules observed by the Head Judge, or
	brought to his or her attention by any of the other judges or officials, can be
	grounds for disqualifying or penalizing the offending entry.


\subsection{Driver and Buggy Qualification Requirements}
	No driver or buggy shall be permitted to participate in a Sweepstakes race heat
	unless all qualification requirements are met. Rolls must only be counted toward
	qualification if the the driver is released from hill 2 and passes Scaife Hall
\label{sec:DriverAndBuggyQualificationRequirements}
	without an accident or collision.

\subsubsection{Driver Requirements}
Drivers must successfully satisfy all of the following requirements:

	\begin{itemize}

		\item 15 minimum total rolls for a driver that did not race in a
		Sweepstakes race the previous year

		\item 10 minimum total rolls for a driver that raced in a
		Sweepstakes race the previous year

		\item 8 roll maximum transfer from fall

		\item Performance demonstration of safe and controlled driving
		during freeroll and push practices.

		\item Course walk with all drivers in a given heat, once per scheduled heat.

		\item Course walks for a given heat must take place on the same day as the
		heat and before the first heat of the day.

		\item Notify the Assistant Chairman, or anyone designated by that Chairman,
		immediately before or after walking the course.

	\end{itemize}

\subsubsection{Buggy Driver Combination Requirements}
	Drivers must successfully meet the following minimum requirements during the spring
	semester in the specific buggy he or she will drive in a given heat:

	\begin{itemize}

		\item 10 total rolls, including transfers from fall
		\item 5 spring rolls, with at least one roll using each
		buggy configuration occurring in the sweepstakes races
		\item Braking Capability Test
		\item Field of Vision test
		\item Pass Test

	\end{itemize}

\subsubsection{Buggy Requirements}

	Each buggy must successfully complete a safety inspection in every race
	configuration, in addition to satisfying all of the prior requirements.

\subsubsection{Exemptions}

	In the case of very special circumstances, such as a long duration of inclement
	weather where no freeroll practices are held, an organization may seek
	permission to race a driver and buggy that do not meet these specific
	requirements. In such cases, the organization must submit an appeal in writing
	to the Sweepstakes Committee, explaining that the failure to meet the
	qualifying standards was through no fault of the driver of the organization.
	The Safety Chairman, with counsel of the Sweepstakes Chairman and Assistant
	Chairman, will evaluate the appeal and either approve or reject it. If an
	appeal is approved by the Safety Chairman, it must then be passed on to the
	Buggy Chairmen to be approved by a majority vote.

\subsection{Rules}

\subsubsection{General}

	No organization may use or attempt to use any buggy preparation area that has
	been assigned to another organization on any day of Sweepstakes racing.

	No organization's entry may use or attempt to use any lane that has been
	assigned to another organization's entry in any Sweepstakes race heat.

	Each organization may allow only one of its representatives to ride in the
	follow car during each race in which one of its entries is competing.

	If any member of any entry's team is substituted for by a member of that
	entry's alternate team, that substitution must be declared to the Sweepstakes
	Chairman before the start of the heat preceding the heat in which that entry is
	scheduled to compete.

\subsubsection{Entries}

	In each class of competition, each organization's entries must compete in the
	heats designated, sorted from fastest to slowest in the order "A, B, C, D".
	Any entry violating this order shall be disqualified from the race competition.
	Determination of whether or not an entry has violated this rule shall be made
	collectively by the Sweepstakes Advisor, the Executive Committee, all Judges,
	and anyone else designated by either the Sweepstakes Advisor or the Sweepstakes
	Chairman.

\subsubsection{Pushers}

	No pusher may use any type of mechanical device with moving parts (such as
	roller skates or a skate board) which could cause that pusher to travel faster,
	while that pusher is pushing a buggy during a Sweepstakes race.

\subsubsection{Buggies}

	The combined weight of a buggy and its driver may not change while that buggy
	is competing in any Sweepstakes race. Weight loss during a race shall be
	permitted only if the Head Judge rules that the weight loss was unintentional,
	the weight loss was not caused by a design failure of the buggy, and the weight
	loss did not interfere with any of the other entries in that heat.

	The loss of any part of a buggy's shell, hatch, or cover while that buggy is
	competing in any Sweepstakes race shall be considered to be a design failure of
	that buggy, and will result in the disqualification of that buggy and its entry
	from the race competition, even if that loss did not interfere with any of the
	other entries in that heat.

	The dimensions of a buggy, not including those of that buggy's pushbar, shall
	not change while that buggy is competing in a heat of the Sweepstakes races.

	No buggy may have any changes made to it between the end of the Design
	Competition and the conclusion of any race that it competes in except for
	changes to its wheels, tires, bearings, and windscreens. Damaged parts of a
	buggy may be changed after the Design Competition and before a race, provided
	that the Safety Chairman is informed of the change, is permitted to view the
	damaged parts, and is given a detailed account of how the parts were damaged.

	Each entry's driver must ride in or on that entry's buggy during the entire
	time that entry is competing in a Sweepstakes race heat. No other person is
	permitted in or on that buggy at that same time.

	Each entry's buggy must take a drop brake test immediately after any race in
	which that entry has competed, whether that buggy completed that race or not.
	Failure to successfully complete this test will result in disqualification of
	that entry from the race in which it just competed and will make that entry
	ineligible for a rerun unless the Safety Chairman and the Head Judge determine
	that the buggy failed the drop brake test because of damage sustained due to
	interference that occurred during the race. Any buggy which fails the drop
	brake test after a race but is subsequently granted a rerun, must successfully
	complete a drop brake test BEFORE it will be permitted to take that rerun.
	These drop brake tests shall be administered by the Safety Chairman, or by
	someone designated by that Chairman. Each buggy shall only be given one chance
	to pass the drop brake test after a race unless any of the following occur:

	\begin{itemize}

		\item
		The drop brake test is not properly administered, for example, if the test
		administrator tells the driver to apply the brakes too soon or too late. In
		this event, the test shall be administered again in the proper manner.

		\item
		The driver unintentionally applies the brakes too soon. In this event the
		driver shall be given one and only one more chance to successfully complete the
		test.

	\end{itemize}

	In the unlikely event that the buggy does not roll after it is initially
	released, or if it does not roll a distance of 30 feet after it is initially
	released, the test administrator shall pull the buggy in a forward direction,
	using a spring scale, or other suitable force measuring device, while the
	driver is actuating the buggy's brakes, in order to determine how much braking
	force the buggy's brakes are able to exert in that direction. The force used to
	pull the buggy shall be applied parallel to, and as close to the ground as is
	practicable. If the buggy's brakes are able to exert a minimum braking force of
	25 pounds in the forward direction, and the driver can release and reapply that
	buggy's brakes once, that buggy shall be considered to have passed the drop
	brake test.

	Any entry's buggy that is involved in any type of accident (such as spinning
	out, having parts fall off while it is rolling, hitting another buggy, being
	hit by another buggy, hitting another object, etc.) during any Sweepstakes race
	will not be permitted to race again until that entry's sponsoring organization
	submits, and has approved, an accident report concerning that accident.
	Accident reports must be submitted to the Safety Chairman or the Sweepstakes
	Chairman. Accident reports can only be approved by the Sweepstakes Chairman or
	the Safety Chairman. The form to be used for these reports can be obtained from
	the Sweepstakes Advisor through the Safety Chairman.

	Any device that allows the driver in the buggy communication with individuals
	outside of the buggy and provides information to a driver that otherwise could
	not be known by the driver are illegal to use on Raceday rolls or when
	performing a pass test. These devices may be used during freeroll and push
	practice; however, use of these prohibited devices on raceday or when more than
	one buggy is rolling at the same time will result in immediate
	disqualification. Examples of the devices are headsets, walkie-talkies,
	telemetry systems, and video display units.

\subsubsection{Starting}

	When the Starter signals the start of an entry's heat, that entry's buggy must
	comply with all of the following:

	\begin{itemize}

		\item The nose of the buggy must be at or behind the starting line.

		\item The buggy must not be moving in the forward direction.

		\item The forward motion of the buggy must not be restrained in any way.

	\end{itemize}

	\noindent When the Starter signals the start of an entry's heat, that entry's
	Hill 1 pusher must comply with all of the following:

	\begin{itemize}

		\item He or she must be touching that entry's buggy.

		\item He or she must have both of his or her feet on the ground.

		\item He or she must not be moving in the forward direction.

		\item He or she must not be using starting blocks, or any similar device.

	\end{itemize}

	After the starter announces that there are five seconds remaining to the start
	of a heat, nobody except the drivers and Hill 1 pushers of the entries in that
	heat, and the starter, may be within five feet of any of the buggies in that
	heat, until that heat begins.

	If the nose of an entry's buggy in any heat is beyond the starting line when
	the Starter signals the start of that heat, that entry shall be considered to
	have false started. If an entry false starts more than two times at the start
	of any single heat, that entry shall be disqualified from that heat, shall not
	be permitted to run in that heat, and shall not be eligible for a rerun.

	If an entry in a heat false starts, all of the buggies in that heat shall be
	brought back to the starting line and that heat shall be restarted. The Starter
	shall restart the countdown at his or her discretion. If the time required to
	restart the heat is greater than 1 minute and any of the entries in that heat
	which did not cause the false start so request,the start of that heat may be
	delayed until the time scheduled for the start of the heat after the current
	heat, and all of the remaining heats shall be moved back for the amount of time
	required for one heat to take place. If the delayed heat is the last scheduled
	heat of the day, it shall be rescheduled for a time that is between 5 and 12
	minutes later than its originally scheduled starting time.

	For example, if Heat 5 is delayed because of a false start, it shall be
	rescheduled for the time at which Heat 7 was originally scheduled to take
	place. Heat 6 would be run at its originally scheduled time. No heat would be
	run at the time originally scheduled for Heat 5. Heat 7 would be run at the
	time originally scheduled for Heat 8 and all subsequent heats would be moved
	back to the time originally scheduled for the heat after them. If Heat 5 was
	the last scheduled heat of the day, no heat would be run at the time originally
	scheduled for Heat 5 and Heat 5 would be run at a time that was between 5 and
	12 minutes later than its original starting time, depending on how much time is
	still available for racing.

	If the Starter's gun fails to fire when the Starter tries to signal the start
	of a heat, all of the buggies in that heat shall be brought back to the
	starting line and that heat shall be restarted. The Starter shall restart the
	countdown at his or her discretion. If the time required to restart the heat is
	greater than 1 minute and any of the entries in that heat so request, the start
	of that heat may be delayed until the time scheduled for the start of the heat
	after the current heat, just as might happen if a false start had occurred.

\subsubsection{Lanes}

	A buggy is considered to be in a lane when no portion of that buggy extends
	beyond or above the innermost edges of the lines delineating that lane.

	The judges shall disqualify an entry from the heat that it is competing in if
	all of the wheels of that entry's buggy are simultaneously out of its assigned
	lane at any point between the starting line and the end of that lane, even if
	that entry doesn't interfere with any other entry.

	If some part of, but not all of, an entry's buggy is out of its assigned lane
	at any point between the starting line and the end of that lane,that entry
	shall be disqualified from the heat that it is competing in if it interferes
	with another entry in that heat, or if one of the other entries in that heat
	protests the lane violation (even if no interference is claimed.)

	A pusher is considered to be in a lane when the greatest part of that pusher's
	body is on or above that lane and no part of that pusher is touching anything
	that is not considered to be in that lane also, except for the buggy which that
	pusher is pushing.

	If some part of, but not all of, one of an entry's pushers is out of that
	entry's assigned lane at any point between the starting line and the end of
	that lane, that entry shall be disqualified from the heat that it is competing
	in, if that pusher interferes with any other entry in that heat, and if the
	entry that was interfered with protests that interference.

\subsubsection{Pushing}

	The position of an entry's pusher along the buggy course is determined by the
	location of the forwardmost part of the pusher's forwardmost foot, i.e. the
	foot that is farthest along the course from the starting line.

	For example, if a pusher's forwardmost foot is on Hill 1, that pusher is
	considered to be on Hill 1. If a pusher's forwardmost foot is in the Hill 1-2
	Transition zone, that pusher is considered to be in that Transition zone even
	if that pusher's rearmost foot is still on Hill 1.

	Exactly one pusher must be designated for each Hill numbered 1-5, and
	may only touch that entry's buggy while he or she is either on their
	designated Hill number, or in a transition zone with the same number.
	For example, the Hill 1 pusher may touch the buggy during Hill 1, and the
	Hill 1-2 Transition Zone. Each entry's Hill 5 pusher must also be in contact
	with that entry's buggy when the nose of that buggy crosses the finish line at
	the end of that entry's heat.

	No organization may cause anyone or anything to pace a pusher of any of that
	organization's entries while that pusher is competing in a race. A person or
	device which is not the next pusher in sequence and which is moving near a
	buggy and moving in the same direction in which that buggy is moving, while in
	view of the pusher currently pushing that buggy, is considered to be a pacer.

	While an entry's pusher is pushing that entry's buggy and after that pusher has
	finished pushing that entry's buggy, that pusher is entitled to run, walk,
	stand, or otherwise be, anywhere in the path which that entry's buggy has
	followed. A buggy's path is considered to be a lane that is 6 feet wide which
	is centered on the centerline of that buggy.

\subsubsection{Driving}
%\label{subsubsec:Driving}

	Each entry's buggy in each heat may not bump into or otherwise contact any
	other entry's buggy in that same heat at any time between the start and finish
	of that heat.

	If during a heat, one entry's buggy passes another entry's buggy, the pass
	shall be considered to be complete whenever the rearmost part of the passing
	buggy (or that buggy's pusher, if it is being pushed at the time) is beyond the
	nose of the buggy being passed.

	If during a heat, one entry's buggy (and pusher, if the buggy is being pushed
	at that time) tries to pass another entry's buggy (and pusher),the passing
	buggy (and pusher) has the primary responsibility of ensuring that the pass is
	completed without contact or any other type of foul between the buggies (or
	pushers). If contact or some other type of foul occurs between the buggies (or
	pushers) during an attempted pass, the judges shall determine which buggy (or
	pusher), if any, is at fault.

	If during a heat, two entries' buggies (and pushers, if the buggies are being
	pushed at the time) are traveling beside each other and neither is clearly
	passing the other, both buggies (and pushers) have the responsibility of
	ensuring that no contact or other type of foul occurs between the buggies (and
	pushers). If contact or some other type of foul occurs between the buggies (or
	pushers), the judges shall determine which buggy (or pusher), if any, is at
	fault.

	If during a heat, an entry's driver stops that entry's buggy because that
	driver considered that an accident was about to take place, that entry shall be
	eligible for a rerun provided that the following conditions are met:

		\begin{itemize}

			\item The Head Judge determines that an accident was about to take
			place.

			\item The accident that was about to occur was not due to any failure
			or foul on the part of the buggy that stopped.

			\item The buggy that stopped is taken to the drop brake test area and
			given a drop brake test as soon as possible after the heat is finished.

		\end{itemize}

	If during a heat, an entry's buggy or pusher is interfered with or fouled, that
	buggy or pusher is not required to stop at that point in the race, in order to
	be eligible for a rerun.

\subsection{Protests and Appeals}

	A protest may be filed by one entry against another entry if the protesting
	entry considers that it was fouled or interfered with by the protested entry
	during a Sweepstakes race.

	A protest may be filed by any participating organization against any entry if
	the protesting organization considers that the protested entry has not complied
	with all of the rules and requirements of the Sweepstakes races.

	An appeal may be filed by an entry if that entry considers that it was
	interfered with while it was competing in a Sweepstakes race by someone or
	something other than any of the other entries in that race.

	Protests and appeals shall be filed as follows:

	\begin{itemize}

		\item
		The Buggy Chairman of the protesting or appealing organization shall verbally
		inform the Sweepstakes Chairman and/or the Head Judge that he or she wishes to
		file a protest or appeal on behalf of his or her entry or organization. This
		verbal notification must be accomplished before the start of the next heat
		after the protested or appealed heat. If the protested or appealed heat is the
		last heat of the day, the notification must be accomplished within ten minutes
		after the end of that heat.

		\item
		The Buggy Chairman of the protesting or appealing organization shall verbally
		describe the reason for the protest or appeal and any information pertinent to
		that protest or appeal to the Head Judge of the Sweepstakes races. If possible,
		protests and appeals should also be submitted in written form, using the same
		form that is used for accident reports, and which can be obtained from the
		Safety Chairman.

	\end{itemize}

\subsubsection{Review}

	The Head Judge will confer with the Sweepstakes Chairman, the Assistant
	Sweepstakes Chairman, the Safety Chairman, the Assistant Head Judge, any of the
	course judges that could have witnessed the alleged incident, any of the
	drivers or pushers that were involved in the incident, and/or anyone else who
	might be able to provide information pertinent to the alleged incident, in
	order to determine as many facts about the incident as possible. Video
	tapes of the incident may also be used for determination of any protests
	or appeals.

\subsubsection{Tape Review}

	The Head Judge shall engage in a tape review of all violations marked by the
	course judges, in addition to all protests and appeals filed with the
	Sweepstakes Chairman. This review shall be performed in	consultation with the
	Assistant Head Judge, Sweepstakes Chairman, Safety Chairman, and Assistant
	Chairman.

	Upon the conclusion of a day of Sweepstakes racing, the Sweepstakes Chairman
	shall acquire the following video footage for tape review:

	\begin{itemize}

		\item CMU TV broadcast tape of the day of Sweepstakes racing

		\item CMU TV finish line camera tape

		\item Starting line Course Judge tape

		\item Transition Zones 1-2, 3-4, and 4-5 Course Judge tapes

		\item Lead car and follow car camera footage

	\end{itemize}


	During the second day of Sweepstakes races, all protests and appeals in question
	that may be granted a rerun shall undergo tape review by the Head Judge in
	consultation with the Sweepstakes Chairman, with the help of CMU TV instant
	replay equipment. This will enable any rerun decisions to be rendered before
	the conclusion of the races. Alternatively, a rerun may be granted without
	tape review, on the stipulation that the rerun may be disqualified upon further
	review of the original protest or appeal.


\subsubsection{Final Decision}

	After reviewing all of the available information the Head Judge will render a
	decision concerning the protest or appeal and will inform the Sweepstakes
	Chairman of that decision. This decision should normally be made after all of the
	races are finished for that day, but it may be made before the start of the second
	heat after the protested or appealed heat at the discretion of the Head Judge
	if circumstances warrant such expedience.

	The Sweepstakes Chairman will confer with all of the affected entries and inform them
	of the judges’ decision, then announce the decision of the judges to the remaining
	race participants and to the public at large.

	\noindent If a protest is filed due to an alleged foul or interference during a race by
	another entry, that protest shall be dispositioned as follows:

	\begin{itemize}

		\item
		If the protest is disallowed, the results of the protested heat shall stand as
		they were.

		\item
		If the protest is allowed and the protesting entry was interfered with or
		fouled, that entry will be granted a rerun.

		\item
		If the protest is allowed, the entry which caused the interference or foul will
		be disqualified from the protested heat.

		\item
		If the Head Judge rules that some type of interference or foul did occur but
		that no violation of the rules occurred, the protesting entry will be granted a
		rerun.

	\end{itemize}

	\noindent If a protest is filed due to an alleged violation of the rules and requirements
	of the Sweepstakes races, that protest shall be dispositioned as follows:

	\begin{itemize}

		\item
		If the protest is disallowed, the protested entry shall not be penalized in any
		way.

		\item
		If the protest is allowed the protested entry will be disqualified from the
		heat in which it competed.

	\end{itemize}

	\noindent If an appeal is filed due to an alleged interference during a race by someone
	or something other than any of the other entries in that race, that appeal
	shall be dispositioned as follows:

	\begin{itemize}

		\item
		If the appeal is disallowed, the results of the appealed heat shall stand as
		they were.

		\item
		If the appeal is allowed the appealing entry shall be granted a rerun.

	\end{itemize}


\subsection{Reruns}

	An entry may be granted a rerun by the Head Judge if all of the following
	occur:

	\begin{itemize}

		\item The entry is either interfered with or fouled while it is competing
		in a Sweepstakes race.

		\item The entry files a protest or an appeal in accordance with the
		procedure for filing protests and appeals.

		\item The Head Judge rules that the protesting or appealing entry was
		interfered with or fouled.

	\end{itemize}

	Any entry which is granted a rerun has the option of taking the rerun or of
	allowing its results in the protested or appealed heat to stand as they are. If
	the entry chooses to take the rerun, the results of the rerun race shall then
	be that entry's official results for the protested or appealed heat.

	If an entry is granted and accepts a rerun because of an incident that occurs
	during the preliminary races, that entry's rerun race shall be scheduled to
	take place on the same day that the finals races are scheduled to take place.

	If an entry is granted and accepts a rerun because of an incident that occurs
	during the rerun races, that entry's rerun race shall be scheduled to take
	place after all of the women's finals races are finished, but before the men's
	finals races are started.

	If an entry is granted and accepts a rerun because of an incident that occurs
	during the finals races, that entry's rerun race shall be scheduled to take
	place after all of the finals races are finished. If that entry competed in the
	last scheduled finals race, its rerun race shall be scheduled to start between
	seven and 12 minutes after the end of that last finals race.

	If an entry is granted a rerun and wishes to accept that rerun but is not able
	to do so because of damage to its buggy or some other reason beyond its
	control, that entry's finishing time for its rerun race shall be its fastest
	time from any race that it completed during that Sweepstakes competition.


\subsection{Safety}

\subsubsection{Buggies}

	In order to ensure that all of the buggies competing in the Sweepstakes races
	are in compliance with all applicable safety rules, regulations, and
	requirements, any buggy may be given a spot safety check by the Sweepstakes
	Chairman, the Assistant Sweepstakes Chairman, the Safety Chairman, or anyone
	designated by the Sweepstakes Advisor, immediately after that buggy has
	finished competing in a race and before that buggy's driver has been removed
	from that buggy. Any buggy found to be out of compliance with any of the
	applicable safety rules, regulations, or requirements during one of these spot
	safety checks shall be disqualified from all of the men's races, the women's
	races, and the design competition for that school year.
	\newline

	\noindent The following spot safety checks are required to be administered:

	\begin{itemize}

		\item
		During all of the men's preliminary and rerun races, any entry which finishes
		its race with a time that is faster than the tenth fastest time recorded in the
		men's preliminary and rerun races of the previous year's Sweepstakes
		competition, shall have its buggy inspected before the driver is removed from
		that buggy after that race.

		\item
		During all of the women's preliminary and rerun races, any entry which finishes
		its race with a time that is faster than the sixth fastest time recorded in the
		women's preliminary and rerun races of the previous year's Sweepstakes
		competition, shall have its buggy inspected before the driver is removed from
		that buggy after that race.

		\item
		Any entry which competes in a finals race shall have its buggy inspected before
		the driver is removed from that buggy after that finals race.

	\end{itemize}

\subsubsection{Buggy Preparation Areas}
\label{subsec:Buggy Prep}
%This section needs some formatting work but it's ugly and probably won't change much
%so i'm leaving it for another day.

	In order to reduce the possibility of accidents or injuries in the areas in
	which the buggies are prepared for the races, these areas shall be randomly
	inspected both before and during the Sweepstakes races by fire marshals, the
	Safety Chairman, or anyone else designated by the Dean of Student Affairs or
	the Sweepstakes Advisor. Any organization found to have unsafe or dangerous
	conditions in their buggy preparation area before or during the Sweepstakes
	races by any of these safety inspectors shall be fined the amount of \$100.00
	AND shall have ALL of their buggies immediately disqualified from all of the
	men's races, the women's races, and the design competition for that school
	year. In addition, any organization found to have ANY combustible liquids
	and/or ANY source of open flame in or near their buggy preparation area before
	or during the Sweepstakes races shall not be permitted to participate in ANY
	activity related to the Sweepstakes Competition for a period of 15 months from
	the date on which the violation is discovered. The following safety
	requirements apply AS A MINIMUM, to all buggy preparation areas both during and
	immediately before all Sweepstakes races:

	NO quantity of ANY combustible liquid is permitted in or near the buggy
	preparation areas. (The ONLY exceptions to this requirement shall be quantities
	of lubricating fluid not greater in volume than two fluid ounces used to
	lubricate wheel bearings and the working fluids contained in motor vehicles and
	needed for their proper operation.)

	No source of open flame is permitted in or near the buggy preparation areas.

	All electrical apparatus and power cables MUST be grounded.

	Each organization is required to have a minimum of two fire extinguishers in
	their buggy preparation area. Each of these fire extinguishers must contain a
	minimum of ten pounds of extinguishing agent. These fire extinguishers must be
	rated such that Class A, Class B, and Class C fires can all be extinguished.
	Class A fires consist of ordinary combustible materials, such as wood, paper,
	textiles, etc., and they require cooling and quenching. Class B fires consist
	of flammable liquids and greases, such as gasoline, hexane, oils, paints, etc.,
	and they require smothering. Class C fires consist of electrical equipment,
	such as motors, switches, cables, etc., and they require a nonconducting agent
	capable of extinguishing a fire in materials that might be present. All fire
	extinguishers must be adequately charged and fully operational.

	NO alcoholic beverages are permitted in or near the buggy preparation areas.
	Alcoholic beverages to be used in post race celebrations may be stored near the
	buggy preparation areas, such as in the cab section of a truck, provided that
	none of them are opened or consumed until AFTER all of the races for that day
	have been completed.

	No member of an organization who is in that organization's buggy preparation
	area may be intoxicated or under the influence of any type of mind altering
	substance.

	If any organization uses any type of enclosed or semi-enclosed structure as a
	buggy preparation area, such as a truck, a tent, a free standing building,
	etc., the following requirements shall apply to that structure:

	Smoking shall NOT be permitted inside that structure.

	The inside of that structure shall NOT be sealed off from the outside of that
	structure by any type of solid door or cover at ANY time that ANY person is
	inside that structure. Trucks must have their rear doors COMPLETELY open
	whenever there are people inside the rear section of those trucks. Curtains or
	other types of non-sealing partitions are permissible in order to prevent
	people outside the structure from looking inside the structure.

	Whenever a buggy preparation area is completely enclosed and there are people
	inside that area, it is recommended that fresh air be circulated through that
	area with some type of powered ventilating fan or blower.

\section{Final Standings}

	Final standings in each class of competition shall be determined separately,
	but by the same method. If all of the preliminary races in a class of
	competition are not completed, no final standings shall be declared.

\subsection{One Day Races}

	If only one day of racing is completed in a class of competition, such that all
	of the preliminary races in that class of competition are completed but not all
	of the rerun races and finals races in that class of competition are completed,
	the final standings of the Sweepstakes races in that class of competition shall
	be determined as follows:
	\newline

	If no entries were granted reruns, all of the entries that completed a
	preliminary race and were not disqualified shall be ranked in order of
	finishing time. The entry with the fastest finishing time shall be awarded
	first place, the entry with the second fastest finishing time shall be awarded
	second place, and so on, to the entry with the slowest finishing time in the
	preliminary races.

	If one or more entries were granted reruns and all of the granted rerun races
	were completed, the finishing times from the rerun races shall be considered to
	be the preliminary race times for the rerun entries, and the final standings
	shall be determined in the same manner as if no reruns had been granted.

	If one or more entries were granted reruns but not all of the granted rerun
	races were completed, the final standings shall be determined in the same
	manner as if no reruns had been granted, i.e. only the actual finishing times
	from the preliminary races shall be used to determine the final standings.

\subsection{Two Day Races}

	If both days of racing are completed in a class of competition, such that all
	of the preliminary races, rerun races, and finals races are completed in that
	class of competition, the final standings of the Sweepstakes races in that
	class of competition shall be determined as follows:
	\newline

	If no entries were granted reruns, all of the entries that completed a finals
	race and were not disqualified shall be ranked in order of finishing time in
	those finals races. The entry with the fastest finishing time in the finals
	races shall be awarded first place, the entry with the second fastest finishing
	time in the finals races shall be awarded second place, and so on, to the entry
	with the slowest finishing time in the finals races. If any entries were
	disqualified in the finals races, they shall be ranked below all of the entries
	that were not disqualified in the finals races and the ranking among those
	disqualified entries shall be the same as their relative ranking after the
	preliminary races.

	If one or more entries were granted reruns and any rerun races that took place
	were held on the same day that the finals races were held, the finishing times
	from the rerun races shall be considered to be finishing times for the finals
	races, just as if no entries had been granted reruns. If any entries were
	disqualified in the rerun or finals races, they shall be ranked below all of
	the entries that were not disqualified in the rerun or finals races and the
	ranking among those disqualified entries shall be the same as their relative
	ranking after the preliminary races.


