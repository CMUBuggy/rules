\chapter{Race Rules, Regulations, and Procedures}

\section{Sweepstakes Race Schedule}

The Sweepstakes races shall be scheduled to be held at or near the same time that the Carnegie Mellon University Spring Carnival is held each school year. The races will usually be scheduled for two consecutive days.

\subsection{First Day of Racing}

The preliminary races shall be held on the first day of Sweepstakes Racing, with the women's preliminary races being held first, followed by the men's preliminary races.

\subsection{Second Day of Racing}

The alumni/exhibition races, the rerun races, and the finals races shall be held on the second day of Sweepstakes racing. If alumni/exhibition races are to be run, they shall be scheduled first, followed by the women's rerun races (if necessary), the men's rerun races (if necessary), the women's finals races, and the men's finals races, in that order. The order in which these races are scheduled may be changed at the discretion of the Sweepstakes Chairman or the Sweepstakes Advisor, based on weather forecasts and/or time constraints.

\subsection{Cancellation}

Either or both days of Sweepstakes racing may be canceled by the Dean of Student Affairs, the Sweepstakes Advisor, or the Sweepstakes Chairman, due to inclement weather, inadequate police protection, inadequate communications, lack of medical personnel, vehicles parked on the course or on the sidewalks around the course, or any other condition which might endanger the contestants or spectators of the races.

If the preliminary races are canceled on the day that they were originally scheduled to take place (Friday), they shall be rescheduled for the day that the finals races were originally scheduled to take place (Saturday), and no finals races shall be held that year.

If the preliminary races are run as originally scheduled (on Friday) and the finals races are canceled on the day that they were originally scheduled to take place (Saturday), no finals races shall be held that year.

If the preliminary races are canceled on the day that they were originally scheduled to take place (Friday), and then are canceled again on the day for which they were rescheduled (Saturday), the preliminary races may be rescheduled again and held on the Sunday of Spring Carnival or a later date, at the discretion of the Sweepstakes Advisor.

\section{Race Day Procedures}

\subsection{Time and Place}

Sweepstakes races shall only be held on the buggy course and only on the dates and at the times specified by the Sweepstakes Advisor. Sweepstakes races will generally be held between the hours of 6:00 am and 2:00 pm on the Friday and Saturday of Spring Carnival weekend. Sweepstakes races may be held on other days and at other times at the discretion of the Sweepstakes Advisor.

\subsection{Permits}

Sweepstakes races shall only be held with the approval of the City of Pittsburgh and the Department of Parks and Recreation. This approval shall be in the form of permits to use the public streets on campus and in Schenley Park, issued by both the City of Pittsburgh and by the Department of Parks and Recreation.

Applications for these permits should be made by the, Sweepstakes Advisor in cooperation with the Sweepstakes Chairman, at the same time that applications for permits for spring freeroll practices are made.

\subsection{Police}

Sweepstakes races shall only be held with the protection and cooperation of both the Police Department of the City of Pittsburgh, and the Carnegie Mellon University Campus Police Department.

Off-duty City of Pittsburgh police officers, usually from the Park Police Department, are hired by the Sweepstakes Committee and Carnegie Mellon University to provide police protection during all of the Sweepstakes races. Arrangements to have these officers present during the races should be made by the Sweepstakes Chairman, in cooperation with the Sweepstakes Advisor, at the same time that the permits to use the streets are applied for.

Usually a minimum of four officers are needed to provide protection during the races. They should be available during the entire time that the races are underway, and should report to the Sweepstakes Chairman or the Sweepstakes Advisor on each day of racing, at least 30 minutes before the races are scheduled to begin that day. These officers should be stationed as follows:
\begin{itemize}
	\item One on Schenley Drive near the clubhouse for the Schenley Park Golf Course.
	\item One on Circuit Road at its intersection with Schenley Drive, near the George Westinghouse Memorial Pond.
	\item One on Panther Hollow Road at its intersection with Schenley Drive, near the north end of the Panther Hollow Bridge.
	\item One on Schenley Drive at the eastern end of the Schenley Bridge, near its intersection with Frew Street. (This officer might alternatively be stationed at the western end of the Schenley Bridge.)
\end{itemize}

Carnegie Mellon University Campus Police should be available during the entire time that the races are underway in the event that their assistance is needed. Arrangements to have these officers present during the races should be made by the Sweepstakes Chairman, in cooperation with the Sweepstakes Advisor.

\subsection{Lane and Zone Markings}

Lines delineating the lanes on Hills 1 and 2, the starting line, the finish line, and the beginnings and ends of the three transition zones must be painted on the buggy course. At the discretion of the Sweepstakes Chairman, the responsibility of painting these lane and zone markings on the buggy course for the Sweepstakes races may be delegated to one organization. These lines should normally be painted by a date that is at least two weeks before the date scheduled for the preliminary races. Paint and the striper to paint the lines should be obtained through the Sweepstakes Advisor.

If the organization responsible for painting lane and zone markings fails to provide them without adequate reason, such as bad weather, before the last scheduled freeroll practice before the races, that organization shall be fined the amount of \$25.00.

\subsection{Buggy Preparation Areas}

Each organization participating in the Sweepstakes races shall be permitted to select an area near the starting line of the buggy course, in which they may prepare their buggies prior to the races. These areas may be occupied by trucks or any other non-permanent enclosures. The order of selection of these areas shall be the same as the seeding order for the preliminary races. The selection of these areas should be made at least four weeks prior to the date scheduled for the preliminary races. The selection process shall be supervised by the Sweepstakes Chairman, or anyone designated by that Chairman. The Sweepstakes chairman will assign spaces to organizations at least 2 weeks prior to the date schedule for the preliminary races. No organization may use or attempt to use any area that has been selected by another organization on any day of Sweepstakes racing.

\subsection{Electrical Power}

Electrical power may be made available on each day of Sweepstakes racing, to each organization participating in the races. If available, this electrical power will be located near each organization's buggy preparation area, and will nominally be 110 volt alternating current, with a maximum current capacity of 20 amperes. (This will provide a maximum nominal power of 2,200 watts.) Two standard 110 VAC electrical outlets will be provided to each participating organization. Each organization will be responsible for providing adequate grounded cabling to transfer this power from the location of the outlets, to their individual buggy preparation areas.

When available, the electrical power near the buggy preparation areas will be provided by the Physical Plant Department of Carnegie Mellon University. Arrangements to provide this power should be made with the Physical Plant Department by the Sweepstakes Chairman, through the office of the Sweepstakes Advisor, at least six weeks before the races are scheduled to take place.

\subsection{Finish Line}

A platform several feet high should be placed on the northern sidewalk of Frew Street, directly in line with the finish line of the buggy course. This platform will provide a vantage point for the Sweepstakes race timers, so that they have a better view of the finish line of the race.

A flatbed truck is usually used as the platform at the finish line. This truck can usually be obtained from the Physical Plant Department of Carnegie Mellon University. Arrangements to borrow this truck should be made by the Sweepstakes Chairman, approximately six weeks before the preliminary races are scheduled to take place, through the office of the Sweepstakes Advisor.

\subsection{Course Watch}

The night before each scheduled day of Sweepstakes racing, the Sweepstakes Chairman, or anyone designated by that Chairman, shall assign representatives of each participating organization to watch various parts of the buggy course in order to prevent motor vehicles from parking on any part of the course. Each participating organization must provide two people to watch the course and a motor vehicle for those people to ride in. The length of time that each person must watch the course and the area that they must watch shall be determined by the Sweepstakes Chairman.

Any organization that fails to provide the required number of people and vehicles for course watch duty on the night before any day of Sweepstakes racing, shall be fined the amount of \$25.00 for each missing person or vehicle.

\subsection{No-Parking Signs}

Before each scheduled day of Sweepstakes racing, No-Parking signs shall be placed around the buggy course in order to prevent cars and other motor vehicles from parking there. The signs shall be obtained from the Police Department of the City of Pittsburgh, with the assistance of the Sweepstakes Advisor, if necessary. The signs shall be put in place around the buggy course as early as 8:00 pm, but NO LATER THAN 11:00 pm, the night before each day of Sweepstakes racing is scheduled. They shall be removed at the time that the course is officially closed to vehicular traffic for each day of races, or the cancellation of the races for that day, unless otherwise specified by the Sweepstakes Chairman or the Sweepstakes Advisor.

At the discretion of the Sweepstakes Chairman, the responsibility of obtaining, storing, and placing No-Parking signs in position for each scheduled day of Sweepstakes racing may be delegated to one organization. The organization charged with this responsibility may not have to provide sweepers, flaggers, or course marshals for each day of Sweepstakes racing, at the discretion of the Sweepstakes Chairman.

If the organization responsible for the No-Parking signs fails to provide them or remove them for any day of Sweepstakes racing, that organization shall be fined the amount of \$50.00.

\subsection{Course Inspection and Official Notification}

Approximately 30 minutes before the scheduled start of the first race on each scheduled day of Sweepstakes racing, the Sweepstakes Chairman and the Safety Chairman shall inspect the buggy course (if necessary) and decide if Sweepstakes races can be safely held that day. If no races are to be held, representatives of all participating organizations and any other people involved with the running of the races shall be notified by the person or persons who made the decision.

\subsection{Sweepers}

Each organization shall provide two sweepers for each scheduled day of Sweepstakes racing, to help clean debris from the buggy course. Each organization shall equip their sweepers with brooms (preferably large push type brooms) and/or shovels. These sweepers must be available from a time that is four hours before the races are scheduled to start, until the races are finished for that day. The Sweepstakes Chairman, or anyone designated by that Chairman, shall determine when and where the sweepers must report for duty.

Any organization that fails to provide the required number of properly equipped sweepers for any day of Sweepstakes racing shall be penalized by having one entry withdrawn, and the entry fee for that entry forfeited.

At the discretion of the Sweepstakes Chairman, some organizations may not be required to provide sweepers, in lieu of providing other services on each day of Sweepstakes racing.

\subsection{Hay Bales}

Sweepstakes races shall only be held when an adequate number of hay bales are in place around the buggy course. The Safety Chairman, or anyone designated by that Chairman, shall determine how many hay bales are required in order to have a Sweepstakes race, and where around the buggy course those hay bales shall be placed, in order to provide the maximum amount of protection to the Sweepstakes race participants.

Bales of hay, as opposed to bales of straw, are usually used because they tend to hold up better and therefore can be used more times before they start to fall apart. Approximately 120 hay bales (in addition to those already obtained for freeroll practices) should be obtained for use during Sweepstakes races. The hay bales should be placed along both curbs of the western end of Frew Street where it intersects with Schenley Drive. For races, bales should be set up as they are for normal freeroll practices, except that the curbs should be lined two bales deep for added cushion.

At the discretion of the Sweepstakes Chairman, the responsibility of obtaining, storing, and placing hay bales in position for each day of Sweepstakes racing may be delegated to one organization. The hay bales shall be put in place no later than two hours before races are scheduled to start on each day of Sweepstakes racing, and they should be in place before the sweepers clean that part of the course. They shall be removed within 45 minutes of either the end of the races for that day, or the cancellation of the races for that day. The organization charged with this responsibility may not have to provide sweepers, flaggers, or course marshals for each day of Sweepstakes racing, at the discretion of the Sweepstakes Chairman.

If the organization responsible for the hay bales fails to provide them or remove them for any day of Sweepstakes racing, that organization shall be fined the amount of \$50.00.

\subsection{Flaggers}

Each organization shall provide two flaggers for each scheduled day of Sweepstakes racing, to help control vehicular traffic on the buggy course. These flaggers must be available from a time that is one hour before the races are scheduled to start, until the races are finished for that day. Each organization shall equip their flaggers with reflective vests and flags which must be used while they are acting as flaggers. All vests and flags must be approved by the Safety Chairman or anyone designated by that Chairman. The Sweepstakes Chairman, or anyone designated by that Chairman, shall determine when and where the flaggers must report for duty.

Any organization that fails to provide the required number of properly equipped flaggers for any day of Sweepstakes racing shall be penalized by having one entry withdrawn, and the entry fee for that entry forfeited.

At the discretion of the Sweepstakes Chairman, some organizations may not be required to provide flaggers, in lieu of providing other services on each day of Sweepstakes racing.

\subsection{Barricades}

Portable wooden barricades shall be placed at several locations near the buggy course while Sweepstakes races are in progress, to stop and/or redirect vehicular traffic when this traffic tries to approach the area of the buggy course. Warning signs which indicate that the road ahead is, or may be, closed, and that there will be flaggers ahead to stop and/or redirect traffic shall be placed near these barricades also. The barricades shall be put in place no later than one hour before races are scheduled to start on each day of Sweepstakes racing. They shall be removed within 15 minutes of either the end of the races for that day, or the cancellation of the races for that day. The Safety Chairman, or anyone designated by that Chairman, shall determine if enough barricades are in place in order to have a Sweepstakes race. In order to provide the maximum amount of protection to the Sweepstakes race participants, barricades should be placed at least at the following locations:
\begin{itemize}
	\item On Margaret Morrison Street, at its intersection with Tech Street.
	\item On Frew Street on the eastern side of its intersection with Tech Street.
	\item On Schenley Drive, just east of its intersection with Tech Street.
	\item On Circuit Road at its intersection with Schenley Drive, near the George Westinghouse Memorial Pond.
	\item On Panther Hollow Road at its intersection with Schenley Drive, near the northern end of the Panther Hollow Bridge.
	\item On Schenley Drive at the eastern end of the Schenley Bridge, near its intersection with Frew Street. The placement of the barricades at this location should be performed carefully. Some space should be left on the northern side of the bridge so that if a buggy failed to make the turn onto Frew Street it would have adequate room to drive across the bridge in the right-hand lane.
	\item On the Scaife Hall driveway at its intersection with Frew Street.
\end{itemize}

At the discretion of the Sweepstakes Chairman, the responsibility of obtaining, storing, and placing barricades in position for Sweepstakes races may be delegated to one organization. The organization charged with this responsibility may not have to provide sweepers, flaggers, or course marshals for each day of Sweepstakes racing, at the discretion of the Sweepstakes Chairman.

If the organization responsible for the barricades fails to provide them or remove them for any day of Sweepstakes racing, that organization shall be fined the amount of \$50.00.

\subsection{Warning Signs}

Warning signs shall be placed at several locations near the buggy course while Sweepstakes races are in progress, to warn vehicular traffic when this traffic tries to approach the area of the buggy course. The signs shall indicate that the road ahead is, or may be, closed, and that there will be flaggers ahead to stop and/or redirect traffic. The warning signs shall be put in place no later than one hour before races are scheduled to start on each day of Sweepstakes racing. They shall be removed within 15 minutes of either the end of the races for that day, or the cancellation of the races for that day. The Safety Chairman, or anyone designated by that Chairman, shall determine if enough warning signs are in place in order to have a Sweepstakes race. In order to provide the maximum amount of protection to the Sweepstakes race participants, warning signs should be placed at all of the locations where barricades have been placed, plus the following locations:
\begin{itemize}
	\item On Margaret Morrison Street at its intersection with Forbes Avenue.
	\item On Schenley Drive at its intersection with Forbes Avenue, near the clubhouse for the Schenley Park Golf Course.
	\item On Circuit Road at its intersection with Serpentine Drive.
	\item On Panther Hollow Road at the southern end of Panther Hollow Bridge.
	\item On Schenley Drive between the Mary E. Schenley Memorial Fountain and the southwest comer of the Carnegie Museum building.
	\item On the driveway to the rear of Hamburg Hall at its intersection with Forbes Avenue.
\end{itemize}

At the discretion of the Sweepstakes Chairman, the responsibility of obtaining, storing, and placing warning signs in position for Sweepstakes races may be delegated to one organization. The organization charged with this responsibility may not have to provide sweepers, flaggers, or course marshals for each day of Sweepstakes racing, at the discretion of the Sweepstakes Chairman.

If the organization responsible for the warning signs fails to provide them or remove them for any day of Sweepstakes racing, that organization shall be fined the amount of \$50.00.

\subsection{Crowd Control Barriers}

Crowd control barriers should be used to restrain race spectators located along Hill 2 and Hill 5. These barriers can be ropes strung between stanchions, or even held by course marshals. The barriers should be placed along Hill 2 on each side of the buggy course, from the Hill 1-2 Transition Zone to the end of the lanes, and along Hill 5 on only the northern side of the course, from a point that is approximately 200 feet from the finish line to the finish line. On Hill 2 the barriers should be placed about 3 feet outside of the outermost edges of the lanes, and on Hill 5 they should be placed even with the northern curb of Frew Street. The barriers shall be put in place no later than 30 minutes before races are scheduled to start on each day of Sweepstakes racing. They shall be removed within 15 minutes of either the end of the races for that day, or the cancellation of the races for that day.

At the discretion of the Sweepstakes Chairman, the responsibility of obtaining, storing, and placing crowd control barriers in position for Sweepstakes races may be delegated to the same organization responsible for placing warning signs in position for the Sweepstakes races. The organization charged with this responsibility may not have to provide sweepers, flaggers, or course marshals for each day of Sweepstakes racing, at the discretion of the Sweepstakes Chairman.

If the organization responsible for the crowd control barriers fails to provide them or remove them for any day of Sweepstakes racing, that organization shall be fined the amount of \$50.00.

\subsection{Course Marshals}

Each organization shall provide at least two course marshals for each scheduled day of Sweepstakes racing, to help control the race spectators on or near the buggy course. These course marshals must be available from a time that is one hour before the races are scheduled to start, until the races are finished for that day. Each organization shall equip at least two of the course marshals which it provides with reflective vests and flags which must be used while they are acting as marshals. Additional marshals must be equipped with some sort of brightly colored distinguishing attire, such as vests, hats, arm-bands, etc. so that they may be easily recognized by the race participants and spectators. The Sweepstakes Chairman, or anyone designated by that Chairman, shall determine when and where the course marshals must report for duty.

Any organization that fails to provide the required number of properly equipped course marshals for any day of Sweepstakes racing, shall be fined the amount of \$25.00 for each missing or improperly equipped course marshal.

At the discretion of the Sweepstakes Chairman, organizations charged with other responsibilities, such as hay bales, no-parking signs, barricades, warning signs, etc., may not have to provide course marshals for each day of Sweepstakes racing.

\subsection{Course Communications}

Sweepstakes races shall only be held when adequate radio communication equipment is available to provide voice communications around the buggy course for automobile traffic control, buggy traffic control, and emergency situation assistance. Radio communication equipment, and the personnel to operate it, are usually available through the Carnegie Mellon University Radio Club. Any personnel helping to provide radio communications should not be responsible for making decisions concerning what happens during the Sweepstakes races, but instead should be providing information to the Sweepstakes Chairman and his or her assistants, in order that they may make any necessary decisions.

\subsection{Traffic Control}
Control of vehicular traffic on the buggy course on each day of Sweepstakes racing will be handled by the City of Pittsburgh police officers, usually from the Park Police Department, who are hired by the Sweepstakes Committee and Carnegie Mellon University to provide police protection during all of the Sweepstakes races. These officers, with the assistance of the flaggers provided by the participating organizations, will stop vehicular traffic from entering the buggy course while the races are in progress. If possible, the buggy course will be completely closed to vehicular traffic from the time that the races begin each day, until all of the races are finished for that day. If necessary, the officers will open the buggy course to traffic one or more times during the course of the day's races to relieve traffic buildup on the streets around the buggy course. In this event, adequate notice must be given to the Sweepstakes Chairman and the official Starter of the races so that no race heat is started while there is vehicular traffic on the buggy course.

\subsection{Lead Car}

During each Sweepstakes race a motor vehicle known as the lead car shall drive around the buggy course ahead of the leading buggy in that race. The lead car should stay several hundred feet in front of the leading buggy at all times so as not to interfere with that buggy during the race. The Head Judge and the Sweepstakes Chairman shall ride in the rear portion of the lead car so that they may observe each race from that vantage point. Other people involved with the operation of the Sweepstakes races may ride in the lead car at the discretion of the Sweepstakes Chairman. The Sweepstakes Chairman shall be responsible for giving verbal instructions to the driver of the lead car during each race, so that the car may be kept at a safe and consistent (from race to race) distance in front of the leading buggy in that race.

The vehicle used as the lead car should be either a convertible top car or some sort of pick-up truck, in order to afford the best view of the race possible to the officials riding in it. If this vehicle is a pick-up truck, its tail-gate must always be closed while the vehicle is moving in order to reduce the chances of anyone falling out of it. The vehicle used as the lead car shall be obtained by the Sweepstakes Chairman with the assistance of the Sweepstakes Advisor. The driver of the lead car shall be determined by the Sweepstakes Advisor.

In the event of an accident during a heat, the lead car should continue along the buggy course, as long as there is at least one buggy still continuing with the race. If all of the buggies in the heat come to a stop, the lead car should stop near the leading buggy in order to render any necessary assistance.

\subsection{Follow Car}

During each Sweepstakes race a motor vehicle known as the follow car shall drive around the buggy course behind the trailing buggy in that race. The follow car should stay approximately one hundred feet behind the trailing buggy at all times so as not to interfere with that buggy during the race. The Assistant Head Judge and one representative of each entry in the heat underway shall ride in the follow car so that they may observe that heat from that vantage point and provide any assistance that may be necessary during that heat. Other people involved with the Sweepstakes races may ride in the follow car at the discretion of the Sweepstakes Chairman.

The vehicle used as the follow car should be either a convertible top car or some sort of pick-up truck, in order to afford the best view of the race possible to the people riding in it. If this vehicle is a pickup truck, its tail gate must always be closed while the vehicle is moving in order to reduce the chances of anyone falling out of it. It is also advisable to have a framework type of structure extending up from the sides of the pick-up truck bed, in order to help prevent anyone from falling out over the sides of the truck bed. The vehicle used as the follow car shall be obtained by the Sweepstakes Chairman with the assistance of the Sweepstakes Advisor. The driver of the follow car shall be determined by the Sweepstakes Advisor.

In the event of an accident during a heat, the follow car should be stopped near the scene of the accident in order that the people in the follow car may render assistance to those involved in the accident. The members of the racing organizations riding in the follow car MUST have any tools or devices that are necessary to quickly remove any of their drivers from their buggies. If an organization fails to have these tools or devices in the car following the buggies during any race, that organization shall be fined the amount of \$25.00 and the buggy in question shall be disqualified from that race.

\subsection{Signal Flaggers}

Each organization participating in the Sweepstakes races must provide a signal flagger for each of its entry's buggy drivers. These required signal flaggers shall be known as chute flaggers. Chute flaggers should provide a signal to the buggy drivers so that the drivers know when to start the right hand turn from Schenley Drive onto Frew Street. Chute flaggers should usually be positioned on the southern curb of Schenley Drive, just east of the intersection of Frew Street. Chute flaggers are not permitted to be on the street portion of Schenley Drive during any Sweepstakes race, unless they receive specific approval from the Safety Chairman. In general, if these flaggers need to position their flags more than an arms length away from the curb, their signal flags should be attached to extension poles so that they may hold these flags out over the street while still standing on the curb.

Each organization's chute flagger must be able to provide an alternate signal to their buggy drivers (such as waving the flag in a particular manner or using different colored flags) which will indicate to those drivers that there is a problem farther ahead and that the driver should start slowing down in a controlled manner.

People providing course communications (usually the Carnegie Mellon University Radio Club) may also signal buggy drivers that there is a problem farther ahead. This will be accomplished by waving a YELLOW flag so that the buggy driver can see it. This yellow flag indicates that the buggy driver should start slowing down in a controlled manner so that he or she may stop as soon as it is possible to do so safely. No organization may use a yellow colored flag for any purpose other than to indicate a problem ahead.

Each organization's chute flagger must have adequate experience relative to chute flagging. Each chute flagger should talk to the drivers that he or she will be flagging for, in order to determine where those drivers would like the flag to be placed. If possible each chute flagger should walk the buggy course with those drivers before the Sweepstakes races start each day that they will be flagging.

If a chute flagger is considered to be acting in an unsafe or inexperienced manner, the Safety Chairman, or anyone designated by that Chairman, may require that flagger to be replaced with another more experienced flagger.

If an organization fails to provide a chute flagger for any of its buggy drivers during any Sweepstakes race, that organization shall be fined the amount of \$25.00 each time one of its buggies attempts to make the turn from Schenley Drive onto Frew Street without a chute flagger.

Each organization participating in the Sweepstakes races may also provide additional signal flaggers for its drivers, in order to provide them with additional information about the buggy course or about the race that they are competing in, such as where the other buggies in that heat are at that time. These flaggers may be located anywhere around the buggy course,such as just after Hill 2 or near the entrance to Phipps Conservatory. These flaggers are not permitted to stand anywhere on the course whenever a race is in progress, unless they receive specific approval from the Safety Chairman.

Each organization is encouraged to provide these additional flaggers to help their drivers during their heats. They are also encouraged to discuss flagging strategies with the other entries in their heats, so as not to confuse each others drivers.

\subsection{Accidents}

If an accident occurs during the Sweepstakes races any medical personnel in attendance will be dispatched to the scene of that accident as quickly as possible. They alone will have the responsibility of determining the condition of any victims and of providing any needed first aid.

Before the medical personnel arrive at the scene of an accident, common sense should be observed. The victim should not be moved or disturbed in any way unless there is a greater danger in not doing so. If a buggy is involved in an accident and comes to rest in a position where it might be impacted by another buggy in the same heat, that buggy should be immediately moved, as quickly and as gently as possible, to a position of safety.

If a buggy needs to be immediately moved from a dangerous position after an accident, anyone near it may do so, even if they are not members of the buggy's sponsoring organization. If there is no immediate danger however, the buggy and its driver should not be moved or disturbed until the medical personnel arrive and assess the situation. In general, a buggy driver should not be removed from a buggy until medical personnel advise that it is wise to do so. In ABSOLUTELY ABSOLUTELY NO CASE should an unconscious driver be removed from a buggy without the supervision of medical personnel.

\subsection{Medical Personnel}

Medical personnel should be available at all times while the Sweepstakes races are underway. off duty paramedics from the City of Pittsburgh can usually be obtained to provide any necessary medical assistance for each day of racing. Arrangements to have paramedics available during the races should be made by the Sweepstakes Chairman, approximately six weeks before the preliminary races are scheduled to take place, through the office of the Sweepstakes Advisor.

If an active Emergency Medical Service exists on the Carnegie Mellon University campus, members of that organization may also be obtained to help provide medical assistance during the races. Arrangements to have members of a campus Emergency Medical Service available during the races should be made by the Sweepstakes Chairman, approximately six weeks before the preliminary races are scheduled to take place, with the assistance of the Sweepstakes Advisor.

\subsection{Clean-Up}

After each day of Sweepstakes racing has been completed, all debris on the buggy course and on the sidewalks around the course must be cleaned up and disposed of properly. Special care must be taken to ensure that all debris left by any race participants is removed, such as empty food and beverage containers, duct tape, buggy preparation materials, hay that has fallen off hay bales, No-Parking signs that have been misplaced, etc.

\subsection{officials}

All of the Sweepstakes race officials should be given some sort of distinguishing attire to wear during the races, so that they may be easily recognized by the race participants and race spectators. This attire should be brightly colored so that it is readily visible. It may consist of armbands, hats, shirts, vests, or any other type of distinguishing attire. This attire shall be provided by the Sweepstakes Chairman in conjunction with the Sweepstakes Advisor.

The following Sweepstakes officials shall wear this distinguishing attire: the Sweepstakes Advisor, the Sweepstakes Chairman, the Assistant Chairman, the Safety Chairman, all assistants to the above people, the Starter,the Head Judge, the Assistant Head Judge, all course judges, all timers,and all course marshals.

\subsection{Starter}

A person shall be appointed by the Sweepstakes Chairman, with the approval of the Sweepstakes Advisor, to be the official Starter of all of the Sweepstakes races. The Starter shall be located near the starting line of the buggy course before and during all of the races, usually just to the right of Lane 1 when looking south from the starting line. The Starter shall announce the time remaining until the start of each heat before each heat and shall announce any delays or holds in the countdown to each heat. The Starter shall use a traditional starting gun, or other similar device,to indicate the start of each heat. Two starting guns should always be available in the event that the first one used to start a heat misfires.

The Starter shall also act as a judge for the race by observing the race as far up Hills 1 and 2 as he or she can see it, watching for fouls by or interference between the competitors.

\subsection{Judges}

Judges shall be utilized to observe the Sweepstakes races and to point out and rule on possible violations of the rules and regulations by the race participants.

\subsubsection{Head Judge}

A person shall be appointed by the Sweepstakes Chairman, with the approval of the Sweepstakes Advisor, to be the Head Judge of the Sweepstakes races. The duties of the Head Judge shall be as follows:

The Head Judge shall observe each heat while riding in the rear of the lead car, watching for fouls by or interference between the entries.

The Head Judge shall hear all protests and appeals by any entry or organization and gather all information available from other judges, officials,or race participants pertinent to those protests and appeals. The Head Judge may use a portable tape recorder in order to record verbal protests and appeals, for later review and consideration.

The Head Judge shall render the final decisions concerning all protests and appeals based on any information that he or she has gathered relative to those protests and appeals, and on his or her interpretation of these Rules, Regulations, and Procedures.

Approximately 30 minutes before the Sweepstakes races begin each day, the Head Judge shall assign the available course judges to various positions around the buggy course such that all portions of the races are observable by at least one course judge. The Head Judge should instruct the course judges concerning what to look for during the races and should provide each course judge with a copy of the race rules and regulations which are pertinent to what they will be observing. The recommended positions for the course judges around the buggy course are as follows:
\begin{itemize}
	\item One judge on Hill 1 on the Lane 3 side of the course, halfway between the starting line and the Hill 1-2 Transition Zone.
	\item Two judges on Tech Street at the Hill 1-2 Transition Zone, one on each side of the course.
	\item Two judges on Schenley Drive just before the end of the lanes, one on each side of the course.
	\item One judge on Schenley Drive near the Westinghouse Memorial Pond.
	\item One judge on Schenley Drive near the end of the first transition on the buggy course.
	\item One judge on Schenley Drive near the Panther Hollow Bridge.
	\item One judge on Schenley Drive in front of the entrance to Phipps Conservatory, near the end of the second transition on the buggy course.
	\item One judge up on the base of the Edward Manning Bigelow monument in the middle of Schenley Drive near Phipps Conservatory.
	\item One judge on Schenley Drive near the chute flaggers at the beginning of the turn onto Frew Street .
	\item Two judges at the intersection of Schenley Drive and Frew Street, one on each side of the course.
	\item One judge on Frew Street, halfway between Schenley Drive and Scaife Hall.
	\item One judge on Frew Street near the driveway beside Scaife Hall.
	\item One judge on Frew Street, halfway between the driveway beside Scaife Hall and the Hill 3-4 Transition Zone.
	\item Two judges on Frew Street at the Hill 3-4 Transition Zone, one on each side of the course.
	\item One judge on Frew Street, halfway between the Hill 3-4 Transition Zone and the Hill 4-5 Transition Zone.
	\item Two judges on Frew Street at the Hill 4-5 Transition Zone, one on each side of the course.
	\item One judge on Frew Street, halfway between the Hill 4-5 Transition Zone and the finish line.
	\item Two judges on Frew Street at the finish line, one on each side of the course.
\end{itemize}

\subsubsection{Assistant Head Judge}

A person shall be appointed by the Sweepstakes Chairman, with the approval of the Sweepstakes Advisor, to be the Assistant Head Judge of the Sweepstakes races. The duties of the Assistant Head Judge shall be as follows:
\begin{itemize}
	\item The Assistant Head Judge shall observe each race while riding in the follow car, watching for fouls by or interference between the competitors.
	\item The Assistant Head Judge shall ensure that only authorized people are permitted to ride in the follow car during each of the Sweepstakes races.
	\item The Assistant Head Judge shall assist the Head Judge with any of his or her duties, when so requested.
\end{itemize}

\subsubsection{Course Judges}

Each organization participating in the Sweepstakes races must provide two course judges for each day of Sweepstakes racing. These judges should be alumni of the organizations providing them. Course judges may also be members of the Carnegie Mellon University faculty and staff, as selected by Sweepstakes Advisor.

The course judges must report to the Head Judge approximately 30 minutes before the races are scheduled to begin on each day of Sweepstakes racing, so that they can be assigned to positions around the buggy course from where they should observe the Sweepstakes races. No course judge shall be permitted to provide information or comments on any heat in which an entry from the organization which provided that judge is competing. The duties of each course judge shall be as follows:
\begin{itemize}
	\item Each course judge shall watch each Sweepstakes race from the location assigned to that course judge by the Head Judge.
	\item Each course judge shall watch for fouls by or interference between the competitors in each of the Sweepstakes races that he or she observes.
	\item Each course judge shall provide to the Head Judge, when so requested, any and all information that they may have, that might be pertinent to any alleged fouls or incidences of interference that may have occurred during any of the Sweepstakes races that they observed.
\end{itemize}

\subsection{Timers}

Timers shall be utilized to measure the time required for each entry's buggy to travel from the starting line to the finish line of the buggy course during the Sweepstakes races. The timers shall be appointed by the Sweepstakes Advisor. A minimum of two timers shall be required for each entry in a heat. The timers shall use stop watches or other suitable measuring devices to determine the time taken by each buggy to travel the buggy course. Sweepstakes Advisor shall appoint one timer as the Head Timer. The Head Timer will coordinate the efforts of all of the other timers.

The timers should be located on the northern sidewalk of Frew Street, directly in line with the finish line of the buggy course. They should be positioned on a platform several feet above the sidewalk so that they have a better view of the finish line of the race.

The timers will usually use the sound of the starter's gun, as broadcast over the Carnegie Mellon University radio station (WRCT), as a signal to start their timing devices at the beginning of a race. If the radio station is not broadcasting the start of the race, another method of starting the timing devices must be devised.

The timers shall stop their timing devices when the nose of the buggy that they are timing reaches the finish line of the buggy course. The finishing time for each entry shall be the average of all of the times determined by the timers for that entry. All finishing times shall be announced by the Head Timer, whether or not they are subsequently invalidated because of a disqualification.

The Head Timer will be responsible for recording official and/or unofficial finishing times for all entries in all Sweepstakes races that take place on each day of racing. These times shall be recorded on Sweepstakes Race Timing Forms. Along with the finishing times, notations concerning disqualifications, accidents, protests, appeals, brake tests, etc. for each entry shall also be recorded on the Sweepstakes Race Timing Forms by the Head Timer. At the end of each day of racing the Head Timer shall present all of the timing forms for that day's races to the Sweepstakes Chairman.

At the discretion of the Sweepstakes Advisor, alternate methods of automatic or semiautomatic timing of the race entrants may be employed, provided that these timing methods can be shown to be as accurate or more accurate than the usual timing method. If any automatic or semi-automatic timing methods are used, the manual timing method described above should also be used as a back-up system

\section{Race Schedule}

\subsection{Starting Times}

The time between the start of one scheduled heat and the start of the next scheduled heat shall usually be 8 minutes during the preliminary races, the alumni/exhibition races, and the rerun races, and 15 minutes during the finals races. The actual time intervals between heats shall be determined by the Sweepstakes Chairman, based on the number of heats to be run and on the amount of time available in which to run them. The actual time intervals to be used for the different races shall be announced to the participating organizations by the Sweepstakes Chairman sometime before the first scheduled day of racing.

The intervals between heats shall be timed by the Starter. Before each scheduled heat starts the Starter shall announce the time remaining to the start of that heat. These announcements shall usually be made when there are 10 minutes (finals races only), five minutes, two minutes, one minute, 30 seconds, and 15 seconds remaining before the start of the next scheduled heat. The last 10 seconds before the start of each heat shall be counted off by the Starter, in a manner such that all competitors located near the starting line are able to hear that count. After the Starter's countdown reaches the count of ``ONE,'' the final warnings to the competitors before the heat starts shall be -- ``READY, -- SET,'' after which the Starter shall fire the starting gun to start the heat.

Each entry must be ready to start its scheduled heat on time. No requests for extensions to delay the scheduled start of any heat shall be granted. Any entry not in position at the starting line when the Starter indicates that there are five seconds left in the countdown until the start of that entry's heat, will not be permitted to start that heat and will not be eligible for a rerun.

\subsection{Delays and Holds}

Delays and holds shall be handled as follows:

If a delay or hold of the countdown to the start of a heat occurs immediately after the end of the previous heat, when the delay or hold is over, the countdown to the start of that next heat shall resume at a time that is between four and seven minutes before the start of that next heat if that heat is a preliminary race, an alumni/exhibition race, or a rerun race. For a finals race, the countdown to the next heat shall resume in between seven and 12 minutes.

If a delay or hold of the countdown to the start of a heat during the preliminary races, the alumni/exhibition races, or the rerun races occurs after there is less than five minutes remaining to the start of that heat, the countdown may be held for up to one minute, and then resumed at the time at which it was held. If the delay or hold is longer than one minute, when that delay or hold is over, the countdown to the start of that next heat shall resume at a time that is between four and seven minutes before the start of that next heat.

If a delay or hold of the countdown to the start of a heat during the finals races occurs after there is less than eight minutes remaining to the start of that heat, the countdown may be held for up to one minute, and then resumed at the time at which it was held. If the delay or hold is longer than one minute, when that delay or hold is over, the countdown to the start of that next heat shall resume at a time that is between seven and 12 minutes before the start of that next heat.

Whenever a delay or hold occurs, the Starter will announce that the countdown has stopped and how much time will remain when the count resumes.

Whenever a delay or hold ends, the Starter will announce that the countdown has resumed and how much time remains until the start of the next heat.

\subsection{Starting Positions}

Starting positions in the preliminary races, rerun races, and finals races in each class of competition shall be determined separately, but by the same method.

\subsection{Preliminary Races}

\subsubsection{Heats}

The preliminary races shall be run in heats, with a maximum of three entries in each heat. The number of heats in each class of competition shall be determined by dividing the total number of entries by three and rounding up to the nearest whole number.

\subsection{Seeding}

1. Entries for each organization shall be identified according to their letter designations, not by the buggy used by each entry. An organization’s “A” entry should be its fastest entry, its “B” entry should be its next fastest entry, and so on.

2. The order of seeding shall be based on the weighted average of an entry’s last three year’s finishing times in the preliminary races, with the time for three years ago being multiplied by one, the time from two years ago being multiplied by two, and the time from last year being multiplied by three. These three numbers would be added together and then divided by six to determine that entry’s seeding time. For example, for race day 2002 the entry’s race day 2001 preliminary time would be multiplied by three and added to the entry’s race day 2000 preliminary time multiplied by two, and then finally added to the entry’s race day 1999 preliminary time. This total would then be divided by six to obtain that entry’s seeding time. The entry with the fastest seeding time will then be placed in the last heat; the entry with the second fastest seeding time will then be placed in the second last heat. This process will be repeated until every heat has one team in it, then the process will continue by going back to the last heat of the day and assigning another team to that heat. By continuing this process each heat will end up with a minimum of two teams and a maximum of three teams. The next step in the seeding process will be the lane selection phase. For every heat the team with the best seeding time will choose their lane first, then the team with the second best seeding time will choose one of the two available lanes, and finally the team with the third best seeding time will be assigned to the remaining available lane.

3. The finishing times used for seeding purposes need not be official times. If an entry finishes the race, but is subsequently disqualified (for example by failing the brake test), its finishing time shall still be used for seeding purposes. In addition if an entry does not finish its preliminary race, but is subsequently granted an official or unofficial reroll, the time from that reroll shall be used for seeding purposes. If an entry does not finish its preliminary race and does not have an official or unofficial preliminary time for one or more of the three most recent years, that year shall be thrown out of that entry’s seeding time. That entry’s seeding time will be computed, using the method stated above, however the year that the entry did not finish will not be entered into the average, and the total time will be divided by the appropriate number.

4. If for any reason the Sweepstakes Executive Committee feels that a heat contains an unsafe grouping of buggies, the Sweepstakes Executive Committee will discuss the matter with the University’s staff Sweepstakes Advisor, and at their approval make the changes necessary to provide a safe race.

\subsubsection{Heat Selection}

The purpose of the heat selection procedure is to help ensure that each heat of the preliminary races has entries that are as far apart, with respect to probable finishing times, as is possible. This should help to reduce the chances of an accident during any of these heats. After all of the entries in each class of competition have been seeded, they shall select heats for the preliminary races using the following procedure:

Heat selections shall be made at a meeting of the Sweepstakes Committee. The heat selection procedure shall be supervised by the Sweepstakes Chairman, or anyone designated by that Chairman. This supervisor shall maintain order during the selection process and shall keep records of all of the selections that are made. If any entered organization is not represented at this meeting, the supervisor shall be empowered to make any required selections for the unrepresented organization.

The seeded entries shall be divided into three groups. If the number of preliminary heats scheduled is $n$, then the first group will have $n$ entries, starting with the highest seeded entry and ending with the $n^{th}$ seeded entry. The second group will also have $n$ entries, starting with the $(n+1)^{th}$ seeded entry and ending with the $(n+n)^{th}$ seeded entry. The third group will have all of the remaining seeded entries.

Representatives of each of the entries in the first group shall select the heats that their entries shall compete in using the following procedure: a) The highest seeded entry in the first group shall select the heat that it wants to compete in during the preliminary races. b) The next highest seeded entry in the first group shall select the heat that it wants to compete in during the preliminary races, with the provision that it cannot select a heat which already has one or more entries from the first group in it,unless all of those other first group entries grant it permission to do so.) The preceding step shall be repeated until all of the seeded entries in the first group have selected heats.

Representatives of each of the entries in the second group shall select the heats that their entries shall compete in using the following procedure: a) The highest seeded entry in the second group shall select the heat that it wants to compete in during the preliminary races. b) The next highest seeded entry in the second group shall select the heat that it wants to compete in during the preliminary races, with the provision that it cannot select a heat which already has one or more entries from the second group in it, unless all of those other second group entries grant it permission to do so. c) The preceding step shall be repeated until all of the seeded entries in the second group have selected heats.

Representatives of each of the entries in the third group shall select the heats that their entries shall compete in using the following procedure:

The highest seeded entry in the third group shall select the heat that it wants to compete in during the preliminary races.

The next highest seeded entry in the third group shall select the heat that it wants to compete in during the preliminary races.

The preceding step shall be repeated until all of the seeded entries in the third group have selected heats.

Heat selections shall be made by a date that is no later than the second day of Truck Weekend (the weekend just prior to the date scheduled for the preliminary races). Within two days of the day that heats and lanes are selected, the Sweepstakes Chairman shall distribute a schedule of all of the heat and lane selections for the preliminary races, to each of the participating organizations.

\subsubsection{Lane Selection}

After heats have been selected for all entries, the lane that each entry will occupy on Hills I and 2 shall be chosen. The entry seeded first shall choose first. The entry seeded second shall choose second. The entry seeded third shall choose third, and so on until all of the entries have chosen lanes. No entry may choose a lane that is already occupied by a higher seeded entry.

Lane selection shall take place at the same meeting at which heats are selected.

\subsubsection{Heat and Lane Switching}

Any entry may switch their heat and/or lane selections with any other entry provided that they comply with the following:

All of the entries in each heat in which a change is to take place must approve the change.

The Sweepstakes Chairman and the Safety Chairman must approve the change.

The change must be made within 24 hours of the meeting at which the heats and lanes were selected.

\subsubsection{Entry Withdrawals}

If an entry withdraws from the Sweepstakes competition before the preliminary races begin, the heat to which that entry was assigned will not have three entries competing in it. If more than one heat has less than three entries competing in it, the Sweepstakes Chairman may reassign entries in heats with less than three entries to different heats if that reassignment will reduce the total number of heats in the preliminary races. A reassignment may only be made if the Sweepstakes Chairman and the Safety Chairman determine that that reassignment would not increase the likelihood of an accident in any heat, and if the reassignment does not cause any reassigned entry to start the race in a lane other than the one that they were originally scheduled to be in, unless that entry agrees to the different lane assignment. No entry of an organization may be reassigned to a heat in a class of race competition that is less than two heats away from any heat in that same class of race competition in which another of that organization's entries is scheduled to compete.

Any organization withdrawing an entry shall forfeit all entry fees for that entry. If an organization withdraws an entry after heats and lanes have been selected for the preliminary races, they shall only be permitted to enter as many entries in the following year's competition as they had competing in the year that they made the late withdrawal. Each organization shall always be permitted to have at least one entry regardless of the number of late withdrawals they had during the previous year's competition.

Exceptions to this rule may be made if an entry is withdrawn involuntarily, such as if the entry's buggy is damaged beyond repair during a practice session. However, the evidence that the withdrawal was involuntary must be very strong and must be agreed upon by the Sweepstakes Chairman, the Assistant Sweepstakes Chairman, and the Safety Chairman. Failure of an entry's driver to successfully complete a passing test or to complete the minimum required number of rolls down the buggy course during freeroll practice sessions shall NOT be considered as valid reasons for the involuntary withdrawal of that entry. Failure of an entry's buggy to complete the minimum required number of rolls down the buggy course during freeroll practice sessions shall NOT be considered as a valid reason for the involuntary withdrawal of that entry.

\subsection{Alumni/Exhibition Races}

Alumni/exhibition races are not considered to be official Sweepstakes races. If alumni/exhibition races are held, all safety related rules, regulations, and requirements applicable to the Sweepstakes races shall also apply to the alumni/exhibition races. This includes the requirement for a drop brake test for each participating buggy after each heat of the alumni/exhibition races, and all of the requirements for drivers. (This also includes the requirement that all drivers be currently enrolled, Activities Fee paying,full-time students of Carnegie Mellon University.) Graduate students may participate in alumni/exhibition races.

\subsubsection{Starting Positions}

If alumni races are scheduled, heat and lane assignments shall be made by the Sweepstakes Chairman, or anyone designated by that Chairman. The procedure used to assign heats and lanes for the alumni/exhibition races shall be determined by the Sweepstakes Chairman.

\subsection{Rerun Races}
Any rerun races that are held shall be run in heats, with a maximum of three entries in each heat. The minimum number of rerun heats in each class of competition shall be determined by dividing the number of rerun entries by three and rounding up to the nearest whole number.

\subsubsection{Heat Assignments}

Heat assignments for rerun races shall be made by the Sweepstakes Chairman and the Safety Chairman. The purpose of the heat assignments is to help ensure that each heat of the rerun races has entries that are as far apart with respect to probable finishing times, as is possible, in order to reduce the chances of an accident during any of the rerun heats. The seeding order used to assign heats in the preliminary races shall be used as a guideline in assigning entries to heats for rerun races. Rerun race heats may be run with less than three entries each, if the Sweepstakes Chairman and the Safety Chairman, determine that the safety of those heats might be substantially increased by doing so. The method of assigning entries to heats for the rerun races shall be determined by the Sweepstakes Chairman and the Safety Chairman. Heat assignments shall be completed as soon as is practical after the preliminary races have ended.

No rerun heat shall have more than one entry from any single organization in it.

\subsubsection{Lane Selection}
After all rerun entries have been assigned to a heat, the lane that each entry will occupy on Hills 1 and 2 shall be chosen. The entry seeded highest before the preliminary heats shall choose first. The entry seeded next highest before the preliminary heats shall choose second, and so on until all of the rerun entries have chosen lanes. No rerun entry may choose a lane that is already occupied by a higher seeded rerun entry.

Lane selection shall be completed as soon as is practical after heat assignments have been made.

\subsection{Finals Races}

\subsubsection{Rankings}

After the preliminary races are finished, and before any rerun races have taken place, all of the entries that finished a preliminary race and were not disqualified or granted a rerun, shall be ranked according to their finishing times in those preliminary races for each class of competition. The entry with the fastest time shall be ranked first, the entry with the second fastest time shall be ranked second, and so on to the entry with the slowest time, which shall be ranked last.

The six highest ranked entries from the women's preliminary races shall be eligible to compete in the women's finals races.

The ten highest ranked entries from the men's preliminary races shall be eligible to compete in the men's finals races.

In the event of ties, one or more extra entries may be eligible to compete in the women's or men's finals races. The rankings of tied entries shall be determined by the Sweepstakes Chairman, or anyone designated by that Chairman.

\subsubsection{Heats}

The finals races shall be run in heats, with a maximum of two entries in each heat. The number of finals heats in each class of competition shall be determined by dividing the number of entries eligible to compete in the finals races by two and rounding up to the nearest whole number.

\subsubsection{Heat Assignment}

The entries eligible to compete in the finals races will be assigned to heats for those races, by the Sweepstakes Chairman, or anyone designated by that Chairman. The purpose of the heat assignments is to help ensure that each heat of the finals races has entries that are as far apart with respect to probable finishing times as is possible, in order to reduce the chance of an accident during any of the finals heats. Heat assignments shall be completed as soon as is practical after the preliminary races have ended.

Heat assignments for each class of competition shall be made as follows:

The entry ranked first after the preliminary races shall be assigned to the last finals heat, the entry ranked second shall be assigned to the next to last finals heat, the entry ranked third shall be assigned to the third to last finals heat, and so on, until one entry has been assigned to each finals heat.

After all of the finals heats have one entry assigned to each of them, the highest ranked entry of the remaining entries shall be assigned to the last finals heat, the next highest ranked of the remaining entries shall be assigned to the next to last finals heat, and so on, until all of the entries eligible to compete in the finals races have been assigned to finals heats.

In the event that this procedure assigns more than one entry from any organization into the same finals heat, the heat assignments shall not be changed.

\subsubsection{Lane Selection}

After each entry eligible to compete in the finals races has been assigned to a finals heat, the lane that each entry will occupy on Hills 1 and 2 shall be chosen. Lanes for each finals heat shall be selected by each entry, with the entry having the higher ranking after the preliminary races choosing first. No finals entry may choose a lane that is already occupied by a higher ranked finals entry.

Lane selection shall be completed as soon as is practical after heat assignments have been made.

\section{Race Rules, Regulations, and Requirements}

If the Head Judge determines that any entry has violated any of the Race Rules, Regulations, and Requirements listed in this section of this document, that entry shall be disqualified from the race competition. The Head Judge does not require a formal protest from a competitor in order to disqualify an entry from the races. Any infraction of the race rules observed by the Head Judge, or brought to his or her attention by any of the other judges or officials, can be grounds for disqualifying or penalizing the offending entry.

\subsection{Regulations and Requirements}

\subsubsection{Drivers}

No driver shall be permitted to participate in a Sweepstakes race unless he or she satisfies all of the following requirements:

He or she has walked the buggy course earlier that same day with the drivers of the other buggies that are scheduled to be in his or her heat. Each driver must notify the Assistant Sweepstakes Chairman, or anyone designated by that Chairman, immediately before or after he or she walks the course. If a driver is scheduled to race in more than one heat on the same day, that driver must walk the course once for each of the heats that he or she is scheduled to race in, with the other drivers of each of those heats. Drivers are only permitted to walk the course BEFORE any races are held on any day of Sweepstakes racing.

He or she has previously successfully completed a braking capability test in the buggy that he or she will be driving in that Sweepstakes race.

He or she has previously successfully completed a field of vision test in the buggy that he or she will be driving in that Sweepstakes race.

If a driver raced in a Sweepstakes race the previous year (an ``OLD'' driver), he or she must have a minimum total of ten freerolls for the school year in which they are racing. A minimum of five of those ten freerolls must be completed during the Spring semester in which the race is to take place, with one being the pass test.

If a driver did not race in a Sweepstakes race during the previous school year (a ``NEW'' driver), he or she must have a minimum total of fifteen freerolls. A minimum of seven of those fifteen freerolls must be completed during the Spring semester in which the race is to take place, with one being a pass test.

All drivers, new and old, must have a minimum total of ten freerolls in any buggy they will be driving during races. (i.e. If an organization brings out t new buggy in the Spring semester, its raceday driver will still need a minimum of ten freerolls in that buggy in order to qualify to drive that buggy during races.)

In the case of very special circumstances, such as a long duration of inclement weather where no freeroll practices are held, an organization may seek permission to race a driver and buggy that do not meet these specific requirements. In such cases, the organization must submit an appeal in writing to the Sweepstakes Committee, explaining that the failure to meet the qualifying standards was through no fault of the driver of the organization. The Safety Chairman, with counsel of the Sweepstakes Chairman and Assistant Chairman, will evaluate the appeal and either approve or reject it. If an appeal is approved by the Safety Chairman, it must then be passed on the the Buggy Chairmen to be approved by a majority vote.

He or she has successfully completed a pass test, during a freeroll practice session that occurred during the same school semester in which that race is scheduled, in the buggy that he or she will be driving in that race.

He or she has demonstrated by his or her performance at both freeroll and push practices that he or she is capable of driving a buggy in a safe and controlled manner.

\subsubsection{Buggies}

No buggy shall be permitted to participate in a Sweepstakes race unless it satisfies all of the following requirements:

It has successfully completed a safety inspection and been driven down the buggy course at least once, in any configuration that it will be in, during that race.

It has successfully completed a braking capability test with the driver who will be driving it during that race.

It has successfully completed a field of vision test with the driver who will be driving it during that race.

It has previously been driven down the buggy course at least five times during freeroll practice sessions.

It has been the passing buggy in a successfully completed pass test, during a freeroll practice session that occurred during the same school semester in which that race is scheduled.

\subsection{Rules}

\subsubsection{General}

No organization may use or attempt to use any buggy preparation area that has been assigned to another organization on any day of Sweepstakes racing.

No organization's entry may use or attempt to use any lane that has been assigned to another organization's entry in any Sweepstakes race heat.

Each organization may allow only one of its representatives to ride in the follow car during each race in which one of its entries is competing.

If any member of any entry's team is substituted for by a member of that entry's alternate team, that substitution must be declared to the Sweepstakes Chairman before the start of the heat preceding the heat in which that entry is scheduled to compete.

\subsubsection{Entries}

In each class of competition, each organization's entries must compete in the heats designated for those entries during the heat selection process as follows:

No organization may permit its fastest entry to compete in any heat of the Sweepstakes races other than the heat designated as the heat for that organization's ``A'' entry.

No organization may permit its second fastest entry to compete in any heat of the Sweepstakes races other than the heat designated as the heat for that organization's ``B'' entry.

No organization may permit its third fastest entry to compete in any heat of the Sweepstakes races other than the heat designated as the heat for that organization's ``C'' entry.

No organization may permit its fourth fastest entry to compete in any heat of the Sweepstakes races other than the heat designated as the heat for that organization's ``D'' entry.

Any entry which competes in a heat that is not the heat designated for that entry shall be disqualified from the race competition. Determination of whether or not an entry has violated this rule shall be made collectively by the Sweepstakes Advisor, the Sweepstakes Chairman, the Assistant Sweepstakes Chairman, the Safety Chairman, the Head Judge, the Assistant Head Judge, and anyone else designated by either the Sweepstakes Advisor or the Sweepstakes Chairman.

\subsubsection{Pushers}

No pusher may use any type of mechanical device with moving parts (such as roller skates or a skate board) which could cause that pusher to travel faster, while that pusher is pushing a buggy during a Sweepstakes race.

\subsubsection{Buggies}

The combined weight of a buggy and its driver may not change while that buggy is competing in any Sweepstakes race. Weight loss during a race shall be permitted only if the Head Judge rules that the weight loss was unintentional, the weight loss was not caused by a design failure of the buggy, and the weight loss did not interfere with any of the other entries in that heat.

The loss of any part of a buggy's shell, hatch, or cover while that buggy is competing in any Sweepstakes race shall be considered to be a design failure of that buggy, and will result in the disqualification of that buggy and its entry from the race competition, even if that loss did not interfere with any of the other entries in that heat.

The dimensions of a buggy, not including those of that buggy's pushbar, shall not change while that buggy is competing in a heat of the Sweepstakes races.

No buggy may have any changes made to it between the end of the Design Competition and the conclusion of any race that it competes in except for changes to its wheels, tires, bearings, and windscreens. Damaged parts of a buggy may be changed after the Design Competition and before a race,provided that the Safety Chairman is informed of the change, is permitted to view the damaged parts, and is given a detailed account of how the parts were damaged.

Each entry's driver must ride in or on that entry's buggy during the entire time that entry is competing in a Sweepstakes race heat. No other person is permitted in or on that buggy at that same time.

Each entry's buggy must take a drop brake test immediately after any race in which that entry has competed, whether that buggy completed that race or not. Failure to successfully complete this test will result in disqualification of that entry from the race in which it just competed and will make that entry ineligible for a rerun unless the Safety Chairman and the Head Judge determine that the buggy failed the drop brake test because of damage sustained due to interference that occurred during the race. Any buggy which fails the drop brake test after a race but is subsequently granted a rerun, must successfully complete a drop brake test BEFORE it will be permitted to take that rerun. These drop brake tests shall be administered by the Safety Chairman, or by someone designated by that Chairman. Each buggy shall only be given one chance to pass the drop brake test after a race unless any of the following occur:

The drop brake test is not properly administered, for example, if the test administrator tells the driver to apply the brakes too soon or too late. In this event, the test shall be administered again in the proper manner.

The driver unintentionally applies the brakes too soon. In this event the driver shall be given one and only one more chance to successfully complete the test. In the unlikely event that the buggy does not roll after it is initially released, or if it does not roll a distance of 30 feet after it is initially released, the test administrator shall pull the buggy in a forward direction, using a spring scale, or other suitable force measuring device, while the driver is actuating the buggy's brakes, in order to determine how much braking force the buggy's brakes are able to exert in that direction. The force used to pull the buggy shall be applied parallel to, and as close to the ground as is practicable. If the buggy's brakes are able to exert a minimum braking force of 25 pounds in the forward direction, and the driver can release and reapply that buggy's brakes once, that buggy shall be considered to have passed the drop brake test.

Any entry's buggy that is involved in any type of accident (such as spinning out, having parts fall off while it is rolling, hitting another buggy, being hit by another buggy, hitting another object, etc.) during any Sweepstakes race will not be permitted to race again until that entry's sponsoring organization submits, and has approved, an accident report concerning that accident. Accident reports must be submitted to the Safety Chairman or the Sweepstakes Chairman. Accident reports can only be approved by the Sweepstakes Chairman or the Safety Chairman. The form to be used for these reports can be obtained from the Sweepstakes Advisor through the Safety Chairman.

Any device that allows the driver in the buggy communication with individuals outside of the buggy and provides information to a driver that otherwise could not be known by the driver are illegal to use on Raceday rolls or when performing a pass test. These devices may be used during freeroll and push practice; however, use of these prohibited devices on raceday or when more than one buggy is rolling at the same time will result in immediate disqualification. Examples of the devices are headsets, walkie-talkies,telemetry systems, and video display units.

\subsubsection{Starting}

When the Starter signals the start of an entry's heat, that entry's buggy must comply with all of the following:
\begin{itemize}
	\item The nose of the buggy must be at or behind the starting line.
	\item The buggy must not be moving in the forward direction.
	\item The forward motion of the buggy must not be restrained in any way.
	\item When the Starter signals the start of an entry's heat, that entry's Hill 1 pusher must comply with all of the following:
	\item He or she must be touching that entry's buggy.
	\item He or she must have both of his or her feet on the ground.
	\item He or she must not be moving in the forward direction.
	\item He or she must not be using starting blocks, or any similar device.
\end{itemize}

After the starter announces that there are five seconds remaining to the start of a heat, nobody except the drivers and Hill 1 pushers of the entries in that heat, and the starter, may be within five feet of any of the buggies in that heat, until that heat begins.

If the nose of an entry's buggy in any heat is beyond the starting line when the Starter signals the start of that heat, that entry shall be considered to have false started. If an entry false starts more than two times at the start of any single heat, that entry shall be disqualified from that heat, shall not be permitted to run in that heat, and shall not be eligible for a rerun.

If an entry in a heat false starts, all of the buggies in that heat shall be brought back to the starting line and that heat shall be restarted. The Starter shall restart the countdown at his or her discretion. If the time required to restart the heat is greater than 1 minute and any of the entries in that heat which did not cause the false start so request,the start of that heat may be delayed until the time scheduled for the start of the heat after the current heat, and all of the remaining heats shall be moved back for the amount of time required for one heat to take place. If the delayed heat is the last scheduled heat of the day, it shall be rescheduled for a time that is between 5 and 12 minutes later than its originally scheduled starting time.

For example, if Heat 5 is delayed because of a false start, it shall be rescheduled for the time at which Heat 7 was originally scheduled to take place. Heat 6 would be run at its originally scheduled time. No heat would be run at the time originally scheduled for Heat 5. Heat 7 would be run at the time originally scheduled for Heat 8 and all subsequent heats would be moved back to the time originally scheduled for the heat after them. If Heat 5 was the last scheduled heat of the day, no heat would be run at the time originally scheduled for Heat 5 and Heat 5 would be run at a time that was between 5 and 12 minutes later than its original starting time, depending on how much time is still available for racing.

If the Starter's gun fails to fire when the Starter tries to signal the start of a heat, all of the buggies in that heat shall be brought back to the starting line and that heat shall be restarted. The Starter shall restart the countdown at his or her discretion. If the time required to restart the heat is greater than 1 minute and any of the entries in that heat so request, the start of that heat may be delayed until the time scheduled for the start of the heat after the current heat, just as might happen if a false start had occurred.

\subsubsection{Lanes}

A buggy is considered to be in a lane when no portion of that buggy extends beyond or above the innermost edges of the lines delineating that lane.

The judges shall disqualify an entry from the heat that it is competing in if all of the wheels of that entry's buggy are simultaneously out of its assigned lane at any point between the starting line and the end of that lane, even if that entry doesn't interfere with any other entry.

If some part of, but not all of, an entry's buggy is out of its assigned lane at any point between the starting line and the end of that lane,that entry shall be disqualified from the heat that it is competing in if it interferes with another entry in that heat, or if one of the other entries in that heat protests the lane violation (even if no interference is claimed.)

A pusher is considered to be in a lane when the greatest part of that pusher's body is on or above that lane and no part of that pusher is touching anything that is not considered to be in that lane also, except for the buggy which that pusher is pushing.

If some part of, but not all of, one of an entry's pushers is out of that entry's assigned lane at any point between the starting line and the end of that lane, that entry shall be disqualified from the heat that it is competing in if that pusher interferes with any other entry in that heat, and if the entry that was interfered with protests that interference.

\subsubsection{Pushing}

The position of an entry's pusher along the buggy course is determined by the location of the forwardmost part of the pusher's forwardmost foot, i.e. the foot that is farthest along the course from the starting line.

For example, if a pusher's forwardmost foot is on Hill 1, that pusher is considered to be on Hill 1. If a pusher's forwardmost foot is in the Hill 1-2 Transition zone, that pusher is considered to be in that Transition zone even if that pusher's rearmost foot is still on Hill 1.

Each entry's Hill 1 pusher may only touch that entry's buggy while he or she is either on Hill 1 or in the Hill 1-2 Transition Zone.

Each entry's Hill 2 pusher may only touch that entry's buggy while he or she is either in the Hill 1-2 Transition Zone or on Hill 2.

Each entry's Hill 3 pusher may only touch that entry's buggy while he or she is either on Hill 3 or in the Hill 3-4 Transition Zone.

Each entry's Hill 4 pusher may only touch that entry's buggy while he or she is either in the Hill 3-4 Transition Zone, on Hill 4, or in the Hill 4-5 Transition Zone.

Each entry's Hill 5 pusher may only touch that entry's buggy while he or she is in the Hill 4-5 Transition Zone, on Hill 5, or crossing the finish line.

Each entry's Hill 5 pusher must be in contact with that entry's buggy when the nose of that buggy crosses the finish line at the end of that entry's heat.

No organization may cause anyone or anything to pace a pusher of any of that organization's entries while that pusher is competing in a race. A person or device which is not the next pusher in sequence and which is moving near a buggy and moving in the same direction in which that buggy is moving, while in view of the pusher currently pushing that buggy, is considered to be a pacer.

While an entry's pusher is pushing that entry's buggy and after that pusher has finished pushing that entry's buggy, that pusher is entitled to run, walk, stand, or otherwise be, anywhere in the path which that entry's buggy has followed. A buggy's path is considered to be a lane that is 6 feet wide which is centered on the centerline of that buggy.

\subsubsection{Driving}

Each entry's buggy in each heat may not bump into or otherwise contact any other entry's buggy in that same heat at any time between the start and finish of that heat.

If during a heat, one entry's buggy passes another entry's buggy, the pass shall be considered to be complete whenever the rearmost part of the passing buggy (or that buggy's pusher, if it is being pushed at the time) is beyond the nose of the buggy being passed.

If during a heat, one entry's buggy (and pusher, if the buggy is being pushed at that time) tries to pass another entry's buggy (and pusher),the passing buggy (and pusher) has the primary responsibility of ensuring that the pass is completed without contact or any other type of foul between the buggies (or pushers). If contact or some other type of foul occurs between the buggies (or pushers) during an attempted pass, the judges shall determine which buggy (or pusher), if any, is at fault.

If during a heat, two entries' buggies (and pushers, if the buggies are being pushed at the time) are traveling beside each other and neither is clearly passing the other, both buggies (and pushers) have the responsibility of ensuring that no contact or other type of foul occurs between the buggies (and pushers). If contact or some other type of foul occurs between the buggies (or pushers), the judges shall determine which buggy (or pusher), if any, is at fault.

If during a heat, an entry's driver stops that entry's buggy because that driver considered that an accident was about to take place, that entry shall be eligible for a rerun provided that the following conditions are met:

The Head Judge determines that an accident was about to take place.

The accident that was about to occur was not due to any failure or foul on the part of the buggy that stopped.

The buggy that stopped is taken to the drop brake test area and given a drop brake test as soon as possible after the heat is finished.

If during a heat, an entry's buggy or pusher is interfered with or fouled, that buggy or pusher is not required to stop at that point in the race, in order to be eligible for a rerun.

\subsection{Protests and Appeals}

A protest may be filed by one entry against another entry if the protesting entry considers that it was fouled or interfered with by the protested entry during a Sweepstakes race.

A protest may be filed by any participating organization against any entry if the protesting organization considers that the protested entry has not complied with all of the rules and requirements of the Sweepstakes races.

An appeal may be filed by an entry if that entry considers that it was interfered with while it was competing in a Sweepstakes race by someone or something other than any of the other entries in that race.

Protests and appeals shall be filed as follows:

The Buggy Chairman of the protesting or appealing organization shall verbally inform the Sweepstakes Chairman and/or the Head Judge that he or she wishes to file a protest or appeal on behalf of his or her entry or organization. This verbal notification must be accomplished before the start of the next heat after the protested or appealed heat. If the protested or appealed heat is the last heat of the day, the notification must be accomplished within ten minutes after the end of that heat.

The Buggy Chairman of the protesting or appealing organization shall verbally describe the reason for the protest or appeal and any information pertinent to that protest or appeal to the Head Judge of the Sweepstakes races. If possible, protests and appeals should also be submitted in written form, using the same form that is used for accident reports, and which can be obtained from the Safety Chairman.

The Head Judge will confer with the Sweepstakes Chairman, the Assistant Sweepstakes Chairman, the Safety Chairman, the follow car judge, any of the course judges that could have witnessed the alleged incident, any of the drivers or pushers that were involved in the incident, and/or anyone else who might be able to provide information pertinent to the alleged incident, in order to determine as many facts about the incident as possible. (If video tapes of the incident are available, they may be used as an additional source of information. )

After reviewing all of the available information the Head Judge will render a decision concerning the protest or appeal and will inform the Sweepstakes Chairman of that decision. This decision should normally be made before the start of the second heat after the protested or appealed heat, but it may be delayed until all of the races are finished for that day at the discretion of the Head Judge, if circumstances warrant such a delay.

The Sweepstakes Chairman will confer with all of the affected entries and inform them of the judges' decision.

The Sweepstakes Chairman will announce the decision of the judges to the remaining race participants and to the public at large.

If a protest is filed due to an alleged foul or interference during a race by another entry, that protest shall be dispositioned as follows:

If the protest is disallowed, the results of the protested heat shall stand as they were.

If the protest is allowed and the protesting entry was interfered with or fouled, that entry will be granted a rerun.

If the protest is allowed, the entry which caused the interference or foul will be disqualified from the protested heat.

If the Head Judge rules that some type of interference or foul did occur but that no violation of the rules occurred, the protesting entry will be granted a rerun.

If a protest is filed due to an alleged violation of the rules and requirements of the Sweepstakes races, that protest shall be dispositioned as follows:

If the protest is disallowed, the protested entry shall not be penalized in any way.

If the protest is allowed the protested entry will be disqualified from the heat in which it competed.

If an appeal is filed due to an alleged interference during a race by someone or something other than any of the other entries in that race, that appeal shall be dispositioned as follows:

If the appeal is disallowed, the results of the appealed heat shall stand as they were.

If the appeal is allowed the appealing entry shall be granted a rerun.

\subsection{Reruns}

An entry may be granted a rerun by the Head Judge if all of the following occur:

The entry is either interfered with or fouled while it is competing in a Sweepstakes race.

The entry files a protest or an appeal in accordance with the procedure for filing protests and appeals .

The Head Judge rules that the protesting or appealing entry was interfered with or fouled.

Any entry which is granted a rerun has the option of taking the rerun or of allowing its results in the protested or appealed heat to stand as they are. If the entry chooses to take the rerun, the results of the rerun race shall then be that entry's official results for the protested or appealed heat.

If an entry is granted and accepts a rerun because of an incident that occurs during the preliminary races, that entry's rerun race shall be scheduled to take place on the same day that the finals races are scheduled to take place.

If an entry is granted and accepts a rerun because of an incident that occurs during the rerun races, that entry's rerun race shall be scheduled to take place after all of the women's finals races are finished, but before the men's finals races are started.

If an entry is granted and accepts a rerun because of an incident that occurs during the finals races, that entry's rerun race shall be scheduled to take place after all of the finals races are finished. If that entry competed in the last scheduled finals race, its rerun race shall be scheduled to start between seven and 12 minutes after the end of that last finals race.

If an entry is granted a rerun and wishes to accept that rerun but is not able to do so because of damage to its buggy or some other reason beyond its control, that entry's finishing time for its rerun race shall be its fastest time from any race that it completed during that Sweepstakes competition.

\subsection{Safety}

\subsubsection{Buggies With Drivers}

No buggy that has a driver in it may be left unattended at ANY time during the Sweepstakes races. When any buggy is outdoors with a driver in it and it is not being used in a race or a drop brake test, it MUST have someone attending it by holding onto the buggy's pushbar.

\subsubsection{Buggies}

In order to ensure that all of the buggies competing in the Sweepstakes races are in compliance with all applicable safety rules, regulations, and requirements, any buggy may be given a spot safety check by the Sweepstakes Chairman, the Assistant Sweepstakes Chairman, the Safety Chairman, or anyone designated by the Sweepstakes Advisor, immediately after that buggy has finished competing in a race and before that buggy's driver has been removed from that buggy. Any buggy found to be out of compliance with any of the applicable safety rules, regulations, or requirements during one of these spot safety checks shall be disqualified from all of the men's races, the women's races, and the design competition for that school year.

The following spot safety checks are required to be administered:

During all of the men's preliminary and rerun races, any entry which finishes its race with a time that is faster than the tenth fastest time recorded in the men's preliminary and rerun races of the previous year's Sweepstakes competition, shall have its buggy inspected before the driver is removed from that buggy after that race.

During all of the women's preliminary and rerun races, any entry which finishes its race with a time that is faster than the sixth fastest time recorded in the women's preliminary and rerun races of the previous year's Sweepstakes competition, shall have its buggy inspected before the driver is removed from that buggy after that race.

Any entry which competes in a finals race shall have its buggy inspected before the driver is removed from that buggy after that finals race.

\subsubsection{Buggy Preparation Areas}

In order to reduce the possibility of accidents or injuries in the areas in which the buggies are prepared for the races, these areas shall be randomly inspected both before and during the Sweepstakes races by fire marshals, the Safety Chairman, or anyone else designated by the Dean of Student Affairs or the Sweepstakes Advisor. Any organization found to have unsafe or dangerous conditions in their buggy preparation area before or during the Sweepstakes races by any of these safety inspectors shall be fined the amount of \$100.00 AND shall have ALL of their buggies immediately disqualified from all of the men's races, the women's races,and the design competition for that school year. In addition, any organization found to have ANY combustible liquids and/or ANY source of open flame in or near their buggy preparation area before or during the Sweepstakes races shall not be permitted to participate in ANY activity related to the Sweepstakes Competition for a period of 15 months from the date on which the violation is discovered. The following safety requirements apply AS A MINIMUM, to all buggy preparation areas both during and immediately before all Sweepstakes races:

NO quantity of ANY combustible liquid is permitted in or near the buggy preparation areas. (The ONLY exceptions to this requirement shall be quantities of lubricating fluid not greater in volume than two fluid ounces used to lubricate wheel bearings and the working fluids contained in motor vehicles and needed for their proper operation.)

No source of open flame is permitted in or near the buggy preparation areas.

All electrical apparatus and power cables MUST be grounded.

Each organization is required to have a minimum of two fire extinguishers in their buggy preparation area. Each of these fire extinguishers must contain a minimum of ten pounds of extinguishing agent. These fire extinguishers must be rated such that Class A, Class B, and Class C fires can all be extinguished. Class A fires consist of ordinary combustible materials,such as wood, paper, textiles, etc., and they require cooling and quenching. Class B fires consist of flammable liquids and greases, such as gasoline,hexane, oils, paints, etc., and they require smothering. Class C fires consist of electrical equipment, such as motors, switches, cables, etc.,and they require a nonconducting agent capable of extinguishing a fire in materials that might be present. All fire extinguishers must be adequately charged and fully operational.

NO alcoholic beverages are permitted in or near the buggy preparation areas. Alcoholic beverages to be used in post race celebrations may be stored near the buggy preparation areas, such as in the cab section of a truck, provided that none of them are opened or consumed until AFTER all of the races for that day have been completed.

No member of an organization who is in that organization's buggy preparation area may be intoxicated or under the influence of any type of mind altering substance.

If any organization uses any type of enclosed or semi-enclosed structure as a buggy preparation area, such as a truck, a tent, a free standing building, etc., the following requirements shall apply to that structure:

Smoking shall NOT be permitted inside that structure.

The inside of that structure shall NOT be sealed off from the outside of that structure by any type of solid door or cover at ANY time that ANY person is inside that structure. Trucks must have their rear doors COMPLETELY open whenever there are people inside the rear section of those trucks. Curtains or other types of non-sealing partitions are permissible in order to prevent people outside the structure from looking inside the structure.

Whenever a buggy preparation area is completely enclosed and there are people inside that area, it is recommended that fresh air be circulated through that area with some type of powered ventilating fan or blower.

\section{Final Standings}

Final standings in each class of competition shall be determined separately, but by the same method. If all of the preliminary races in a class of competition are not completed, no final standings shall be declared.

\subsection{One Day Races}

If only one day of racing is completed in a class of competition, such that all of the preliminary races in that class of competition are completed but not all of the rerun races and finals races in that class of competition are completed, the final standings of the Sweepstakes races in that class of competition shall be determined as follows:

If no entries were granted reruns, all of the entries that completed a preliminary race and were not disqualified shall be ranked in order of finishing time. The entry with the fastest finishing time shall be awarded first place, the entry with the second fastest finishing time shall be awarded second place, and so on, to the entry with the slowest finishing time in the preliminary races.

If one or more entries were granted reruns and all of the granted rerun races were completed, the finishing times from the rerun races shall be considered to be the preliminary race times for the rerun entries, and the final standings shall be determined in the same manner as if no reruns had been granted.

If one or more entries were granted reruns but not all of the granted rerun races were completed, the final standings shall be determined in the same manner as if no reruns had been granted, i.e. only the actual finishing times from the preliminary races shall be used to determine the final standings.

\subsection{Two Day Races}

If both days of racing are completed in a class of competition, such that all of the preliminary races, rerun races, and finals races are completed in that class of competition, the final standings of the Sweepstakes races in that class of competition shall be determined as follows:

If no entries were granted reruns, all of the entries that completed a finals race and were not disqualified shall be ranked in order of finishing time in those finals races. The entry with the fastest finishing time in the finals races shall be awarded first place, the entry with the second fastest finishing time in the finals races shall be awarded second place, and so on, to the entry with the slowest finishing time in the finals races. If any entries were disqualified in the finals races, they shall be ranked below all of the entries that were not disqualified in the finals races and the ranking among those disqualified entries shall be the same as their relative ranking after the preliminary races.

If one or more entries were granted reruns and any rerun races that took place were held on the same day that the finals races were held, the finishing times from the rerun races shall be considered to be finishing times for the finals races, just as if no entries had been granted reruns. If any entries were disqualified in the rerun or finals races, they shall be ranked below all of the entries that were not disqualified in the rerun or finals races and the ranking among those disqualified entries shall be the same as their relative ranking after the preliminary races.


