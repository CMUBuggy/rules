\chapter{General Safety Requirements and Procedures}

\section{Buggy Safety}

\subsection{Safety Inspection Requirements}

	The requirements for the safety inspection of buggies shall be as follows:
	\newline

	Each buggy must successfully complete a safety inspection each school semester,
	before it shall be permitted to participate in any freeroll practice, push
	practice, or Sweepstakes race during that semester at Carnegie Mellon
	University. The purpose of this safety inspection is to determine if the buggy
	being inspected complies with all of the Buggy Construction and Performance
	Requirements specified by these Rules, Regulations, and Procedures. Each
	buggy's sponsoring organization has the burden of proof to demonstrate that
	that buggy meets or exceeds ALL of these requirements.

	There shall be no limit to the number of times that a buggy may try to
	successfully complete a safety inspection.

	Safety inspections shall be conducted by the Safety Chairman or anyone
	designated by the Sweepstakes Advisor.

	Any buggy that is involved in any accident during a freeroll practice or push
	practice, may be required to again successfully complete a safety inspection,
	if the Sweepstakes Chairman or the Safety Chairman, so requests, before that
	buggy shall be permitted to participate in any future practice sessions at
	Carnegie Mellon University.

	Any buggy that is modified in any way which might affect the safety of that
	buggy after it has successfully completed a safety inspection, must
	successfully complete an additional safety inspection before it shall be
	permitted to participate in any freeroll practice, push practice, or
	Sweepstakes race at Carnegie Mellon University. Any buggy found to be
	participating in any type of practice session or race competition after a
	safety related modification, and before this additional safety inspection,
	shall be penalized by being banned from any further participation in any type
	of practice sessions and by being disqualified from all race competition for
	that entire school year.
	\newline

	\noindent Modifications considered to affect the safety of a buggy include, but are
	not limited to, the following:
	
	\begin{itemize}
		\item Changing the configuration of the braking system.
		\item Changing the driver's personal protection equipment, such as modifying or
				changing the helmet, or changing the attachment method or construction of the
				safety harness.
		\item Changing the driver's protective cage.
		\item Any change that could affect the driver's field of vision, such as a change to the shell or windscreen.
		\item Changing the configuration of the steering system.
		\item Changing the diameter of any wheel or tire by more than 0.5 inches.
		\item Changing the position of any wheel relative to the frame or the other wheels.
		\item Changing the tire on any wheel from a solid tire to a pneumatic tire or vice versa.
		\item Changing the attachment method for any wheel.
		\item Adding or removing an aerodynamic covering or shroud to any wheel.
		\item Changing the attachment method or transparency of any windscreen.
	\end{itemize}

	\noindent Any buggy that varies its configuration while it is on the buggy course, must
	be capable of successfully completing a safety inspection, in any of its
	different configurations. These variations can include, but are not limited to,
	the following:

	\begin{itemize}
		\item Changing any wheel position relative to the frame, the other wheels, or the center of mass of the buggy.
		\item Changing the number of wheels in contact with the ground.
		\item Changing the position of the driver relative to the frame of the buggy.
		\item Changing the position of any part of the pushbar relative to the frame of the buggy.
	\end{itemize}

\subsection{Safety Inspection Procedures}

	Each safety inspection shall consist of two parts, the design inspection and
	the performance demonstration.

\subsubsection{Design Inspection}

	The design inspection shall be a demonstration to the inspector that the buggy
	meets or exceeds all of the buggy construction requirements specified by these
	Rules, Regulations, and Procedures.
	\newline

	\noindent Each design inspection shall be conducted as follows:

	\begin{itemize}

		\item
		A representative of the organization desiring to have a buggy design inspected
		shall contact the Safety Chairman and make an appointment for that inspection.

		\item
		At the appointed time and place, the Safety Chairman shall perform a design
		inspection of the organization's buggy. This design inspection shall be
		performed in accordance with the Design Inspection Checklist provided and kept
		by the Safety Chairman. A copy of the Design Inspection Checklist may be
		obtained from the Sweepstakes Advisor through the Safety Chairman.

		\item
		Representatives of each buggy's sponsoring organization shall demonstrate at
		that buggy's design inspection, that the buggy complies with all of the
		construction criteria specified in the Buggy Construction and Performance
		Requirements section of these Rules, Regulations, and Procedures. This
		demonstration can be either theoretical or experimental, i.e. the design
		features of a buggy may be shown by engineering analysis, drawings, and
		calculations, or they may be shown by an actual demonstration of those
		features, such as actually showing the inspector the buggy and its hardware and
		how it operates. In either case the buggy under inspection must be presented to
		the inspector for actual viewing.

		\item
		Successful completion of a prior design inspection by a buggy shall NOT be
		considered as a contributing factor in evaluating a buggy during a design
		inspection. Each buggy MUST comply with all of the CURRENT construction
		requirements in order to successfully complete a design inspection.

		\item
		A driver that fits in each buggy being design inspected shall be present during
		the inspection in order to demonstrate the adequacy of the driver's personal
		protection features of the buggy, and the fit of the driver inside the driver's
		protective cage.

		\item
		The inspector shall permanently record on the Design Inspection Checklist any
		and all information pertinent to the condition of the buggy being inspected.
		The inspector shall also record some identifiable feature or marking that is
		on, or is a part of, a permanent structure of that buggy. This feature or
		marking will be used during any subsequent safety inspections, spot safety
		checks, braking capability tests, or field of vision tests to uniquely identify
		that buggy. This information shall be kept on file by the Sweepstakes Advisor
		for future reference, as long as that particular buggy is active in Sweepstakes
		practice sessions or races. This information shall be used by present and/or
		future Safety Chairmen and design inspectors to evaluate changes made to a
		buggy after its initial design inspection.

	\end{itemize}

\subsubsection{Performance Demonstration}

	The performance demonstration shall be a demonstration to the inspector that
	the buggy can successfully complete the field of vision test, the braking
	capability test, and the drop brake test specified in the Buggy Construction
	and Performance Requirements section of these Rules, Regulations, and
	Procedures. It shall also be an on-going demonstration to the Safety Chairman,
	of the buggy's actual performance during both freeroll and push practices.
	\newline

	\noindent The performance demonstration portion of the safety inspection shall consist of
	the following:

	\begin{itemize}	

		\item
		Successful completion of the braking capability test specified in the Buggy
		Construction and Performance Requirements section of these Rules, Regulations,
		and Procedures. The braking capability test shall be conducted at a time and
		placed arranged by appointment with the Safety Chairman or anyone designated by
		that Chairman.

		\item
		Successful completion of the field of vision test specified in the Buggy
		Construction and Performance Requirements section of these Rules,Regulations,
		and Procedures. The field of vision test can usually be performed at the same
		time and place as the braking capability test.

		\item
		Successful completion, when required, of the drop brake test specified in the
		Buggy Construction and Performance Requirements section of these Rules,
		Regulations, and Procedures. The drop brake test shall be conducted whenever
		required before freeroll practices, after races, or during spot safety checks.
		The Safety Chairman may require an additional design inspection and braking
		capability test if a buggy frequently fails the drop brake test on its first
		attempt of the day.

		\item
		An on-going demonstration that the buggy's steering and control systems can be
		operated in a stable manner. The stability and performance of a buggy's
		steering and control systems and of that buggy's driver shall be observed by
		the Safety Chairman, and/or anyone designated by that Chairman, whenever that
		buggy is participating in any type of practice session at Carnegie Mellon
		University.

		If the Safety Chairman determines that a buggy and/or its driver is not stable,
		that buggy and/or driver shall not be permitted to participate in any
		subsequent freeroll practices or push practices until the cause of that
		instability has been eliminated. The Safety Chairman may require an additional
		design inspection or may ban a buggy from any future practices or races if that
		buggy consistently exhibits erratic control or performance at any practice
		session.

		\item
		An on-going demonstration of the buggy's static and dynamic lateral stability
		during freeroll and push practices (i.e. the buggy must demonstrate by its
		performance during practices that it is laterally stable while it is both at
		rest and in motion, and that it can be used during practice sessions without
		tipping over, rolling over, or having one or more wheels lift off the ground
		while it is cornering).

		The static and dynamic lateral stability of a buggy shall be observed by the
		Safety Chairman, and/or anyone designated by that Chairman, whenever that buggy
		is participating in any type of practice session at Carnegie Mellon University.
		If the Safety Chairman determines that the static or dynamic lateral stability
		of a buggy is not adequate (due to tipping over, rolling over, or having one or
		more wheels lift off the ground when cornering), that buggy shall not be
		permitted to participate in any subsequent freeroll practices or push practices
		until the cause of that instability has been eliminated. The Safety Chairman
		may require an additional design inspection, or may ban a buggy from any future
		practices or races if that buggy continues to exhibit erratic static or dynamic
		lateral stability at any practice session.

	\end{itemize}

\subsection{Spot Safety Checks}

	At any time during any type of buggy practice session at Carnegie Mellon
	University, the Sweepstakes Chairman, Assistant Chairman, Safety Chairman, or
	anyone designated by the Sweepstakes Advisor, may perform a spot safety check
	on any buggy participating in that practice session. A spot safety check shall
	consist of any or all of the following:

	\begin{itemize}

		\item Performance of a drop brake test to determine if the buggy's braking
		system is working properly .

		\item A check of the buggy driver's required head protection, including the
		manner in which the driver's helmet is secured in order to prevent it from
		obscuring the driver's vision.

		\item A check of the buggy driver's required safety harness.

		\item A check of the buggy driver's required eye protection, including goggles
		and how they are secured, windscreen, and where required, an interior
		windscreen.

		\item A check of the buggy driver's required gloves.

		\item A check of the buggy's lights and reflectors, when they are required,
		such as during after dark drop brake tests and push practices.

		\item A check of the general condition of the buggy to ensure that it is in as
		good, or better condition, in regards to safety related items, as it was during
		its safety inspection, and that no modifications have been made to the buggy
		since its inspection which could reduce its safety.

	\end{itemize}

	Any buggy found to be out of compliance with any applicable safety rules and
	regulations during a spot safety check shall not be permitted to participate in
	any type of practice session for the remainder of that day and that buggy's
	sponsoring organization shall be fined the amount of \$25.00 for the first
	occurrence and \$50.00 for any additional occurrences by that organization
	during that school year. If a buggy is found to be out of compliance with any
	applicable safety rules and regulations during a spot safety check on more than
	one occasion during any one school year, that buggy shall not be permitted to
	participate in any type of practice session or Sweepstakes race for the
	remainder of that school year.

\section{Driver Safety}

\subsection{Education Program}

	Each buggy driver MUST participate in a driver education program before they
	will be permitted to drive a buggy during any type of buggy practice session at
	Carnegie Mellon University. The driver education program shall consist of the
	following:
	\newline

	Before any freeroll practice takes place during the fall semester of each
	school year, a meeting of all of the people who want to be buggy drivers shall
	be held. This meeting shall be conducted by the Sweepstakes Chairman, the
	Assistant Chairman, the Safety Chairman, and/or anyone designated by the
	Sweepstakes Chairman. Buggy Chairmen from individual organizations shall not be
	permitted to attend the first at least a portion (usually the first hour) of
	this meeting unless they are also buggy drivers.

	Topics that shall be discussed at this meeting shall include, but not be
	limited to, the following:

	\begin{itemize}
		\item Potential dangers of buggy driving.
		\item Rules, Regulations, and Procedures relative to driving a buggy.
		\item Driver's personal safety and protective equipment.
		\item Driver training methods.
		\item Driving techniques for both normal and emergency situations.
		\item Use of signal flags, including the required chute flaggers and the yellow colored emergency flag. 
	\end{itemize}

	In addition to the discussions at this meeting, each potential driver shall be
	given a written summary of the topics covered during this meeting.

	Topics that shall be discussed at the spring semester driver's education
	meeting shall include, but not be limited to, all of the same topics covered
	during the fall semester buggy driver's education meeting.

	Approximately one week before the preliminary races are scheduled each school
	year, a meeting of all of the people who are to drive buggies during those
	races shall be held. This meeting shall be conducted by the Sweepstakes
	Chairman, the Assistant Chairman, the Safety Chairman, and/or anyone designated
	by the Sweepstakes Chairman. Buggy Chairmen from individual organizations will
	usually, but not always, be permitted to attend all of this meeting. Topics
	that shall be discussed at this meeting shall include, but not be limited to, a
	review of all of the topics covered during the previous buggy driver's
	education meetings, plus a detailed discussion and review of race day
	procedures and regulations, and driving techniques that can be employed during
	the races.

	Additional driver education meetings may be held at the discretion of the
	Sweepstakes Advisor, the Sweepstakes Chairman, or the Safety Chairman.

	Attendance at ALL of the required driver education meetings is MANDATORY for
	anyone who wants to drive a buggy at Carnegie Mellon University. If a person
	who wants to be a buggy driver fails to attend any of the required meetings, he
	or she must submit a request (either written or verbal) for a review of the
	content of each missed meeting to the Sweepstakes Chairman. This request must
	state the reason that each meeting in question was missed. The Sweepstakes
	Chairman shall then require that person to review the content of each missed
	meeting with someone that he or she designates.

\subsection{Liability Waiver}

	Each potential buggy driver must sign a Waiver of liability form (this is
	usually done at the buggy driver's education meeting). No person shall be
	permitted to drive a buggy at any type of practice session or Sweepstakes race
	at Carnegie Mellon University, until they have signed this form. These waiver
	forms shall be kept on file by the Sweepstakes Advisor. A copy of this waiver
	form can be obtained from the Sweepstakes Advisor through the Safety Chairman.
	If the buggy driver is less than 18 years of age, that driver's parent or legal
	guardian must also sign that waiver form.

\subsection{Pass Test}

	Each buggy driver must successfully complete a pass test during a spring
	semester freeroll practice session in each buggy that he or she will be driving
	during the Sweepstakes races before he or she will be permitted to compete in
	those races. The purpose of this pass test is to demonstrate that each driver
	has certain minimum driving skills while driving the buggy that they will be
	racing in. The procedure for the pass test shall be as follows:
	\newline

	The test shall take place during a freeroll practice session while the driver
	and buggy being tested are freerolling on the buggy course. Arrangements to
	have the test observed by a qualified observer must be made by the organization
	taking the test with the Safety Chairman, or anyone designated by that
	Chairman, prior to the test.
	\newline

	\noindent The test shall be conducted as follows:

	\begin{itemize}

		\item A buggy and driver not taking the test shall be allowed to freeroll
		down the buggy course. If the organization being tested has more than one
		buggy, it must use one of its own buggies as the buggy being passed. If it only
		has one buggy, it may use a buggy from another organization, but that buggy may
		only be a buggy that is already qualified for the races, i.e. that buggy must
		have already been driven down the buggy course at least five times and have
		successfully completed its own pass test.

		\item The buggy and driver taking the test shall be allowed to freeroll
		down the buggy course after the first buggy, at a speed that is greater than
		that of the first buggy, so that the buggy and driver taking the test may
		eventually overtake the first buggy.

		\item The buggy and driver taking the test shall pass the first buggy and
		driver somewhere between the George Westinghouse Memorial Pond and the service
		driveway that extends eastward onto Flagstaff Hill from Frew Street,
		approximately halfway between Schenley Drive and Scaife Hall.

		\item The test shall be observed by the Safety Chairman, or anyone
		designated by that Chairman. This observer shall ride in a follow car
		immediately behind the two buggies participating in the test.

	\end{itemize}

	\noindent To successfully complete the test, the following requirements must be met:

	\begin{itemize}

		\item The test observer must agree that the pass was adequate, based on
		that observer's judgment and experience.

		\item The buggy being passed must be moving at a reasonable speed, as
		determined by the test observer.

		\item The pass must take place somewhere between the George Westinghouse
		Memorial Pond and the service driveway that extends eastward onto Flagstaff
		Hill from Frew Street, approximately halfway between Schenley Drive and Scaife
		Hall.

		\item The pass must demonstrate the skill of the passing driver (i.e. the
		passing buggy must come up behind the slower buggy, smoothly pull out to one
		side of it, pass it, and then smoothly pull over in front of the passed buggy).
		The passing buggy must appear to be in control at all times during the pass and
		must not exhibit any instability attributable to either the driver or the
		buggy. The driver of the passing buggy must leave adequate clearance when
		passing the slower buggy and must return to his or her desired course smoothly
		after the pass has been completed. The manner in which the pass is made must
		demonstrate to the observer that the passing driver is able to adequately judge
		the position of his or her buggy relative to the buggy being passed.

	\end{itemize}


