\section{Fire Safety Rules}

% DRAFT 2/20/2012

In order to reduce the possibility of accidents or injuries in the areas in which the buggies are prepared for the races, these areas shall be randomly inspected both before and during the Sweepstakes races by the Sweepstakes Fire Marshal, the Safety Chairman, or anyone else designated by the Dean of Student Affairs or the Sweepstakes Advisor.  Any organization found to have unsafe or dangerous conditions in their buggy preparation area shall be issued a written statement indicating the cause of the hazard, the level of hazard that exists, and any necessary actions to mitigate this hazard.

As the nature of any danger in the buggy preparation area will vary, the Fire Marshal and Safety Chairman will work together to determine the level of danger that exists in the buggy preparation area, pursuant to the table below. In determining the level of hazard, the Fire Marshal and Safety Chairman should consider the details of the case at hand, including intent, immediacy of danger, and any prior rule violations within the last 3 years by the organization in question.

\begin{table}[h]
\begin{tabular}{|p{2cm}|p{5cm}|p{5cm}|}
  \hline
  & Violation & Penalty \\
  \hline
  Minor & Greater than  4 oz of  Class II combustible liquid OR other small hazards (e.g. trip hazards, etc) & Remedial safety training \\
  \hline
  Moderate & Greater than  4 oz of  Class II combustible liquid OR other small hazards (e.g. trip hazards, etc) in a compromising situation that would result in injury & Immediate disqualification from all Sweepstakes activities for the remainder of the year, Remedial safety training, Probationary Status for next year of racing \\
  \hline
  Severe & Liquids not found in their original container, any quantity of flammable material (all Class I liquids), excessive quantities of dangerous or combustible material,  fire or other injury likely & Immediate disqualification from all Sweepstakes activities for the remainder of the year, Remedial safety training, 15-month suspension from all Sweepstakes activities \\
  \hline
\end{tabular}
\end{table}

Organizations should keep in mind that every violation will be unique, and violations that appear similar at first glance may be quite different upon closer inspection.  For example, if an organization has a hazard that is deemed a minor violation one day of racing has the same hazard present the next day, they may find the hazard is now considered a moderate violation.  The fire marshal and safety chairman may reach this decision because of the pattern of disregard for safety, the reason for having the hazard, or other factors.

An organization that is disqualified from all Sweepstakes activities for the remainder of the year shall be disqualified from all of the men’s races, all of the women’s races, the design competition, shall be ineligible for any award that year, and shall be prohibited from participating in any exhibition races.
