\section{Yearly Revisions}

This appendix is meant to hold changes to the Bylaws that should be revisited and revised annually. These are meant to be temporary changes, or those that require tuning and rapid iteration. 

\subsection*{All Gender Division} \label{All Gender}

The All Gender Division was designed to help bring more students to the sport of buggy, who might have been excluded otherwise. The following regulations should be reviewed annually to ensure that they do not remain stagnant while the format of these races stabilizes. 

Organizations will be limited to entering 1 team in the All Gender Division, although this may be adjusted as the division matures and more time is allocated for the event. 

The rosters of these teams should be considered open. Neither teams nor Sweepstakes should regulate rosters based on sex, gender, appearance, or similar. In the case of Greek teams, they may draw pushers from the general student body and are not limited to the members of their chapter. 

Since we don't have historical data for this division, seeding order for heats will use a blended format. If there is historical data, this will be used, otherwise, an average of the men's and women's team times will be used as a substitute. The seeding can then be determined in the same format as 10.4.4

In order to ensure this division does not exclude students from pushing, Sweepstakes will also collect pusher division preferences with final rosters. If an organization chooses to not run an all gender team, this preference sheet will not be required. If an organization does run an all gender team, their final roster and pusher division preferences must meet the following requirements: 
\begin{itemize}
    \item If 5 or fewer students rank the all gender division as their first choice division, these students must be run in the all gender division 
    \item If fewer than 5 students rank the all gender division as their first choice, the remaining roster should be filled in such a way that accommodates a student's highest preference. So the roster should use those who ranked the all gender division second before those who ranked it third, before those who chose not to rank it. 
    \item If more than 5 students rank the all gender division as their first choice, all students on the all gender team should come from this set of students. Teams are free to pick from among these students however they see fit and may then place the remaining students not on the all gender team in accordance with their division preferences. 
\end{itemize}


Sweepstakes has the option of providing a warning to teams if there is a roster violation, however the rosters actually run on Raceday must meet the above requirements. 

The entries placing in the top three final standings of the All Gender races shall receive awards. 

For now, the All Gender Division will have no fee, as long as that organization has at least one entry in the Men's or Women's divisions. If they do not, the entry fee will be \$50.75. 

\subsection*{Greek Rosters} \label{Greek}
For Raceday 2023, organizations should strive to have their push teams limited to pledges and initiated members of that chapter and must be, or have been, listed on an IFC membership roster or Panhellenic roster. However, organization may request an exemption for one or more of their push teams. These teams must still have 80\% of their team come from their pledge or membership roster. Exemption requests will be evaluated by Sweepstakes and are due at the same time as preliminary rosters. 

\subsection*{Highmark Center Construction} \label{Seeding Area}
Each organization participating in the Sweepstakes races shall be permitted to select an area near the starting line of the buggy course, in which they may prepare their buggies prior to the races. These areas may be occupied by trucks or any other non-permanent enclosures. While truck space is limited by the development of the Highmark Center for Health, Wellness, and Athletics (HCHWA), the picking order will first be determined by the number of buggies an organization is planning to race that year, with the picking order then determined by their seeding order for preliminary races. The selection of these areas should be made at least four weeks prior to the date scheduled for the preliminary races. The selection process shall be supervised by the Sweepstakes Chairman, or anyone designated by that Chairman. The Sweepstakes chairman will assign spaces to organizations at least 2 weeks prior to the date schedule for the preliminary races. No organization may use or attempt to use any area that has been selected by another organization on any day of Sweepstakes racing.

\subsection*{Initial Equipment Retirement Dates}
If a team is able to supply proof of purchase for their safety gear, that date will be used to calculate the 5 year retirement date. Otherwise that piece of equipment will be tagged to retire after Raceday 2024