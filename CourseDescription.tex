\chapter{Course Description}

	The race shall be held on the public streets of the Carnegie Mellon University campus and of Schenley Park. The course starts near the intersection of Margaret Morrison and Tech Streets and proceeds south along Tech Street, past the intersection of Tech and Frew Streets, and onto Schenley Drive. The race shall then proceed west along Schenley Drive (passing the George Westinghouse Memorial Pond, the Panther Hollow bridge, and through the area between the Edward Manning Bigelow monument and Phipps Conservatory), turns toward the northeast onto Frew Street, and then east along Frew Street to the finish line. Maps showing the race course and all of its boundaries and zones can be obtained from the Sweepstakes Advisor through the Sweepstakes Chairman.

\section{Lines}

	The zones and lanes of the course shall be delineated by lines painted onto the course. These lines shall be not less than 2 inches nor more than 6 inches in width and be of a color that is easily visible, usually white. All measurements made to these lines shall be made to their centerlines, except for the starting and finishing lines. The location of the starting line shall be measured to the edge that is farthest away from Hill 1 and the location of the finishing line shall be made to the edge that is closest to Hill 5.

\section{Lanes}

	There shall be three lanes painted on the course from the starting line to a line that is approximately 58 feet (as measured along the curve of the curb) east of the centerline of the gravel footpath which runs from the southern edge of Schenley Drive to the northern edge of Circuit Road. (As a second reference point, this line is also approximately 49 feet east of the centerline of the eastern-most cross country course marker pole on the southern side of Schenley Drive.)

	The lanes shall be 7 feet wide. Each lane line shall be between 3.0-3.5 inches thick. A single line, perpendicular to Schenley Drive, shall mark the ends of the lanes. The lane farthest to the west shall be designated as Lane 1, then Lane 2 in the center, and Lane 3 in the east. (ordered 3, 2, 1, from left to right). The western (right) edge of Lane 1 shall be parallel to and 6 feet 6 inches away from the curb on the western (right) side of Tech Street. Lanes 1 and 2 shall have a common central edge, as will Lanes 2 and 3.

	The western (right) edge of Lane 1 shall be 8 feet to the south of the northern curb of Schenley Drive. The western (right) edge of Lane 1 shall be 4 feet to the east (left) of the western (right) curb of Tech Street at its closest approach to that curb, which should be approximately halfway between Frew Street and Schenley Drive.

\section{Zones}

	The course shall be divided into five pushing zones and three transition zones. Each successive pushing zone shall be labeled Hill 1, Hill 2, Hill 3, Hill 4, and Hill 5 in the order each is reached during a race. Each transition zone shall start at the previous pushing zone's end line and extend to an end line parallel to and 45 feet from the end line of that pushing zone. The Transition Zones shall be labeled Hill 1-2 Transition Zone, Hill 3-4 Transition Zone, and Hill 4-5 Transition Zone in that order. The course shall start at a line known as the starting line, and end at a line known as the finish line.

\subsection{Starting Line}

	The starting line for Lane 1 shall be perpendicular to Tech Street and 12 feet 4 inches north of a line formed by extending the northern-most face of Skibo Gymnasium. The starting line for Lane 2 shall be 6 feet south of lane 1, and the starting line for Lane 3 shall be 12 feet south of Lane 1.

\subsection{Hill 1}

	Hill 1 is a pushing zone that shall extend from the starting lines to lines perpendicular to Tech Street and approximately 410 feet south of the starting lines. The end lines for the first pushing zone in each of the three lanes shall be located as follows:

	\begin{itemize}
	
		\item
		The end line in Lane 1 shall be perpendicular to Tech Street and 32 feet 3 inches north of a line formed by extending the southern-most face of Skibo Gymnasium.

		\item
		The end line in Lane 2 shall be parallel to, and 6 feet to the south of, the end line in Lane 1.
		
		\item
		The end line in Lane 3 shall be parallel to, and 12 feet to the south of, the end line in Lane 1

	\end{itemize}

\subsubsection{Hill 1-2 Transition Zone} 


\subsection{Hill 2}

	Hill 2 is a pushing zone that shall start at the end lines for the first transition zone, and extend from those lines and proceed along the course to the area between the Edward Manning Bigelow monument and the sidewalk in front of Phipps Conservatory. The second pushing zone shall end at a line that is perpendicular to Schenley Drive and which passes through the center of the Edward Manning Bigelow monument. The end of the second pushing zone shall not be marked by a line painted on the course.

\subsection{Hill 3}

	Hill 3 is a pushing zone that shall start at the end of the second pushing zone and shall extend from that location and proceed along the course (northwest along Schenley Drive and north and east along Frew Street) to a line that is perpendicular to Frew Street and located 60 feet east of the centerline of the western-most doorway, that is on the southern side of Porter Hall.

\subsubsection{Hill 3-4 Transition Zone} 


\subsection{Hill 4}

	Hill 4 is a pushing zone that shall start at the end line of the second transition zone and extend from that line and proceed along the course (east along Frew Street) to a line that is perpendicular to Frew Street and located 60 feet west of a line formed by extending the eastern-most face of Baker Hall.

\subsubsection{Hill 4-5 Transition Zone} 

\subsection{Hill 5}

	Hill 5 is a pushing zone that shall start at the end line of the third transition zone and extend from that line and proceed along the course (east along Frew Street) to the finish line.

\subsection{Finish Line}

	The course shall end at a line known as the finish line. The finish line shall be located on Frew Street. It shall be perpendicular to Frew Street and located 5 feet 6 inches west of the eastern-most face of the brick wall surrounding the window wells on the southern side of the Hall of Arts (HoA) building. As a second reference, this line is 19 feet 6 inches west of the first corner of the HoA building that is east of the window well wall.
